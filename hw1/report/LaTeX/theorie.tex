\section{Theoretical background}
\subsection{GMRES}
The GMRES abbreviates generalized minimal residuals. It's idea is to solve $Ax = b$ by using a vector
$x_n \in \mathcal{K}_n$ that minimizes the residual $r_n = b - Ax_n$ \footnote{Numerical Linear Algebra, Trefethen, Bau page 266}.
In general the process is implemented using Arnoldi-iterations to compute the Krylov subspace and the Hessenberg representation of the
original problem. The least square problem is then solved by coupling a QR decomposition into the Arnoldi iterations. 

\subsection{BICGSTAB}
The Bi-CGSTAB, method is a modified version of the biconjugate gradient method. The biconjugate gradient method uses recurrences which only
require data from the last iteration, which keeps memory requirements under control. The better memory usage comes at the price of lost monotonic convergence. The original BCG algorithm does not converge nicely. The main added benefit of the Bi-CGSTAB algorithm is better convergence behavior. 
A table containing the most important methods is given below\footnote{ as found in numerical linear algebra, Trefethen, Bau page 235.}.

\begin{table}
\centering
\begin{tabular}{|l|c|r|} \hline
   						    & $Ax = b$ & $Ax = \lambda x$ \\ \hline
$A = A*$ (hermitian)        & CG	   &  Lanczos		   \\ \hline
$A \neq A*$ (non-hermitian) & GMRES, CGN, BCG et al. & Arnoldi \\ \hline
\end{tabular}
\caption{Krylov subspace methods}
\label{tab:KrySub}
\end{table}


\subsection{Preconditioning}
The convergence of iterative methods depends on the properties of the matrix A, such as the position of it's eigenvalues or normality. It the spectrum is spread out or A is too far from normal iterative schemes might fail to converge. To fix such problems instead of solving:
\begin{equation}
Ax = b.
\end{equation}
A matrix $M^{-1}$, the preconditioner is multiplied from the left to both sides of the equation \footnote{Numerical linear algebra, Trefethen, Bau page 313.}:
\begin{equation}
M^{-1} Ax = M^{-1}b. 
\end{equation} 
The resulting system $M^{-1}A = B$ should be close to normal $(\bar{B}^TB \approx B\bar{B}^T)$ and have a clustured spectrum. 