\section{Abstract}
When solving $\|\mathbf{Ax} - \mathbf{b}\|$ for very large $\mathbf{A}$
the memory requirements do not allow it to let the \texttt{GMRES} algorithm iterate until convergence, as memory and computational cost increases with each step. Therefore a modified version of \texttt{GMRES} called restarted \texttt{GMRES(m)} has been proposed by Saad and Schultz in their original paper.\footnote{A Generalized minimal residual algorithm for solving nonsymmetric linear systems, SIAM J. Sci. Stat. Comput. Vol 7, No.3 1986} After a fixed amount of iterations \texttt{m} the algorithm is restarted using $\mathbf{x}_m$ as initial condition for the next run. This changed version  avoids memory problems at the cost of convergence. One would expect that the larger $m$ is chosen the better the convergence properties of the restarted algorithm would be, since without restart when the largest possible \texttt{m} is chosen convergence is guaranteed. However this is not always the case. In this report curious behavior of \texttt{GMRES(m)} will be investigated following the paper \textquotedblleft The Tortoise and the Hare, Restart GMRES\textquotedblright   by Marc Embree.\footnote{SIAM Rev., 45(2), 259–266. (8 pages) The Tortoise and the Hare Restart GMRES, Marc Embree,
\url{http://epubs.siam.org/doi/abs/10.1137/S003614450139961}}
