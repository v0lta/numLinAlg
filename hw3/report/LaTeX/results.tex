\section{Results}
\subsection{Tikhonov regularization}
Tikhonov regularization can be implemented using the singular value decomposition of $\mathbf{A}$. The idea is to filter out the very small singular values, which are greatly influenced by the noise contribution using
\begin{equation}
\mathbf{x}_{reg} = \sum_{i=1}^{n}f_i \frac{\mathbf{u}_{i}^T b}{\sigma_i}\mathbf{v}_{i} .
\label{eq:tikh}
\end{equation}
The filter factors $f_i$ are computed using
\begin{equation}
f_i = \sigma_i^2/(\sigma_i^2 + \lambda).
\label{eq:filter}
\end{equation}
Where filtering takes place if $\sigma_i < \lambda$. On the other hand a singular value $\sigma$ remains unfiltered if $\sigma_i > \lambda$. 
The equations above can be implemented without a \texttt{for} loop in matlab:
\begin{lstlisting}[language=matlab]
     [U,S,V] = svd(A)
     sigma = diag(S);
     fVec = sigma.^2 ./ ( sigma.^2 + ones(n,1)*lambda.^2);
     F = diag(fVec);
     x = sum(U*F*diag(sigma.^(-1))*V'*b,2);
\end{lstlisting}
For this method to work it is essential to choose a good value for $\lambda$. 
\subsubsection{The L-curve}
The L-curve is a plot of the residual norm $\|\mathbf{Ax} - \mathbf{b}\|$ and the norm of the regularized solution $\|\mathbf{Lx}\|$. It displays the trade off between the fit to regularized data and the size of the solution and gives insight into the properties of the underlying regularization method.\footnote{The L-curve and its use in the numerical treatment of inverse problems, P. C. Hansen, Department of Mathematical Modelling, Technical University of Denmark, DK-2800 Lyngby, Denmark} Using Tikhonov regularization as described in equation~\ref{eq:tikh} and the filter of equation~\ref{eq:filter}. The plots in figure~\ref{fig:A1LTihk} have been computed. 
\begin{figure}
\centering
% This file was created by matlab2tikz.
% Minimal pgfplots version: 1.3
%
%The latest updates can be retrieved from
%  http://www.mathworks.com/matlabcentral/fileexchange/22022-matlab2tikz
%where you can also make suggestions and rate matlab2tikz.
%
\documentclass[tikz]{standalone}
\usepackage{pgfplots}
\usepackage{grffile}
\pgfplotsset{compat=newest}
\usetikzlibrary{plotmarks}
\usepackage{amsmath}

\begin{document}
\definecolor{mycolor1}{rgb}{0.00000,0.44700,0.74100}%
\definecolor{mycolor2}{rgb}{0.85000,0.32500,0.09800}%
\definecolor{mycolor3}{rgb}{0.92900,0.69400,0.12500}%
%
\begin{tikzpicture}

\begin{axis}[%
width=2in,
height=2in,
at={(2.5in,0in)},
scale only axis,
xmin=0,
xmax=1,
ymin=-1,
ymax=2.5
]
\addplot [color=mycolor1,solid,forget plot]
  table[row sep=crcr]{%
0	0.641724596964442\\
0.00334448160535117	0.168786586549343\\
0.00668896321070234	-0.210419101060613\\
0.0100334448160535	-0.483188648345283\\
0.0133779264214047	-0.654208081277431\\
0.0167224080267559	-0.738179501105754\\
0.020066889632107	-0.754496025687768\\
0.0234113712374582	-0.723549852143192\\
0.0267558528428094	-0.664319923081695\\
0.0301003344481605	-0.59294146706583\\
0.0334448160535117	-0.522009441895614\\
0.0367892976588629	-0.460411935244745\\
0.040133779264214	-0.413528205031025\\
0.0434782608695652	-0.383659630741644\\
0.0468227424749164	-0.370590772030897\\
0.0501672240802676	-0.372202375891921\\
0.0535117056856187	-0.385078923778416\\
0.0568561872909699	-0.405070546202618\\
0.0602006688963211	-0.427783226454606\\
0.0635451505016722	-0.448982525937754\\
0.0668896321070234	-0.464904933553135\\
0.0702341137123746	-0.472477693995045\\
0.0735785953177258	-0.469452907234952\\
0.0769230769230769	-0.454465094125293\\
0.0802675585284281	-0.427023548071434\\
0.0836120401337793	-0.387451873749542\\
0.0869565217391304	-0.336787361000271\\
0.0903010033444816	-0.276652442248315\\
0.0936454849498328	-0.209109599508408\\
0.096989966555184	-0.136509864956845\\
0.100334448160535	-0.0613436193440302\\
0.103678929765886	0.0138991620670624\\
0.107023411371237	0.0868536509251169\\
0.110367892976589	0.155385876009759\\
0.11371237458194	0.217676838705109\\
0.117056856187291	0.272283794413252\\
0.120401337792642	0.318178626012332\\
0.123745819397993	0.35476422530094\\
0.127090301003344	0.381870604001768\\
0.130434782608696	0.399733077533052\\
0.133779264214047	0.408955307285831\\
0.137123745819398	0.410460262072733\\
0.140468227424749	0.405432281214648\\
0.1438127090301	0.39525340739097\\
0.147157190635452	0.381437027014144\\
0.150501672240803	0.365561629120206\\
0.153846153846154	0.349207191190756\\
0.157190635451505	0.333896342454278\\
0.160535117056856	0.321042061297546\\
0.163879598662207	0.311903249800669\\
0.167224080267559	0.307549114348355\\
0.17056856187291	0.308832877216598\\
0.173913043478261	0.316374965510835\\
0.177257525083612	0.330555477850782\\
0.180602006688963	0.351515425635307\\
0.183946488294314	0.379165989703384\\
0.187290969899666	0.413204825845346\\
0.190635451505017	0.453138299800878\\
0.193979933110368	0.498308429804946\\
0.197324414715719	0.547923261067251\\
0.20066889632107	0.601089395239445\\
0.204013377926421	0.656845428933069\\
0.207357859531773	0.714195134658252\\
0.210702341137124	0.772139323369915\\
0.214046822742475	0.829705454684681\\
0.217391304347826	0.88597421617444\\
0.220735785953177	0.94010245194004\\
0.224080267558528	0.991341994618465\\
0.22742474916388	1.03905411443456\\
0.230769230769231	1.08271947754754\\
0.234113712374582	1.1219436474282\\
0.237458193979933	1.15645830731031\\
0.240802675585284	1.18611851569779\\
0.244147157190635	1.21089639388918\\
0.247491638795987	1.2308717377649\\
0.250836120401338	1.24622011060889\\
0.254180602006689	1.25719899542444\\
0.25752508361204	1.26413262144845\\
0.260869565217391	1.26739606015534\\
0.264214046822742	1.26739916824256\\
0.267558528428094	1.26457090891528\\
0.270903010033445	1.25934454437715\\
0.274247491638796	1.25214409797222\\
0.277591973244147	1.24337244418635\\
0.280936454849498	1.23340125944098\\
0.284280936454849	1.22256304538468\\
0.287625418060201	1.21114527257994\\
0.290969899665552	1.19938671372398\\
0.294314381270903	1.18747584400692\\
0.297658862876254	1.17555123535244\\
0.301003344481605	1.16370372664227\\
0.304347826086957	1.15198016943722\\
0.307692307692308	1.14038842375168\\
0.311036789297659	1.1289033794349\\
0.31438127090301	1.11747369064122\\
0.317725752508361	1.10602885268496\\
0.321070234113712	1.09448641899784\\
0.324414715719064	1.08275901113086\\
0.327759197324415	1.07076094841718\\
0.331103678929766	1.05841424128214\\
0.334448160535117	1.04565367575166\\
0.337792642140468	1.03243114856413\\
0.341137123745819	1.01871885676875\\
0.344481605351171	1.0045114413344\\
0.347826086956522	0.989827163417072\\
0.351170568561873	0.974707918965318\\
0.354515050167224	0.959218529622982\\
0.357859531772575	0.943444927002366\\
0.361204013377926	0.927492012233205\\
0.364548494983278	0.911480530437239\\
0.367892976588629	0.895543829986395\\
0.37123745819398	0.879824243576203\\
0.374581939799331	0.864469545065948\\
0.377926421404682	0.84962911946416\\
0.381270903010033	0.835450387872986\\
0.384615384615385	0.822075942230281\\
0.387959866220736	0.809640067241776\\
0.391304347826087	0.798266615022084\\
0.394648829431438	0.78806688929816\\
0.397993311036789	0.779137601244928\\
0.40133779264214	0.771559716161186\\
0.404682274247492	0.765397609642512\\
0.408026755852843	0.760698497587951\\
0.411371237458194	0.757491775161231\\
0.414715719063545	0.755789245994892\\
0.418060200668896	0.755584925205753\\
0.421404682274247	0.7568551012882\\
0.424749163879599	0.759558408040643\\
0.42809364548495	0.763635602579421\\
0.431438127090301	0.769010861305356\\
0.434782608695652	0.775590025819012\\
0.438127090301003	0.78326229886145\\
0.441471571906354	0.791899762916608\\
0.444816053511706	0.801357349573766\\
0.448160535117057	0.811473376813331\\
0.451505016722408	0.822070819317851\\
0.454849498327759	0.832958283915391\\
0.45819397993311	0.843930411893796\\
0.461538461538462	0.854771447442442\\
0.464882943143813	0.865256457390057\\
0.468227424749164	0.875155595804701\\
0.471571906354515	0.884238612790055\\
0.474916387959866	0.892277978616909\\
0.478260869565217	0.8990535585636\\
0.481605351170569	0.904362371379038\\
0.48494983277592	0.908016891561108\\
0.488294314381271	0.909861862936877\\
0.491638795986622	0.909770112516229\\
0.494983277591973	0.907656605792632\\
0.498327759197324	0.903480386481673\\
0.501672240802676	0.897247006317214\\
0.505016722408027	0.88902123881118\\
0.508361204013378	0.87892361272123\\
0.511705685618729	0.867128206470982\\
0.51505016722408	0.853874745298617\\
0.518394648829431	0.839455535830271\\
0.521739130434783	0.824213880024452\\
0.525083612040134	0.808554869163261\\
0.528428093645485	0.792908530440642\\
0.531772575250836	0.777751799311814\\
0.535117056856187	0.763572898257877\\
0.538461538461538	0.750889880344067\\
0.54180602006689	0.740213689555246\\
0.545150501672241	0.732039256495942\\
0.548494983277592	0.726852325896859\\
0.551839464882943	0.72510024475021\\
0.555183946488294	0.727184321478758\\
0.558528428093645	0.733456982263336\\
0.561872909698997	0.744210809658504\\
0.565217391304348	0.759653497387554\\
0.568561872909699	0.779955238580301\\
0.57190635451505	0.805183886020526\\
0.575250836120401	0.835331848728571\\
0.578595317725752	0.870341126210692\\
0.581939799331104	0.910065599424643\\
0.585284280936455	0.95428120348991\\
0.588628762541806	1.00273460421784\\
0.591973244147157	1.05506027135741\\
0.595317725752508	1.11087679006074\\
0.59866220735786	1.16976107896744\\
0.602006688963211	1.23119836661138\\
0.605351170568562	1.29467727406452\\
0.608695652173913	1.35962214595696\\
0.612040133779264	1.42538875674283\\
0.615384615384615	1.49136311862892\\
0.618729096989967	1.55683357467479\\
0.622073578595318	1.62101855189192\\
0.625418060200669	1.68324800741376\\
0.62876254180602	1.74259834284814\\
0.632107023411371	1.7983660865088\\
0.635451505016722	1.84964365607569\\
0.638795986622074	1.89566195782009\\
0.642140468227425	1.93565540373597\\
0.645484949832776	1.96887644722704\\
0.648829431438127	1.9947640689018\\
0.652173913043478	2.01276936988855\\
0.655518394648829	2.0226914391335\\
0.658862876254181	2.02443415479943\\
0.662207357859532	2.01815175567449\\
0.665551839464883	2.00431122162641\\
0.668896321070234	1.98362444223398\\
0.672240802675585	1.95724019757348\\
0.675585284280936	1.92645295556747\\
0.678929765886288	1.89295569518315\\
0.682274247491639	1.85873297376894\\
0.68561872909699	1.82548952343617\\
0.688963210702341	1.7956091399717\\
0.692307692307692	1.77108646600635\\
0.695652173913043	1.75369963019429\\
0.698996655518395	1.74530504832678\\
0.702341137123746	1.74676879575411\\
0.705685618729097	1.7591160243962\\
0.709030100334448	1.78203009090697\\
0.712374581939799	1.81520761457924\\
0.71571906354515	1.85713728667757\\
0.719063545150502	1.9063298030475\\
0.722408026755853	1.9596417311588\\
0.725752508361204	2.01450015881439\\
0.729096989966555	2.06805453730071\\
0.732441471571906	2.1165185920764\\
0.735785953177257	2.15713938039299\\
0.739130434782609	2.18783571763803\\
0.74247491638796	2.20645311688763\\
0.745819397993311	2.2128277515869\\
0.749163879598662	2.2074113629439\\
0.752508361204013	2.19150127650027\\
0.755852842809365	2.16753230076219\\
0.759197324414716	2.13975863561569\\
0.762541806020067	2.11158824889389\\
0.765886287625418	2.08728359787896\\
0.769230769230769	2.07211875749676\\
0.77257525083612	2.06758586754029\\
0.775919732441472	2.07673289708877\\
0.779264214046823	2.10010026291898\\
0.782608695652174	2.13581715163831\\
0.785953177257525	2.1813251456343\\
0.789297658862876	2.23261325571811\\
0.792642140468227	2.28342728234754\\
0.795986622073579	2.32737779315159\\
0.79933110367893	2.35762330784539\\
0.802675585284281	2.37228471039825\\
0.806020066889632	2.36751472517902\\
0.809364548494983	2.34134205896496\\
0.812709030100334	2.29811650823742\\
0.816053511705686	2.24078515507287\\
0.819397993311037	2.17871366875084\\
0.822742474916388	2.11581144507365\\
0.826086956521739	2.06099493689675\\
0.82943143812709	2.01823834302851\\
0.832775919732441	1.99535126014031\\
0.836120401337793	1.98229883087956\\
0.839464882943144	1.98366058008113\\
0.842809364548495	1.98520930261272\\
0.846153846153846	1.98519276983815\\
0.849498327759197	1.97601498467688\\
0.852842809364548	1.95450896360666\\
0.8561872909699	1.92384945482035\\
0.859531772575251	1.89016852716285\\
0.862876254180602	1.86533491593262\\
0.866220735785953	1.86277151650962\\
0.869565217391304	1.89904209603578\\
0.872909698996655	1.95676923908522\\
0.876254180602007	2.05421620740893\\
0.879598662207358	2.14874798391178\\
0.882943143812709	2.23376578297829\\
0.88628762541806	2.26173320071057\\
0.889632107023411	2.22754245636132\\
0.892976588628763	2.11993865645829\\
0.896321070234114	1.94247273755019\\
0.899665551839465	1.74536256977005\\
0.903010033444816	1.56633079515917\\
0.906354515050167	1.43966890642502\\
0.909698996655518	1.37297290192584\\
0.91304347826087	1.35620004432459\\
0.916387959866221	1.34794748021864\\
0.919732441471572	1.32182660660044\\
0.923076923076923	1.28200114456922\\
0.926421404682274	1.25623029880262\\
0.929765886287625	1.32482181258092\\
0.933110367892977	1.51101910606667\\
0.936454849498328	1.77945236186497\\
0.939799331103679	1.89859112375416\\
0.94314381270903	1.7605864959769\\
0.946488294314381	1.33248082920909\\
0.949832775919732	0.81198750063777\\
0.953177257525084	0.525320045650005\\
0.956521739130435	0.653105393052101\\
0.959866220735786	1.12365508079529\\
0.963210702341137	1.2295286655426\\
0.966555183946488	1.00660467147827\\
0.969899665551839	0.563481330871582\\
0.973244147157191	0.143348693847656\\
0.976588628762542	-0.223487854003906\\
0.979933110367893	-0.10284423828125\\
0.983277591973244	1.01129150390625\\
0.986622073578595	0.3909912109375\\
0.989966555183946	0.29736328125\\
0.993311036789298	0.66796875\\
0.996655518394649	-0.4375\\
1	-0.796875\\
};
\addplot [color=mycolor2,solid,forget plot]
  table[row sep=crcr]{%
0	-0.5817200624972\\
0.00334448160535117	-0.552136084881755\\
0.00668896321070234	-0.522869644302756\\
0.0100334448160535	-0.493921045928198\\
0.0133779264214047	-0.465290569931929\\
0.0167224080267559	-0.436978471036511\\
0.020066889632107	-0.408984978050614\\
0.0234113712374582	-0.381310293400917\\
0.0267558528428094	-0.353954592658499\\
0.0301003344481605	-0.326918024059692\\
0.0334448160535117	-0.300200708021393\\
0.0367892976588629	-0.273802736650805\\
0.040133779264214	-0.247724173249598\\
0.0434782608695652	-0.221965051812488\\
0.0468227424749164	-0.196525376520206\\
0.0501672240802676	-0.171405121226871\\
0.0535117056856187	-0.14660422894175\\
0.0568561872909699	-0.122122611305399\\
0.0602006688963211	-0.0979601480602068\\
0.0635451505016722	-0.0741166865153203\\
0.0668896321070234	-0.0505920410059758\\
0.0702341137123746	-0.0273859923472403\\
0.0735785953177258	-0.00449828728217516\\
0.0769230769230769	0.0180713620755576\\
0.0802675585284281	0.0403232788046294\\
0.0836120401337793	0.0622578217444915\\
0.0869565217391304	0.0838753860693087\\
0.0903010033444816	0.105176403867346\\
0.0936454849498328	0.12616134472577\\
0.096989966555184	0.146830716320829\\
0.100334448160535	0.167185065013359\\
0.103678929765886	0.187224976449565\\
0.107023411371237	0.20695107616703\\
0.110367892976589	0.22636403020588\\
0.11371237458194	0.245464545725038\\
0.117056856187291	0.264253371623502\\
0.120401337792642	0.28273129916656\\
0.123745819397993	0.300899162616856\\
0.127090301003344	0.318757839870221\\
0.130434782608696	0.336308253096159\\
0.133779264214047	0.353551369382896\\
0.137123745819398	0.370488201386857\\
0.140468227424749	0.387119807986469\\
0.1438127090301	0.403447294940148\\
0.147157190635452	0.419471815548328\\
0.150501672240803	0.435194571319397\\
0.153846153846154	0.450616812639369\\
0.157190635451505	0.465739839445125\\
0.160535117056856	0.480565001901065\\
0.163879598662207	0.495093701078952\\
0.167224080267559	0.509327389640775\\
0.17056856187291	0.523267572524409\\
0.173913043478261	0.536915807631851\\
0.177257525083612	0.550273706519802\\
0.180602006688963	0.563342935092336\\
0.183946488294314	0.576125214295415\\
0.187290969899666	0.588622320812953\\
0.190635451505017	0.60083608776416\\
0.193979933110368	0.612768405401846\\
0.197324414715719	0.624421221811374\\
0.20066889632107	0.635796543609926\\
0.204013377926421	0.64689643664573\\
0.207357859531773	0.657723026696869\\
0.210702341137124	0.668278500169293\\
0.214046822742475	0.678565104793627\\
0.217391304347826	0.688585150320339\\
0.220735785953177	0.698341009212832\\
0.224080267558528	0.707835117337982\\
0.22742474916388	0.717069974653636\\
0.230769230769231	0.726048145892554\\
0.234113712374582	0.734772261242265\\
0.237458193979933	0.743245017020264\\
0.240802675585284	0.751469176343967\\
0.244147157190635	0.759447569794821\\
0.247491638795987	0.767183096075904\\
0.250836120401338	0.774678722662364\\
0.254180602006689	0.781937486444002\\
0.25752508361204	0.788962494359234\\
0.260869565217391	0.795756924019716\\
0.264214046822742	0.802324024324798\\
0.267558528428094	0.808667116065002\\
0.270903010033445	0.814789592513635\\
0.274247491638796	0.820694920005662\\
0.277591973244147	0.826386638502869\\
0.280936454849498	0.831868362144355\\
0.284280936454849	0.837143779781317\\
0.287625418060201	0.84221665549508\\
0.290969899665552	0.847090829097237\\
0.294314381270903	0.851770216610766\\
0.297658862876254	0.856258810730909\\
0.301003344481605	0.860560681264567\\
0.304347826086957	0.864679975546893\\
0.307692307692308	0.868620918833744\\
0.311036789297659	0.872387814668568\\
0.31438127090301	0.875985045222263\\
0.317725752508361	0.879417071604466\\
0.321070234113712	0.882688434144712\\
0.324414715719064	0.885803752641772\\
0.327759197324415	0.888767726579486\\
0.331103678929766	0.891585135307284\\
0.334448160535117	0.894260838183545\\
0.337792642140468	0.896799774679873\\
0.341137123745819	0.899206964444264\\
0.344481605351171	0.901487507321111\\
0.347826086956522	0.903646583325862\\
0.351170568561873	0.905689452572088\\
0.354515050167224	0.90762145514865\\
0.357859531772575	0.909448010944502\\
0.361204013377926	0.911174619418666\\
0.364548494983278	0.912806859312727\\
0.367892976588629	0.914350388303172\\
0.37123745819398	0.915810942590744\\
0.374581939799331	0.917194336423904\\
0.377926421404682	0.918506461553384\\
0.381270903010033	0.91975328661468\\
0.384615384615385	0.920940856435253\\
0.387959866220736	0.922075291263056\\
0.391304347826087	0.92316278591289\\
0.394648829431438	0.924209608826981\\
0.397993311036789	0.925222101046013\\
0.40133779264214	0.926206675086728\\
0.404682274247492	0.927169813722085\\
0.408026755852843	0.928118068659778\\
0.411371237458194	0.929058059114813\\
0.414715719063545	0.929996470271664\\
0.418060200668896	0.930940051631378\\
0.421404682274247	0.931895615238847\\
0.424749163879599	0.932870033785265\\
0.42809364548495	0.933870238580675\\
0.431438127090301	0.934903217391274\\
0.434782608695652	0.935976012135996\\
0.438127090301003	0.937095716436719\\
0.441471571906354	0.938269473016207\\
0.444816053511706	0.939504470937758\\
0.448160535117057	0.940807942680286\\
0.451505016722408	0.942187161042375\\
0.454849498327759	0.943649435868644\\
0.45819397993311	0.945202110591539\\
0.461538461538462	0.946852558581428\\
0.464882943143813	0.948608179297706\\
0.468227424749164	0.950476394233315\\
0.471571906354515	0.952464642644929\\
0.474916387959866	0.954580377060755\\
0.478260869565217	0.956831058557691\\
0.481605351170569	0.959224151799341\\
0.48494983277592	0.961767119826125\\
0.488294314381271	0.964467418588475\\
0.491638795986622	0.967332491213872\\
0.494983277591973	0.970369761998191\\
0.498327759197324	0.973586630111605\\
0.501672240802676	0.976990463009004\\
0.505016722408027	0.98058858953466\\
0.508361204013378	0.984388292710574\\
0.511705685618729	0.988396802197729\\
0.51505016722408	0.992621286419187\\
0.518394648829431	0.997068844333715\\
0.521739130434783	1.00174649684842\\
0.525083612040134	1.00666117785857\\
0.528428093645485	1.0118197249026\\
0.531772575250836	1.01722886942004\\
0.535117056856187	1.02289522659992\\
0.538461538461538	1.02882528480698\\
0.54180602006689	1.03502539457283\\
0.545150501672241	1.04150175713903\\
0.548494983277592	1.04826041253899\\
0.551839464882943	1.05530722720523\\
0.555183946488294	1.06264788108886\\
0.558528428093645	1.0702878542775\\
0.561872909698997	1.07823241309844\\
0.565217391304348	1.08648659569326\\
0.568561872909699	1.0950551970506\\
0.57190635451505	1.10394275348372\\
0.575250836120401	1.11315352653945\\
0.578595317725752	1.12269148632567\\
0.581939799331104	1.13256029424439\\
0.585284280936455	1.14276328511814\\
0.588628762541806	1.15330344869755\\
0.591973244147157	1.16418341053861\\
0.595317725752508	1.17540541223881\\
0.59866220735786	1.18697129102188\\
0.602006688963211	1.19888245866185\\
0.605351170568562	1.21113987973799\\
0.608695652173913	1.22374404921353\\
0.612040133779264	1.23669496933206\\
0.615384615384615	1.24999212582714\\
0.618729096989967	1.26363446344239\\
0.622073578595318	1.27762036076092\\
0.625418060200669	1.29194760434541\\
0.62876254180602	1.30661336219222\\
0.632107023411371	1.32161415650582\\
0.635451505016722	1.33694583580256\\
0.638795986622074	1.35260354635627\\
0.642140468227425	1.3685817030018\\
0.645484949832776	1.38487395931668\\
0.648829431438127	1.4014731772056\\
0.652173913043478	1.41837139591741\\
0.655518394648829	1.43555980053002\\
0.658862876254181	1.45302868994443\\
0.662207357859532	1.47076744443606\\
0.665551839464883	1.48876449281897\\
0.668896321070234	1.5070072792866\\
0.672240802675585	1.52548223000179\\
0.675585284280936	1.54417471951857\\
0.678929765886288	1.56306903712911\\
0.682274247491639	1.58214835324104\\
0.68561872909699	1.60139468590329\\
0.688963210702341	1.62078886761291\\
0.692307692307692	1.64031051255055\\
0.695652173913043	1.65993798440972\\
0.698996655518395	1.6796483650028\\
0.702341137123746	1.69941742384743\\
0.705685618729097	1.71921958895859\\
0.709030100334448	1.73902791909545\\
0.712374581939799	1.75881407773828\\
0.71571906354515	1.77854830909851\\
0.719063545150502	1.79819941649576\\
0.722408026755853	1.81773474346865\\
0.725752508361204	1.83712015802188\\
0.729096989966555	1.85632004045069\\
0.732441471571906	1.8752972752255\\
0.735785953177257	1.89401324746429\\
0.739130434782609	1.91242784456835\\
0.74247491638796	1.93049946364899\\
0.745819397993311	1.9481850254276\\
0.749163879598662	1.96543999535091\\
0.752508361204013	1.98221841272569\\
0.755852842809365	1.99847292874383\\
0.759197324414716	2.01415485433913\\
0.762541806020067	2.02921421889079\\
0.765886287625418	2.04359984086621\\
0.769230769230769	2.05725941157574\\
0.77257525083612	2.07013959329514\\
0.775919732441472	2.082186133096\\
0.779264214046823	2.09334399380986\\
0.782608695652174	2.10355750363664\\
0.785953177257525	2.11277052599092\\
0.789297658862876	2.12092665125776\\
0.792642140468227	2.12796941220142\\
0.795986622073579	2.13384252483067\\
0.79933110367893	2.13849015657051\\
0.802675585284281	2.14185722361544\\
0.806020066889632	2.1438897193378\\
0.809364548494983	2.14453507558757\\
0.812709030100334	2.14374255863744\\
0.816053511705686	2.14146370138579\\
0.819397993311037	2.13765277321563\\
0.822742474916388	2.13226728860015\\
0.826086956521739	2.12526855512284\\
0.82943143812709	2.11662226101369\\
0.832775919732441	2.10629910155933\\
0.836120401337793	2.09427544278255\\
0.839464882943144	2.08053401955623\\
0.842809364548495	2.06506466375806\\
0.846153846153846	2.04786505611425\\
0.849498327759197	2.02894149293556\\
0.852842809364548	2.00830965591617\\
0.8561872909699	1.9859953694216\\
0.859531772575251	1.96203532509334\\
0.862876254180602	1.93647774797529\\
0.866220735785953	1.90938297152515\\
0.869565217391304	1.88082388058702\\
0.872909698996655	1.85088617141523\\
0.876254180602007	1.81966836587482\\
0.879598662207358	1.78728150269903\\
0.882943143812709	1.75384841185748\\
0.88628762541806	1.7195024583988\\
0.889632107023411	1.68438561936717\\
0.892976588628763	1.64864573148873\\
0.896321070234114	1.61243271847943\\
0.899665551839465	1.57589357569542\\
0.903010033444816	1.53916585782854\\
0.906354515050167	1.50236938503349\\
0.909698996655518	1.46559585871328\\
0.91304347826087	1.42889606752025\\
0.916387959866221	1.39226437867516\\
0.919732441471572	1.35562026781742\\
0.923076923076923	1.31878677052084\\
0.926421404682274	1.28146598319297\\
0.929765886287625	1.24321216443976\\
0.933110367892977	1.2034036846273\\
0.936454849498328	1.1612161781202\\
0.939799331103679	1.11560096285732\\
0.94314381270903	1.06527536935369\\
0.946488294314381	1.00873540620022\\
0.949832775919732	0.944306577774618\\
0.953177257525084	0.870256024346187\\
0.956521739130435	0.784998554842334\\
0.959866220735786	0.687439797215855\\
0.963210702341137	0.577508661673773\\
0.966555183946488	0.456930915993892\\
0.969899665551839	0.330268544024368\\
0.973244147157191	0.206160007101838\\
0.976588628762542	0.0984774774579287\\
0.979933110367893	0.0266566932600707\\
0.983277591973244	0.0136058898482677\\
0.986622073578595	0.0782969357277907\\
0.989966555183946	0.218902844011593\\
0.993311036789298	0.383984799814483\\
0.996655518394649	0.444138657327278\\
1	0.238059059160716\\
};
\addplot [color=mycolor3,solid,forget plot]
  table[row sep=crcr]{%
0	0.28362684333106\\
0.00334448160535117	0.287090465067145\\
0.00668896321070234	0.290568261986188\\
0.0100334448160535	0.294060323121017\\
0.0133779264214047	0.297566738227722\\
0.0167224080267559	0.301087597791921\\
0.020066889632107	0.304622993035051\\
0.0234113712374582	0.308173015920711\\
0.0267558528428094	0.311737759161017\\
0.0301003344481605	0.315317316223004\\
0.0334448160535117	0.318911781335056\\
0.0367892976588629	0.322521249493364\\
0.040133779264214	0.326145816468414\\
0.0434782608695652	0.329785578811505\\
0.0468227424749164	0.333440633861291\\
0.0501672240802676	0.337111079750351\\
0.0535117056856187	0.340797015411778\\
0.0568561872909699	0.344498540585798\\
0.0602006688963211	0.348215755826406\\
0.0635451505016722	0.351948762508016\\
0.0668896321070234	0.355697662832137\\
0.0702341137123746	0.359462559834059\\
0.0735785953177258	0.363243557389556\\
0.0769230769230769	0.367040760221595\\
0.0802675585284281	0.370854273907063\\
0.0836120401337793	0.37468420488349\\
0.0869565217391304	0.378530660455784\\
0.0903010033444816	0.382393748802968\\
0.0936454849498328	0.38627357898491\\
0.096989966555184	0.390170260949058\\
0.100334448160535	0.394083905537161\\
0.103678929765886	0.398014624491983\\
0.107023411371237	0.401962530464004\\
0.110367892976589	0.405927737018108\\
0.11371237458194	0.409910358640243\\
0.117056856187291	0.413910510744066\\
0.120401337792642	0.417928309677552\\
0.123745819397993	0.42196387272958\\
0.127090301003344	0.426017318136472\\
0.130434782608696	0.430088765088503\\
0.133779264214047	0.434178333736357\\
0.137123745819398	0.438286145197532\\
0.140468227424749	0.442412321562698\\
0.1438127090301	0.446556985901981\\
0.147157190635452	0.450720262271192\\
0.150501672240803	0.454902275717973\\
0.153846153846154	0.459103152287874\\
0.157190635451505	0.463323019030331\\
0.160535117056856	0.467562004004562\\
0.163879598662207	0.471820236285361\\
0.167224080267559	0.476097845968777\\
0.17056856187291	0.480394964177689\\
0.173913043478261	0.484711723067248\\
0.177257525083612	0.489048255830193\\
0.180602006688963	0.493404696702017\\
0.183946488294314	0.497781180965996\\
0.187290969899666	0.502177844958044\\
0.190635451505017	0.506594826071405\\
0.193979933110368	0.51103226276116\\
0.197324414715719	0.515490294548544\\
0.20066889632107	0.519969062025055\\
0.204013377926421	0.52446870685635\\
0.207357859531773	0.528989371785908\\
0.210702341137124	0.533531200638452\\
0.214046822742475	0.538094338323114\\
0.217391304347826	0.542678930836326\\
0.220735785953177	0.54728512526443\\
0.224080267558528	0.551913069785975\\
0.22742474916388	0.556562913673708\\
0.230769230769231	0.561234807296216\\
0.234113712374582	0.56592890211922\\
0.237458193979933	0.57064535070649\\
0.240802675585284	0.575384306720373\\
0.244147157190635	0.580145924921899\\
0.247491638795987	0.584930361170459\\
0.250836120401338	0.58973777242301\\
0.254180602006689	0.594568316732816\\
0.25752508361204	0.599422153247666\\
0.260869565217391	0.60429944220757\\
0.264214046822742	0.609200344941889\\
0.267558528428094	0.614125023865872\\
0.270903010033445	0.619073642476594\\
0.274247491638796	0.624046365348221\\
0.277591973244147	0.629043358126617\\
0.280936454849498	0.634064787523222\\
0.284280936454849	0.639110821308184\\
0.287625418060201	0.644181628302714\\
0.290969899665552	0.649277378370606\\
0.294314381270903	0.654398242408901\\
0.297658862876254	0.659544392337636\\
0.301003344481605	0.664716001088656\\
0.304347826086957	0.669913242593424\\
0.307692307692308	0.675136291769783\\
0.311036789297659	0.680385324507635\\
0.31438127090301	0.685660517653472\\
0.317725752508361	0.690962048993703\\
0.321070234113712	0.696290097236728\\
0.324414715719064	0.701644841993699\\
0.327759197324415	0.707026463757893\\
0.331103678929766	0.712435143882652\\
0.334448160535117	0.7178710645578\\
0.337792642140468	0.723334408784491\\
0.341137123745819	0.728825360348383\\
0.344481605351171	0.734344103791081\\
0.347826086956522	0.739890824379773\\
0.351170568561873	0.745465708074944\\
0.354515050167224	0.751068941496112\\
0.357859531772575	0.756700711885471\\
0.361204013377926	0.762361207069344\\
0.364548494983278	0.768050615417364\\
0.367892976588629	0.773769125799244\\
0.37123745819398	0.779516927539049\\
0.374581939799331	0.785294210366838\\
0.377926421404682	0.791101164367568\\
0.381270903010033	0.796937979927099\\
0.384615384615385	0.802804847675212\\
0.387959866220736	0.808701958425455\\
0.391304347826087	0.814629503111688\\
0.394648829431438	0.820587672721174\\
0.397993311036789	0.826576658224045\\
0.40133779264214	0.832596650498976\\
0.404682274247492	0.838647840254881\\
0.408026755852843	0.84473041794846\\
0.411371237458194	0.850844573697372\\
0.414715719063545	0.856990497188862\\
0.418060200668896	0.86316837758359\\
0.421404682274247	0.869378403414464\\
0.424749163879599	0.875620762480216\\
0.42809364548495	0.881895641733487\\
0.431438127090301	0.888203227163147\\
0.434782608695652	0.894543703670579\\
0.438127090301003	0.900917254939635\\
0.441471571906354	0.907324063299965\\
0.444816053511706	0.913764309583389\\
0.448160535117057	0.920238172972988\\
0.451505016722408	0.926745830844555\\
0.454849498327759	0.933287458600028\\
0.45819397993311	0.939863229492541\\
0.461538461538462	0.946473314442643\\
0.464882943143813	0.953117881845299\\
0.468227424749164	0.959797097367185\\
0.471571906354515	0.966511123733819\\
0.474916387959866	0.973260120506019\\
0.478260869565217	0.980044243845173\\
0.481605351170569	0.986863646266741\\
0.48494983277592	0.993718476381441\\
0.488294314381271	1.00060887862347\\
0.491638795986622	1.00753499296512\\
0.494983277591973	1.01449695461714\\
0.498327759197324	1.02149489371408\\
0.501672240802676	1.02852893498387\\
0.505016722408027	1.03559919740085\\
0.508361204013378	1.04270579382143\\
0.511705685618729	1.04984883060143\\
0.51505016722408	1.05702840719421\\
0.518394648829431	1.06424461572861\\
0.521739130434783	1.07149754056558\\
0.525083612040134	1.07878725783259\\
0.528428093645485	1.08611383493436\\
0.531772575250836	1.09347733003896\\
0.535117056856187	1.10087779153789\\
0.538461538461538	1.10831525747865\\
0.54180602006689	1.11578975496858\\
0.545150501672241	1.12330129954826\\
0.548494983277592	1.13084989453287\\
0.551839464882943	1.13843553031995\\
0.555183946488294	1.14605818366156\\
0.558528428093645	1.15371781689898\\
0.561872909698997	1.16141437715806\\
0.565217391304348	1.16914779550279\\
0.568561872909699	1.17691798604518\\
0.57190635451505	1.1847248450086\\
0.575250836120401	1.19256824974249\\
0.578595317725752	1.20044805768534\\
0.581939799331104	1.20836410527331\\
0.585284280936455	1.21631620679125\\
0.588628762541806	1.22430415316315\\
0.591973244147157	1.23232771067823\\
0.595317725752508	1.24038661964933\\
0.59866220735786	1.24848059299952\\
0.602006688963211	1.25660931477294\\
0.605351170568562	1.26477243856536\\
0.608695652173913	1.27296958586987\\
0.612040133779264	1.28120034433279\\
0.615384615384615	1.28946426591435\\
0.618729096989967	1.2977608649487\\
0.622073578595318	1.30608961609715\\
0.625418060200669	1.31444995218826\\
0.62876254180602	1.32284126193801\\
0.632107023411371	1.33126288754279\\
0.635451505016722	1.33971412213742\\
0.638795986622074	1.34819420711\\
0.642140468227425	1.3567023292646\\
0.645484949832776	1.36523761782268\\
0.648829431438127	1.37379914125271\\
0.652173913043478	1.38238590391768\\
0.655518394648829	1.39099684252853\\
0.658862876254181	1.39963082239151\\
0.662207357859532	1.40828663343603\\
0.665551839464883	1.41696298600891\\
0.668896321070234	1.42565850641989\\
0.672240802675585	1.43437173222199\\
0.675585284280936	1.44310110720951\\
0.678929765886288	1.45184497611454\\
0.682274247491639	1.46060157898227\\
0.68561872909699	1.46936904520312\\
0.688963210702341	1.47814538717859\\
0.692307692307692	1.48692849359582\\
0.695652173913043	1.49571612228378\\
0.698996655518395	1.50450589262225\\
0.702341137123746	1.51329527747218\\
0.705685618729097	1.52208159459381\\
0.709030100334448	1.53086199751611\\
0.712374581939799	1.53963346581832\\
0.71571906354515	1.54839279478118\\
0.719063545150502	1.55713658436187\\
0.722408026755853	1.5658612274433\\
0.725752508361204	1.57456289730387\\
0.729096989966555	1.58323753424961\\
0.732441471571906	1.5918808313458\\
0.735785953177257	1.60048821917959\\
0.739130434782609	1.60905484957962\\
0.74247491638796	1.61757557821194\\
0.745819397993311	1.62604494596484\\
0.749163879598662	1.63445715902734\\
0.752508361204013	1.64280606755766\\
0.755852842809365	1.65108514282893\\
0.759197324414716	1.65928745272902\\
0.762541806020067	1.66740563548019\\
0.765886287625418	1.67543187143225\\
0.769230769230769	1.68335785276878\\
0.77257525083612	1.69117475095157\\
0.775919732441472	1.69887318171136\\
0.779264214046823	1.70644316737522\\
0.782608695652174	1.7138740963001\\
0.785953177257525	1.72115467916016\\
0.789297658862876	1.72827290181052\\
0.792642140468227	1.73521597442253\\
0.795986622073579	1.74197027655517\\
0.79933110367893	1.74852129779339\\
0.802675585284281	1.75485357354637\\
0.806020066889632	1.76095061555691\\
0.809364548494983	1.76679483662641\\
0.812709030100334	1.77236746900801\\
0.816053511705686	1.77764847586239\\
0.819397993311037	1.78261645510629\\
0.822742474916388	1.78724853491147\\
0.826086956521739	1.79152026003145\\
0.82943143812709	1.7954054680434\\
0.832775919732441	1.798876154492\\
0.836120401337793	1.80190232580986\\
0.839464882943144	1.80445183876389\\
0.842809364548495	1.80649022503671\\
0.846153846153846	1.8079804993965\\
0.849498327759197	1.80888294973425\\
0.852842809364548	1.80915490705429\\
0.8561872909699	1.80875049328888\\
0.859531772575251	1.80762034457031\\
0.862876254180602	1.80571130733215\\
0.866220735785953	1.80296610432461\\
0.869565217391304	1.79932296731786\\
0.872909698996655	1.79471523293198\\
0.876254180602007	1.78907089767793\\
0.879598662207358	1.78231212792643\\
0.882943143812709	1.77435472015297\\
0.88628762541806	1.76510750645649\\
0.889632107023411	1.75447170004584\\
0.892976588628763	1.7423401751779\\
0.896321070234114	1.72859667598208\\
0.899665551839465	1.7131149488216\\
0.903010033444816	1.69575779347367\\
0.906354515050167	1.67637602968804\\
0.909698996655518	1.65480737794217\\
0.91304347826087	1.6308752569549\\
0.916387959866221	1.60438750649984\\
0.919732441471572	1.57513505339774\\
0.923076923076923	1.54289055295868\\
0.926421404682274	1.50740706015792\\
0.929765886287625	1.4684168183847\\
0.933110367892977	1.42563030473792\\
0.936454849498328	1.37873574893059\\
0.939799331103679	1.32739946260396\\
0.94314381270903	1.27126750051859\\
0.946488294314381	1.2099694619407\\
0.949832775919732	1.14312568986054\\
0.953177257525084	1.07035983607359\\
0.956521739130435	0.991319894646999\\
0.959866220735786	0.905712637077451\\
0.963210702341137	0.813359367945021\\
0.966555183946488	0.714285837254475\\
0.969899665551839	0.608867317088335\\
0.973244147157191	0.498063522260972\\
0.976588628762542	0.383801012714216\\
0.979933110367893	0.26959921075896\\
0.983277591973244	0.161600110543721\\
0.986622073578595	0.0702656044904304\\
0.989966555183946	0.0131673134114213\\
0.993311036789298	0.0195197185831033\\
0.996655518394649	0.137351455399399\\
1	0.444227909364375\\
};
\end{axis}

\begin{axis}[%
width=2in,
height=2in,
scale only axis,
xmin=0,
xmax=1,
ymin=-2,
ymax=4
]
\addplot [color=mycolor1,solid,forget plot]
  table[row sep=crcr]{%
0	0\\
0.00334448160535117	-0.0520354680790495\\
0.00668896321070234	-0.0989263715591333\\
0.0100334448160535	-0.140892126690317\\
0.0133779264214047	-0.178146671719105\\
0.0167224080267559	-0.210898544570521\\
0.020066889632107	-0.239350960040487\\
0.0234113712374582	-0.263701886498478\\
0.0267558528428094	-0.284144122100472\\
0.0301003344481605	-0.300865370512177\\
0.0334448160535117	-0.31404831614256\\
0.0367892976588629	-0.323870698887648\\
0.040133779264214	-0.330505388384626\\
0.0434782608695652	-0.334120457776215\\
0.0468227424749164	-0.334879256985344\\
0.0501672240802676	-0.332940485500104\\
0.0535117056856187	-0.328458264668988\\
0.0568561872909699	-0.321582209506425\\
0.0602006688963211	-0.312457500008592\\
0.0635451505016722	-0.301224951979521\\
0.0668896321070234	-0.288021087367487\\
0.0702341137123746	-0.272978204111685\\
0.0735785953177258	-0.256224445499198\\
0.0769230769230769	-0.237883869032242\\
0.0802675585284281	-0.218076514805713\\
0.0836120401337793	-0.196918473395003\\
0.0869565217391304	-0.17452195325412\\
0.0903010033444816	-0.150995347624083\\
0.0936454849498328	-0.126443300951609\\
0.096989966555184	-0.100966774818089\\
0.100334448160535	-0.0746631133788444\\
0.103678929765886	-0.0476261083126772\\
0.107023411371237	-0.0199460632817028\\
0.110367892976589	0.00829014209852687\\
0.11371237458194	0.0369989887786181\\
0.117056856187291	0.066100255284636\\
0.120401337792642	0.095516955217126\\
0.123745819397993	0.125175275239841\\
0.127090301003344	0.15500451355819\\
0.130434782608696	0.184937018887388\\
0.133779264214047	0.214908129910331\\
0.137123745819398	0.244856115225178\\
0.140468227424749	0.274722113782645\\
0.1438127090301	0.304450075813013\\
0.147157190635452	0.333986704242853\\
0.150501672240803	0.363281396601459\\
0.153846153846154	0.392286187416995\\
0.157190635451505	0.420955691102351\\
0.160535117056856	0.449247045330733\\
0.163879598662207	0.477119854900928\\
0.167224080267559	0.504536136092323\\
0.17056856187291	0.531460261509599\\
0.173913043478261	0.557858905417176\\
0.177257525083612	0.583700989563339\\
0.180602006688963	0.608957629494092\\
0.183946488294314	0.633602081356728\\
0.187290969899666	0.657609689193095\\
0.190635451505017	0.680957832722601\\
0.193979933110368	0.703625875614909\\
0.197324414715719	0.725595114252361\\
0.20066889632107	0.746848726982091\\
0.204013377926421	0.767371723857889\\
0.207357859531773	0.787150896871743\\
0.210702341137124	0.806174770675116\\
0.214046822742475	0.824433553789919\\
0.217391304347826	0.841919090309212\\
0.220735785953177	0.858624812087605\\
0.224080267558528	0.874545691421407\\
0.22742474916388	0.889678194218394\\
0.230769230769231	0.904020233657441\\
0.234113712374582	0.917571124337712\\
0.237458193979933	0.930331536917678\\
0.240802675585284	0.94230345324377\\
0.244147157190635	0.953490121968799\\
0.247491638795987	0.963896014660065\\
0.250836120401338	0.973526782397173\\
0.254180602006689	0.982389212859571\\
0.25752508361204	0.99049118790382\\
0.260869565217391	0.997841641630531\\
0.264214046822742	1.00445051894106\\
0.267558528428094	1.01032873458389\\
0.270903010033445	1.01548813269074\\
0.274247491638796	1.01994144680237\\
0.277591973244147	1.02370226038411\\
0.280936454849498	1.02678496783115\\
0.284280936454849	1.02920473596339\\
0.287625418060201	1.03097746601021\\
0.290969899665552	1.03211975608482\\
0.294314381270903	1.0326488641483\\
0.297658862876254	1.03258267146351\\
0.301003344481605	1.03193964653851\\
0.304347826086957	1.03073880955987\\
0.307692307692308	1.02899969731554\\
0.311036789297659	1.02674232860757\\
0.31438127090301	1.02398717015447\\
0.317725752508361	1.02075510298329\\
0.321070234113712	1.01706738931138\\
0.324414715719064	1.01294563991798\\
0.327759197324415	1.00841178200537\\
0.331103678929766	1.00348802754984\\
0.334448160535117	0.998196842142353\\
0.337792642140468	0.992560914318849\\
0.341137123745819	0.986603125380424\\
0.344481605351171	0.980346519702999\\
0.347826086956522	0.973814275536928\\
0.351170568561873	0.967029676296132\\
0.354515050167224	0.960016082337059\\
0.357859531772575	0.952796903227364\\
0.361204013377926	0.945395570504176\\
0.364548494983278	0.937835510922251\\
0.367892976588629	0.930140120191702\\
0.37123745819398	0.922332737205475\\
0.374581939799331	0.914436618756667\\
0.377926421404682	0.906474914745253\\
0.381270903010033	0.898470643874911\\
0.384615384615385	0.890446669839241\\
0.387959866220736	0.882425677997883\\
0.391304347826087	0.874430152542281\\
0.394648829431438	0.866482354151134\\
0.397993311036789	0.858604298135593\\
0.40133779264214	0.850817733074265\\
0.404682274247492	0.843144119937692\\
0.408026755852843	0.835604611702781\\
0.411371237458194	0.828220033456859\\
0.414715719063545	0.821010862991364\\
0.418060200668896	0.813997211885395\\
0.421404682274247	0.807198807078858\\
0.424749163879599	0.800634972935354\\
0.42809364548495	0.794324613794846\\
0.431438127090301	0.78828619701596\\
0.434782608695652	0.782537736507975\\
0.438127090301003	0.777096776752652\\
0.441471571906354	0.771980377315677\\
0.444816053511706	0.767205097847845\\
0.448160535117057	0.762786983575923\\
0.451505016722408	0.758741551283165\\
0.454849498327759	0.755083775779948\\
0.45819397993311	0.751828076863427\\
0.461538461538462	0.748988306767515\\
0.464882943143813	0.746577738102235\\
0.468227424749164	0.744609052282908\\
0.471571906354515	0.74309432844914\\
0.474916387959866	0.742045032873256\\
0.478260869565217	0.741472008858642\\
0.481605351170569	0.741385467127905\\
0.48494983277592	0.741794976700433\\
0.488294314381271	0.742709456259984\\
0.491638795986622	0.744137166011735\\
0.494983277591973	0.746085700029226\\
0.498327759197324	0.748561979090883\\
0.501672240802676	0.751572244006456\\
0.505016722408027	0.755122049432815\\
0.508361204013378	0.759216258179858\\
0.511705685618729	0.763859036005915\\
0.51505016722408	0.769053846902873\\
0.518394648829431	0.774803448870939\\
0.521739130434783	0.781109890183492\\
0.525083612040134	0.787974506141015\\
0.528428093645485	0.79539791631548\\
0.531772575250836	0.803380022283687\\
0.535117056856187	0.811920005851013\\
0.538461538461538	0.821016327764386\\
0.54180602006689	0.830666726915137\\
0.545150501672241	0.840868220031634\\
0.548494983277592	0.85161710186156\\
0.551839464882943	0.862908945843856\\
0.555183946488294	0.874738605270469\\
0.558528428093645	0.887100214937817\\
0.561872909698997	0.899987193287776\\
0.565217391304348	0.913392245038626\\
0.568561872909699	0.927307364305761\\
0.57190635451505	0.941723838211544\\
0.575250836120401	0.956632250985791\\
0.578595317725752	0.972022488555183\\
0.581939799331104	0.987883743622639\\
0.585284280936455	1.00420452123658\\
0.588628762541806	1.0209726448497\\
0.591973244147157	1.03817526286768\\
0.595317725752508	1.0557988556871\\
0.59866220735786	1.07382924322375\\
0.602006688963211	1.09225159293016\\
0.605351170568562	1.11105042830305\\
0.608695652173913	1.13020963788042\\
0.612040133779264	1.14971248472844\\
0.615384615384615	1.16954161641781\\
0.618729096989967	1.18967907549031\\
0.622073578595318	1.21010631041447\\
0.625418060200669	1.23080418703164\\
0.62876254180602	1.25175300049079\\
0.632107023411371	1.27293248767419\\
0.635451505016722	1.29432184011197\\
0.638795986622074	1.31589971738664\\
0.642140468227425	1.33764426102731\\
0.645484949832776	1.35953310889388\\
0.648829431438127	1.38154341005055\\
0.652173913043478	1.40365184012913\\
0.655518394648829	1.42583461718203\\
0.658862876254181	1.4480675180256\\
0.662207357859532	1.47032589507212\\
0.665551839464883	1.49258469365228\\
0.668896321070234	1.51481846982709\\
0.672240802675585	1.53700140868937\\
0.675585284280936	1.55910734315562\\
0.678929765886288	1.58110977324658\\
0.682274247491639	1.60298188585788\\
0.68561872909699	1.62469657502169\\
0.688963210702341	1.64622646265541\\
0.692307692307692	1.6675439198029\\
0.695652173913043	1.68862108836321\\
0.698996655518395	1.70942990331051\\
0.702341137123746	1.72994211540315\\
0.705685618729097	1.75012931438188\\
0.709030100334448	1.76996295265929\\
0.712374581939799	1.789414369497\\
0.71571906354515	1.80845481567458\\
0.719063545150502	1.82705547864643\\
0.722408026755853	1.84518750818928\\
0.725752508361204	1.86282204253997\\
0.729096989966555	1.87993023502099\\
0.732441471571906	1.89648328115815\\
0.735785953177257	1.91245244628627\\
0.739130434782609	1.92780909364483\\
0.74247491638796	1.94252471296413\\
0.745819397993311	1.95657094954052\\
0.749163879598662	1.96991963380134\\
0.752508361204013	1.98254281135924\\
0.755852842809365	1.99441277355768\\
0.759197324414716	2.00550208850449\\
0.762541806020067	2.01578363259593\\
0.765886287625418	2.02523062252989\\
0.769230769230769	2.03381664781016\\
0.77257525083612	2.04151570373872\\
0.775919732441472	2.04830222489836\\
0.779264214046823	2.05415111912654\\
0.782608695652174	2.05903780197509\\
0.785953177257525	2.06293823166511\\
0.789297658862876	2.06582894452612\\
0.792642140468227	2.06768709092826\\
0.795986622073579	2.06849047170444\\
0.79933110367893	2.06821757505891\\
0.802675585284281	2.06684761396951\\
0.806020066889632	2.06436056407699\\
0.809364548494983	2.06073720206578\\
0.812709030100334	2.05595914453161\\
0.816053511705686	2.05000888734403\\
0.819397993311037	2.04286984549188\\
0.822742474916388	2.03452639342515\\
0.826086956521739	2.02496390588109\\
0.82943143812709	2.01416879920379\\
0.832775919732441	2.00212857315154\\
0.836120401337793	1.9888318531946\\
0.839464882943144	1.9742684333026\\
0.842809364548495	1.95842931922093\\
0.846153846153846	1.94130677223808\\
0.849498327759197	1.92289435344287\\
0.852842809364548	1.90318696846811\\
0.8561872909699	1.88218091272904\\
0.859531772575251	1.85987391714838\\
0.862876254180602	1.83626519437041\\
0.866220735785953	1.81135548546652\\
0.869565217391304	1.78514710713173\\
0.872909698996655	1.75764399936594\\
0.876254180602007	1.72885177365086\\
0.879598662207358	1.69877776161204\\
0.882943143812709	1.66743106417385\\
0.88628762541806	1.63482260120219\\
0.889632107023411	1.60096516163577\\
0.892976588628763	1.56587345411319\\
0.896321070234114	1.5295641580808\\
0.899665551839465	1.49205597539662\\
0.903010033444816	1.45336968242316\\
0.906354515050167	1.41352818260856\\
0.909698996655518	1.37255655955585\\
0.91304347826087	1.3304821305868\\
0.916387959866221	1.28733450079108\\
0.919732441471572	1.24314561756688\\
0.923076923076923	1.19794982565122\\
0.926421404682274	1.15178392263948\\
0.929765886287625	1.10468721499586\\
0.933110367892977	1.05670157455143\\
0.936454849498328	1.0078714954937\\
0.939799331103679	0.958244151846429\\
0.94314381270903	0.907869455435105\\
0.946488294314381	0.856800114348403\\
0.949832775919732	0.805091691883774\\
0.953177257525084	0.752802665985655\\
0.956521739130435	0.699994489174401\\
0.959866220735786	0.646731648959303\\
0.963210702341137	0.593081728749667\\
0.966555183946488	0.539115469248543\\
0.969899665551839	0.484906830338332\\
0.973244147157191	0.430533053460636\\
0.976588628762542	0.376074724476098\\
0.979933110367893	0.321615837024865\\
0.983277591973244	0.267243856367884\\
0.986622073578595	0.21304978372325\\
0.989966555183946	0.159128221089929\\
0.993311036789298	0.105577436563692\\
0.996655518394649	0.0524994301380843\\
1	5.6843418860808e-14\\
};
\addplot [color=mycolor2,solid,forget plot]
  table[row sep=crcr]{%
0	0.779078681682136\\
0.00334448160535117	-0.116876026810613\\
0.00668896321070234	-0.507812529176049\\
0.0100334448160535	0.00336109561543477\\
0.0133779264214047	-0.743836192904562\\
0.0167224080267559	0.229759988089724\\
0.020066889632107	-0.711457808918791\\
0.0234113712374582	-0.574497848415356\\
0.0267558528428094	-1.27967029321384\\
0.0301003344481605	-0.776842007351622\\
0.0334448160535117	-0.185073818590505\\
0.0367892976588629	0.275227913567449\\
0.040133779264214	0.223978002825794\\
0.0434782608695652	-1.14045381506217\\
0.0468227424749164	-0.513194078284398\\
0.0501672240802676	-0.33724871492542\\
0.0535117056856187	-0.335765243503209\\
0.0568561872909699	-0.213038381370316\\
0.0602006688963211	-0.222362134084484\\
0.0635451505016722	-0.561483154716828\\
0.0668896321070234	-0.407812040445127\\
0.0702341137123746	-0.61948581918225\\
0.0735785953177258	-0.598679838557238\\
0.0769230769230769	0.228417511337146\\
0.0802675585284281	-0.822352012604582\\
0.0836120401337793	-0.377311225324433\\
0.0869565217391304	-0.765706692319596\\
0.0903010033444816	0.17133345339393\\
0.0936454849498328	-0.670859736075343\\
0.096989966555184	-0.950204049715909\\
0.100334448160535	-0.719842922488165\\
0.103678929765886	0.225187263556392\\
0.107023411371237	1.04772870459695\\
0.110367892976589	-0.280139230569036\\
0.11371237458194	0.449593515620052\\
0.117056856187291	0.872652333085613\\
0.120401337792642	0.703478046386008\\
0.123745819397993	0.932130926644352\\
0.127090301003344	-0.144964888026493\\
0.130434782608696	0.70497702341891\\
0.133779264214047	0.0927047092221992\\
0.137123745819398	-0.497677259600256\\
0.140468227424749	0.0966328002747983\\
0.1438127090301	0.481711855368307\\
0.147157190635452	1.25144314139754\\
0.150501672240803	0.812127630578683\\
0.153846153846154	0.518437337572297\\
0.157190635451505	0.912969609255538\\
0.160535117056856	0.191966583106087\\
0.163879598662207	-0.178206962677953\\
0.167224080267559	0.160878304084209\\
0.17056856187291	-0.0612720481157575\\
0.173913043478261	1.04905321137439\\
0.177257525083612	0.516027705388994\\
0.180602006688963	0.013937645636661\\
0.183946488294314	0.529947268918011\\
0.187290969899666	0.234802096013732\\
0.190635451505017	0.523810269166624\\
0.193979933110368	1.41576603128982\\
0.197324414715719	1.47133354132305\\
0.20066889632107	0.37482453064553\\
0.204013377926421	0.580183233489758\\
0.207357859531773	0.999204342278761\\
0.210702341137124	0.86323262507081\\
0.214046822742475	0.568884574987627\\
0.217391304347826	1.35308770061039\\
0.220735785953177	0.149032890459909\\
0.224080267558528	2.14788535749426\\
0.22742474916388	0.710700443072848\\
0.230769230769231	1.10467944701739\\
0.234113712374582	0.564473709499841\\
0.237458193979933	0.419999839477269\\
0.240802675585284	1.50928386819936\\
0.244147157190635	0.972762787982191\\
0.247491638795987	1.07935237788835\\
0.250836120401338	1.03639354256773\\
0.254180602006689	1.33053201946369\\
0.25752508361204	1.86022801364133\\
0.260869565217391	1.33053033443026\\
0.264214046822742	1.34821642137691\\
0.267558528428094	1.62981309222932\\
0.270903010033445	1.46481966649758\\
0.274247491638796	0.444892408202876\\
0.277591973244147	0.891612028913818\\
0.280936454849498	0.214194741665319\\
0.284280936454849	1.32002603545623\\
0.287625418060201	1.982223830536\\
0.290969899665552	1.82216689430372\\
0.294314381270903	1.49316339163886\\
0.297658862876254	0.563631441355292\\
0.301003344481605	0.663346540851877\\
0.304347826086957	1.1206292025641\\
0.307692307692308	0.963084460278867\\
0.311036789297659	1.46070577534022\\
0.31438127090301	1.68118311236445\\
0.317725752508361	0.895195327151878\\
0.321070234113712	0.981187633242824\\
0.324414715719064	1.56054999541559\\
0.327759197324415	1.25305196589635\\
0.331103678929766	0.163411178959312\\
0.334448160535117	0.59243411230258\\
0.337792642140468	0.616448468912216\\
0.341137123745819	0.323643578923731\\
0.344481605351171	0.889606092211076\\
0.347826086956522	1.03952201901497\\
0.351170568561873	1.66394465071635\\
0.354515050167224	1.31312279372596\\
0.357859531772575	1.53422030855224\\
0.361204013377926	0.568551708650521\\
0.364548494983278	1.09879786207026\\
0.367892976588629	1.74799222123946\\
0.37123745819398	0.59843547575806\\
0.374581939799331	1.25106826978946\\
0.377926421404682	1.54325266431465\\
0.381270903010033	0.378564733183654\\
0.384615384615385	1.13442168306121\\
0.387959866220736	0.429048875951002\\
0.391304347826087	0.637078615048329\\
0.394648829431438	0.533438990171034\\
0.397993311036789	0.607594490557053\\
0.40133779264214	0.696374723162189\\
0.404682274247492	0.424693471617786\\
0.408026755852843	0.521442664819016\\
0.411371237458194	1.07867176230748\\
0.414715719063545	1.65613766912257\\
0.418060200668896	0.043132160865398\\
0.421404682274247	0.671194246008195\\
0.424749163879599	0.971458341744076\\
0.42809364548495	1.0365080780996\\
0.431438127090301	1.26462830556506\\
0.434782608695652	0.848212801741729\\
0.438127090301003	-0.0938598599123668\\
0.441471571906354	1.10586103284419\\
0.444816053511706	1.21628303625958\\
0.448160535117057	0.488129204487624\\
0.451505016722408	0.59999141991674\\
0.454849498327759	0.463737120108976\\
0.45819397993311	0.669722611307785\\
0.461538461538462	0.281421681387047\\
0.464882943143813	1.34063144208084\\
0.468227424749164	0.13821489852469\\
0.471571906354515	0.552505071150945\\
0.474916387959866	1.40337173200236\\
0.478260869565217	0.934145147137245\\
0.481605351170569	0.560474545665527\\
0.48494983277592	0.528606645715257\\
0.488294314381271	0.936278425833911\\
0.491638795986622	1.19552564190776\\
0.494983277591973	1.10497333497798\\
0.498327759197324	1.61660234635529\\
0.501672240802676	0.247155773805896\\
0.505016722408027	0.665102347332268\\
0.508361204013378	-0.25405758914103\\
0.511705685618729	0.968327956555486\\
0.51505016722408	1.13621640666948\\
0.518394648829431	0.815787165149711\\
0.521739130434783	-0.062148619801414\\
0.525083612040134	-0.242881157853704\\
0.528428093645485	1.70586649395694\\
0.531772575250836	0.178814249750749\\
0.535117056856187	1.28007807923349\\
0.538461538461538	0.735236967742326\\
0.54180602006689	0.791482162702684\\
0.545150501672241	0.535683615881723\\
0.548494983277592	1.13605488125844\\
0.551839464882943	0.784310351718553\\
0.555183946488294	0.34048573019508\\
0.558528428093645	0.73083913134306\\
0.561872909698997	0.891425748984127\\
0.565217391304348	0.847226107757241\\
0.568561872909699	1.14965006809544\\
0.57190635451505	1.14259311650308\\
0.575250836120401	-0.300462857315262\\
0.578595317725752	0.577776118114563\\
0.581939799331104	0.930875432324413\\
0.585284280936455	1.2412730087435\\
0.588628762541806	1.36343073489099\\
0.591973244147157	1.3080604826317\\
0.595317725752508	0.865574382451056\\
0.59866220735786	1.79535759316347\\
0.602006688963211	0.687602185624133\\
0.605351170568562	0.375434899252645\\
0.608695652173913	1.34292067092577\\
0.612040133779264	1.63180885026527\\
0.615384615384615	1.65474987457155\\
0.618729096989967	1.45083585522095\\
0.622073578595318	1.64749579126733\\
0.625418060200669	1.30983738256443\\
0.62876254180602	2.03459729937771\\
0.632107023411371	1.94931137392325\\
0.635451505016722	1.34058592779069\\
0.638795986622074	1.70716136157192\\
0.642140468227425	1.8926564581587\\
0.645484949832776	2.1095063770903\\
0.648829431438127	1.31535298643079\\
0.652173913043478	2.0325516863047\\
0.655518394648829	1.97248100120026\\
0.658862876254181	1.9356781535426\\
0.662207357859532	2.07046880562013\\
0.665551839464883	1.43894301161318\\
0.668896321070234	1.82456809175814\\
0.672240802675585	2.08248566259781\\
0.675585284280936	1.33808990492284\\
0.678929765886288	0.94071317543162\\
0.682274247491639	1.93237363450394\\
0.68561872909699	1.7252112865628\\
0.688963210702341	1.87123626789699\\
0.692307692307692	1.6610095078387\\
0.695652173913043	1.26538004268846\\
0.698996655518395	1.95078909491662\\
0.702341137123746	2.19650254553044\\
0.705685618729097	1.89597962620639\\
0.709030100334448	1.66326612595028\\
0.712374581939799	1.7690616615951\\
0.71571906354515	1.69444067538925\\
0.719063545150502	1.13650308755138\\
0.722408026755853	1.97771959940121\\
0.725752508361204	2.39385864183772\\
0.729096989966555	1.90920434057769\\
0.732441471571906	1.61418300937433\\
0.735785953177257	2.24377786516881\\
0.739130434782609	2.53234601151007\\
0.74247491638796	2.16993545456056\\
0.745819397993311	2.2033566152992\\
0.749163879598662	2.08072994557332\\
0.752508361204013	1.6952207559342\\
0.755852842809365	2.0605478866073\\
0.759197324414716	2.56982198361084\\
0.762541806020067	1.98778068879149\\
0.765886287625418	2.3783635603795\\
0.769230769230769	2.0035411943712\\
0.77257525083612	2.33663996490065\\
0.775919732441472	1.73849860328873\\
0.779264214046823	1.45901108187204\\
0.782608695652174	0.841359759698586\\
0.785953177257525	2.42894101888679\\
0.789297658862876	2.2392351479771\\
0.792642140468227	2.43173060795068\\
0.795986622073579	2.35407030667778\\
0.79933110367893	1.17943043988948\\
0.802675585284281	2.11853700651562\\
0.806020066889632	3.03178443195966\\
0.809364548494983	2.52607370366299\\
0.812709030100334	1.98054182385542\\
0.816053511705686	1.81590935863238\\
0.819397993311037	1.86022650463404\\
0.822742474916388	2.16359364770581\\
0.826086956521739	2.11764383896818\\
0.82943143812709	1.56688627101495\\
0.832775919732441	1.77491620583234\\
0.836120401337793	1.8473238104731\\
0.839464882943144	2.52251926817715\\
0.842809364548495	2.46732774566551\\
0.846153846153846	2.09258422737918\\
0.849498327759197	1.29243054640074\\
0.852842809364548	2.02326250635497\\
0.8561872909699	2.27389985823916\\
0.859531772575251	1.76336388042607\\
0.862876254180602	1.18276019331846\\
0.866220735785953	1.42855564718457\\
0.869565217391304	2.1764238403333\\
0.872909698996655	1.72331271966245\\
0.876254180602007	1.92045999990337\\
0.879598662207358	1.85449219801494\\
0.882943143812709	2.04345975317656\\
0.88628762541806	2.11341848550115\\
0.889632107023411	1.82171710833444\\
0.892976588628763	1.84244634013485\\
0.896321070234114	1.71315821696353\\
0.899665551839465	0.91970749546602\\
0.903010033444816	1.91711165614357\\
0.906354515050167	1.93739559714564\\
0.909698996655518	1.53710439901333\\
0.91304347826087	1.21064513878297\\
0.916387959866221	1.00341621216014\\
0.919732441471572	0.460840652288592\\
0.923076923076923	1.21280882306977\\
0.926421404682274	1.45537283297421\\
0.929765886287625	1.58483229303974\\
0.933110367892977	0.99393327509009\\
0.936454849498328	0.76658960533194\\
0.939799331103679	1.81844915799435\\
0.94314381270903	1.76368098259752\\
0.946488294314381	2.0527763939093\\
0.949832775919732	0.336470588855975\\
0.953177257525084	0.110881921099071\\
0.956521739130435	0.781292699742966\\
0.959866220735786	0.841246795229484\\
0.963210702341137	0.935578318445592\\
0.966555183946488	0.293205739960825\\
0.969899665551839	0.385506111099671\\
0.973244147157191	0.276380587690381\\
0.976588628762542	-0.609424775267481\\
0.979933110367893	-0.0894473400057751\\
0.983277591973244	0.300905523470179\\
0.986622073578595	0.00523564127995996\\
0.989966555183946	0.580027869684013\\
0.993311036789298	0.402330926335885\\
0.996655518394649	0.202003836613747\\
1	0.310163078041398\\
};
\end{axis}
\end{tikzpicture}%
\end{document}
\caption{The L-curve in relation to curve fitting problems. On the left a known curve $\mathbf{b}_{true}$ is shown in blue. $\mathbf{b} = \mathbf{b}_{true} + \mathbf{b}_{noise}$ is shown in red. The right plot shows efforts to recover the noise free solution. The effects of no filtering (blue, $\|\mathbf{Ax}\| = 21.77$, $\|\mathbf{Ax}-\mathbf{b}_{true}\| =7.59$), optimal filtering (red, $\|\mathbf{Ax}\| = 19.44$, $\|\mathbf{Ax}-\mathbf{b}_{true}\| = 3.33$) and over-damping (yellow, $\|\mathbf{Ax}\| =  18.3362$, $\|\mathbf{Ax}-\mathbf{b}_{true}\| = 5.08$) are shown.} 
\label{fig:knownFitFilt}
\end{figure} 
\begin{figure}
\centering
% This file was created by matlab2tikz.
% Minimal pgfplots version: 1.3
%
%The latest updates can be retrieved from
%  http://www.mathworks.com/matlabcentral/fileexchange/22022-matlab2tikz
%where you can also make suggestions and rate matlab2tikz.
%
\documentclass[tikz]{standalone}
\usepackage{pgfplots}
\usepackage{grffile}
\pgfplotsset{compat=newest}
\usetikzlibrary{plotmarks}
\usepackage{amsmath}

\begin{document}
\definecolor{mycolor1}{rgb}{0.00000,0.44700,0.74100}%
\definecolor{mycolor2}{rgb}{0.85000,0.32500,0.09800}%
\definecolor{mycolor3}{rgb}{0.92900,0.69400,0.12500}%
\definecolor{mycolor4}{rgb}{0.49400,0.18400,0.55600}%
%
\begin{tikzpicture}

\begin{axis}[%
width=4.0in,
height=2.0in,
at={(0.758333in,0.48125in)},
scale only axis,
xmode=log,
xmin=1e-20,
xmax=100000,
xminorticks=true,
xlabel={$\lambda$},
ymode=log,
ymin=1e-05,
ymax=1e+15,
yminorticks=true,
ylabel={norm magnitude},
legend style={legend cell align=left,align=left,draw=white!15!black}
]
\addplot [color=mycolor1,solid]
  table[row sep=crcr]{%
1.87538253595704e-16	0.0193038738470416\\
2.12461726656173e-16	0.0185090239982361\\
2.40697481331118e-16	0.0189429358760381\\
2.72685713473941e-16	0.0180029271976985\\
3.08925120119979e-16	0.0183563234880474\\
3.49980674181025e-16	0.0180723180565316\\
3.96492432381782e-16	0.018098760751986\\
4.49185513754134e-16	0.0179955590412866\\
5.08881404254101e-16	0.0180920637748833\\
5.76510763740636e-16	0.0179900880618026\\
6.53127934977265e-16	0.0180069571996471\\
7.39927380852124e-16	0.0181788301346268\\
8.38262306072912e-16	0.0180470248323015\\
9.49665753649293e-16	0.0180465064202844\\
1.07587450505717e-15	0.0180670192553766\\
1.21885615668887e-15	0.0180490192730289\\
1.38083979471139e-15	0.018060774607695\\
1.56435074655435e-15	0.0180518060427566\\
1.77224995080377e-15	0.0180536444583133\\
2.00777855927902e-15	0.0180501877891216\\
2.27460846663999e-15	0.01805539261113\\
2.57689955528173e-15	0.018057863565329\\
2.91936454796561e-15	0.0180574257643346\\
3.30734247924019e-15	0.0180577052735867\\
3.74688193107274e-15	0.0180583034305718\\
4.24483533033586e-15	0.0180592879039267\\
4.8089657782488e-15	0.0180598732976866\\
5.44806807724583e-15	0.0180607706350695\\
6.17210584208268e-15	0.0180612331281622\\
6.9923668327451e-15	0.0180618655544502\\
7.92163893080218e-15	0.0180620995643229\\
8.97440950267838e-15	0.0180623774386299\\
1.01670912579208e-14	0.0180625253358517\\
1.15182781235957e-14	0.0180627259468327\\
1.30490351238998e-14	0.0180629199521028\\
1.47832267842144e-14	0.0180629796137325\\
1.67478891794262e-14	0.018063107539685\\
1.89736514267543e-14	0.0180631718251209\\
2.14952131941625e-14	0.0180632308426069\\
2.43518856687217e-14	0.0180632804143862\\
2.75882044186257e-14	0.0180633046909893\\
3.125462370339e-14	0.0180633355579015\\
3.54083030565416e-14	0.0180633615010352\\
4.01139984036312e-14	0.0180633761684266\\
4.54450716081196e-14	0.0180633885375935\\
5.14846341839652e-14	0.0180633998597478\\
5.83268430054169e-14	0.018063415704104\\
6.60783682141439e-14	0.0180634132651175\\
7.48600562083991e-14	0.0180634136748239\\
8.48088136402428e-14	0.0180634340959456\\
9.6079741792372e-14	0.0180634393318133\\
1.08848554609523e-13	0.0180634509334252\\
1.23314318081598e-13	0.0180634759672453\\
1.39702553685533e-13	0.0180635054195379\\
1.58268754268623e-13	0.0180635407974204\\
1.79302367185974e-13	0.0180635922095068\\
2.03131306789261e-13	0.0180637067459649\\
2.30127066616556e-13	0.0180638557074881\\
2.60710510982351e-13	0.0180640825675037\\
2.95358436258748e-13	0.0180644518704336\\
3.34611004138294e-13	0.0180649606567773\\
3.7908016276316e-13	0.0180657092731537\\
4.29459187006148e-13	0.0180667318083721\\
4.86533486636738e-13	0.0180681117168792\\
5.51192850871559e-13	0.0180698700265487\\
6.24445320202089e-13	0.018072044388082\\
7.07432901761553e-13	0.0180745998601917\\
8.01449373233844e-13	0.0180774719411406\\
9.07960452867686e-13	0.0180805663402366\\
1.02862665004687e-12	0.0180837436444513\\
1.16532915265731e-12	0.0180868751950086\\
1.32019915483536e-12	0.0180898484684424\\
1.49565108231746e-12	0.0180925825351158\\
1.69442023337484e-12	0.0180950189388918\\
1.91960542215596e-12	0.0180971352766312\\
2.17471728924723e-12	0.0180989389461052\\
2.46373303261411e-12	0.0181004477978187\\
2.7911584121792e-12	0.0181016914554518\\
3.16209799468923e-12	0.0181027058509076\\
3.5823347339899e-12	0.018103525445822\\
4.05842012736601e-12	0.0181041822898649\\
4.59777635348579e-12	0.0181047055377142\\
5.20881198428144e-12	0.0181051200258215\\
5.90105307471603e-12	0.0181054481966439\\
6.68529167412812e-12	0.0181057063322298\\
7.57375407444839e-12	0.0181059093222394\\
8.58029141828051e-12	0.0181060688474578\\
9.72059563842917e-12	0.0181061942471563\\
1.10124440953759e-11	0.0181062930406997\\
1.2475976726605e-11	0.0181063711722556\\
1.41340100285409e-11	0.0181064338855667\\
1.60123927660819e-11	0.018106485459539\\
1.81404089552453e-11	0.0181065297522824\\
2.0551234401432e-11	0.0181065704246783\\
2.32824539107472e-11	0.0181066111829758\\
2.63766472377103e-11	0.018106656177585\\
2.98820529042884e-11	0.0181067098431584\\
3.3853320239203e-11	0.0181067768836964\\
3.83523613618113e-11	0.018106862077281\\
4.34493163930081e-11	0.0181069693973531\\
4.92236469407986e-11	0.0181071012874678\\
5.57653749079995e-11	0.0181072579651559\\
6.31764859351011e-11	0.0181074366535289\\
7.1572519357984e-11	0.0181076318608298\\
8.10843694679576e-11	0.0181078361042111\\
9.1860326155794e-11	0.0181080410735101\\
1.04068386753423e-10	0.0181082390846518\\
1.17898875114945e-10	0.0181084240226779\\
1.33567408768466e-10	0.0181085919505398\\
1.51318260396711e-10	0.0181087411979539\\
1.71428166051932e-10	0.0181088720737481\\
1.94210639475258e-10	0.0181089864631304\\
2.2002085978078e-10	0.0181090873250326\\
2.49261208703455e-10	0.0181091782399825\\
2.82387543736592e-10	0.0181092629827193\\
3.19916304957242e-10	0.0181093450767586\\
3.62432566335019e-10	0.0181094273636529\\
4.10599157044353e-10	0.0181095116310456\\
4.65166994981609e-10	0.0181095983919488\\
5.26986793586735e-10	0.0181096868806495\\
5.97022324478994e-10	0.0181097753390293\\
6.76365442671451e-10	0.0181098614619215\\
7.66253108607572e-10	0.0181099429192705\\
8.68086672393724e-10	0.0181100177743172\\
9.83453720869071e-10	0.0181100847352402\\
1.11415282810902e-09	0.0181101432073161\\
1.26222159522298e-09	0.0181101932131666\\
1.42996841658724e-09	0.0181102352354042\\
1.6200084677491e-09	0.0181102700501234\\
1.83530447605426e-09	0.0181102985777661\\
2.07921290961208e-09	0.0181103217761277\\
2.35553630468546e-09	0.0181103405699204\\
2.66858254729017e-09	0.0181103558136603\\
3.0232320332047e-09	0.0181103682760122\\
3.42501375341611e-09	0.0181103786394604\\
3.88019149117531e-09	0.0181103875037787\\
4.39586147447511e-09	0.0181103953932443\\
4.98006300635469e-09	0.0181104027642407\\
5.64190379775879e-09	0.0181104100182647\\
6.39170195689245e-09	0.0181104175316245\\
7.24114684868813e-09	0.0181104257119643\\
8.20348133218639e-09	0.0181104350974241\\
9.29370821691347e-09	0.0181104465128206\\
1.0528824156915e-08	0.0181104613009683\\
1.19280846288559e-08	0.0181104816587315\\
1.35133041252003e-08	0.0181105111211314\\
1.53091962424876e-08	0.018110555257472\\
1.73437589666859e-08	0.0181106226518086\\
1.96487111622276e-08	0.0181107262312567\\
2.22599870695977e-08	0.0181108849489687\\
2.52182965207006e-08	0.0181111256975603\\
2.85697596057712e-08	0.0181114850852605\\
3.23666256862967e-08	0.0181120103547438\\
3.66680879633732e-08	0.018112758337166\\
4.15412063006284e-08	0.018113791144275\\
4.70619526885367e-08	0.0181151676757018\\
5.33163956489282e-08	0.0181169312761498\\
6.04020420445811e-08	0.0181190959312347\\
6.8429357212722e-08	0.0181216353324856\\
7.75234871213494e-08	0.018124479570665\\
8.78262093968165e-08	0.0181275221273165\\
9.9498143639222e-08	0.0181306358842011\\
1.12721255484472e-07	0.01813369322757\\
1.27701693451363e-07	0.0181365841176291\\
1.44673002800189e-07	0.0181392276520529\\
1.63899766506975e-07	0.0181415757863525\\
1.8568173011617e-07	0.0181436106071437\\
2.10358474778347e-07	0.0181453378529988\\
2.38314711325597e-07	0.0181467792936709\\
2.69986277919373e-07	0.0181479657612466\\
3.05866934774194e-07	0.0181489316980804\\
3.4651606188704e-07	0.018149711391972\\
3.92567379780188e-07	0.0181503366823762\\
4.44738829213984e-07	0.0181508357816535\\
5.03843763894432e-07	0.0181512328597907\\
5.70803630670092e-07	0.0181515481094255\\
6.46662334902741e-07	0.0181517980860786\\
7.32602514968158e-07	0.018151996183355\\
8.29963979606747e-07	0.018152153149073\\
9.40264595562587e-07	0.0181522775810241\\
1.06522395114952e-06	0.0181523763664021\\
1.20679016465962e-06	0.018152455049221\\
1.36717025555763e-06	0.0181525181246097\\
1.54886455194861e-06	0.0181525692679642\\
1.75470567073192e-06	0.0181526115118927\\
1.98790267814258e-06	0.0181526473876735\\
2.25209111914369e-06	0.0181526790523969\\
2.55138969562882e-06	0.0181527084261297\\
2.89046447704834e-06	0.0181527373620604\\
3.27460164450466e-06	0.0181527678621793\\
3.70978990239752e-06	0.0181528023314787\\
4.20281384241847e-06	0.0181528438371454\\
4.76135971544073e-06	0.0181528963090774\\
5.39413526028941e-06	0.0181529645884205\\
6.11100545752489e-06	0.0181530542117083\\
6.92314632464288e-06	0.0181531708335259\\
7.84321915036034e-06	0.0181533192713702\\
8.88556788430334e-06	0.0181535023141473\\
1.00664427594039e-05	0.0181537196251135\\
1.14042536332837e-05	0.0181539671784498\\
1.29198569982199e-05	0.0181542375807817\\
1.46368811341833e-05	0.0181545213347005\\
1.65820944740897e-05	0.0181548087464687\\
1.87858229240841e-05	0.0181550919742976\\
2.12824226448884e-05	0.0181553667641572\\
2.41108156648795e-05	0.0181556336545424\\
2.73150966751156e-05	0.018155898674302\\
3.09452204662542e-05	0.0181561736633889\\
3.5057780724513e-05	0.0181564762712204\\
3.97168923281164e-05	0.0181568294810963\\
4.49951908992406e-05	0.0181572602817585\\
5.09749651944919e-05	0.0181577970135374\\
5.7749439987897e-05	0.0181584651281119\\
6.54242294465766e-05	0.0181592817119563\\
7.4118983657251e-05	0.0181602499927355\\
8.39692539729447e-05	0.0181613557124593\\
9.51286062606868e-05	0.018162567125664\\
0.000107771014995758	0.0181638392405917\\
0.000122093575526457	0.0181651212574842\\
0.000138319576793645	0.0181663649367864\\
0.000156701982408788	0.0181675315487439\\
0.000177527374360594	0.0181685960247264\\
0.000201120420832653	0.0181695482547901\\
0.00022784893778546	0.0181703924655812\\
0.000258129623212951	0.0181711459951147\\
0.000292434553470681	0.0181718386796904\\
0.000331298542953538	0.0181725137444152\\
0.000375327481860455	0.0181732307230567\\
0.00042520778203201	0.0181740705475598\\
0.000481717078121666	0.0181751424289965\\
0.000545736350932085	0.0181765913244725\\
0.000618263661919515	0.018178603535278\\
0.000700429712987355	0.0181814064630854\\
0.000793515474146391	0.0181852574446091\\
0.000898972153857128	0.0181904172187573\\
0.00101844382339219	0.018197107492151\\
0.00115379304793301	0.0182054596859145\\
0.00130712992398982	0.0182154709250988\\
0.00148084497583905	0.0182269884037063\\
0.00167764642383389	0.0182397393619062\\
0.00190060240560151	0.0182534112591566\\
0.0021531888083564	0.0182677727709143\\
0.00243934345803585	0.0182828210952883\\
0.00276352751006746	0.0182989500200491\\
0.00313079499885148	0.0183171537844324\\
0.0035468716302354	0.0183393076210796\\
0.00401824404535711	0.0183685924894653\\
0.00455226094748072	0.0184101580678432\\
0.0051572476683945	0.018472143974551\\
0.00584263596046263	0.0185671972248001\\
0.00661911103779082	0.0187146107800328\\
0.00749877815888005	0.0189431132210531\\
0.00849535134779431	0.0192940813319528\\
0.00962436719600857	0.0198244455547327\\
0.0109034270781108	0.0206078582920832\\
0.012352471557505	0.0217321912858221\\
0.0139940912601039	0.0232919413704934\\
0.0158538790625373	0.025376277918259\\
0.0179608290854958	0.0280564156537937\\
0.0203477887125226	0.0313773536731881\\
0.0230519706812316	0.035357352805636\\
0.0261155332304648	0.0399953582537882\\
0.0295862373478896	0.0452845107150037\\
0.0335181913645377	0.0512297671741041\\
0.0379726945045244	0.0578686959190988\\
0.0430191925409049	0.0652959289249479\\
0.0487363604563524	0.0736931660847102\\
0.0552133289873585	0.0833673600982491\\
0.0625510741737984	0.0947983847059136\\
0.070863991576959	0.108692970175765\\
0.0802816797081136	0.126035356579165\\
0.0909509604713219	0.148122125926402\\
0.103038168119195	0.176574701477409\\
0.116731742406472	0.213336742376739\\
0.132245166368717	0.260674321185848\\
0.149820294526163	0.321194660736001\\
0.169731123399425	0.397887798577533\\
0.192288063119511	0.494186994836839\\
0.217842776726586	0.614044755768391\\
0.246793662602209	0.762028366553567\\
0.279592065506294	0.943442401787128\\
0.31674931304892	1.16447590776445\\
0.358844687295691	1.43234351234257\\
0.406534455784135	1.75534972424825\\
0.460562102744801	2.14277982568908\\
0.521769920031932	2.60454593625146\\
0.591112138466545	3.15061504105096\\
0.669669804309735	3.79039496629373\\
0.758667632790667	4.53238223211493\\
0.859493101435968	5.38438525026818\\
0.973718080865916	6.35444876859064\\
1.1031233402818	7.45221389176631\\
1.24972630968536	8.68997319289359\\
1.41581253164384	10.0823834235159\\
1.60397129293407	11.6439772963265\\
1.81713599156346	13.3843872732308\\
2.05862986849043	15.3023106826175\\
2.33221781700257	17.380157225926\\
2.64216507746134	19.5814989148664\\
2.99330373246548	21.8527108855554\\
};
\addlegendentry{$\|\mathbf{Ax} - \mathbf{b}\|$};

\addplot [color=mycolor2,solid]
  table[row sep=crcr]{%
1.87538253595704e-16	47620661818301.5\\
2.12461726656173e-16	41711233097080.8\\
2.40697481331118e-16	35993439851674.7\\
2.72685713473941e-16	30621635463548.1\\
3.08925120119979e-16	25713189471717.9\\
3.49980674181025e-16	21341000826915.2\\
3.96492432381782e-16	17534204382122.4\\
4.49185513754134e-16	14284997029379.6\\
5.08881404254101e-16	11558545690624\\
5.76510763740636e-16	9303225071369.23\\
6.53127934977265e-16	7459396257887.02\\
7.39927380852124e-16	5965977445903.33\\
8.38262306072912e-16	4764818645866.84\\
9.49665753649293e-16	3803281446262.14\\
1.07587450505717e-15	3035514598258.67\\
1.21885615668887e-15	2422836023216.39\\
1.38083979471139e-15	1933504700009.91\\
1.56435074655435e-15	1542073861503.44\\
1.77224995080377e-15	1228484021987.75\\
2.00777855927902e-15	977053168942.971\\
2.27460846663999e-15	775509025587.71\\
2.57689955528173e-15	614165299471.984\\
2.91936454796561e-15	485283735987.673\\
3.30734247924019e-15	382612184564.632\\
3.74688193107274e-15	301060926653.4\\
4.24483533033586e-15	236474289485.504\\
4.8089657782488e-15	185462523723.749\\
5.44806807724583e-15	145270594770.414\\
6.17210584208268e-15	113670480305.771\\
6.9923668327451e-15	88870112132.1243\\
7.92163893080218e-15	69435650620.4201\\
8.97440950267838e-15	54225364245.6017\\
1.01670912579208e-14	42333926185.766\\
1.15182781235957e-14	33046037654.2409\\
1.30490351238998e-14	25798262839.2961\\
1.47832267842144e-14	20147948005.2377\\
1.67478891794262e-14	15748134452.3155\\
1.89736514267543e-14	12327455990.3054\\
2.14952131941625e-14	9674117264.45862\\
2.43518856687217e-14	7623160042.85017\\
2.75882044186257e-14	6046325434.78037\\
3.125462370339e-14	4843905396.18393\\
3.54083030565416e-14	3938061558.61134\\
4.01139984036312e-14	3267224082.01924\\
4.54450716081196e-14	2781439306.72611\\
5.14846341839652e-14	2438883064.45002\\
5.83268430054169e-14	2203898010.90303\\
6.60783682141439e-14	2046494827.3129\\
7.48600562083991e-14	1942552397.85281\\
8.48088136402428e-14	1873780822.05216\\
9.6079741792372e-14	1827056062.01811\\
1.08848554609523e-13	1793325764.77986\\
1.23314318081598e-13	1766454764.58367\\
1.39702553685533e-13	1742249089.13146\\
1.58268754268623e-13	1717731017.22094\\
1.79302367185974e-13	1690644078.17388\\
2.03131306789261e-13	1659138765.88111\\
2.30127066616556e-13	1621594136.88437\\
2.60710510982351e-13	1576544400.60001\\
2.95358436258748e-13	1522692023.87556\\
3.34611004138294e-13	1458994995.75377\\
3.7908016276316e-13	1384813550.07947\\
4.29459187006148e-13	1300090524.3286\\
4.86533486636738e-13	1205522626.20018\\
5.51192850871559e-13	1102665139.23155\\
6.24445320202089e-13	993912066.277343\\
7.07432901761553e-13	882317585.15205\\
8.01449373233844e-13	771271906.574895\\
9.07960452867686e-13	664097578.171209\\
1.02862665004687e-12	563664109.680187\\
1.16532915265731e-12	472111357.030812\\
1.32019915483536e-12	390729908.799184\\
1.49565108231746e-12	319993425.144641\\
1.69442023337484e-12	259698752.693048\\
1.91960542215596e-12	209156285.027035\\
2.17471728924723e-12	167381391.581594\\
2.46373303261411e-12	133256211.868969\\
2.7911584121792e-12	105649390.058388\\
3.16209799468923e-12	83494190.5785279\\
3.5823347339899e-12	65832143.9755846\\
4.05842012736601e-12	51831432.9305083\\
4.59777635348579e-12	40788601.7821536\\
5.20881198428144e-12	32120403.2653107\\
5.90105307471603e-12	25350620.69837\\
6.68529167412812e-12	20094969.66522\\
7.57375407444839e-12	16045838.1259551\\
8.58029141828051e-12	12957689.2345531\\
9.72059563842917e-12	10633454.7480084\\
1.10124440953759e-11	8912269.65820343\\
1.2475976726605e-11	7659397.11136455\\
1.41340100285409e-11	6759594.10405223\\
1.60123927660819e-11	6114473.77468875\\
1.81404089552453e-11	5642600.27044611\\
2.0551234401432e-11	5279864.20503934\\
2.32824539107472e-11	4978465.92662809\\
2.63766472377103e-11	4704523.36105988\\
2.98820529042884e-11	4435240.31518759\\
3.3853320239203e-11	4156448.30825456\\
3.83523613618113e-11	3860819.79852647\\
4.34493163930081e-11	3546636.95309166\\
4.92236469407986e-11	3216811.62429147\\
5.57653749079995e-11	2877854.43659739\\
6.31764859351011e-11	2538625.2286972\\
7.1572519357984e-11	2208893.29251044\\
8.10843694679576e-11	1897904.54570517\\
9.1860326155794e-11	1613214.29128094\\
1.04068386753423e-10	1359976.16913267\\
1.17898875114945e-10	1140730.64357529\\
1.33567408768466e-10	955595.879276588\\
1.51318260396711e-10	802695.294727089\\
1.71428166051932e-10	678672.916065437\\
1.94210639475258e-10	579216.435808406\\
2.2002085978078e-10	499571.548212844\\
2.49261208703455e-10	435042.939296113\\
2.82387543736592e-10	381440.012866269\\
3.19916304957242e-10	335394.160739596\\
3.62432566335019e-10	294494.999801846\\
4.10599157044353e-10	257249.725998324\\
4.65166994981609e-10	222916.159826874\\
5.26986793586735e-10	191275.031270387\\
5.97022324478994e-10	162399.420612697\\
6.76365442671451e-10	136462.50811041\\
7.66253108607572e-10	113605.407748268\\
8.68086672393724e-10	93868.1721697268\\
9.83453720869071e-10	77172.7050329485\\
1.11415282810902e-09	63339.059126756\\
1.26222159522298e-09	52116.3692467074\\
1.42996841658724e-09	43214.0194155048\\
1.6200084677491e-09	36324.7918017275\\
1.83530447605426e-09	31138.0795598596\\
2.07921290961208e-09	27346.9023523382\\
2.35553630468546e-09	24655.4558468611\\
2.66858254729017e-09	22791.4096166494\\
3.0232320332047e-09	21520.334670321\\
3.42501375341611e-09	20655.480133049\\
3.88019149117531e-09	20058.8384259822\\
4.39586147447511e-09	19634.8472830699\\
4.98006300635469e-09	19320.6975293347\\
5.64190379775879e-09	19076.5699994422\\
6.39170195689245e-09	18877.4649015413\\
7.24114684868813e-09	18707.0797088785\\
8.20348133218639e-09	18553.6189398605\\
9.29370821691347e-09	18407.1970868366\\
1.0528824156915e-08	18258.4114348224\\
1.19280846288559e-08	18097.6560814753\\
1.35133041252003e-08	17914.8208079086\\
1.53091962424876e-08	17699.1465478844\\
1.73437589666859e-08	17439.1472299502\\
1.96487111622276e-08	17122.6176541727\\
2.22599870695977e-08	16736.8155218152\\
2.52182965207006e-08	16268.936128812\\
2.85697596057712e-08	15706.9936334449\\
3.23666256862967e-08	15041.1758947568\\
3.66680879633732e-08	14265.6355983531\\
4.15412063006284e-08	13380.5116740225\\
4.70619526885367e-08	12393.7677363138\\
5.33163956489282e-08	11322.2676894107\\
6.04020420445811e-08	10191.5045174726\\
6.8429357212722e-08	9033.65607465525\\
7.75234871213494e-08	7884.13980189994\\
8.78262093968165e-08	6777.3841431085\\
9.9498143639222e-08	5742.84746571426\\
1.12721255484472e-07	4802.21572220248\\
1.27701693451363e-07	3968.25657248297\\
1.44673002800189e-07	3245.25136666929\\
1.63899766506975e-07	2630.52663622546\\
1.8568173011617e-07	2116.47789667372\\
2.10358474778347e-07	1692.57417943563\\
2.38314711325597e-07	1347.02991472668\\
2.69986277919373e-07	1068.02407681261\\
3.05866934774194e-07	844.480778595152\\
3.4651606188704e-07	666.493598480379\\
3.92567379780188e-07	525.494442487374\\
4.44738829213984e-07	414.257962525565\\
5.03843763894432e-07	326.811246071334\\
5.70803630670092e-07	258.296094871198\\
6.46662334902741e-07	204.812975492304\\
7.32602514968158e-07	163.263243939986\\
8.29963979606747e-07	131.198745008399\\
9.40264595562587e-07	106.683718241772\\
1.06522395114952e-06	88.1721396798851\\
1.20679016465962e-06	74.4042991234569\\
1.36717025555763e-06	64.3288023820004\\
1.54886455194861e-06	57.0560267895096\\
1.75470567073192e-06	51.841233248145\\
1.98790267814258e-06	48.0833635085316\\
2.25209111914369e-06	45.3212452878227\\
2.55138969562882e-06	43.2175480913609\\
2.89046447704834e-06	41.5329699004681\\
3.27460164450466e-06	40.0987080955499\\
3.70978990239752e-06	38.79377896547\\
4.20281384241847e-06	37.5298013881936\\
4.76135971544073e-06	36.2427802076725\\
5.39413526028941e-06	34.8900125786382\\
6.11100545752489e-06	33.4499766035446\\
6.92314632464288e-06	31.9232129730426\\
7.84321915036034e-06	30.332371530414\\
8.88556788430334e-06	28.7198015425192\\
1.00664427594039e-05	27.1416188084359\\
1.14042536332837e-05	25.6583219793382\\
1.29198569982199e-05	24.3236004821585\\
1.46368811341833e-05	23.1743042480185\\
1.65820944740897e-05	22.2247099085511\\
1.87858229240841e-05	21.4667493139817\\
2.12824226448884e-05	20.8754136864062\\
2.41108156648795e-05	20.4166391258845\\
2.73150966751156e-05	20.0547404801564\\
3.09452204662542e-05	19.7576437197034\\
3.5057780724513e-05	19.4996387027202\\
3.97168923281164e-05	19.2622627572742\\
4.49951908992406e-05	19.0340953341755\\
5.09749651944919e-05	18.8099990053888\\
5.7749439987897e-05	18.5900202110341\\
6.54242294465766e-05	18.3779698192517\\
7.4118983657251e-05	18.1797077736758\\
8.39692539729447e-05	18.0013110363084\\
9.51286062606868e-05	17.847470832462\\
0.000107771014995758	17.7204869732226\\
0.000122093575526457	17.6200458316065\\
0.000138319576793645	17.5436834558809\\
0.000156701982408788	17.4876222602971\\
0.000177527374360594	17.4476345090281\\
0.000201120420832653	17.4196990048498\\
0.00022784893778546	17.4003742463308\\
0.000258129623212951	17.3869261989826\\
0.000292434553470681	17.3772961060774\\
0.000331298542953538	17.3699910201443\\
0.000375327481860455	17.3639553588172\\
0.00042520778203201	17.3584558619513\\
0.000481717078121666	17.3529932961744\\
0.000545736350932085	17.3472429100169\\
0.000618263661919515	17.341019679124\\
0.000700429712987355	17.3342612065284\\
0.000793515474146391	17.3270191606536\\
0.000898972153857128	17.3194491509747\\
0.00101844382339219	17.3117899742429\\
0.00115379304793301	17.3043274774027\\
0.00130712992398982	17.2973457458121\\
0.00148084497583905	17.2910761664777\\
0.00167764642383389	17.2856586075777\\
0.00190060240560151	17.2811255115118\\
0.0021531888083564	17.2774106858801\\
0.00243934345803585	17.2743753940495\\
0.00276352751006746	17.2718400084435\\
0.00313079499885148	17.2696110803544\\
0.0035468716302354	17.2674986123708\\
0.00401824404535711	17.2653231939327\\
0.00455226094748072	17.2629155727706\\
0.0051572476683945	17.2601120971509\\
0.00584263596046263	17.2567491083796\\
0.00661911103779082	17.2526586469356\\
0.00749877815888005	17.247667246028\\
0.00849535134779431	17.2415992253107\\
0.00962436719600857	17.2342855951055\\
0.0109034270781108	17.2255791142904\\
0.012352471557505	17.2153748667355\\
0.0139940912601039	17.2036337635281\\
0.0158538790625373	17.190403932112\\
0.0179608290854958	17.1758329867761\\
0.0203477887125226	17.1601641584668\\
0.0230519706812316	17.1437123356715\\
0.0261155332304648	17.1268218568296\\
0.0295862373478896	17.1098140333552\\
0.0335181913645377	17.0929356498126\\
0.0379726945045244	17.0763181585803\\
0.0430191925409049	17.0599517162565\\
0.0487363604563524	17.0436716264935\\
0.0552133289873585	17.0271507271803\\
0.0625510741737984	17.0098917375369\\
0.070863991576959	16.9912174244033\\
0.0802816797081136	16.9702601999975\\
0.0909509604713219	16.9459536590482\\
0.103038168119195	16.9170269171482\\
0.116731742406472	16.8820009889756\\
0.132245166368717	16.839186748059\\
0.149820294526163	16.7866857471757\\
0.169731123399425	16.7223961818918\\
0.192288063119511	16.6440245456799\\
0.217842776726586	16.5490987247016\\
0.246793662602209	16.4349727031116\\
0.279592065506294	16.2988119066682\\
0.31674931304892	16.1375580932622\\
0.358844687295691	15.9478966125634\\
0.406534455784135	15.7262790840067\\
0.460562102744801	15.4690697497164\\
0.521769920031932	15.1728581237778\\
0.591112138466545	14.8349038825443\\
0.669669804309735	14.4535750411124\\
0.758667632790667	14.0285628848083\\
0.859493101435968	13.5606736047086\\
0.973718080865916	13.0511470543679\\
1.1031233402818	12.5007065901597\\
1.24972630968536	11.9087794700829\\
1.41581253164384	11.2733840475153\\
1.60397129293407	10.5919740118061\\
1.81713599156346	9.86313839557892\\
2.05862986849043	9.08867148344933\\
2.33221781700257	8.27532165157653\\
2.64216507746134	7.43557013542991\\
2.99330373246548	6.58705798576191\\
};
\addlegendentry{$\|\mathbf{Lx}\|$};

\addplot [color=mycolor2,only marks,mark=asterisk,mark options={solid},forget plot]
  table[row sep=crcr]{%
0.000700429712987355	17.3342612065284\\
};
\addplot [color=mycolor1,only marks,mark=asterisk,mark options={solid},forget plot]
  table[row sep=crcr]{%
0.000700429712987355	0.0181814064630854\\
};
\end{axis}
\end{tikzpicture}%
\end{document}
% This file was created by matlab2tikz.
% Minimal pgfplots version: 1.3
%
%The latest updates can be retrieved from
%  http://www.mathworks.com/matlabcentral/fileexchange/22022-matlab2tikz
%where you can also make suggestions and rate matlab2tikz.
%
\documentclass[tikz]{standalone}
\usepackage{pgfplots}
\usepackage{grffile}
\pgfplotsset{compat=newest}
\usetikzlibrary{plotmarks}
\usepackage{amsmath}

\begin{document}
\definecolor{mycolor1}{rgb}{0.00000,0.44700,0.74100}%
\definecolor{mycolor2}{rgb}{0.85000,0.32500,0.09800}%
%
\begin{tikzpicture}

\begin{axis}[%
width=2.0in,
height=2.0in,
at={(0.758333in,0.48125in)},
scale only axis,
xmode=log,
xmin=0.01,
xmax=100,
xminorticks=true,
xlabel={$\|\mathbf{Ax} - \mathbf{b}\|$},
ymode=log,
ymin=1,
ymax=100000000000000,
yminorticks=true,
ylabel={$\|\mathbf{Lx}\|$}
]
\addplot [color=mycolor1,solid,forget plot]
  table[row sep=crcr]{%
0.0193038738470416	47620661818301.5\\
0.0185090239982361	41711233097080.8\\
0.0189429358760381	35993439851674.7\\
0.0180029271976985	30621635463548.1\\
0.0183563234880474	25713189471717.9\\
0.0180723180565316	21341000826915.2\\
0.018098760751986	17534204382122.4\\
0.0179955590412866	14284997029379.6\\
0.0180920637748833	11558545690624\\
0.0179900880618026	9303225071369.23\\
0.0180069571996471	7459396257887.02\\
0.0181788301346268	5965977445903.33\\
0.0180470248323015	4764818645866.84\\
0.0180465064202844	3803281446262.14\\
0.0180670192553766	3035514598258.67\\
0.0180490192730289	2422836023216.39\\
0.018060774607695	1933504700009.91\\
0.0180518060427566	1542073861503.44\\
0.0180536444583133	1228484021987.75\\
0.0180501877891216	977053168942.971\\
0.01805539261113	775509025587.71\\
0.018057863565329	614165299471.984\\
0.0180574257643346	485283735987.673\\
0.0180577052735867	382612184564.632\\
0.0180583034305718	301060926653.4\\
0.0180592879039267	236474289485.504\\
0.0180598732976866	185462523723.749\\
0.0180607706350695	145270594770.414\\
0.0180612331281622	113670480305.771\\
0.0180618655544502	88870112132.1243\\
0.0180620995643229	69435650620.4201\\
0.0180623774386299	54225364245.6017\\
0.0180625253358517	42333926185.766\\
0.0180627259468327	33046037654.2409\\
0.0180629199521028	25798262839.2961\\
0.0180629796137325	20147948005.2377\\
0.018063107539685	15748134452.3155\\
0.0180631718251209	12327455990.3054\\
0.0180632308426069	9674117264.45862\\
0.0180632804143862	7623160042.85017\\
0.0180633046909893	6046325434.78037\\
0.0180633355579015	4843905396.18393\\
0.0180633615010352	3938061558.61134\\
0.0180633761684266	3267224082.01924\\
0.0180633885375935	2781439306.72611\\
0.0180633998597478	2438883064.45002\\
0.018063415704104	2203898010.90303\\
0.0180634132651175	2046494827.3129\\
0.0180634136748239	1942552397.85281\\
0.0180634340959456	1873780822.05216\\
0.0180634393318133	1827056062.01811\\
0.0180634509334252	1793325764.77986\\
0.0180634759672453	1766454764.58367\\
0.0180635054195379	1742249089.13146\\
0.0180635407974204	1717731017.22094\\
0.0180635922095068	1690644078.17388\\
0.0180637067459649	1659138765.88111\\
0.0180638557074881	1621594136.88437\\
0.0180640825675037	1576544400.60001\\
0.0180644518704336	1522692023.87556\\
0.0180649606567773	1458994995.75377\\
0.0180657092731537	1384813550.07947\\
0.0180667318083721	1300090524.3286\\
0.0180681117168792	1205522626.20018\\
0.0180698700265487	1102665139.23155\\
0.018072044388082	993912066.277343\\
0.0180745998601917	882317585.15205\\
0.0180774719411406	771271906.574895\\
0.0180805663402366	664097578.171209\\
0.0180837436444513	563664109.680187\\
0.0180868751950086	472111357.030812\\
0.0180898484684424	390729908.799184\\
0.0180925825351158	319993425.144641\\
0.0180950189388918	259698752.693048\\
0.0180971352766312	209156285.027035\\
0.0180989389461052	167381391.581594\\
0.0181004477978187	133256211.868969\\
0.0181016914554518	105649390.058388\\
0.0181027058509076	83494190.5785279\\
0.018103525445822	65832143.9755846\\
0.0181041822898649	51831432.9305083\\
0.0181047055377142	40788601.7821536\\
0.0181051200258215	32120403.2653107\\
0.0181054481966439	25350620.69837\\
0.0181057063322298	20094969.66522\\
0.0181059093222394	16045838.1259551\\
0.0181060688474578	12957689.2345531\\
0.0181061942471563	10633454.7480084\\
0.0181062930406997	8912269.65820343\\
0.0181063711722556	7659397.11136455\\
0.0181064338855667	6759594.10405223\\
0.018106485459539	6114473.77468875\\
0.0181065297522824	5642600.27044611\\
0.0181065704246783	5279864.20503934\\
0.0181066111829758	4978465.92662809\\
0.018106656177585	4704523.36105988\\
0.0181067098431584	4435240.31518759\\
0.0181067768836964	4156448.30825456\\
0.018106862077281	3860819.79852647\\
0.0181069693973531	3546636.95309166\\
0.0181071012874678	3216811.62429147\\
0.0181072579651559	2877854.43659739\\
0.0181074366535289	2538625.2286972\\
0.0181076318608298	2208893.29251044\\
0.0181078361042111	1897904.54570517\\
0.0181080410735101	1613214.29128094\\
0.0181082390846518	1359976.16913267\\
0.0181084240226779	1140730.64357529\\
0.0181085919505398	955595.879276588\\
0.0181087411979539	802695.294727089\\
0.0181088720737481	678672.916065437\\
0.0181089864631304	579216.435808406\\
0.0181090873250326	499571.548212844\\
0.0181091782399825	435042.939296113\\
0.0181092629827193	381440.012866269\\
0.0181093450767586	335394.160739596\\
0.0181094273636529	294494.999801846\\
0.0181095116310456	257249.725998324\\
0.0181095983919488	222916.159826874\\
0.0181096868806495	191275.031270387\\
0.0181097753390293	162399.420612697\\
0.0181098614619215	136462.50811041\\
0.0181099429192705	113605.407748268\\
0.0181100177743172	93868.1721697268\\
0.0181100847352402	77172.7050329485\\
0.0181101432073161	63339.059126756\\
0.0181101932131666	52116.3692467074\\
0.0181102352354042	43214.0194155048\\
0.0181102700501234	36324.7918017275\\
0.0181102985777661	31138.0795598596\\
0.0181103217761277	27346.9023523382\\
0.0181103405699204	24655.4558468611\\
0.0181103558136603	22791.4096166494\\
0.0181103682760122	21520.334670321\\
0.0181103786394604	20655.480133049\\
0.0181103875037787	20058.8384259822\\
0.0181103953932443	19634.8472830699\\
0.0181104027642407	19320.6975293347\\
0.0181104100182647	19076.5699994422\\
0.0181104175316245	18877.4649015413\\
0.0181104257119643	18707.0797088785\\
0.0181104350974241	18553.6189398605\\
0.0181104465128206	18407.1970868366\\
0.0181104613009683	18258.4114348224\\
0.0181104816587315	18097.6560814753\\
0.0181105111211314	17914.8208079086\\
0.018110555257472	17699.1465478844\\
0.0181106226518086	17439.1472299502\\
0.0181107262312567	17122.6176541727\\
0.0181108849489687	16736.8155218152\\
0.0181111256975603	16268.936128812\\
0.0181114850852605	15706.9936334449\\
0.0181120103547438	15041.1758947568\\
0.018112758337166	14265.6355983531\\
0.018113791144275	13380.5116740225\\
0.0181151676757018	12393.7677363138\\
0.0181169312761498	11322.2676894107\\
0.0181190959312347	10191.5045174726\\
0.0181216353324856	9033.65607465525\\
0.018124479570665	7884.13980189994\\
0.0181275221273165	6777.3841431085\\
0.0181306358842011	5742.84746571426\\
0.01813369322757	4802.21572220248\\
0.0181365841176291	3968.25657248297\\
0.0181392276520529	3245.25136666929\\
0.0181415757863525	2630.52663622546\\
0.0181436106071437	2116.47789667372\\
0.0181453378529988	1692.57417943563\\
0.0181467792936709	1347.02991472668\\
0.0181479657612466	1068.02407681261\\
0.0181489316980804	844.480778595152\\
0.018149711391972	666.493598480379\\
0.0181503366823762	525.494442487374\\
0.0181508357816535	414.257962525565\\
0.0181512328597907	326.811246071334\\
0.0181515481094255	258.296094871198\\
0.0181517980860786	204.812975492304\\
0.018151996183355	163.263243939986\\
0.018152153149073	131.198745008399\\
0.0181522775810241	106.683718241772\\
0.0181523763664021	88.1721396798851\\
0.018152455049221	74.4042991234569\\
0.0181525181246097	64.3288023820004\\
0.0181525692679642	57.0560267895096\\
0.0181526115118927	51.841233248145\\
0.0181526473876735	48.0833635085316\\
0.0181526790523969	45.3212452878227\\
0.0181527084261297	43.2175480913609\\
0.0181527373620604	41.5329699004681\\
0.0181527678621793	40.0987080955499\\
0.0181528023314787	38.79377896547\\
0.0181528438371454	37.5298013881936\\
0.0181528963090774	36.2427802076725\\
0.0181529645884205	34.8900125786382\\
0.0181530542117083	33.4499766035446\\
0.0181531708335259	31.9232129730426\\
0.0181533192713702	30.332371530414\\
0.0181535023141473	28.7198015425192\\
0.0181537196251135	27.1416188084359\\
0.0181539671784498	25.6583219793382\\
0.0181542375807817	24.3236004821585\\
0.0181545213347005	23.1743042480185\\
0.0181548087464687	22.2247099085511\\
0.0181550919742976	21.4667493139817\\
0.0181553667641572	20.8754136864062\\
0.0181556336545424	20.4166391258845\\
0.018155898674302	20.0547404801564\\
0.0181561736633889	19.7576437197034\\
0.0181564762712204	19.4996387027202\\
0.0181568294810963	19.2622627572742\\
0.0181572602817585	19.0340953341755\\
0.0181577970135374	18.8099990053888\\
0.0181584651281119	18.5900202110341\\
0.0181592817119563	18.3779698192517\\
0.0181602499927355	18.1797077736758\\
0.0181613557124593	18.0013110363084\\
0.018162567125664	17.847470832462\\
0.0181638392405917	17.7204869732226\\
0.0181651212574842	17.6200458316065\\
0.0181663649367864	17.5436834558809\\
0.0181675315487439	17.4876222602971\\
0.0181685960247264	17.4476345090281\\
0.0181695482547901	17.4196990048498\\
0.0181703924655812	17.4003742463308\\
0.0181711459951147	17.3869261989826\\
0.0181718386796904	17.3772961060774\\
0.0181725137444152	17.3699910201443\\
0.0181732307230567	17.3639553588172\\
0.0181740705475598	17.3584558619513\\
0.0181751424289965	17.3529932961744\\
0.0181765913244725	17.3472429100169\\
0.018178603535278	17.341019679124\\
0.0181814064630854	17.3342612065284\\
0.0181852574446091	17.3270191606536\\
0.0181904172187573	17.3194491509747\\
0.018197107492151	17.3117899742429\\
0.0182054596859145	17.3043274774027\\
0.0182154709250988	17.2973457458121\\
0.0182269884037063	17.2910761664777\\
0.0182397393619062	17.2856586075777\\
0.0182534112591566	17.2811255115118\\
0.0182677727709143	17.2774106858801\\
0.0182828210952883	17.2743753940495\\
0.0182989500200491	17.2718400084435\\
0.0183171537844324	17.2696110803544\\
0.0183393076210796	17.2674986123708\\
0.0183685924894653	17.2653231939327\\
0.0184101580678432	17.2629155727706\\
0.018472143974551	17.2601120971509\\
0.0185671972248001	17.2567491083796\\
0.0187146107800328	17.2526586469356\\
0.0189431132210531	17.247667246028\\
0.0192940813319528	17.2415992253107\\
0.0198244455547327	17.2342855951055\\
0.0206078582920832	17.2255791142904\\
0.0217321912858221	17.2153748667355\\
0.0232919413704934	17.2036337635281\\
0.025376277918259	17.190403932112\\
0.0280564156537937	17.1758329867761\\
0.0313773536731881	17.1601641584668\\
0.035357352805636	17.1437123356715\\
0.0399953582537882	17.1268218568296\\
0.0452845107150037	17.1098140333552\\
0.0512297671741041	17.0929356498126\\
0.0578686959190988	17.0763181585803\\
0.0652959289249479	17.0599517162565\\
0.0736931660847102	17.0436716264935\\
0.0833673600982491	17.0271507271803\\
0.0947983847059136	17.0098917375369\\
0.108692970175765	16.9912174244033\\
0.126035356579165	16.9702601999975\\
0.148122125926402	16.9459536590482\\
0.176574701477409	16.9170269171482\\
0.213336742376739	16.8820009889756\\
0.260674321185848	16.839186748059\\
0.321194660736001	16.7866857471757\\
0.397887798577533	16.7223961818918\\
0.494186994836839	16.6440245456799\\
0.614044755768391	16.5490987247016\\
0.762028366553567	16.4349727031116\\
0.943442401787128	16.2988119066682\\
1.16447590776445	16.1375580932622\\
1.43234351234257	15.9478966125634\\
1.75534972424825	15.7262790840067\\
2.14277982568908	15.4690697497164\\
2.60454593625146	15.1728581237778\\
3.15061504105096	14.8349038825443\\
3.79039496629373	14.4535750411124\\
4.53238223211493	14.0285628848083\\
5.38438525026818	13.5606736047086\\
6.35444876859064	13.0511470543679\\
7.45221389176631	12.5007065901597\\
8.68997319289359	11.9087794700829\\
10.0823834235159	11.2733840475153\\
11.6439772963265	10.5919740118061\\
13.3843872732308	9.86313839557892\\
15.3023106826175	9.08867148344933\\
17.380157225926	8.27532165157653\\
19.5814989148664	7.43557013542991\\
21.8527108855554	6.58705798576191\\
};
\addplot [color=mycolor1,only marks,mark=asterisk,mark options={solid},forget plot]
  table[row sep=crcr]{%
0.0181814064630854	17.3342612065284\\
};
\end{axis}
\end{tikzpicture}%
\end{document}
% This file was created by matlab2tikz.
% Minimal pgfplots version: 1.3
%
%The latest updates can be retrieved from
%  http://www.mathworks.com/matlabcentral/fileexchange/22022-matlab2tikz
%where you can also make suggestions and rate matlab2tikz.
%
\documentclass[tikz]{standalone}
\usepackage{pgfplots}
\usepackage{grffile}
\pgfplotsset{compat=newest}
\usetikzlibrary{plotmarks}
\usepackage{amsmath}

\begin{document}
\definecolor{mycolor1}{rgb}{0.00000,0.44700,0.74100}%
\definecolor{mycolor2}{rgb}{0.85000,0.32500,0.09800}%
%
\begin{tikzpicture}

\begin{axis}[%
width=2.0in,
height=2.0in,
at={(0.758333in,0.48125in)},
scale only axis,
xmin=-40,
xmax=5,
xlabel={$\lambda$},
ymin=-50,
ymax=450,
ylabel={curvature $\kappa$}
]
\addplot [color=mycolor1,solid,forget plot]
  table[row sep=crcr]{%
-36.2125488301144	6.09221813763179\\
-36.0877698116034	-5.97415018582107\\
-35.9629907930924	4.36603306252821\\
-35.8382117745814	-2.06301735698658\\
-35.7134327560705	0.900283790775962\\
-35.5886537375595	-0.331232808607055\\
-35.4638747190485	0.472621373823182\\
-35.3390957005376	-0.437814656678314\\
-35.2143166820266	0.25037927298571\\
-35.0895376635156	0.309411862385457\\
-34.9647586450047	-0.593937861355489\\
-34.8399796264937	0.2534356897126\\
-34.7152006079827	0.040474821344752\\
-34.5904215894717	-0.0740462660751898\\
-34.4656425709608	0.0572623601109025\\
-34.3408635524498	-0.0398709522724985\\
-34.2160845339388	0.0207363471802273\\
-34.0913055154279	-0.0100437039176201\\
-33.9665264969169	0.0162096641568897\\
-33.8417474784059	-0.00510236586067235\\
-33.716968459895	-0.00527017663008731\\
-33.592189441384	0.00127127679454228\\
-33.467410422873	0.000547212932726237\\
-33.342631404362	0.000649706517421663\\
-33.2178523858511	-0.00067945839569997\\
-33.0930733673401	0.000511348522553965\\
-32.9682943488291	-0.000718749166347963\\
-32.8435153303182	0.000273474031876721\\
-32.7187363118072	-0.000645529421867146\\
-32.5939572932962	6.96704900042877e-05\\
-32.4691782747852	-0.000208361738306897\\
-32.3443992562743	8.39427712862944e-05\\
-32.2196202377633	-1.040762496761e-05\\
-32.0948412192523	-0.000213698498767569\\
-31.9700622007414	0.000109493966419686\\
-31.8452831822304	-0.000101220461159557\\
-31.7205041637194	-7.50806445875848e-06\\
-31.5957251452085	-1.40490930053897e-05\\
-31.4709461266975	-4.12116729078651e-05\\
-31.3461671081865	1.39019066090379e-05\\
-31.2213880896755	-5.72095957800476e-06\\
-31.0966090711646	-1.99145507969682e-05\\
-30.9718300526536	-7.6724150638199e-07\\
-30.8470510341426	4.61332980511031e-06\\
-30.7222720156317	4.02654591880189e-05\\
-30.5974929971207	-0.000133564304871119\\
-30.4727139786097	3.78321265558738e-05\\
-30.3479349600988	0.000724348185444229\\
-30.2231559415878	-0.000682351267204543\\
-30.0983769230768	0.00118594659670613\\
-29.9735979045658	0.00440045618153702\\
-29.8488188860549	0.00281854697874944\\
-29.7240398675439	0.00263112231178817\\
-29.5992608490329	0.00571000275586538\\
-29.474481830522	0.0207699192354251\\
-29.349702812011	0.00264966779209324\\
-29.2249237935	0.00811438865485744\\
-29.100144774989	0.0110122128388491\\
-28.9753657564781	0.00442668789008433\\
-28.8505867379671	0.00681201594394825\\
-28.7258077194561	0.00419127144712097\\
-28.6010287009452	0.00383986568231801\\
-28.4762496824342	0.00220336768433706\\
-28.3514706639232	0.00155895684371394\\
-28.2266916454123	0.000554760232770241\\
-28.1019126269013	-0.000100345992147451\\
-27.9771336083903	-0.000543154876274834\\
-27.8523545898793	-0.00093628955476088\\
-27.7275755713684	-0.00110207017103588\\
-27.6027965528574	-0.00114965405007815\\
-27.4780175343464	-0.00110402892426873\\
-27.3532385158355	-0.00103258879194002\\
-27.2284594973245	-0.000917190667153013\\
-27.1036804788135	-0.00078115599706141\\
-26.9789014603026	-0.000662200643779638\\
-26.8541224417916	-0.000549234768192029\\
-26.7293434232806	-0.000446045061996189\\
-26.6045644047696	-0.000360662409480899\\
-26.4797853862587	-0.000289607890877613\\
-26.3550063677477	-0.00023014692995196\\
-26.2302273492367	-0.000182228809182102\\
-26.1054483307258	-0.000140880690508111\\
-25.9806693122148	-0.000111396805495181\\
-25.8558902937038	-8.49387335372302e-05\\
-25.7311112751929	-6.4141486975728e-05\\
-25.6063322566819	-4.72206952711894e-05\\
-25.4815532381709	-3.29763826431526e-05\\
-25.3567742196599	-2.07326799362556e-05\\
-25.231995201149	-7.38558160553557e-06\\
-25.107216182638	7.73298625720086e-06\\
-24.982437164127	2.9046292579525e-05\\
-24.8576581456161	6.08903298580817e-05\\
-24.7328791271051	0.000105527265276155\\
-24.6081001085941	0.00016223182028646\\
-24.4833210900832	0.000207166779587538\\
-24.3585420715722	0.000222592517170538\\
-24.2337630530612	0.000208262343199567\\
-24.1089840345502	0.000166761908359042\\
-23.9842050160393	0.000114682051051417\\
-23.8594259975283	6.37059666776792e-05\\
-23.7346469790173	1.7311148686665e-05\\
-23.6098679605064	-1.86789003541832e-05\\
-23.4850889419954	-4.36102373591991e-05\\
-23.3603099234844	-5.83764678517638e-05\\
-23.2355309049735	-6.40470983162839e-05\\
-23.1107518864625	-6.32418759788027e-05\\
-22.9859728679515	-5.79479016285982e-05\\
-22.8611938494405	-5.0029819828102e-05\\
-22.7364148309296	-4.10081278569379e-05\\
-22.6116358124186	-3.17649047529719e-05\\
-22.4868567939076	-2.29881711733246e-05\\
-22.3620777753967	-1.50964980312454e-05\\
-22.2372987568857	-8.6273157032172e-06\\
-22.1125197383747	-4.60197190352032e-06\\
-21.9877407198637	-4.11822437713305e-06\\
-21.8629617013528	-7.44438422902013e-06\\
-21.7381826828418	-1.34128431306694e-05\\
-21.6134036643308	-2.00815246088264e-05\\
-21.4886246458199	-2.55441999920079e-05\\
-21.3638456273089	-2.88754294459783e-05\\
-21.2390666087979	-2.98561639056115e-05\\
-21.114287590287	-2.88465303648591e-05\\
-20.989508571776	-2.63855685073664e-05\\
-20.864729553265	-2.30506031920026e-05\\
-20.739950534754	-1.92852927386687e-05\\
-20.6151715162431	-1.53959567182581e-05\\
-20.4903924977321	-1.15170972488858e-05\\
-20.3656134792211	-7.62399276759472e-06\\
-20.2408344607102	-3.45308463525006e-06\\
-20.1160554421992	1.62567359314013e-06\\
-19.9912764236882	8.92858546486201e-06\\
-19.8664974051773	2.10570786656304e-05\\
-19.7417183866663	4.3089592749938e-05\\
-19.6169393681553	8.45152974927789e-05\\
-19.4921603496443	0.000162078493945149\\
-19.3673813311334	0.00030215285595726\\
-19.2426023126224	0.000540613034647328\\
-19.1178232941114	0.000921189619940612\\
-18.9930442756005	0.00149434144881491\\
-18.8682652570895	0.00231656811391592\\
-18.7434862385785	0.00343040142208491\\
-18.6187072200675	0.00480601976419874\\
-18.4939282015566	0.00628058780607481\\
-18.3691491830456	0.00758759388306072\\
-18.2443701645346	0.00850230910587312\\
-18.1195911460237	0.00894811040807759\\
-17.9948121275127	0.00897389066312171\\
-17.8700331090017	0.00867160425455564\\
-17.7452540904908	0.00812284313634873\\
-17.6204750719798	0.00738542638110644\\
-17.4956960534688	0.00650099826356843\\
-17.3709170349578	0.00550624614266943\\
-17.2461380164469	0.00444165174281292\\
-17.1213589979359	0.00335546986025783\\
-16.9965799794249	0.00230241143695523\\
-16.871800960914	0.00133805941587693\\
-16.747021942403	0.000510408605895772\\
-16.622242923892	-0.000148100713294696\\
-16.4974639053811	-0.000625437459796836\\
-16.3726848868701	-0.000929914102789099\\
-16.2479058683591	-0.00108562029710541\\
-16.1231268498481	-0.00112540392047554\\
-15.9983478313372	-0.00108386201858586\\
-15.8735688128262	-0.000992115077415015\\
-15.7487897943152	-0.00087500448714252\\
-15.6240107758043	-0.00075039449854594\\
-15.4992317572933	-0.000629854377587039\\
-15.3744527387823	-0.000519972434313055\\
-15.2496737202714	-0.000423775549587433\\
-15.1248947017604	-0.000341961430289191\\
-15.0001156832494	-0.000273842412635628\\
-14.8753366647384	-0.00021800332535338\\
-14.7505576462275	-0.000172730517398879\\
-14.6257786277165	-0.000136274335096347\\
-14.5009996092055	-0.00010700091056545\\
-14.3762205906946	-8.34694997222682e-05\\
-14.2514415721836	-6.44574648516033e-05\\
-14.1266625536726	-4.89439048070829e-05\\
-14.0018835351617	-3.60544101326613e-05\\
-13.8771045166507	-2.49594966156811e-05\\
-13.7523254981397	-1.47083683234717e-05\\
-13.6275464796287	-3.96935883171649e-06\\
-13.5027674611178	9.35816702959614e-06\\
-13.3779884426068	2.86908967436622e-05\\
-13.2532094240958	5.94171198938003e-05\\
-13.1284304055849	0.00010932154520109\\
-13.0036513870739	0.000187882539757719\\
-12.8788723685629	0.000302887120851669\\
-12.754093350052	0.000453262038188594\\
-12.629314331541	0.00062143284863955\\
-12.50453531303	0.000775324027114705\\
-12.379756294519	0.000884828940500922\\
-12.2549772760081	0.000938652491956517\\
-12.1301982574971	0.000946157596463981\\
-12.0054192389861	0.000928167700843164\\
-11.8806402204752	0.000908409189577197\\
-11.7558612019642	0.000910258659789876\\
-11.6310821834532	0.000957832808579899\\
-11.5063031649422	0.00108032448318703\\
-11.3815241464313	0.00132030262581064\\
-11.2567451279203	0.00174871901254032\\
-11.1319661094093	0.00249184346213611\\
-11.0071870908984	0.00377855235956079\\
-10.8824080723874	0.00601684966722558\\
-10.7576290538764	0.00988819203664231\\
-10.6328500353655	0.0163610200731254\\
-10.5080710168545	0.0263736356228967\\
-10.3832919983435	0.0400871788877738\\
-10.2585129798325	0.0564895272370124\\
-10.1337339613216	0.074326977287665\\
-10.0089549428106	0.0935538498576371\\
-9.88417592429964	0.115912526295427\\
-9.75939690578866	0.144909160184417\\
-9.6346178872777	0.186180564466739\\
-9.50983886876672	0.248870238663852\\
-9.38505985025575	0.348591351978152\\
-9.26028083174478	0.51319527698798\\
-9.13550181323381	0.793969748532292\\
-9.01072279472284	1.28766909162571\\
-8.88594377621187	2.18040712184725\\
-8.7611647577009	3.83587452409376\\
-8.63638573918993	6.97276690272807\\
-8.51160672067896	13.0138545305035\\
-8.38682770216799	24.7112392107127\\
-8.26204868365702	46.9327519594366\\
-8.13726966514605	86.3457274272016\\
-8.01249064663508	146.976559170558\\
-7.88771162812411	222.654170035361\\
-7.76293260961314	298.011398184918\\
-7.63815359110217	360.376526076378\\
-7.5133745725912	405.746486522671\\
-7.38859555408023	434.516540079604\\
-7.26381653556926	446.432105034218\\
-7.13903751705829	439.469503925818\\
-7.01425849854732	412.102079152361\\
-6.88947948003635	366.667419601104\\
-6.76470046152538	310.481048127001\\
-6.63992144301441	252.995151641012\\
-6.51514242450344	201.565187361554\\
-6.39036340599247	159.457554274747\\
-6.2655843874815	126.526981598365\\
-6.14080536897053	100.75412338966\\
-6.01602635045956	79.4332042456975\\
-5.89124733194859	60.1845342811617\\
-5.76646831343761	42.2315137180986\\
-5.64168929492665	26.8096281998068\\
-5.51691027641568	15.4526920741129\\
-5.39213125790471	8.27379788206151\\
-5.26735223939374	4.22431547262584\\
-5.14257322088277	2.09873757468237\\
-5.01779420237179	1.02787891196496\\
-4.89301518386082	0.499996574954822\\
-4.76823616534985	0.242551826177656\\
-4.64345714683888	0.11755162769285\\
-4.51867812832791	0.056869140710867\\
-4.39389910981695	0.0273010674373336\\
-4.26912009130597	0.0127918425048574\\
-4.144341072795	0.00562935110804931\\
-4.01956205428404	0.00212961561648531\\
-3.89478303577306	0.000530584610580381\\
-3.77000401726209	-3.21415666995049e-05\\
-3.64522499875112	-2.0873092520716e-05\\
-3.52044598024015	0.00026712078364543\\
-3.39566696172918	0.000590457188754978\\
-3.27088794321821	0.000720421046634405\\
-3.14610892470724	0.000436608357775251\\
-3.02132990619627	-0.000446839626993569\\
-2.8965508876853	-0.00202925553870089\\
-2.77177186917433	-0.00427334376300568\\
-2.64699285066336	-0.00701656823503399\\
-2.52221383215239	-0.0100733244865916\\
-2.39743481364142	-0.0133584977830454\\
-2.27265579513045	-0.0169345303090322\\
-2.14787677661948	-0.0209728487104091\\
-2.02309775810851	-0.025690605525539\\
-1.89831873959754	-0.0313093797670349\\
-1.77353972108657	-0.0380442944601847\\
-1.6487607025756	-0.0461150589704604\\
-1.52398168406463	-0.0557695166085355\\
-1.39920266555366	-0.0673104298729809\\
-1.27442364704269	-0.0811125346880275\\
-1.14964462853172	-0.0976137732332702\\
-1.02486561002075	-0.1172707439141\\
-0.90008659150978	-0.140488070111668\\
-0.775307572998809	-0.1675561517084\\
-0.650528554487835	-0.19864591386477\\
-0.525749535976864	-0.233907787308602\\
-0.400970517465898	-0.273710272512796\\
-0.276191498954924	-0.319016338979791\\
-0.151412480443953	-0.371793991233108\\
-0.0266334619329871	-0.435188018067786\\
0.0981455565779833	-0.513058054470466\\
0.222924575088958	-0.60861869402061\\
0.347703593599924	-0.722345869913085\\
0.472482612110894	-0.84968183585494\\
0.597261630621869	-0.979092294355523\\
0.722040649132835	-1.0913479538456\\
0.846819667643806	-1.16211943650845\\
};
\addplot [color=mycolor1,only marks,mark=asterisk,mark options={solid},forget plot]
  table[row sep=crcr]{%
-7.26381653556926	446.432105034218\\
};
\end{axis}
\end{tikzpicture}%
\end{document}
\caption{Plots of $\|\mathbf{Ax} - \mathbf{b}\|$, $\|\mathbf{Lx}\|$ and the curvature of the L-curve $\kappa$. For the given \texttt{A1}, \texttt{berr1} pair.} 
\label{fig:A1LTihk}
\end{figure} 
We are interested in obtaining a result as close to the noise free data as possible. The trade off shown in the L-curve is illustrated in another way in figure~\ref{fig:knownFitFilt}.\footnote{following a similar example in: The L-curve and its use in the numerical treatment of inverse problems, P. C. Hansen} When choosing a good filter parameter it is important to remove noise contributions, which is done by minimizing $\|\mathbf{Lx}\|$, while at the same time making sure that the data points are still being followed in a satisfactory way, which is done by keeping $\|\mathbf{Ax} - \mathbf{b}\|$ as small as possible. In practice $\mathbf{b}_{true}$ is unknown, it has been used in the experiment shown in figure~\ref{fig:knownFitFilt} only to verify that the L-curve criterion which has been computed using the noisy $\mathbf{b}$ is indeed working. 
The optimal point can be found at the pointy edge of the L-curve, ideally the curvature will display a maximal value here. Following Hansen once more the curvature $\kappa$ has been computed using\footnote{The L-curve and its use in the numerical treatment of inverse problems, P. C. Hansen}
\begin{equation}
\kappa = 2 \cdot \frac{\hat{\rho}'\hat{\eta}''- \hat{\rho}''\hat{\eta}' }{((\hat{\rho}')^2)+ (\hat{\eta}')^2)^{3/2} }.
\end{equation}
with $\hat{\eta} = log(\|x_{\lambda}\|_2^2)$ and $\hat{\rho} = log(\|Ax_{\lambda} - b\|_2^2)$. The derivatives have been implemented by simply using right forward differences
\begin{equation}
\hat{\rho}' = \frac{(\hat{\rho}_{k+1} - \hat{\rho}_{k})}{(\lambda_{k+1} - \lambda_{k})}.
\end{equation}
For the second derivatives the same process has simply been applied again. $\hat{\eta}'$ and $\hat{\eta}''$ have been computed in the same manner. The plot of the curvature which has been obtained by using the formulas above is shown in figure~\ref{fig:A1LTihk} in the bottom left. In all plots in the figure the location of the maximum of $\kappa$ has been indicated with stars(*). It is important to notice that $\lambda$ increases when following the L-curve from the top left to the bottom right.
\begin{figure}
% This file was created by matlab2tikz.
% Minimal pgfplots version: 1.3
%
%The latest updates can be retrieved from
%  http://www.mathworks.com/matlabcentral/fileexchange/22022-matlab2tikz
%where you can also make suggestions and rate matlab2tikz.
%
\documentclass[tikz]{standalone}
\usepackage{pgfplots}
\usepackage{grffile}
\pgfplotsset{compat=newest}
\usetikzlibrary{plotmarks}
\usepackage{amsmath}

\begin{document}
\definecolor{mycolor1}{rgb}{0.00000,0.44700,0.74100}%
\definecolor{mycolor2}{rgb}{0.85000,0.32500,0.09800}%
%
\begin{tikzpicture}

\begin{axis}[%
width=2in,
height=2in,
scale only axis,
xmode=log,
xmin=0.01,
xmax=1,
xminorticks=true,
xlabel={$\|\mathbf{Ax} - \mathbf{b}\|$},
ymode=log,
ymin=0.01,
ymax=1000000000000,
yminorticks=true,
ylabel={$\|\mathbf{Lx}\|$}
]
\addplot [color=mycolor1,solid,forget plot]
  table[row sep=crcr]{%
0.0677209112985524	102674453519.548\\
0.0708211621906349	95560119862.2172\\
0.073998236298928	88746909934.8235\\
0.0772384016645485	82245349326.9593\\
0.0805278670700113	76063479833.7524\\
0.0838530051074213	70206000635.8289\\
0.0872009209211088	64673966608.5742\\
0.0905597165024013	59464978230.7549\\
0.0939184822636006	54573692697.2523\\
0.0972670663705365	49992446083.7069\\
0.100596002037309	45711816142.4135\\
0.103896433025172	41721045974.5615\\
0.107160226849168	38008340755.6555\\
0.110379755371384	34561102279.6935\\
0.1135479684492	31366167056.4379\\
0.11665838755775	28410078703.4839\\
0.119704813384484	25679382812.5716\\
0.122681173135013	23160905703.8753\\
0.125581352617351	20841975944.3129\\
0.128399083048332	18710563560.2835\\
0.131127873485326	16755334391.3767\\
0.13376112526393	14965635249.1534\\
0.136292228254776	13331434151.9525\\
0.138714779699656	11843239087.0118\\
0.141022803087094	10492012106.4963\\
0.143210985248772	9269087658.9314\\
0.145274925836848	8166098425.64045\\
0.147211363486804	7174909958.99871\\
0.149018376948866	6287566384.12429\\
0.150695546571406	5496251493.16168\\
0.152244034655025	4793270947.78911\\
0.153666577154421	4171060883.38272\\
0.15496736341347	3622225520.79303\\
0.156151818413933	3139601586.89261\\
0.15722629648338	2716341341.05425\\
0.158197726537082	2346000603.37669\\
0.159073275328322	2022615612.14112\\
0.159860054274092	1740754375.87697\\
0.160564927808634	1495534276.54089\\
0.161194409068703	1282605958.17264\\
0.161754652430275	1098110997.31757\\
0.162251487752149	938625082.011949\\
0.162690483901544	801098653.083512\\
0.163076988619057	682804068.550238\\
0.163416146670087	581294067.974825\\
0.163712883811099	494372257.654074\\
0.16397186599402	420073465.450678\\
0.164197449229787	356650382.10711\\
0.164393634453581	302562727.734536\\
0.164564035355229	256465898.61941\\
0.164711868282652	217197269.73086\\
0.16483996369539	183759670.305987\\
0.16495079451998	155302671.928747\\
0.165046517210537	131102994.737134\\
0.165129015750039	110545480.589174\\
0.16519994333476	93105805.7028362\\
0.16526075923936	78335613.431379\\
0.16531275760293	65850246.0711828\\
0.16535708926459	55318878.3593602\\
0.165394778284342	46456650.3491019\\
0.165426734153766	39018341.6806388\\
0.165453761716577	32793169.9666522\\
0.165476569924366	27600379.4759425\\
0.165495779978837	23285374.1400322\\
0.165511933270126	19716221.7382133\\
0.165525499701956	16780411.0836142\\
0.165536885301035	14381790.5824998\\
0.165546440315096	12437671.0489399\\
0.165554466807453	10876151.4185243\\
0.165561225938359	9633815.18292674\\
0.165566945134044	8654001.73581837\\
0.165571824953616	7885809.35016967\\
0.165576045765199	7283806.4060083\\
0.165579774181041	6808197.49301796\\
0.165583169308421	6425067.86724632\\
0.165586388890424	6106399.61235847\\
0.165589594462119	5829748.23587805\\
0.165592955835401	5577649.04332748\\
0.165596653490977	5336903.95165589\\
0.165600878152298	5097887.63843603\\
0.16560582676367	4853955.79042916\\
0.165611693463508	4600979.27820055\\
0.165618655938722	4336985.53947878\\
0.165626857539573	4061865.78448309\\
0.165636387460178	3777101.2065744\\
0.165647262423432	3485469.43850414\\
0.165659414137488	3190710.04761961\\
0.165672686010476	2897150.04367896\\
0.165686841095448	2609311.29468717\\
0.165701581431616	2331535.63524695\\
0.165716574752098	2067666.56807934\\
0.165731484712531	1820818.60168716\\
0.165745998904177	1593249.80879011\\
0.165759850748334	1386335.52212892\\
0.165772833058187	1200626.34762363\\
0.165784802899285	1035964.98906684\\
0.165795679204378	891634.288511179\\
0.165805435250529	766511.950680791\\
0.16581408830622	659213.33599163\\
0.165821688520086	568210.351346895\\
0.165828308392408	491920.513643413\\
0.165834033748084	428765.333111122\\
0.165838956548609	377201.58938336\\
0.165843169592546	335733.229933463\\
0.165846762906902	302914.994076315\\
0.165849821609023	277359.535754654\\
0.165852424922691	257755.667129204\\
0.165854646148846	242896.977130179\\
0.165856553419045	231712.166620001\\
0.165858211065141	223286.272703286\\
0.165859681643848	216866.292328239\\
0.165861028589514	211851.455004726\\
0.165862319617391	207773.049707042\\
0.165863631	204269.827920502\\
0.165865052888995	201063.725253228\\
0.165866695833445	197938.663554473\\
0.165868698595931	194723.552203375\\
0.165871237147311	191279.597303995\\
0.16587453446031	187491.550664379\\
0.165878870222042	183262.381007101\\
0.165884588894122	178510.850010479\\
0.165892103766993	173171.50647326\\
0.165901893857294	167196.605455392\\
0.165914490022947	160559.387296419\\
0.165930447008968	153258.01890317\\
0.165950299597087	145319.343789847\\
0.165974504135651	136801.474427776\\
0.166003370958239	127794.273195172\\
0.166036997819025	118416.978124751\\
0.166075217654397	108812.65967215\\
0.16611757402497	99139.788258185\\
0.166163333478291	89561.8110283718\\
0.166211536384095	80236.1000430784\\
0.166261078870552	71303.7926177478\\
0.166310811336969	62881.8450519947\\
0.166359636127048	55058.133650387\\
0.16640658887517	47889.8219593143\\
0.166450893615859	41404.6499886146\\
0.166491988596191	35604.4199332139\\
0.166529525669132	30469.8002664913\\
0.166563349849813	25965.6168517637\\
0.166593466866337	22045.9739802093\\
0.166620005882061	18658.7716873172\\
0.166643182869945	15749.3979693518\\
0.166663268124279	13263.5417496657\\
0.166680559625106	11149.1842176469\\
0.166695362655198	9357.8873948231\\
0.166707975227703	7845.5215013346\\
0.166718678454078	6572.57014508884\\
0.166727730837096	5504.13579629821\\
0.166735365509701	4609.74559968072\\
0.166741789568857	3863.0344768463\\
0.16674718481567	3241.36149435973\\
0.166751709374219	2725.39784973647\\
0.16675549980388	2298.71090435852\\
0.166758673437903	1947.35849556622\\
0.166761330770833	1659.50140345384\\
0.166763557786265	1425.03966750488\\
0.166765428163459	1235.28060174805\\
0.166767005333598	1082.65165044836\\
0.166768344373618	960.47570285935\\
0.166769493731526	862.823862037094\\
0.166770496772066	784.446836069024\\
0.166771393116133	720.765401672674\\
0.166772219725073	667.885280831953\\
0.166773011655123	622.602631960084\\
0.166773802387672	582.381411647736\\
0.166774623640433	545.301740176634\\
0.166775504598023	509.989186858522\\
0.166776470580437	475.536544849423\\
0.166777541291667	441.425891648742\\
0.166778728936286	407.454046195098\\
0.16678003661068	373.661558379538\\
0.166781457411313	340.264671351996\\
0.166782974611379	307.59062518444\\
0.166784563043008	276.018138094774\\
0.166786191540848	245.92593199701\\
0.166787826049862	217.652205350998\\
0.166789432861047	191.466965029963\\
0.166790981454458	167.557500996032\\
0.166792446575777	146.02559372499\\
0.166793809384744	126.893773026827\\
0.166795057716071	110.117343213003\\
0.16679618563468	95.5989408278925\\
0.166797192529576	83.2029233788092\\
0.166798081986202	72.7676952587576\\
0.166798860630816	64.1150147839117\\
0.166799537078442	57.0563121248436\\
0.16680012105692	51.3969685832379\\
0.166800622733551	46.9400962627294\\
0.166801052240525	43.4912372828552\\
0.166801419379128	40.8644405953268\\
0.166801733477252	38.8888607472433\\
0.16680200337663	37.4142536496477\\
0.166802237532334	36.3140224624647\\
0.166802444215758	35.4854445602313\\
0.166802631822027	34.847635409069\\
0.166802809292915	34.338219429425\\
0.166802986676201	33.9096098920696\\
0.16680317585112	33.5255060904227\\
0.166803391455508	33.157914527796\\
0.166803652050722	32.7847877203499\\
0.166803981551072	32.3882576220279\\
0.166804410918874	31.9533938525084\\
0.166804980076392	31.4674097990381\\
0.166805739903304	30.9192498986106\\
0.166806754067432	30.2995048856216\\
0.166808100280958	29.6006102316396\\
0.166809870405221	28.817281748346\\
0.166812168692445	27.9471293166954\\
0.166815107430329	26.9913656295034\\
0.16681879944248	25.9554959184161\\
0.16682334737571	24.8498452955962\\
0.166828830481822	23.6897646121142\\
0.166835290549189	22.4953666273664\\
0.166842719474186	21.2906911222339\\
0.166851051310032	20.1022809735724\\
0.16686016118769	18.9572603263765\\
0.166869872229681	17.8811202495416\\
0.166879969793561	16.8955111533619\\
0.166890220675316	16.0163889770018\\
0.166900393867829	15.2528402719221\\
0.166910279423685	14.6068044697956\\
0.166919702829154	14.0737275992684\\
0.166928533657151	13.6439675762159\\
0.166936688642246	13.3046058574613\\
0.166944130345546	13.0412718039251\\
0.166950863109304	12.8396646337056\\
0.166956928097161	12.6866101614953\\
0.166962399040971	12.5706404704427\\
0.166967380052613	12.4821837152225\\
0.166972006655004	12.4134883833024\\
0.166976451134326	12.3583987088061\\
0.166980933465721	12.3120693033195\\
0.166985739444947	12.27067522568\\
0.166991248290008	12.231147632414\\
0.166997972885657	12.1909472701068\\
0.167006617061732	12.1478774807572\\
0.167018155852139	12.099933108426\\
0.167033946578006	12.0451798192383\\
0.16705588077254	11.9816584558443\\
0.167086589218025	11.9073102224923\\
0.167129714259393	11.8199202062195\\
0.167190264268008	11.7170787026233\\
0.167275063254918	11.5961618823485\\
0.167393302017763	11.454335392935\\
0.167557182854022	11.2885863808042\\
0.167782624178378	11.0957908668833\\
0.168089951005267	10.8728239776978\\
0.168504441094981	10.6167196001264\\
0.169056529116081	10.3248828430689\\
0.169781407016443	9.99535254983526\\
0.170717726293435	9.62710168490271\\
0.171905148191802	9.22035123969537\\
0.17338064367948	8.77686021629205\\
0.175173735323806	8.30014365024545\\
0.177301260145106	7.7955670268812\\
0.179762601491458	7.27027309163559\\
0.182536519335302	6.73291788413554\\
0.185580554228595	6.19322444323618\\
0.188833463736126	5.66139773388413\\
0.192220417297416	5.14747255885754\\
0.195659992648123	4.66067809108034\\
0.199071636546969	4.20889408664159\\
0.202382284863083	3.79824847251521\\
0.205531212975835	3.43287422615416\\
0.208472722110047	3.11481809449696\\
0.211176765985768	2.84408393553183\\
0.213627959812496	2.61879911074063\\
0.215823556893134	2.43550164580339\\
0.217770962473862	2.28954158571641\\
0.219485245134518	2.17556432458011\\
0.220986965500274	2.08801017963928\\
0.222300514016999	2.02154964624442\\
0.223453056238962	1.97139241346515\\
0.224474131417996	1.9334494408714\\
0.225395934518199	1.9043668457955\\
0.226254325414868	1.88147143142857\\
0.227090643181832	1.86266940084735\\
0.227954448580629	1.84633030796814\\
0.228907362235549	1.83117608839048\\
0.230028191331405	1.81618491419924\\
0.23141951396697	1.80051301779992\\
0.233215769403671	1.78343410037783\\
0.235592615007794	1.76429451419443\\
0.238776775311595	1.74248220359661\\
0.243054770300566	1.71740778282747\\
0.248777826974138	1.68849672311623\\
0.256359252632505	1.65519216771579\\
0.266260218573018	1.61696821726458\\
0.27896110039414	1.57335348681913\\
0.294918707842347	1.52396421083445\\
0.314514230045011	1.46854509311257\\
0.338000535060489	1.40701450219497\\
0.365458487354662	1.33950871624131\\
0.396769719061464	1.26641817517295\\
0.431609404267157	1.18840775925994\\
0.469459136325933	1.10641369202443\\
0.509637861802388	1.02161223412223\\
0.551347483494916	0.935359798509158\\
};
\addplot [color=mycolor1,only marks,mark=asterisk,mark options={solid},forget plot]
  table[row sep=crcr]{%
0.228907362235549	1.83117608839048\\
};
\end{axis}
\end{tikzpicture}%
\end{document}
% This file was created by matlab2tikz.
% Minimal pgfplots version: 1.3
%
%The latest updates can be retrieved from
%  http://www.mathworks.com/matlabcentral/fileexchange/22022-matlab2tikz
%where you can also make suggestions and rate matlab2tikz.
%
\documentclass[tikz]{standalone}
\usepackage{pgfplots}
\usepackage{grffile}
\pgfplotsset{compat=newest}
\usetikzlibrary{plotmarks}
\usepackage{amsmath}

\begin{document}
\definecolor{mycolor1}{rgb}{0.00000,0.44700,0.74100}%
\definecolor{mycolor2}{rgb}{0.85000,0.32500,0.09800}%
%
\begin{tikzpicture}

\begin{axis}[%
width=2in,
height=2in,
at={(1.5in,0.48125in)},
scale only axis,
xmin=-30,
xmax=0,
xlabel={$\lambda$},
ymin=-2,
ymax=16,
ylabel={curvature $\kappa$}
]
\addplot [color=mycolor1,solid,forget plot]
  table[row sep=crcr]{%
-28.27153926164	-0.480928078288111\\
-28.1790700236978	-0.483907089613483\\
-28.0866007857556	-0.480772305570318\\
-27.9941315478134	-0.473605507532208\\
-27.9016623098712	-0.46298326979872\\
-27.809193071929	-0.449680587355254\\
-27.7167238339868	-0.434522316100088\\
-27.6242545960445	-0.418376921725417\\
-27.5317853581023	-0.401579375651604\\
-27.4393161201601	-0.384485734957791\\
-27.3468468822179	-0.367167945935578\\
-27.2543776442757	-0.350282138163843\\
-27.1619084063335	-0.333831827188816\\
-27.0694391683913	-0.317817487469803\\
-26.9769699304491	-0.302500350729436\\
-26.8845006925069	-0.287784341737248\\
-26.7920314545647	-0.273568133051587\\
-26.6995622166224	-0.259726461397606\\
-26.6070929786802	-0.246174725792284\\
-26.514623740738	-0.232735754112278\\
-26.4221545027958	-0.219375057826842\\
-26.3296852648536	-0.206018747538434\\
-26.2372160269114	-0.192667033747006\\
-26.1447467889692	-0.179350663027839\\
-26.052277551027	-0.166122253855825\\
-25.9598083130848	-0.153071577027812\\
-25.8673390751426	-0.140305899587203\\
-25.7748698372003	-0.127938380360063\\
-25.6824005992581	-0.11609153653661\\
-25.5899313613159	-0.104875689532403\\
-25.4974621233737	-0.0943851157789202\\
-25.4049928854315	-0.0846808001820199\\
-25.3125236474893	-0.0757923496510977\\
-25.2200544095471	-0.0677159730033908\\
-25.1275851716049	-0.0604099061831369\\
-25.0351159336627	-0.053816539500559\\
-24.9426466957205	-0.047860877425631\\
-24.8501774577782	-0.0424747833549737\\
-24.757708219836	-0.0375931459813676\\
-24.6652389818938	-0.0331700566990484\\
-24.5727697439516	-0.0291670960973022\\
-24.4803005060094	-0.0255611639489127\\
-24.3878312680672	-0.0223311001566058\\
-24.295362030125	-0.0194578741523348\\
-24.2028927921828	-0.0169207208278361\\
-24.1104235542406	-0.0146948353555048\\
-24.0179543162984	-0.0127513803309909\\
-23.9254850783561	-0.0110599997161391\\
-23.8330158404139	-0.0095893452441766\\
-23.7405466024717	-0.00830944259488782\\
-23.6480773645295	-0.00719350224171422\\
-23.5556081265873	-0.00621785528823999\\
-23.4631388886451	-0.00536321819828484\\
-23.3706696507029	-0.00461405760743194\\
-23.2782004127607	-0.00395743576251853\\
-23.1857311748185	-0.00338293531822431\\
-23.0932619368762	-0.00288166235318492\\
-23.000792698934	-0.00244563667534277\\
-22.9083234609918	-0.00206771481280004\\
-22.8158542230496	-0.00174123209456976\\
-22.7233849851074	-0.00145993294124937\\
-22.6309157471652	-0.00121800306677743\\
-22.538446509223	-0.00101003699080817\\
-22.4459772712808	-0.000830819180372288\\
-22.3535080333386	-0.000675645319254764\\
-22.2610387953964	-0.000539658807793088\\
-22.1685695574541	-0.000417899280098576\\
-22.0761003195119	-0.000304752315005042\\
-21.9836310815697	-0.000193143080727899\\
-21.8911618436275	-7.37118785566473e-05\\
-21.7986926056853	6.62263289412935e-05\\
-21.7062233677431	0.000243122320200229\\
-21.6137541298009	0.000475826850809898\\
-21.5212848918587	0.000780118592870182\\
-21.4288156539165	0.0011537628755551\\
-21.3363464159743	0.00156174922046007\\
-21.2438771780321	0.00193069106004154\\
-21.1514079400898	0.00217547632403348\\
-21.0589387021476	0.00224250845332103\\
-20.9664694642054	0.00212719178491318\\
-20.8740002262632	0.00186604958607178\\
-20.781530988321	0.00151053024984461\\
-20.6890617503788	0.00111040955574115\\
-20.5965925124366	0.000706244932114651\\
-20.5041232744944	0.000328189266376556\\
-20.4116540365522	-3.69435743341553e-06\\
-20.3191847986099	-0.000277958935239528\\
-20.2267155606677	-0.000489591293010184\\
-20.1342463227255	-0.000639885001677567\\
-20.0417770847833	-0.000733820743660762\\
-19.9493078468411	-0.000779265878888181\\
-19.8568386088989	-0.000785363274852055\\
-19.7643693709567	-0.000761388330135527\\
-19.6719001330145	-0.000715993204968539\\
-19.5794308950723	-0.000656600345284624\\
-19.4869616571301	-0.000589192883035037\\
-19.3944924191878	-0.000518275929579216\\
-19.3020231812456	-0.000446920080764559\\
-19.2095539433034	-0.000376934490815388\\
-19.1170847053612	-0.000308933553697744\\
-19.024615467419	-0.000242346522182853\\
-18.9321462294768	-0.000175209077765325\\
-18.8396769915346	-0.000103674266467264\\
-18.7472077535924	-2.10103979410284e-05\\
-18.6547385156502	8.40657453952184e-05\\
-18.562269277708	0.000230223235811012\\
-18.4698000397657	0.00044803400552833\\
-18.3773308018235	0.000786542079370946\\
-18.2848615638813	0.00132207909496028\\
-18.1923923259391	0.00216499400499033\\
-18.0999230879969	0.00345186193572004\\
-18.0074538500547	0.00529548916717856\\
-17.9149846121125	0.00767414991220263\\
-17.8225153741703	0.0103219199184459\\
-17.7300461362281	0.0127748545059351\\
-17.6375768982859	0.0145999041612075\\
-17.5451076603436	0.0155946662027515\\
-17.4526384224014	0.0157904073239939\\
-17.3601691844592	0.0153360390028396\\
-17.267699946517	0.0143979891206215\\
-17.1752307085748	0.0131163676315221\\
-17.0827614706326	0.0115994495802853\\
-16.9902922326904	0.00993278461017954\\
-16.8978229947482	0.00818863758664494\\
-16.805353756806	0.00643276684591271\\
-16.7128845188638	0.00472663891078151\\
-16.6204152809215	0.00312674198869545\\
-16.5279460429793	0.00168182438499421\\
-16.4354768050371	0.000429561550624308\\
-16.3430075670949	-0.000606175913522232\\
-16.2505383291527	-0.00141660786009023\\
-16.1580690912105	-0.0020074435015021\\
-16.0655998532683	-0.00239658542881655\\
-15.9731306153261	-0.00261063916441489\\
-15.8806613773839	-0.00268101295776264\\
-15.7881921394417	-0.00264026488911659\\
-15.6957229014994	-0.00251920245505476\\
-15.6032536635572	-0.0023449400548604\\
-15.510784425615	-0.00213989767397776\\
-15.4183151876728	-0.00192157104718785\\
-15.3258459497306	-0.00170284173368791\\
-15.2333767117884	-0.00149259479935064\\
-15.1409074738462	-0.00129646600122678\\
-15.048438235904	-0.00111758805606488\\
-14.9559689979618	-0.000957259717153324\\
-14.8634997600196	-0.000815506511966857\\
-14.7710305220773	-0.000691522492359028\\
-14.6785612841351	-0.00058400370489724\\
-14.5860920461929	-0.000491388698294027\\
-14.4936228082507	-0.000412024450175424\\
-14.4011535703085	-0.000344275041430615\\
-14.3086843323663	-0.000286587760689535\\
-14.2162150944241	-0.000237528169761131\\
-14.1237458564819	-0.000195791960366639\\
-14.0312766185397	-0.000160199406599886\\
-13.9388073805975	-0.000129673742185039\\
-13.8463381426552	-0.000103204341482541\\
-13.753868904713	-7.97943929232056e-05\\
-13.6613996667708	-5.83987617776094e-05\\
-13.5689304288286	-3.78702975660292e-05\\
-13.4764611908864	-1.69602144812467e-05\\
-13.3839919529442	5.53741004803971e-06\\
-13.291522715002	3.0366001662176e-05\\
-13.1990534770598	5.70410490314935e-05\\
-13.1065842391176	8.30273330715771e-05\\
-13.0141150011754	0.000103734693880404\\
-12.9216457632331	0.000114124132825957\\
-12.8291765252909	0.000111217318129991\\
-12.7367072873487	9.55185707533106e-05\\
-12.6442380494065	7.03965080043746e-05\\
-12.5517688114643	4.03796308683e-05\\
-12.4592995735221	9.69335233177688e-06\\
-12.3668303355799	-1.84666715536259e-05\\
-12.2743610976377	-4.20830237404705e-05\\
-12.1818918596955	-6.01780187371245e-05\\
-12.0894226217533	-7.25773303791306e-05\\
-11.996953383811	-7.96644740584203e-05\\
-11.9044841458688	-8.21600493642862e-05\\
-11.8120149079266	-8.09398367861109e-05\\
-11.7195456699844	-7.68965187416004e-05\\
-11.6270764320422	-7.08443814424562e-05\\
-11.5346071941	-6.34609920767341e-05\\
-11.4421379561578	-5.52562759330898e-05\\
-11.3496687182156	-4.65565305505571e-05\\
-11.2571994802734	-3.74891300570723e-05\\
-11.1647302423311	-2.79512300767629e-05\\
-11.0722610043889	-1.75418920642154e-05\\
-10.9797917664467	-5.42982179496669e-06\\
-10.8873225285045	9.88430602202681e-06\\
-10.7948532905623	3.09761250215343e-05\\
-10.7023840526201	6.21814552252613e-05\\
-10.6099148146779	0.000110714269705694\\
-10.5174455767357	0.000188415386148496\\
-10.4249763387935	0.000314235373596163\\
-10.3325071008513	0.000516951321753036\\
-10.240037862909	0.000835619945479012\\
-10.1475686249668	0.00131116373120895\\
-10.0550993870246	0.00196047058244699\\
-9.96263014908242	0.002739276932051\\
-9.87016091114021	0.0035329211552614\\
-9.777691673198	0.0042069061242873\\
-9.68522243525579	0.00467598320593802\\
-9.59275319731358	0.00492610578122413\\
-9.50028395937137	0.00498902446959805\\
-9.40781472142916	0.00490966035759132\\
-9.31534548348695	0.00472811535587912\\
-9.22287624554474	0.0044750509055606\\
-9.13040700760252	0.00417327389059004\\
-9.03793776966032	0.00384079527252882\\
-8.94546853171811	0.00349344314219404\\
-8.8529992937759	0.00314656658308354\\
-8.76053005583369	0.00281590458304099\\
-8.66806081789148	0.00251786259513442\\
-8.57559157994926	0.00226950849676468\\
-8.48312234200706	0.00208866088395246\\
-8.39065310406484	0.00199451505650902\\
-8.29818386612264	0.00200934048503238\\
-8.20571462818043	0.00216191856094713\\
-8.11324539023821	0.00249364003508923\\
-8.020776152296	0.00306866938540976\\
-7.9283069143538	0.00399049849754957\\
-7.83583767641158	0.00542885172933661\\
-7.74336843846938	0.00766373328451395\\
-7.65089920052717	0.0111581396990123\\
-7.55842996258495	0.016678629875597\\
-7.46596072464274	0.025494820149601\\
-7.37349148670054	0.0397055301478402\\
-7.28102224875832	0.0627565063991701\\
-7.18855301081612	0.100208484123583\\
-7.0960837728739	0.160697190072193\\
-7.00361453493169	0.256561206463226\\
-6.91114529698948	0.402452843073063\\
-6.81867605904727	0.608938233333657\\
-6.72620682110506	0.87048120967725\\
-6.63373758316286	1.15719150710849\\
-6.54126834522064	1.42473026977888\\
-6.44879910727843	1.63770328912313\\
-6.35632986933622	1.78350198768569\\
-6.26386063139401	1.86802714888431\\
-6.17139139345181	1.90451539989715\\
-6.07892215550959	1.90569857916105\\
-5.98645291756738	1.8808962160215\\
-5.89398367962517	1.83591747902714\\
-5.80151444168296	1.7739065283167\\
-5.70904520374075	1.6962685423915\\
-5.61657596579855	1.60343206276427\\
-5.52410672785633	1.49544751718018\\
-5.43163748991412	1.37248008647489\\
-5.33916825197191	1.23524164066389\\
-5.2466990140297	1.08536823286747\\
-5.15422977608749	0.92570505110242\\
-5.06176053814529	0.760424062859191\\
-4.96929130020307	0.594890076136758\\
-4.87682206226087	0.435227718657606\\
-4.78435282431865	0.287627282754666\\
-4.69188358637644	0.157531491891542\\
-4.59941434843423	0.0489120848487915\\
-4.50694511049202	-0.036171211494471\\
-4.41447587254981	-0.0976332906861989\\
-4.3220066346076	-0.137075701599947\\
-4.22953739666539	-0.157265576624233\\
-4.13706815872318	-0.161525709744328\\
-4.04459892078097	-0.153153761209766\\
-3.95212968283876	-0.134934587899607\\
-3.85966044489655	-0.108748830280877\\
-3.76719120695434	-0.0752243339051108\\
-3.67472196901213	-0.0333280957629275\\
-3.58225273106992	0.020249215714742\\
-3.48978349312771	0.0921195906740127\\
-3.3973142551855	0.193921861206192\\
-3.30484501724329	0.34515239633017\\
-3.21237577930108	0.577640684110866\\
-3.11990654135887	0.942197848958447\\
-3.02743730341666	1.51733147941008\\
-2.93496806547445	2.41728629970313\\
-2.84249882753224	3.7888999285558\\
-2.75002958959003	5.77037213893567\\
-2.65756035164782	8.37128893286657\\
-2.56509111370561	11.2826904663556\\
-2.4726218757634	13.7978670945054\\
-2.38015263782119	15.098673310469\\
-2.28768339987898	14.7720070962817\\
-2.19521416193677	13.0367016845144\\
-2.10274492399456	10.5087926009284\\
-2.01027568605235	7.8271895021591\\
-1.91780644811014	5.42874006102203\\
-1.82533721016793	3.50934845774791\\
-1.73286797222572	2.0870265107347\\
-1.64039873428351	1.08628788718925\\
-1.5479294963413	0.402712441634223\\
-1.45546025839909	-0.0611702628225401\\
-1.36299102045688	-0.381945094256227\\
-1.27052178251467	-0.613556873168935\\
-1.17805254457246	-0.790462032057077\\
-1.08558330663025	-0.931776396030481\\
-0.993114068688038	-1.04501664822517\\
-0.900644830745826	-1.12948876238639\\
-0.808175592803618	-1.17992088953912\\
};
\addplot [color=mycolor1,only marks,mark=asterisk,mark options={solid},forget plot]
  table[row sep=crcr]{%
-2.38015263782119	15.098673310469\\
};
\end{axis}
\end{tikzpicture}%
\end{document}
\caption{Tikhonov regularization L-curve and curvature for the matrix $\mathbf{A}_{2}$ and vector $\mathbf{b}_{err2}$.}
\label{fig:A2Tikh}
\end{figure}
\begin{figure}
% This file was created by matlab2tikz.
% Minimal pgfplots version: 1.3
%
%The latest updates can be retrieved from
%  http://www.mathworks.com/matlabcentral/fileexchange/22022-matlab2tikz
%where you can also make suggestions and rate matlab2tikz.
%
\documentclass[tikz]{standalone}
\usepackage{pgfplots}
\usepackage{grffile}
\pgfplotsset{compat=newest}
\usetikzlibrary{plotmarks}
\usepackage{amsmath}

\begin{document}
\definecolor{mycolor1}{rgb}{0.00000,0.44700,0.74100}%
\definecolor{mycolor2}{rgb}{0.85000,0.32500,0.09800}%
%
\begin{tikzpicture}

\begin{axis}[%
width=2in,
height=2in,
scale only axis,
xmode=log,
xmin=0.0172459449054614,
xmax=0.0184781931290211,
xminorticks=true,
xlabel={$\|\mathbf{Ax} - \mathbf{b}\|$},
ymode=log,
ymin=0.0165667803766244,
ymax=223954091.612629,
yminorticks=true,
ylabel={$\|\mathbf{Lx}\|$}
]
\addplot [color=mycolor1,solid,forget plot]
  table[row sep=crcr]{%
0.0177125384232676	196951682897319\\
0.0177880643315343	172893708155922\\
0.018000906539671	149539377096934\\
0.0173808456265003	127519406974167\\
0.0175855439373241	107320655007013\\
0.0174385369559518	89254284087660.2\\
0.0174488267697341	73456104482266.5\\
0.0175207242324966	59911585589857.9\\
0.0175034641722402	48494084096570.9\\
0.0174416593919824	39005561380951.2\\
0.0174376228603456	31212431204491.1\\
0.0175363204570928	24873012189443.8\\
0.0174522340939544	19756037596593.1\\
0.0174528032912632	15651417432902.6\\
0.0174554994427612	12375144407740.6\\
0.0174426814685518	9770257195723.82\\
0.017441066304396	7705454749367.32\\
0.0174401311351775	6072522343482.35\\
0.0174414252199266	4783317980861.23\\
0.0174413912685086	3766744033779.3\\
0.0174419501804182	2965913946527.5\\
0.0174411984790429	2335604714100.77\\
0.01744009551029	1840030636265.91\\
0.0174401648324428	1450948035349.31\\
0.0174403795340453	1146081708607.04\\
0.0174402713589406	907845571050.445\\
0.0174404771895851	722315987216.996\\
0.0174404191480583	578411705493.736\\
0.0174407152741679	467240713285.522\\
0.017440748761486	381589214024.821\\
0.0174412943656234	315544123233.223\\
0.0174414421713898	264246171160.669\\
0.0174420255044457	223756460570.554\\
0.0174426129599272	190992316911.597\\
0.0174433176439871	163673466250.411\\
0.0174441969155043	140234560461.051\\
0.0174454016957506	119693969488.397\\
0.0174463912620663	101496894504.832\\
0.0174477940919009	85360726247.6364\\
0.0174489884169807	71145709983.1087\\
0.0174501525054337	58762856683.0905\\
0.0174513066096883	48119936949.8025\\
0.0174523377254354	39098616180.7152\\
0.0174532078414604	31552401182.2321\\
0.0174539820243722	25315425032.5157\\
0.0174546549950066	20214600951.4567\\
0.0174552187511788	16080705405.8705\\
0.0174556454511848	12756463952.2906\\
0.0174560258643356	10101340478.0309\\
0.0174563246710077	7993548784.20627\\
0.017456558632257	6330067612.85746\\
0.0174567572994089	5025396855.80495\\
0.017456892705382	4009621988.4588\\
0.0174570152965497	3226159232.77954\\
0.0174571075422012	2629391468.45893\\
0.0174571829379883	2182315080.50271\\
0.0174572410963941	1854337966.59056\\
0.0174572843883883	1619484196.52549\\
0.0174573220297865	1455314719.60973\\
0.0174573515809013	1342646431.44496\\
0.0174573760474628	1265722224.94278\\
0.0174573962836572	1212293866.6096\\
0.0174574143493396	1173327029.27891\\
0.0174574311754361	1142399423.27722\\
0.0174574543180396	1115012970.73033\\
0.0174574786388042	1087990482.31417\\
0.0174575154360988	1059026858.15083\\
0.0174575632240251	1026398006.91156\\
0.0174576363050821	988803888.190207\\
0.0174577374850595	945316521.128421\\
0.0174578778283912	895403515.136123\\
0.0174580709798813	838995006.359143\\
0.0174583257130654	776556121.798722\\
0.0174586495040954	709122741.298516\\
0.017459044770311	638262655.252245\\
0.0174595035276771	565942703.609212\\
0.0174600161999088	494313223.130118\\
0.0174605630073729	425452686.81368\\
0.0174611207314459	361132209.943101\\
0.0174616680846195	302652492.214212\\
0.0174621854145397	250779078.446304\\
0.0174626596435793	205770063.581626\\
0.0174630807521008	167467925.091119\\
0.0174634470292359	135420132.240577\\
0.0174637586547328	108998775.70977\\
0.0174640191357905	87500804.2081504\\
0.0174642342261616	70221491.1525681\\
0.0174644098348403	56501398.0916279\\
0.0174645517644089	45750923.4749164\\
0.0174646658778273	37457562.2443541\\
0.0174647569379595	31180811.1294361\\
0.0174648293922133	26539751.9521567\\
0.0174648869054174	23199368.3999841\\
0.0174649325339704	20862063.9350279\\
0.0174649687432505	19267644.6630972\\
0.0174649978250197	18198620.5000952\\
0.0174650215171354	17483942.3215461\\
0.0174650415062699	16996880.2809063\\
0.0174650593822802	16647761.1886584\\
0.0174650770327976	16374724.2222578\\
0.0174650965721182	16135123.7219437\\
0.0174651208958233	15898755.7291139\\
0.0174651541403352	15643043.2126937\\
0.0174652020778525	15349879.2971233\\
0.0174652730645552	15003763.4193865\\
0.0174653786365885	14590945.9716336\\
0.017465534571596	14099395.4159396\\
0.0174657610567612	13519458.6140134\\
0.0174660826010047	12845073.390898\\
0.0174665261856377	12075306.2030409\\
0.017467118045549	11215849.8137781\\
0.0174678784522931	10279989.7745326\\
0.0174688159845901	9288533.12349945\\
0.0174699222949913	8268376.72562806\\
0.0174711703997731	7249778.65287656\\
0.0174725164381622	6262851.29428621\\
0.0174739061493581	5334098.98332268\\
0.0174752831610492	4483807.92329714\\
0.0174765971386334	3724771.54069258\\
0.0174778094329013	3062379.88712376\\
0.0174788954352	2495735.49259956\\
0.0174798440422747	2019301.64748174\\
0.0174806551755779	1624630.58832346\\
0.0174813366002627	1301868.35485758\\
0.0174819008032609	1040896.8817631\\
0.0174823624768238	832097.189972862\\
0.0174827366826286	666785.925953861\\
0.0174830377157505	537400.746090273\\
0.0174832784546655	437506.08524476\\
0.0174834701199082	361677.532032119\\
0.0174836222463699	305315.727415209\\
0.0174837428140057	264447.609195683\\
0.0174838384455053	235582.192751912\\
0.0174839146472935	215666.343825061\\
0.0174839760638059	202121.382920622\\
0.01748402675472	192886.717917187\\
0.0174840704944363	186408.910846224\\
0.0174841111299826	181569.949580387\\
0.017484153012317	177585.792419812\\
0.0174842015485946	173907.399758095\\
0.0174842638996756	170141.348079099\\
0.0174843498302214	165994.037936792\\
0.0174844726582775	161237.193280692\\
0.0174846501473724	155690.70403293\\
0.0174849050194142	149219.049587657\\
0.0174852645989592	141737.796782709\\
0.0174857589853826	133226.252970261\\
0.017486417299228	123741.292692987\\
0.0174872620627602	113426.325365859\\
0.017488302684526	102509.439367117\\
0.0174895299510498	91286.9483162739\\
0.0174909137526013	80092.9229293992\\
0.0174924055167498	69260.3637973448\\
0.017493945102849	59083.0369759641\\
0.0174954701376969	49786.8217890843\\
0.0174969249638256	41515.8308390014\\
0.0174982668902962	34333.5450445496\\
0.0174994687989133	28235.1583496013\\
0.0175005185035962	23165.5705386833\\
0.0175014160268693	19037.8633252215\\
0.0175021700536011	15748.6792813259\\
0.0175027945072425	13188.7751557872\\
0.0175033057645176	11248.7378487491\\
0.0175037206707601	9821.39399158572\\
0.0175040553036072	8803.47487277726\\
0.0175043243411054	8098.67096936251\\
0.0175045408784954	7622.05480742179\\
0.017504716571475	7303.81595630115\\
0.0175048620314191	7090.42166559258\\
0.0175049874530177	6943.0894866812\\
0.0175051035117791	6834.75847103833\\
0.0175052226287461	6746.81914244203\\
0.0175053607632936	6666.32078421017\\
0.0175055399616805	6583.87059037776\\
0.0175057919389275	6492.1746368803\\
0.0175061629759712	6385.09057128501\\
0.0175067202947472	6257.07179287804\\
0.0175075597272316	6102.92112185199\\
0.0175088137761057	5917.80859737582\\
0.0175106579774437	5697.52927761913\\
0.0175133119251916	5438.97485131691\\
0.0175170299504061	5140.76305848756\\
0.0175220764305176	4803.91413931089\\
0.0175286835623424	4432.40070577087\\
0.017536995983012	4033.36081765917\\
0.0175470153372757	3616.79522133922\\
0.0175585641507971	3194.69066476483\\
0.0175712865603948	2779.69466471496\\
0.0175846916641805	2383.63413520616\\
0.017598228558432	2016.23272468816\\
0.0176113699869243	1684.30406451525\\
0.0176236806511069	1391.52477090038\\
0.0176348556333632	1138.71250386821\\
0.0176447272615936	924.425089451241\\
0.0176532482930072	745.677479772513\\
0.0176604627694761	598.618590141805\\
0.0176664744701574	479.078687840319\\
0.0176714192734203	382.958652900734\\
0.0176754441618642	306.471586293495\\
0.0176786931065824	246.265062310179\\
0.0176812987663449	199.455258370688\\
0.0176833785206147	163.599919045763\\
0.017685033439769	136.632567639745\\
0.0176863490913953	116.780583685109\\
0.017687397417957	102.493946644733\\
0.0176882392198117	92.4090024691694\\
0.0176889270140509	85.3515131587433\\
0.0176895082130245	80.3561119286346\\
0.0176900286784944	76.6721561508816\\
0.0176905367489699	73.7439259186449\\
0.0176910877789036	71.1733877670254\\
0.0176917490161071	68.6799546219913\\
0.0176926042164766	66.0671899414995\\
0.0176937567317913	63.2000704573643\\
0.0176953290236511	59.9923036790408\\
0.0176974560306899	56.4012497795508\\
0.0177002702182937	52.4271662625189\\
0.017703878183513	48.1131787166112\\
0.0177083324625563	43.5426011299871\\
0.0177136063946339	38.8312689177534\\
0.0177195818791557	34.1144627130131\\
0.0177260572031007	29.5302742418901\\
0.0177327749974523	25.2029234428662\\
0.017739462365752	21.2297477631399\\
0.0177458709377679	17.6742835115068\\
0.0177518061602353	14.5658056911294\\
0.0177571408130224	11.9039175361614\\
0.0177618138412662	9.66592901307352\\
0.0177658193293202	7.81485536023919\\
0.0177691912061301	6.30651354246193\\
0.0177719880464304	5.09495221011338\\
0.0177742804566469	4.13604034772168\\
0.0177761419174355	3.38938368244898\\
0.0177776429321536	2.81889888328599\\
0.0177788478624312	2.3924753701582\\
0.0177798137375471	2.08127282058663\\
0.0177805904253549	1.85925285430067\\
0.0177812217253895	1.70330152330672\\
0.0177817471199503	1.59377007775045\\
0.0177822040615851	1.51489933441311\\
0.0177826307643955	1.4547480798073\\
0.0177830694779181	1.40465799052356\\
0.0177835701196591	1.35852364908489\\
0.0177841938806658	1.31211401013786\\
0.0177850159718264	1.26256844510079\\
0.0177861261036725	1.20808847954739\\
0.0177876248200063	1.14779113241975\\
0.017789613893873	1.08166486746065\\
0.0177921802044272	1.01055859828536\\
0.0177953750794489	0.936133890542888\\
0.0177991943157452	0.86072430025775\\
0.0178035661680978	0.787076616499844\\
0.0178083535125993	0.717991493406012\\
0.0178133716948147	0.65592232318219\\
0.0178184174111577	0.602617482941666\\
0.0178232998683288	0.558894289631408\\
0.0178278656429636	0.524607973819114\\
0.0178320124609015	0.498822520165432\\
0.0178356919192757	0.480117426109442\\
0.0178389045994692	0.466917777270892\\
0.017841692242862	0.457748675098566\\
0.0178441312478222	0.451373215783508\\
0.0178463307403727	0.446828830779579\\
0.0178484376897754	0.443401258452363\\
0.017850651352076	0.440572717972909\\
0.017853249726243	0.437967559453745\\
0.0178566314384772	0.435306367293036\\
0.017861377018682	0.432371885849802\\
0.0178683330112594	0.428986634345293\\
0.0178787193254198	0.425001034581083\\
0.0178942526404195	0.420290900151846\\
0.017917264438224	0.41476329590107\\
0.0179507709140948	0.408369459946563\\
0.0179984282603343	0.40112231302219\\
0.018064294417556	0.393114028200396\\
0.0181523407544066	0.384526785584127\\
0.0182657368541555	0.375628623825728\\
0.0184060656783843	0.366748194878691\\
0.0185727618502379	0.358228462925714\\
0.0187631104322326	0.350368581871348\\
0.0189730326027579	0.343370645115896\\
0.0191986585726668	0.337308197857119\\
0.0194384905536963	0.332124775559277\\
0.0196959286038841	0.32765813013912\\
0.0199820880365048	0.323676954824559\\
0.0203190599360095	0.319916057604142\\
0.0207438697839178	0.31610128781692\\
0.0213131813949385	0.311962336776075\\
0.022108108579801	0.307236288948167\\
0.0232373146681715	0.301666699319262\\
0.0248353649758358	0.295002924602573\\
0.0270532937583908	0.287003575361274\\
0.0300408236316281	0.277446880145129\\
0.0339237805567347	0.266149460070009\\
0.0387825063232961	0.252993170825715\\
0.0446354294422953	0.237956999281624\\
0.0514290100753482	0.221147710696976\\
0.059034515346827	0.202820075101273\\
0.0672532081849943	0.183376868874386\\
0.0758317847655904	0.163342163986433\\
};
\addplot [color=mycolor1,only marks,mark=asterisk,mark options={solid},forget plot]
  table[row sep=crcr]{%
0.0196959286038841	0.32765813013912\\
};
\end{axis}
\end{tikzpicture}%
\end{document}
% This file was created by matlab2tikz.
% Minimal pgfplots version: 1.3
%
%The latest updates can be retrieved from
%  http://www.mathworks.com/matlabcentral/fileexchange/22022-matlab2tikz
%where you can also make suggestions and rate matlab2tikz.
%
\documentclass[tikz]{standalone}
\usepackage{pgfplots}
\usepackage{grffile}
\pgfplotsset{compat=newest}
\usetikzlibrary{plotmarks}
\usepackage{amsmath}

\begin{document}
\definecolor{mycolor1}{rgb}{0.00000,0.44700,0.74100}%
\definecolor{mycolor2}{rgb}{0.85000,0.32500,0.09800}%
%
\begin{tikzpicture}

\begin{axis}[%
width=2in,
height=2in,
scale only axis,
xmin=-40,
xmax=0,
xlabel={$\lambda$},
ymin=-4,
ymax=10,
ylabel={curvature $\kappa$}
]
\addplot [color=mycolor1,solid,forget plot]
  table[row sep=crcr]{%
-37.6591698078069	0.744130891669897\\
-37.5359125190664	-3.99881436563151\\
-37.412655230326	3.22321885084528\\
-37.2893979415856	-1.23465088104324\\
-37.1661406528452	0.490799285827435\\
-37.0428833641047	0.162829999172825\\
-36.9196260753643	-0.223359116099126\\
-36.7963687866238	-0.0997574419212052\\
-36.6731114978834	0.126354902820779\\
-36.549854209143	0.209262985572445\\
-36.4265969204025	-0.360950103145844\\
-36.3033396316621	0.162966239944378\\
-36.1800823429217	0.00396307283877022\\
-36.0568250541812	-0.0285263076764704\\
-35.9335677654408	0.0204248596525376\\
-35.8103104767004	0.00123185427460661\\
-35.6870531879599	0.00398794595955894\\
-35.5637958992195	-0.00236680022324491\\
-35.4405386104791	0.0010528341016628\\
-35.3172813217386	-0.00232494255784362\\
-35.1940240329982	-0.000626315093211638\\
-35.0707667442578	0.00208186711215008\\
-34.9475094555173	0.000262035180647328\\
-34.8242521667769	-0.000583653274133958\\
-34.7009948780365	0.000582388628810425\\
-34.577737589296	-0.00050060976399242\\
-34.4544803005556	0.000722676609861356\\
-34.3312230118152	-0.000550662809562708\\
-34.2079657230747	0.0012711304110468\\
-34.0847084343343	-0.00101478194836508\\
-33.9614511455939	0.00143235518641595\\
-33.8381938568534	0.000117822787688632\\
-33.714936568113	0.00053344704352965\\
-33.5916792793726	0.000738720468609128\\
-33.4684219906321	0.00128861801090992\\
-33.3451647018917	-0.00107060452226606\\
-33.2219074131513	0.00135562336205319\\
-33.0986501244108	-0.000952018685088213\\
-32.9753928356704	-0.000274071195358937\\
-32.85213554693	-0.000172844013509037\\
-32.7288782581895	-0.000426540615102346\\
-32.6056209694491	-0.00045825471955581\\
-32.4823636807087	-0.000263265473972747\\
-32.3591063919682	-0.000246483764531634\\
-32.2358491032278	-0.000241151142655417\\
-32.1125918144873	-0.000278592589470526\\
-31.9893345257469	-9.36600708174829e-05\\
-31.8660772370065	-0.000153794650836879\\
-31.742819948266	-0.000118246980411899\\
-31.6195626595256	-6.09722850399513e-05\\
-31.4963053707852	-0.000112097785839197\\
-31.3730480820447	-1.5447805755561e-05\\
-31.2497907933043	-4.94826367608025e-05\\
-31.1265335045639	-2.09503578080788e-05\\
-31.0032762158234	-2.25334324325547e-05\\
-30.880018927083	-1.9352574008492e-05\\
-30.7567616383426	1.91719249090033e-05\\
-30.6335043496021	1.04051065753635e-05\\
-30.5102470608617	4.41337599621071e-05\\
-30.3869897721213	6.84089117458172e-05\\
-30.2637324833808	0.000148985092694742\\
-30.1404751946404	0.000195005103974442\\
-30.0172179059	0.00111537057767516\\
-29.8939606171595	0.000158545632208845\\
-29.7707033284191	0.00169096240407517\\
-29.6474460396787	0.000710698088971147\\
-29.5241887509382	0.00166495402256828\\
-29.4009314621978	0.000952387251334399\\
-29.2776741734574	0.000916729612091107\\
-29.1544168847169	0.000853618825953398\\
-29.0311595959765	0.000601999414393378\\
-28.9079023072361	0.000416061717964674\\
-28.7846450184956	0.000245483891617058\\
-28.6613877297552	6.61913192990713e-05\\
-28.5381304410148	-2.50333260306359e-05\\
-28.4148731522743	-0.000119127353247324\\
-28.2916158635339	-0.000178835981329334\\
-28.1683585747935	-0.000202656007565629\\
-28.045101286053	-0.000211600476712923\\
-27.9218439973126	-0.000200737141559544\\
-27.7985867085722	-0.0001880631524998\\
-27.6753294198317	-0.000162318192268054\\
-27.5520721310913	-0.000140590222684439\\
-27.4288148423508	-0.000118082071731319\\
-27.3055575536104	-9.59806532317001e-05\\
-27.18230026487	-7.72674456883877e-05\\
-27.0590429761295	-6.11882908287282e-05\\
-26.9357856873891	-4.63332003502758e-05\\
-26.8125283986487	-3.43788943735899e-05\\
-26.6892711099082	-2.27555140130581e-05\\
-26.5660138211678	-1.1553837360083e-05\\
-26.4427565324274	1.26398720563005e-06\\
-26.3194992436869	1.83824151138948e-05\\
-26.1962419549465	4.94068414144009e-05\\
-26.0729846662061	0.000101937319160547\\
-25.9497273774656	0.000206944910158518\\
-25.8264700887252	0.000404873577087189\\
-25.7032127999848	0.000800730866166575\\
-25.5799555112443	0.00140899379824835\\
-25.4566982225039	0.00221848833354355\\
-25.3334409337635	0.00303044197893515\\
-25.210183645023	0.0035226940652182\\
-25.0869263562826	0.00373284642487992\\
-24.9636690675422	0.00364569191825109\\
-24.8404117788017	0.00341520646563779\\
-24.7171544900613	0.00304823150256979\\
-24.5938972013209	0.00261771588457599\\
-24.4706399125804	0.00213587268186858\\
-24.34738262384	0.00163962403979746\\
-24.2241253350996	0.00115032594683344\\
-24.1008680463591	0.000698225161702748\\
-23.9776107576187	0.000302640693356236\\
-23.8543534688783	-1.57999981903146e-05\\
-23.7310961801378	-0.000253651087686976\\
-23.6078388913974	-0.000410517473420003\\
-23.484581602657	-0.000496803625855622\\
-23.3613243139165	-0.000526362411912167\\
-23.2380670251761	-0.000514914674523741\\
-23.1148097364357	-0.000477075056490069\\
-22.9915524476952	-0.000424754246487423\\
-22.8682951589548	-0.000366913822839594\\
-22.7450378702144	-0.000309447029758616\\
-22.6217805814739	-0.0002559345060426\\
-22.4985232927335	-0.000208087913801868\\
-22.3752660039931	-0.000166474690744924\\
-22.2520087152526	-0.000130824374360999\\
-22.1287514265122	-0.000100493636214563\\
-22.0054941377718	-7.44778608166672e-05\\
-21.8822368490313	-5.15388464602032e-05\\
-21.7589795602909	-2.98755925583329e-05\\
-21.6357222715504	-6.58142018788497e-06\\
-21.51246498281	2.36246038748461e-05\\
-21.3892076940696	7.07299535209988e-05\\
-21.2659504053291	0.000154091296846485\\
-21.1426931165887	0.000310347108910984\\
-21.0194358278483	0.000602981487679976\\
-20.8961785391078	0.00111149842099902\\
-20.7729212503674	0.00185402027954897\\
-20.649663961627	0.00267352467593689\\
-20.5264066728865	0.00330166084446995\\
-20.4031493841461	0.00358399194819761\\
-20.2798920954057	0.00354180996634823\\
-20.1566348066652	0.00326805584902029\\
-20.0333775179248	0.00284927937567092\\
-19.9101202291844	0.00234812818917569\\
-19.7868629404439	0.00181130270762364\\
-19.6636056517035	0.00127719818393038\\
-19.5403483629631	0.000779157702245402\\
-19.4170910742226	0.000344615731984889\\
-19.2938337854822	-7.50131052846203e-06\\
-19.1705764967418	-0.000268641229384475\\
-19.0473192080013	-0.000440733013817876\\
-18.9240619192609	-0.000534184235787254\\
-18.8008046305205	-0.00056453391850573\\
-18.67754734178	-0.000548912666854275\\
-18.5542900530396	-0.000503213441378957\\
-18.4310327642992	-0.000440414013067595\\
-18.3077754755587	-0.000369947796212001\\
-18.1845181868183	-0.000297790051660424\\
-18.0612608980779	-0.000226801578516624\\
-17.9380036093374	-0.00015686116732546\\
-17.814746320597	-8.42596888008719e-05\\
-17.6914890318566	3.52034215736099e-07\\
-17.5682317431161	0.000116645265168855\\
-17.4449744543757	0.000304356369880828\\
-17.3217171656353	0.000643343767718524\\
-17.1984598768948	0.00129421153763808\\
-17.0752025881544	0.00257485498357593\\
-16.9519452994139	0.00506899350976855\\
-16.8286880106735	0.00961557825962562\\
-16.7054307219331	0.0166990610594263\\
-16.5821734331926	0.0252186119846947\\
-16.4589161444522	0.0326221020407469\\
-16.3356588557118	0.0370828751851551\\
-16.2124015669713	0.0385108879253818\\
-16.0891442782309	0.0376664456278837\\
-15.9658869894905	0.0352940332762731\\
-15.84262970075	0.031890832914618\\
-15.7193724120096	0.0277711280008388\\
-15.5961151232692	0.0231695218105431\\
-15.4728578345287	0.0183096033921279\\
-15.3496005457883	0.0134325971357654\\
-15.2263432570479	0.00879355579989981\\
-15.1030859683074	0.00463508524698328\\
-14.979828679567	0.00115138440476588\\
-14.8565713908266	-0.00154275058633465\\
-14.7333141020861	-0.00342581201470874\\
-14.6100568133457	-0.00456173437145417\\
-14.4867995246053	-0.0050747773772667\\
-14.3635422358648	-0.00511729884851851\\
-14.2402849471244	-0.0048406983578974\\
-14.117027658384	-0.00437571730482196\\
-13.9937703696435	-0.00382330392546388\\
-13.8705130809031	-0.00325383370845605\\
-13.7472557921627	-0.00271128626825838\\
-13.6239985034222	-0.00221936916579452\\
-13.5007412146818	-0.00178760385255975\\
-13.3774839259414	-0.00141637591857584\\
-13.2542266372009	-0.00110061705943701\\
-13.1309693484605	-0.000832082989225336\\
-13.0077120597201	-0.00060016988623791\\
-12.8844547709796	-0.000390944137903461\\
-12.7611974822392	-0.000183646529359522\\
-12.6379401934988	5.66151041102964e-05\\
-12.5146829047583	0.000392205565575583\\
-12.3914256160179	0.000935192094876225\\
-12.2681683272775	0.00187400957387444\\
-12.144911038537	0.00345799309518712\\
-12.0216537497966	0.00580642496103212\\
-11.8983964610562	0.00852374165273391\\
-11.7751391723157	0.0106882525704396\\
-11.6518818835753	0.0115710220891567\\
-11.5286245948349	0.0111205548944217\\
-11.4053673060944	0.00970898416867601\\
-11.282110017354	0.00776131577221543\\
-11.1588527286135	0.00561419117180822\\
-11.0355954398731	0.00351189277848346\\
-10.9123381511327	0.00162688775005541\\
-10.7890808623922	7.0587283226905e-05\\
-10.6658235736518	-0.00110303964831471\\
-10.5425662849114	-0.00189230801127303\\
-10.4193089961709	-0.00233682654348667\\
-10.2960517074305	-0.00250182631070061\\
-10.1727944186901	-0.00246184070327648\\
-10.0495371299496	-0.00228759115801612\\
-9.92627984120921	-0.00203798144855873\\
-9.80302255246877	-0.00175703239130979\\
-9.67976526372834	-0.00147436905840148\\
-9.55650797498791	-0.00120760731806904\\
-9.43325068624747	-0.000965300702190854\\
-9.30999339750704	-0.000749572621135795\\
-9.1867361087666	-0.000557889733012075\\
-9.06347882002617	-0.000383512400264934\\
-8.94022153128573	-0.0002140025068686\\
-8.8169642425453	-2.68348699769011e-05\\
-8.69370695380487	0.00021947879536996\\
-8.57044966506443	0.000601367953053257\\
-8.447192376324	0.00125456187932844\\
-8.32393508758356	0.00239001407820679\\
-8.20067779884313	0.00422078735194762\\
-8.07742051010269	0.00668068013014718\\
-7.95416322136226	0.00915168146381509\\
-7.83090593262183	0.0108057513774378\\
-7.70764864388139	0.0112734357472524\\
-7.58439135514096	0.0107301057631463\\
-7.46113406640052	0.00955374274651019\\
-7.33787677766009	0.00809057198042237\\
-7.21461948891965	0.00660233081687332\\
-7.09136220017922	0.00528112056451999\\
-6.96810491143879	0.00426925772050605\\
-6.84484762269835	0.00367463897554492\\
-6.72159033395792	0.0035901615360712\\
-6.59833304521748	0.00412865176095851\\
-6.47507575647705	0.00548659032131846\\
-6.35181846773662	0.00805826559074402\\
-6.22856117899618	0.0126442743173587\\
-6.10530389025575	0.0208475303074897\\
-5.98204660151531	0.0358511105471126\\
-5.85878931277488	0.063969928291402\\
-5.73553202403445	0.117711839626507\\
-5.61227473529402	0.221388793006169\\
-5.48901744655358	0.418785765104297\\
-5.36576015781314	0.772199906134906\\
-5.24250286907271	1.32120213683349\\
-5.11924558033228	1.99990325803597\\
-4.99598829159184	2.63919495135227\\
-4.8727310028514	3.10660328117125\\
-4.74947371411097	3.38504084033926\\
-4.62621642537054	3.52337653802841\\
-4.50295913663011	3.57614469355774\\
-4.37970184788967	3.58296907741414\\
-4.25644455914923	3.57045914133869\\
-4.1331872704088	3.55904753351839\\
-4.00992998166837	3.5688406911646\\
-3.88667269292794	3.62366029711912\\
-3.7634154041875	3.75352865727206\\
-3.64015811544707	3.99572863570726\\
-3.51690082670664	4.39397015851236\\
-3.3936435379662	4.9936922255963\\
-3.27038624922577	5.82850777662518\\
-3.14712896048532	6.88876451268715\\
-3.02387167174489	8.06232559679716\\
-2.90061438300446	9.055625157277\\
-2.77735709426403	9.37820157568369\\
-2.65409980552359	8.58118846868781\\
-2.53084251678316	6.72929554526414\\
-2.40758522804273	4.48955416297463\\
-2.28432793930229	2.56319918529408\\
-2.16107065056186	1.23041348393226\\
-2.03781336182142	0.421706346462991\\
-1.91455607308099	-0.0420078932829469\\
-1.79129878434056	-0.315313561756009\\
-1.66804149560012	-0.498667645352382\\
-1.54478420685969	-0.648680127080005\\
-1.42152691811925	-0.792783850581067\\
-1.29826962937882	-0.938045427201314\\
-1.17501234063839	-1.07461923297838\\
-1.05175505189795	-1.17878050897885\\
};
\addplot [color=mycolor1,only marks,mark=asterisk,mark options={solid},forget plot]
  table[row sep=crcr]{%
-2.77735709426403	9.37820157568369\\
};
\end{axis}
\end{tikzpicture}%
\end{document}
\caption{Tikhonov regularization, L-curve and curvature for the matrix $\mathbf{A}_{3}$ and vector $\mathbf{b}_{err3}$.}
\label{fig:A3Tikh}
\end{figure}
\begin{figure}
% This file was created by matlab2tikz.
% Minimal pgfplots version: 1.3
%
%The latest updates can be retrieved from
%  http://www.mathworks.com/matlabcentral/fileexchange/22022-matlab2tikz
%where you can also make suggestions and rate matlab2tikz.
%
\documentclass[tikz]{standalone}
\usepackage{pgfplots}
\usepackage{grffile}
\pgfplotsset{compat=newest}
\usetikzlibrary{plotmarks}
\usepackage{amsmath}

\begin{document}
\definecolor{mycolor1}{rgb}{0.00000,0.44700,0.74100}%
\definecolor{mycolor2}{rgb}{0.85000,0.32500,0.09800}%
%
\begin{tikzpicture}

\begin{axis}[%
width=2in,
height=2in,
at={(0.758333in,0.48125in)},
scale only axis,
xmode=log,
xmin=0.0001,
xmax=10,
xminorticks=true,
xlabel={$\|\mathbf{Ax} - \mathbf{b}\|$},
ymode=log,
ymin=0.01,
ymax=1000000000000,
yminorticks=true,
ylabel={$\|\mathbf{Lx}\|$}
]
\addplot [color=mycolor1,solid,forget plot]
  table[row sep=crcr]{%
0.000384027890814747	743500821101.263\\
0.000385447432510019	676155254895.383\\
0.000381987127080822	611520170814.035\\
0.000387353764078907	550754288730.837\\
0.000394216431799186	494626592258.456\\
0.000401692700134262	443491641244.576\\
0.000414751260125575	397345890162.792\\
0.000427909043240456	355930466354.342\\
0.000441777863354013	318841855864.669\\
0.000453868798989714	285623137568.929\\
0.00047025575991135	255824610634.816\\
0.000485537584175856	229035552173.279\\
0.000503032891399228	204895397977.969\\
0.000518275109191441	183093686341.516\\
0.000536163301374921	163365806025.068\\
0.000554109688885454	145488119665.364\\
0.000571609852732594	129273171856.078\\
0.000588167026347474	114564462607.478\\
0.000605854199988932	101230576275.648\\
0.000623816484577194	89159276839.9105\\
0.000641220161235284	78252467249.8039\\
0.000658198682608486	68422461340.4207\\
0.000675439079053507	59589416090.1494\\
0.000691436314497076	51679580285.3664\\
0.000707012071379498	44624192076.4838\\
0.000722018069072014	38358976109.6992\\
0.000736262500792227	32824021223.7828\\
0.000749555632026515	27963561880.5254\\
0.000761962421125444	23725187364.5913\\
0.000773430994943271	20058363127.7042\\
0.000783890353265997	16912636182.0592\\
0.000793358258497581	14236184797.5186\\
0.000801875325254155	11975313136.0416\\
0.000809520186675673	10075170397.9925\\
0.000816355663234585	8481574797.94941\\
0.000822539214725467	7143481506.82984\\
0.000828123766435992	6015423730.35089\\
0.00083321693802785	5059243555.36166\\
0.000837863310535716	4244660797.73266\\
0.000842106600033454	3548640810.59659\\
0.000845975813623817	2953923992.17945\\
0.000849457610981324	2447274286.6506\\
0.000852573810052974	2017935063.26637\\
0.000855331352442333	1656549417.321\\
0.000857740052302947	1354564482.13195\\
0.000859821583185747	1103998434.48551\\
0.000861604204186245	897417182.262705\\
0.000863122988940828	728000441.139572\\
0.000864405011024584	589623630.816262\\
0.000865485708876924	476916666.225935\\
0.000866395554908363	385281280.006612\\
0.000867157629846471	310862261.363927\\
0.000867794398753791	250479488.877017\\
0.000868326576304309	201535995.091383\\
0.000868769197020489	161919816.305616\\
0.000869136850528671	129913652.00541\\
0.000869440488945298	104119280.053997\\
0.000869691170338768	83397255.7559758\\
0.000869896248522231	66819051.987793\\
0.000870064046461856	53628323.3554455\\
0.000870199932176283	43208820.1193909\\
0.000870310247639021	35057259.4498918\\
0.000870398976878009	28759808.5644919\\
0.000870470277181159	23971197.8213569\\
0.000870527648061129	20396494.4570335\\
0.000870573834108022	17777178.1855881\\
0.000870611661739135	15883899.7398944\\
0.000870643262166464	14516313.7660312\\
0.000870670740914407	13506598.4483667\\
0.000870696422880757	12721692.0461756\\
0.00087072248114511	12061619.3030508\\
0.000870751842064248	11454675.1583029\\
0.000870787274185134	10851692.925655\\
0.000870831979601829	10221124.0145389\\
0.000870889514335383	9545554.87126655\\
0.000870963084693589	8819444.40505906\\
0.000871054930541817	8047454.60071501\\
0.000871165922417228	7242693.73761202\\
0.000871294851891283	6424409.70109059\\
0.000871438449236497	5615063.10151283\\
0.000871591531079947	4837120.98823263\\
0.000871748214196866	4110157.2906526\\
0.000871902050092965	3448814.24157712\\
0.000872047799482256	2861909.60179837\\
0.000872181426527803	2352628.11358995\\
0.000872300482166606	1919479.77811208\\
0.000872403995430151	1557620.19053736\\
0.000872492181206447	1260184.83412948\\
0.000872566039881508	1019414.53411701\\
0.000872627077680546	827473.965518925\\
0.00087267693942677	676953.659043754\\
0.000872717352630143	561095.987008413\\
0.00087274991515181	473814.774436151\\
0.000872776072820075	409605.138358783\\
0.000872797128110895	363459.221116786\\
0.00087281422092995	330876.112291744\\
0.000872828377779907	307962.980918983\\
0.000872840575668105	291537.554284651\\
0.000872851804378324	279145.515960269\\
0.000872863148875945	268980.097625565\\
0.000872875887637065	259748.831806473\\
0.000872891614641111	250538.828722352\\
0.00087291230762658	240710.653820926\\
0.000872940391709611	229829.50746143\\
0.00087297870269551	217629.873133199\\
0.000873030271125157	204003.999263513\\
0.000873097940602713	189001.983813064\\
0.000873183805266014	172830.416321257\\
0.000873288572336398	155838.031658823\\
0.000873411063166874	138481.614709706\\
0.000873548065721145	121273.109433709\\
0.000873694645337868	104716.978944899\\
0.000873844870077205	89251.6131265291\\
0.000873992726933359	75207.5676784111\\
0.00087413297586651	62789.3820633814\\
0.000874261701800862	52080.0241746977\\
0.000874376513644115	43061.222391436\\
0.000874476444111076	35640.7818461698\\
0.000874561666674543	29679.0205885948\\
0.000874633152897007	25009.3138333594\\
0.000874692345297704	21451.2615033764\\
0.000874740911978656	18818.4738824219\\
0.000874780575250463	16925.1826170329\\
0.000874813020545509	15594.9600208712\\
0.000874839871739128	14670.8465681741\\
0.000874862724994816	14022.9962625837\\
0.000874883231957467	13550.8740378196\\
0.000874903234795669	13180.3611147232\\
0.000874924961838042	12858.2182345406\\
0.000874951281324253	12546.2707550198\\
0.000874986018421378	12216.6360817378\\
0.000875034304267812	11848.4112305444\\
0.00087510290028779	11425.7654295641\\
0.000875200365344259	10937.2087198124\\
0.0008753368861756	10375.749883363\\
0.000875523548917971	9739.60503227158\\
0.000875770894571484	9033.04210157061\\
0.000876086809344917	8266.88277966416\\
0.000876474144172252	7458.20933824314\\
0.000876928781808961	6629.00807177653\\
0.00087743896787572	5803.81773341996\\
0.000877986385935057	5006.82437651052\\
0.000878548819483504	4259.07424000569\\
0.000879103603004721	3576.4392767225\\
0.000879630825442135	2968.68986426562\\
0.000880115481562126	2439.66124041543\\
0.00088054828331104	1988.21780953104\\
0.000880925319377854	1609.61055875963\\
0.000881247004918733	1296.86882073137\\
0.000881516775068231	1041.99263078334\\
0.000881739851420666	836.842357633934\\
0.000881922253399447	673.717138624922\\
0.000882070101938568	545.664397941477\\
0.000882189189725795	446.580182958761\\
0.000882284761511706	371.162202476506\\
0.000882361445400847	314.780726227773\\
0.00088242328566234	273.34108608954\\
0.000882473841277314	243.20693945792\\
0.00088251632611709	221.210218014894\\
0.000882553773815641	204.707415850353\\
0.000882589210773637	191.612718439686\\
0.000882625813864916	180.37121383014\\
0.000882667015412885	169.88512632604\\
0.000882716501443067	159.426692727585\\
0.000882778041898897	148.562493580378\\
0.000882855109971558	137.096644114229\\
0.000882950307925163	125.02772729231\\
0.000883064713559896	112.50963490526\\
0.000883197354448412	99.8081609686592\\
0.000883345039783694	87.250821677965\\
0.000883502685625159	75.1739072177402\\
0.000883664084311492	63.8750023675712\\
0.000883822891656153	53.5791905392609\\
0.000883973537883274	44.4233211759313\\
0.000884111837000655	36.4574869534576\\
0.000884235214978939	29.658863471175\\
0.000884342612044681	23.951622608696\\
0.000884434184534794	19.2274521246893\\
0.000884510934898311	15.3631657745006\\
0.000884574362607374	12.233886241643\\
0.000884626184187896	9.72173345072184\\
0.000884668135371678	7.72073337964578\\
0.000884701847539274	6.13891964021595\\
0.0008847287819896	4.89854178304064\\
0.000884750204564176	3.9350952735346\\
0.000884767185808648	3.19566443174455\\
0.000884780615607714	2.63689748537517\\
0.000884791224737016	2.22287061081627\\
0.000884799608525488	1.92316194660471\\
0.000884806249760762	1.71153200245591\\
0.000884811539192677	1.56544779667082\\
0.00088481579273776	1.46623597964422\\
0.000884819264951798	1.39929783167656\\
0.000884822158760304	1.35394273632145\\
0.000884824631977512	1.32280003387814\\
0.000884826801750614	1.30102840427451\\
0.000884828748523357	1.28554855422026\\
0.000884830521034949	1.27442157795579\\
0.000884832143111614	1.26640384850493\\
0.000884833621868123	1.26065953860207\\
0.000884834956016491	1.25659245470824\\
0.000884836142796894	1.25375642342405\\
0.000884837182598229	1.2518102299731\\
0.000884838081179467	1.25049378613001\\
0.000884838850090758	1.24961253557527\\
0.000884839506195232	1.24902454951075\\
0.000884840071154404	1.24862885175094\\
0.000884840571563048	1.2483550970209\\
0.000884841040250615	1.24815497279366\\
0.000884841519163693	1.24799546763158\\
0.000884842064262226	1.24785389038387\\
0.000884842752832547	1.24771437852759\\
0.000884843693531583	1.24756559910647\\
0.000884845039047646	1.24739937616724\\
0.000884847000317441	1.2472100285124\\
0.000884849859519498	1.24699423778375\\
0.000884853976646257	1.24675126995746\\
0.000884859781944192	1.24648333975193\\
0.000884867745612843	1.24619585599825\\
0.000884878319375639	1.24589726248653\\
0.000884891854032734	1.24559825458754\\
0.000884908511922742	1.2453103446345\\
0.000884928207301962	1.24504402880711\\
0.000884950611496515	1.24480704178895\\
0.000884975247411572	1.24460321687151\\
0.000885001674712229	1.24443224723128\\
0.000885029748245637	1.24429028679574\\
0.00088505993362632	1.24417104215994\\
0.000885093690126888	1.24406692658891\\
0.000885133975472721	1.24396995865744\\
0.00088518598046678	1.2438722776723\\
0.000885258259892295	1.24376630663545\\
0.000885364492831723	1.24364467572023\\
0.000885526183735454	1.24350003626796\\
0.000885776699026595	1.2433248799655\\
0.000886167093950071	1.2431114574531\\
0.000886774153029138	1.2428518774532\\
0.000887710820622937	1.24253846003055\\
0.000889138551097445	1.24216440399217\\
0.000891279864257203	1.24172479006235\\
0.000894427506052603	1.2412178580741\\
0.000898944528621309	1.24064635779957\\
0.000905248669481221	1.24001859925329\\
0.000913776988357551	1.23934869050307\\
0.000924935182227325	1.2386554682921\\
0.000939050216888145	1.23795990763481\\
0.000956359720496604	1.23728132806644\\
0.000977077891367307	1.23663328067663\\
0.00100156923052178	1.23602025716306\\
0.00103064193013246	1.23543608447943\\
0.00106595355773076	1.23486417181015\\
0.00111050844205948	1.23427906522417\\
0.00116920756877463	1.23364842304354\\
0.00124936671804924	1.2329346470784\\
0.00136104924496009	1.23209580766755\\
0.00151703290799603	1.23108592870681\\
0.00173235075065774	1.22985500183823\\
0.00202359783502913	1.22834925973995\\
0.00240833384797581	1.2265123039569\\
0.00290473088024552	1.22428768477834\\
0.00353128248676465	1.22162344199861\\
0.00430623084306251	1.21847884344671\\
0.00524647234300286	1.21483297659519\\
0.0063659555456435	1.21069388095001\\
0.00767389922745198	1.20610567514398\\
0.00917350230388063	1.20115010747196\\
0.0108621328123916	1.19593897527504\\
0.0127341346267285	1.19059572584351\\
0.0147872104636232	1.18522826769181\\
0.0170327342331192	1.1798991356029\\
0.0195093738080221	1.17460120875709\\
0.0222982457539649	1.16924546759898\\
0.0255366935900139	1.16366251413495\\
0.0294270245057018	1.1576146597932\\
0.034236859911973	1.15081312097494\\
0.0402898907437977	1.14293597932512\\
0.0479491817173111	1.1336456796491\\
0.0575971512353649	1.12260792976103\\
0.0696148437922719	1.10951551266987\\
0.0843595509475537	1.09412010717513\\
0.102138010889903	1.07627253444351\\
0.123174569101008	1.05596692304409\\
0.147579227821669	1.03337777465943\\
0.175327313374725	1.00887295941265\\
0.206267756074147	0.982984290054965\\
0.240177476058065	0.956325045576296\\
0.276872368946822	0.929461922682164\\
0.316370445326132	0.902771338562272\\
0.35908263463658	0.876323981174198\\
0.405986614181663	0.849835839633071\\
0.45872425484415	0.822699819857905\\
0.51956075906348	0.79408413535927\\
0.59116311031725	0.763068177714891\\
0.676202954101578	0.728788099867023\\
0.776850906064735	0.690574714716072\\
0.894272159356436	0.648074137526996\\
1.0282346481496	0.601341392655325\\
1.17691301930834	0.550891607813543\\
1.33693825712574	0.497690043167036\\
1.50370647347027	0.443068371809084\\
};
\addplot [color=mycolor1,only marks,mark=asterisk,mark options={solid},forget plot]
  table[row sep=crcr]{%
0.000885133975472721	1.24396995865744\\
};
\end{axis}
\end{tikzpicture}%
\end{document}
% This file was created by matlab2tikz.
% Minimal pgfplots version: 1.3
%
%The latest updates can be retrieved from
%  http://www.mathworks.com/matlabcentral/fileexchange/22022-matlab2tikz
%where you can also make suggestions and rate matlab2tikz.
%
\documentclass[tikz]{standalone}
\usepackage{pgfplots}
\usepackage{grffile}
\pgfplotsset{compat=newest}
\usetikzlibrary{plotmarks}
\usepackage{amsmath}

\begin{document}
\definecolor{mycolor1}{rgb}{0.00000,0.44700,0.74100}%
\definecolor{mycolor2}{rgb}{0.85000,0.32500,0.09800}%
%
\begin{tikzpicture}

\begin{axis}[%
width=2in,
height=2in,
at={(0.758333in,0.48125in)},
scale only axis,
xmin=-40,
xmax=5,
xlabel={$\lambda$},
ymin=-500,
ymax=3000,
ylabel={curvature $\kappa$}
]
\addplot [color=mycolor1,solid,forget plot]
  table[row sep=crcr]{%
-35.6840203428776	-2.52866718219788\\
-35.5607558214447	4.0397007267329\\
-35.4374913000118	0.508310982486185\\
-35.3142267785789	0.140913548286635\\
-35.190962257146	1.85818070130469\\
-35.0676977357131	-0.106056372359343\\
-34.9444332142802	0.0875461960325155\\
-34.8211686928473	-0.63266191597387\\
-34.6979041714144	1.12768941899492\\
-34.5746396499815	-0.455762071362318\\
-34.4513751285486	0.409733057022145\\
-34.3281106071157	-0.728558535881027\\
-34.2048460856828	0.464548946831439\\
-34.0815815642499	-0.18823546023159\\
-33.958317042817	-0.289918605532512\\
-33.8350525213841	-0.369917939092663\\
-33.7117879999512	0.0421658524628182\\
-33.5885234785183	-0.126102409158861\\
-33.4652589570854	-0.254478866254122\\
-33.3419944356525	-0.211547643517017\\
-33.2187299142196	-0.0977500859252488\\
-33.0954653927867	-0.284073482717354\\
-32.9722008713538	-0.154940317723393\\
-32.8489363499209	-0.155223524109803\\
-32.725671828488	-0.157357039600514\\
-32.6024073070551	-0.156353177641714\\
-32.4791427856222	-0.130802644391972\\
-32.3558782641893	-0.117972579384299\\
-32.2326137427564	-0.108244973196421\\
-32.1093492213235	-0.093907470856709\\
-31.9860846998906	-0.0812547930450456\\
-31.8628201784577	-0.0693251640250379\\
-31.7395556570248	-0.0617616501535549\\
-31.6162911355919	-0.049850241621338\\
-31.493026614159	-0.0470874801558925\\
-31.3697620927261	-0.0408268418322121\\
-31.2464975712932	-0.0386957358881428\\
-31.1232330498603	-0.0356007752648415\\
-30.9999685284274	-0.0324657898177313\\
-30.8767040069945	-0.0312410682868664\\
-30.7534394855616	-0.0273974826948738\\
-30.6301749641287	-0.0245440846823299\\
-30.5069104426958	-0.0218358274438967\\
-30.3836459212629	-0.018948571241534\\
-30.26038139983	-0.0162206490378194\\
-30.1371168783971	-0.0136069605066502\\
-30.0138523569642	-0.0117290969426987\\
-29.8905878355313	-0.00971074284434647\\
-29.7673233140984	-0.00809080917030379\\
-29.6440587926655	-0.00690018618273968\\
-29.5207942712326	-0.00578361687666702\\
-29.3975297497997	-0.00477081280786461\\
-29.2742652283668	-0.00402355478804884\\
-29.1510007069339	-0.00331029511060089\\
-29.027736185501	-0.00276755207223115\\
-28.9044716640681	-0.00223554560696997\\
-28.7812071426352	-0.00187121313954897\\
-28.6579426212023	-0.00147857683087449\\
-28.5346780997694	-0.00121770904342735\\
-28.4114135783365	-0.000923066412600601\\
-28.2881490569036	-0.000732645143755706\\
-28.1648845354707	-0.000534742131596577\\
-28.0416200140378	-0.000358541691777898\\
-27.9183554926049	-0.000206097012442107\\
-27.795090971172	-1.50507216767801e-06\\
-27.6718264497391	0.000217076160918809\\
-27.5485619283062	0.000544517189785829\\
-27.4252974068733	0.00111816567765689\\
-27.3020328854404	0.00181416096773618\\
-27.1787683640075	0.00293885632073104\\
-27.0555038425746	0.00356387233696913\\
-26.9322393211417	0.00380430716622086\\
-26.8089747997088	0.00366331409151755\\
-26.6857102782759	0.00303809290239784\\
-26.562445756843	0.00215862594619505\\
-26.4391812354101	0.00129107862585661\\
-26.3159167139772	0.000478644884911733\\
-26.1926521925443	-0.000173863526691466\\
-26.0693876711114	-0.000669759413003251\\
-25.9461231496785	-0.000959165237798107\\
-25.8228586282456	-0.00114486550414959\\
-25.6995941068127	-0.00117652455318723\\
-25.5763295853798	-0.00113623724594451\\
-25.4530650639469	-0.00104107569834352\\
-25.329800542514	-0.0009167151211075\\
-25.2065360210811	-0.000781527037345303\\
-25.0832714996482	-0.000649358279565317\\
-24.9600069782153	-0.000524338774389237\\
-24.8367424567824	-0.000413831511544012\\
-24.7134779353495	-0.000311935923269186\\
-24.5902134139166	-0.000220790608599665\\
-24.4669488924837	-0.000133867068801464\\
-24.3436843710508	-3.96796845132219e-05\\
-24.2204198496179	7.76110229393102e-05\\
-24.097155328185	0.000251763405156988\\
-23.9738908067521	0.000544866454898443\\
-23.8506262853192	0.00105331431454039\\
-23.7273617638863	0.00188381017582317\\
-23.6040972424534	0.00303052028994921\\
-23.4808327210205	0.00424846116507346\\
-23.3575681995876	0.00506520938333592\\
-23.2343036781547	0.00526643293422564\\
-23.1110391567218	0.00491830953363469\\
-22.9877746352889	0.00418741659076853\\
-22.864510113856	0.00326725873071502\\
-22.7412455924231	0.00229554118200493\\
-22.6179810709902	0.00136835752986749\\
-22.4947165495573	0.000556203375674196\\
-22.3714520281244	-9.68982602369896e-05\\
-22.2481875066915	-0.000574323197820882\\
-22.1249229852586	-0.00088110896611129\\
-22.0016584638257	-0.00103924344810893\\
-21.8783939423928	-0.00107855498974403\\
-21.7551294209599	-0.00103254040558003\\
-21.631864899527	-0.000931195994619841\\
-21.5086003780941	-0.000798254661287654\\
-21.3853358566612	-0.000650226258564591\\
-21.2620713352283	-0.00049609601044232\\
-21.1388068137954	-0.000337704738449236\\
-21.0155422923625	-0.000166718451399487\\
-20.8922777709296	3.9237342119535e-05\\
-20.7690132494967	0.000328226218540609\\
-20.6457487280638	0.000795584396000782\\
-20.5224842066309	0.00162493025187424\\
-20.399219685198	0.00314764709182573\\
-20.2759551637651	0.00586198933909271\\
-20.1526906423322	0.0101827116797276\\
-20.0294261208993	0.0156907714804115\\
-19.9061615994664	0.0207823203817392\\
-19.7828970780335	0.023812891220019\\
-19.6596325566006	0.0243428579296813\\
-19.5363680351677	0.0228599891692451\\
-19.4131035137348	0.0200650709912819\\
-19.2898389923019	0.0165262401911477\\
-19.166574470869	0.0126625679766985\\
-19.0433099494361	0.00879989191376292\\
-18.9200454280032	0.00520493906407475\\
-18.7967809065703	0.00208545328065327\\
-18.6735163851374	-0.000420822349915907\\
-18.5502518637045	-0.00225811688797687\\
-18.4269873422716	-0.00344866925649436\\
-18.3037228208387	-0.00407621329467629\\
-18.1804582994058	-0.00425956684657115\\
-18.0571937779729	-0.00412652825185547\\
-17.93392925654	-0.00379373238507507\\
-17.8106647351071	-0.00335551352651654\\
-17.6874002136742	-0.00288053038721978\\
-17.5641356922413	-0.00241382424336812\\
-17.4408711708084	-0.00198141358668401\\
-17.3176066493755	-0.00159546447386263\\
-17.1943421279426	-0.00125884268999694\\
-17.0710776065097	-0.0009684890627197\\
-16.9478130850768	-0.000717396719069068\\
-16.8245485636439	-0.000494908119184825\\
-16.701284042211	-0.000285026757552941\\
-16.5780195207781	-6.2446098097943e-05\\
-16.4547549993452	0.000213294788222611\\
-16.3314904779123	0.000601850132066607\\
-16.2082259564794	0.00116869095502204\\
-16.0849614350465	0.00192146399393367\\
-15.9616969136136	0.00270644542921058\\
-15.8384323921807	0.00322491612083822\\
-15.7151678707478	0.00325841750212148\\
-15.5919033493149	0.00282579112989471\\
-15.468638827882	0.00210125418764623\\
-15.3453743064491	0.00127501777105382\\
-15.2221097850162	0.000488906472062368\\
-15.0988452635833	-0.00017124929958144\\
-14.9755807421504	-0.000665863093573972\\
-14.8523162207175	-0.000989822786152735\\
-14.7290516992846	-0.00116117199784384\\
-14.6057871778517	-0.00121060309979291\\
-14.4825226564188	-0.00117262827618742\\
-14.3592581349859	-0.0010793214892759\\
-14.235993613553	-0.000956867882587587\\
-14.1127290921201	-0.000824474599326245\\
-13.9894645706872	-0.000694838807137999\\
-13.8662000492543	-0.000575381925901056\\
-13.7429355278214	-0.000469673043311122\\
-13.6196710063885	-0.000378709761012589\\
-13.4964064849556	-0.000301924677799391\\
-13.3731419635227	-0.000237900695454906\\
-13.2498774420898	-0.000184834473265303\\
-13.1266129206569	-0.000140796211528349\\
-13.003348399224	-0.000103812065484003\\
-12.8800838777911	-7.17499801532315e-05\\
-12.7568193563582	-4.19231955061317e-05\\
-12.6335548349253	-1.02375890878418e-05\\
-12.5102903134924	3.04391470507824e-05\\
-12.3870257920595	9.33347942738299e-05\\
-12.2637612706266	0.000203453870965084\\
-12.1404967491937	0.000407612316578735\\
-12.0172322277608	0.000790512811658689\\
-11.8939677063278	0.00149793600831768\\
-11.770703184895	0.00277029294127176\\
-11.6474386634621	0.00500177876426867\\
-11.5241741420291	0.0088642932185428\\
-11.4009096205963	0.0155663253570431\\
-11.2776450991634	0.0273675790734949\\
-11.1543805777304	0.048587733638632\\
-11.0311160562976	0.0876074768007146\\
-10.9078515348647	0.160891450043928\\
-10.7845870134318	0.301154573061637\\
-10.6613224919989	0.574023895648592\\
-10.5380579705659	1.11176736022051\\
-10.414793449133	2.17830569348852\\
-10.2915289277002	4.27244435153555\\
-10.1682644062672	8.18583170904009\\
-10.0449998848344	14.6756254336597\\
-9.92173536340145	23.497488421752\\
-9.79847084196855	32.9731305236013\\
-9.67520632053565	41.3615064948216\\
-9.55194179910275	48.1976716527977\\
-9.42867727766985	54.058704726234\\
-9.30541275623695	59.8466267994801\\
-9.18214823480405	66.4696404388069\\
-9.05888371337115	74.8688767481852\\
-8.93561919193825	86.1900351388434\\
-8.81235467050535	102.039735536647\\
-8.68909014907245	124.865014664198\\
-8.56582562763955	158.559475786613\\
-8.44256110620665	209.463708003427\\
-8.31929658477375	287.994655814347\\
-8.19603206334085	411.104167923982\\
-8.07276754190795	605.161833855718\\
-7.94950302047505	906.09288623496\\
-7.82623849904215	1345.38732201667\\
-7.70297397760925	1900.28986159567\\
-7.57970945617635	2417.39260113834\\
-7.45644493474345	2635.78016326041\\
-7.33318041331055	2407.97551200296\\
-7.20991589187765	1859.88304046079\\
-7.08665137044475	1250.12242777972\\
-6.96338684901185	757.216003465418\\
-6.84012232757895	426.952271194982\\
-6.71685780614605	230.093695625311\\
-6.59359328471315	120.908551153581\\
-6.47032876328025	62.8592461044344\\
-6.34706424184735	32.6842023198584\\
-6.22379972041445	17.1422143458738\\
-6.10053519898155	9.13619347073635\\
-5.97727067754865	4.98283539718212\\
-5.85400615611575	2.80041165490599\\
-5.73074163468285	1.6327499980774\\
-5.60747711324995	0.993185575870266\\
-5.48421259181705	0.632249571600708\\
-5.36094807038415	0.4204539420185\\
-5.23768354895125	0.289323752471637\\
-5.11441902751835	0.201901346244557\\
-4.99115450608545	0.138440146802035\\
-4.86788998465255	0.0897019573429255\\
-4.74462546321965	0.0527610188101506\\
-4.62136094178675	0.0268127266648718\\
-4.49809642035385	0.0102489653765714\\
-4.37483189892095	0.000400413113059156\\
-4.25156737748805	-0.00539918795282119\\
-4.12830285605515	-0.00910743560410175\\
-4.00503833462225	-0.0119081833446357\\
-3.88177381318935	-0.0144176651966337\\
-3.75850929175645	-0.0168911499098576\\
-3.63524477032355	-0.0193543988233582\\
-3.51198024889065	-0.0216739257300706\\
-3.38871572745775	-0.0235995802478734\\
-3.26545120602485	-0.0248098373042436\\
-3.14218668459195	-0.0249819510145252\\
-3.01892216315905	-0.0238971194274488\\
-2.89565764172615	-0.0215741892038493\\
-2.77239312029325	-0.0184075541601099\\
-2.64912859886035	-0.0152681566010115\\
-2.52586407742745	-0.013494246397096\\
-2.40259955599454	-0.0146286784678313\\
-2.27933503456165	-0.0197794960904729\\
-2.15607051312875	-0.0289154417052758\\
-2.03280599169585	-0.0409099736880586\\
-1.90954147026295	-0.0544864421662522\\
-1.78627694883005	-0.0690510697973908\\
-1.66301242739714	-0.0847182845372331\\
-1.53974790596425	-0.101849646833216\\
-1.41648338453135	-0.120596766174175\\
-1.29321886309845	-0.140600448251741\\
-1.16995434166555	-0.160819311230019\\
-1.04668982023265	-0.179490690016273\\
-0.923425298799741	-0.194314456361817\\
-0.800160777366845	-0.202994551427409\\
-0.676896255933949	-0.204209097273909\\
-0.553631734501041	-0.198887609549247\\
-0.430367213068145	-0.191407416730595\\
-0.307102691635245	-0.190025289902381\\
-0.183838170202349	-0.205522955333876\\
-0.0605736487694452	-0.247296026504519\\
0.0626908726634548	-0.318569238228649\\
0.185955394096351	-0.415405491056711\\
0.309219915529255	-0.531111021743454\\
0.432484436962155	-0.660776858691072\\
0.555748958395051	-0.800946117791777\\
0.679013479827955	-0.945069743713509\\
0.802278001260855	-1.07841746724902\\
0.925542522693751	-1.17680768199543\\
};
\addplot [color=mycolor1,only marks,mark=asterisk,mark options={solid},forget plot]
  table[row sep=crcr]{%
-7.45644493474345	2635.78016326041\\
};
\end{axis}
\end{tikzpicture}%
\end{document}
\caption{Tikhonov regularization, L-curve and curvature for the matrix $\mathbf{A}_{4}$ and vector $\mathbf{b}_{err4}$.}
\label{fig:A4Tikh}
\end{figure}
\begin{figure}
% This file was created by matlab2tikz.
% Minimal pgfplots version: 1.3
%
%The latest updates can be retrieved from
%  http://www.mathworks.com/matlabcentral/fileexchange/22022-matlab2tikz
%where you can also make suggestions and rate matlab2tikz.
%
\documentclass[tikz]{standalone}
\usepackage{pgfplots}
\usepackage{grffile}
\pgfplotsset{compat=newest}
\usetikzlibrary{plotmarks}
\usepackage{amsmath}

\begin{document}
\definecolor{mycolor1}{rgb}{0.00000,0.44700,0.74100}%
\definecolor{mycolor2}{rgb}{0.85000,0.32500,0.09800}%
%
\begin{tikzpicture}

\begin{axis}[%
width=2in,
height=2in,
scale only axis,
xmode=log,
xmin=0.0001,
xmax=10000,
xminorticks=true,
xlabel={$\|\mathbf{Ax} - \mathbf{b}\|$},
ymode=log,
ymin=1,
ymax=1000000000000,
yminorticks=true,
ylabel={$\|\mathbf{Lx}\|$}
]
\addplot [color=mycolor1,solid,forget plot]
  table[row sep=crcr]{%
0.000789147999723165	74188475088.9781\\
0.000821572301693219	65021953614.3971\\
0.000782656713114355	56131373996.4269\\
0.000783702641439355	47760339681.4693\\
0.000784023028530464	40095719346.9744\\
0.000791518446110522	33255486403.0428\\
0.000782568243724834	27289318434.645\\
0.000781034117145177	22188837347.0588\\
0.000781527618564922	17902852640.5956\\
0.000781935435404986	14353348145.8207\\
0.000781432921722013	11449365978.8688\\
0.000781873864287092	9097494872.8983\\
0.000781522230913377	7208838512.17827\\
0.000781402375728534	5702990240.96278\\
0.00078170299166396	4509770861.23152\\
0.000781604978241085	3569461719.88999\\
0.000781637730068396	2832128770.80341\\
0.000781633880020603	2256477213.22895\\
0.000781837768296601	1808546812.51121\\
0.000781848565312015	1460466127.2074\\
0.000782014859499689	1189414174.17413\\
0.000782178914123818	976861356.76202\\
0.000782360186016512	808062605.06056\\
0.000782647371296919	671682933.503484\\
0.00078294302651438	559406984.515451\\
0.000783259381031987	465443502.887793\\
0.00078361933294886	385936745.361963\\
0.000783946366948332	318367950.701837\\
0.000784291387053906	261038685.417249\\
0.000784618533082697	212691071.092777\\
0.000784901304614162	172272218.553212\\
0.000785179926298839	138816173.176088\\
0.000785405352028932	111403483.758485\\
0.000785602066799948	89161452.8582881\\
0.000785762228862213	71278744.0576541\\
0.000785896410404918	57019485.6049404\\
0.000786007890878817	45730878.6145279\\
0.00078609944713387	36843867.0454238\\
0.000786173457076853	29869219.8823745\\
0.000786236081051855	24392127.9972113\\
0.000786288914462028	20067588.9856114\\
0.00078633301784106	16616977.4073512\\
0.000786372270042741	13824354.2762478\\
0.000786407794119244	11530533.6786545\\
0.000786438953961266	9624040.85813619\\
0.000786467869875797	8029858.85385462\\
0.000786493621837752	6697932.08097949\\
0.000786517204922552	5593275.28221592\\
0.000786538113798533	4688651.75270523\\
0.000786555928654173	3959845.26348543\\
0.000786571554568911	3383022.37889509\\
0.000786585067026108	2933642.07910238\\
0.000786596744742713	2586573.36699445\\
0.000786607217902807	2317162.98796251\\
0.000786616941186616	2102809.26039114\\
0.000786626332643297	1924412.91913918\\
0.000786636127918206	1767219.36837574\\
0.000786647011885157	1620954.48472345\\
0.000786659299124315	1479447.26899706\\
0.000786673565128756	1339979.2371813\\
0.000786689725718461	1202510.0754423\\
0.000786707789848156	1068847.21323217\\
0.000786727149332803	941807.525690871\\
0.000786747451531658	824437.910901388\\
0.000786767653278072	719372.427119893\\
0.000786787271949639	628383.68576024\\
0.000786805705996953	552146.882331837\\
0.000786822682976417	490204.490095843\\
0.000786838093709323	441110.219755598\\
0.000786852095317468	402722.38266846\\
0.000786865036345892	372585.86084428\\
0.000786877471321028	348306.81593013\\
0.000786890115092965	327830.680007221\\
0.000786903734820923	309587.540982043\\
0.000786919144194431	292524.787219484\\
0.000786937177794038	276068.222529991\\
0.000786958413485896	260043.560062826\\
0.000786983084677588	244573.170114259\\
0.000787011223589871	229954.531961561\\
0.000787042344474459	216530.283571538\\
0.000787075852295958	204569.047936763\\
0.000787111180133819	194181.409961815\\
0.000787148063545241	185288.816186957\\
0.000787186949907559	177645.494745962\\
0.000787229267097226	170895.025507098\\
0.000787277736560026	164635.708723104\\
0.000787336714384676	158475.126218057\\
0.000787412353261338	152066.918272708\\
0.000787512832714215	145132.801980775\\
0.00078764804423445	137476.417213784\\
0.000787829048059749	128993.91718028\\
0.000788066587355112	119682.234311859\\
0.000788369056479564	109642.168706228\\
0.000788740146874091	99071.6261995533\\
0.000789176837555339	88245.4692838261\\
0.000789668506705965	77482.357330841\\
0.000790197938662955	67103.9374618951\\
0.000790743817374261	57395.1569020764\\
0.00079128418926596	48574.315151183\\
0.000791799761799977	40777.8515435646\\
0.000792276196867555	34059.8433696091\\
0.000792705057780873	28402.2972448611\\
0.000793083631884483	23730.9704562383\\
0.00079341396594314	19932.493473471\\
0.000793701595188731	16870.7181161423\\
0.000793954228007021	14401.922017741\\
0.000794180528622922	12388.7384097087\\
0.000794388981858553	10711.7428460504\\
0.000794586893721421	9276.94387300093\\
0.000794779541214855	8018.05430126088\\
0.000794969671284699	6893.94154952155\\
0.000795157441804115	5882.91041952119\\
0.000795340884137871	4975.85401878159\\
0.00079551674851576	4169.96613269876\\
0.000795681478730523	3464.03041203895\\
0.000795832050713028	2855.59745144341\\
0.000795966490754766	2339.8235261479\\
0.000796084045875918	1909.46465383387\\
0.000796185078991107	1555.48413139974\\
0.00079627080778601	1267.85469543518\\
0.000796342997523048	1036.31359763022\\
0.000796403669270905	850.980352675733\\
0.000796454861837589	702.831960773381\\
0.000796498450575703	584.046585870914\\
0.000796536026293775	488.204827190225\\
0.000796568835476533	410.323942806596\\
0.000796597780469228	346.721776114875\\
0.000796623473984962	294.750953881948\\
0.000796646330617415	252.476647423061\\
0.000796666668997416	218.374073367984\\
0.000796684803900146	191.101650025826\\
0.000796701111361691	169.377312102333\\
0.000796716063669109	151.956042194405\\
0.000796730234520773	137.679083641205\\
0.000796744276084151	125.548986540809\\
0.000796758868402952	114.789460855972\\
0.000796774641726454	104.869766399068\\
0.000796792080776282	95.4934476589041\\
0.000796811431428227	86.560574900157\\
0.00079683264157748	78.1135561116287\\
0.000796855364960451	70.275321571692\\
0.000796879040211658	63.1883383993088\\
0.000796903028384706	56.9630290509655\\
0.000796926771951852	51.6433237797135\\
0.000796949934175678	47.1943681806604\\
0.00079697248832676	43.5125524069934\\
0.000796994746692485	40.4519384556145\\
0.000797017328200031	37.8566754233191\\
0.000797041069583123	35.5890397545754\\
0.000797066884902358	33.5469792637582\\
0.000797095587098639	31.6700640710187\\
0.000797127701872173	29.9357976372888\\
0.000797163323493465	28.3492439375768\\
0.000797202068965899	26.9291841974667\\
0.000797243165151172	25.6942813682552\\
0.000797285658276642	24.6526652428333\\
0.000797328692557592	23.7972602685334\\
0.000797371780068753	23.1070625563202\\
0.000797414994473845	22.5524533098743\\
0.000797459047103759	22.1017096482158\\
0.000797505228507424	21.726403721087\\
0.000797555211712236	21.4046168111988\\
0.00079761073156449	21.1219460015186\\
0.000797673178154486	20.8707984505863\\
0.000797743187299552	20.6485993147678\\
0.00079782034687866	20.4555363756942\\
0.000797903132513429	20.2924293045178\\
0.000797989123809586	20.1591983396122\\
0.000798075447676047	20.0541686899829\\
0.000798159315458043	19.9741450998331\\
0.000798238502566048	19.9149647931241\\
0.000798311673494137	19.8721809121537\\
0.000798378539246377	19.8416220761911\\
0.000798439903475095	19.8197219854615\\
0.000798497691354488	19.8036310044458\\
0.000798555060098873	19.7911778352591\\
0.00079861668529605	19.7807555671005\\
0.000798689304210887	19.7711883069188\\
0.000798782582488735	19.7616121058939\\
0.000798910332824404	19.7513861165877\\
0.0007990920366959	19.7400384227814\\
0.000799354478489525	19.7272441187064\\
0.000799733088195838	19.7128287075141\\
0.000800272341655625	19.6967859846303\\
0.000801024422522156	19.6792956988275\\
0.000802045544681321	19.6607236942031\\
0.000803390133180049	19.6415889665833\\
0.000805104494827263	19.6224911739729\\
0.00080722316955633	19.6040082627774\\
0.000809771797713763	19.5865906775326\\
0.000812779063309408	19.5704859852832\\
0.000816297012692341	19.5557191240548\\
0.000820424993141676	19.542132396969\\
0.000825329275099548	19.529467552328\\
0.000831249273903773	19.5174606478023\\
0.00083848272463703	19.5059214625186\\
0.000847346931879918	19.4947783482739\\
0.00085812218647119	19.4840809268028\\
0.000870995700958258	19.4739642076237\\
0.000886034983308022	19.4645876364315\\
0.000903220786705124	19.4560693297609\\
0.00092255767579969	19.4484360230433\\
0.000944260456022836	19.4416016336027\\
0.000968999544326252	19.435375064289\\
0.00099818562786579	19.4294873666353\\
0.00103427835867735	19.4236244912415\\
0.00108109937584316	19.4174547252467\\
0.00114410189053527	19.4106461398221\\
0.00123050081299879	19.4028752697304\\
0.00134913623267811	19.3938319577429\\
0.00150998892272122	19.3832266281431\\
0.001723402035521	19.3708055817843\\
0.00199918742727847	19.3563774454901\\
0.00234578406410116	19.3398497128434\\
0.00276952495268161	19.3212686968664\\
0.00327402324039969	19.3008503438652\\
0.00385978320599582	19.2789857637029\\
0.00452428705018191	19.2562076586358\\
0.00526285423006131	19.233114814061\\
0.00607041339252486	19.2102697418367\\
0.00694395953193575	19.1881012653039\\
0.00788500171675741	19.166848123605\\
0.00890095732746626	19.1465658124882\\
0.0100044712376792	19.127192658472\\
0.0112102027432634	19.1086465230603\\
0.0125296657469256	19.0909126466105\\
0.0139658255306155	19.0740892160655\\
0.0155097230363011	19.0583756753177\\
0.0171409843010678	19.0440116409438\\
0.0188328315297276	19.0311929492568\\
0.0205608031898219	19.019998278608\\
0.0223135003197429	19.010351870512\\
0.0241035826339094	19.0020297423603\\
0.0259776832385632	18.9946988348108\\
0.0280242739257161	18.9879697738078\\
0.0303782089707012	18.9814461710195\\
0.0332196971622655	18.9747618970527\\
0.0367648047147891	18.9676062327994\\
0.0412459281257039	18.9597414038506\\
0.0468845124738951	18.9510170737603\\
0.0538623589246208	18.9413831481169\\
0.0622989903310456	18.930897655061\\
0.0722407729923958	18.9197227551115\\
0.0836658653027225	18.9081014647666\\
0.0965089525338173	18.8963121329251\\
0.11070915526905	18.8846064574773\\
0.126281105176815	18.8731458795133\\
0.14340293654526	18.8619547517034\\
0.162507987665224	18.8509034004342\\
0.184361962080961	18.8397225943107\\
0.21010629807046	18.8280399491333\\
0.241253483396197	18.815424432418\\
0.279632064737968	18.8014281958189\\
0.327294915438388	18.7856215897198\\
0.386415516667634	18.7676224337085\\
0.459197458316691	18.7471215214941\\
0.547816334610195	18.7239032012241\\
0.6544096962322	18.697855647905\\
0.781129915510641	18.6689642421289\\
0.930265992906048	18.6372861360235\\
1.10441490228178	18.6029135102309\\
1.30665009493215	18.5659410898917\\
1.5406249116596	18.5264521192974\\
1.81059713248179	18.4845229422738\\
2.12147300345355	18.4402264155697\\
2.47909473728327	18.3936029066471\\
2.89104629440448	18.3445760420476\\
3.36815986434145	18.292816970345\\
3.92667701009591	18.2375896356232\\
4.59074768017153	18.1776206173985\\
5.39478071141323	18.1110223779148\\
6.38521222115566	18.0352691605538\\
7.62153973309294	17.9472015467327\\
9.17681609117762	17.8430332112485\\
11.1379595817325	17.7183495982091\\
13.6060962361536	17.568109558744\\
16.6968298060319	17.3866749962168\\
20.5400535194324	17.16789657619\\
25.2787893141482	16.9052787419829\\
31.0665627548722	16.5922377068323\\
38.0629774595468	16.2224513592303\\
46.4274078427226	15.7902806351717\\
56.3109402580207	15.2912263771545\\
67.8466344821961	14.7223884476628\\
81.1378233793216	14.0829182210285\\
96.2440135130231	13.3744756928272\\
113.164841506294	12.601673943675\\
131.824744804497	11.7724037320445\\
152.063229489933	10.8978436404526\\
173.635482649474	9.99198638509857\\
196.224419214788	9.07069298687226\\
219.460158066803	8.15051649861752\\
242.94061822126	7.24762570105955\\
266.249699663169	6.37701954971973\\
288.974773067855	5.55198001091083\\
310.727552028522	4.7835876600195\\
331.169196051035	4.08021657168354\\
350.034695376577	3.44711389795193\\
367.1487792313	2.88626135319397\\
382.428319052096	2.396633132412\\
};
\addplot [color=mycolor1,only marks,mark=asterisk,mark options={solid},forget plot]
  table[row sep=crcr]{%
0.000798910332824404	19.7513861165877\\
};
\end{axis}
\end{tikzpicture}%
\end{document}
% This file was created by matlab2tikz.
% Minimal pgfplots version: 1.3
%
%The latest updates can be retrieved from
%  http://www.mathworks.com/matlabcentral/fileexchange/22022-matlab2tikz
%where you can also make suggestions and rate matlab2tikz.
%
\documentclass[tikz]{standalone}
\usepackage{pgfplots}
\usepackage{grffile}
\pgfplotsset{compat=newest}
\usetikzlibrary{plotmarks}
\usepackage{amsmath}

\begin{document}
\definecolor{mycolor1}{rgb}{0.00000,0.44700,0.74100}%
\definecolor{mycolor2}{rgb}{0.85000,0.32500,0.09800}%
%
\begin{tikzpicture}

\begin{axis}[%
width=2in,
height=2in,
scale only axis,
xmin=-35,
xmax=5,
xlabel={$\lambda$},
ymin=-50,
ymax=350,
ylabel={curvature $\kappa$}
]
\addplot [color=mycolor1,solid,forget plot]
  table[row sep=crcr]{%
-32.8768468297196	-8.29975831331613\\
-32.7526364789922	3.82300018043879\\
-32.6284261282649	-0.0702764740927605\\
-32.5042157775376	0.524024151239461\\
-32.3800054268102	-1.0777224456229\\
-32.2557950760829	0.446792544667484\\
-32.1315847253555	0.110036628719627\\
-32.0073743746282	-0.00493315978158074\\
-31.8831640239008	-0.0425586409287666\\
-31.7589536731735	0.0421049742473888\\
-31.6347433224461	-0.0341014722733605\\
-31.5105329717188	0.00977394407014115\\
-31.3863226209915	0.0173179042540163\\
-31.2621122702641	-0.0163010692936906\\
-31.1379019195368	0.00536221822462906\\
-31.0136915688094	-0.00152009196398775\\
-30.8894812180821	0.00908710483540918\\
-30.7652708673547	-0.00859104869452696\\
-30.6410605166274	0.00770805410139214\\
-30.5168501659001	0.000243818717352702\\
-30.3926398151727	0.00134893383298987\\
-30.2684294644454	0.00693090938925579\\
-30.144219113718	0.000751773605560133\\
-30.0200087629907	0.00128121455174843\\
-29.8957984122633	0.00250414140924679\\
-29.771588061536	-0.00276013937531195\\
-29.6473777108087	0.000457404752209763\\
-29.5231673600813	-0.00165404600284221\\
-29.398957009354	-0.00289643498249165\\
-29.2747466586266	-0.000566135986246061\\
-29.1505363078993	-0.00282395440572955\\
-29.0263259571719	-0.00146629868813422\\
-28.9021156064446	-0.00170454677004465\\
-28.7779052557172	-0.00114605933282156\\
-28.6536949049899	-0.000954877183054126\\
-28.5294845542626	-0.000814731976282984\\
-28.4052742035352	-0.000717906040751984\\
-28.2810638528079	-0.000449960184095657\\
-28.1568535020805	-0.000410785918954812\\
-28.0326431513532	-0.000411749310651137\\
-27.9084328006258	-0.000237097915083997\\
-27.7842224498985	-0.000211251141207145\\
-27.6600120991712	-0.000288552071494925\\
-27.5358017484438	-0.000158674661418349\\
-27.4115913977165	-0.000219703132968224\\
-27.2873810469891	-0.000137138876620456\\
-27.1631706962618	-0.00015042673192358\\
-27.0389603455344	-0.000159319475944462\\
-26.9147499948071	-7.68048807849287e-05\\
-26.7905396440797	-5.74235488346769e-05\\
-26.6663292933524	-2.8719032189468e-05\\
-26.5421189426251	3.85060205360831e-05\\
-26.4179085918977	8.9156599438434e-05\\
-26.2936982411704	0.000121803138240899\\
-26.169487890443	0.000219449723852137\\
-26.0452775397157	0.000294729548918421\\
-25.9210671889883	0.000234342002694492\\
-25.796856838261	0.000254869467201822\\
-25.6726464875337	0.000129316201905229\\
-25.5484361368063	9.04080077300488e-05\\
-25.424225786079	-6.40103196163311e-06\\
-25.3000154353516	-8.70517571301531e-06\\
-25.1758050846243	-7.51140553273425e-05\\
-25.0515947338969	-5.09095337029803e-05\\
-24.9273843831696	-4.05582206059495e-05\\
-24.8031740324422	2.2881493879682e-06\\
-24.6789636817149	5.61831755870303e-05\\
-24.5547533309876	0.000142176683219342\\
-24.4305429802602	0.000265259460380528\\
-24.3063326295329	0.000453988335248723\\
-24.1821222788055	0.000723277258604648\\
-24.0579119280782	0.00102169332992154\\
-23.9337015773508	0.00131723802773276\\
-23.8094912266235	0.00160288862033835\\
-23.6852808758962	0.00174787808599287\\
-23.5610705251688	0.00181368610271251\\
-23.4368601744415	0.00199684444351132\\
-23.3126498237141	0.00216096419802843\\
-23.1884394729868	0.00254731753333603\\
-23.0642291222594	0.00319508054771457\\
-22.9400187715321	0.00421711554729911\\
-22.8158084208047	0.00586118808073022\\
-22.6915980700774	0.00828713912320807\\
-22.5673877193501	0.0115231395447518\\
-22.4431773686227	0.0152040216277968\\
-22.3189670178954	0.0182012963693647\\
-22.194756667168	0.019676792726224\\
-22.0705463164407	0.0190618895993463\\
-21.9463359657133	0.016858609589573\\
-21.822125614986	0.0135627706466608\\
-21.6979152642587	0.00981720836110109\\
-21.5737049135313	0.00611789165082413\\
-21.449494562804	0.00280706012790349\\
-21.3252842120766	8.89778425540067e-05\\
-21.2010738613493	-0.00191748447262724\\
-21.0768635106219	-0.00322775257238475\\
-20.9526531598946	-0.00391939446687207\\
-20.8284428091673	-0.00411767858140216\\
-20.7042324584399	-0.00396296825229674\\
-20.5800221077126	-0.0035874268958669\\
-20.4558117569852	-0.00309727718568823\\
-20.3316014062579	-0.00257403664211491\\
-20.2073910555305	-0.00207222330232614\\
-20.0831807048032	-0.00162699028960432\\
-19.9589703540758	-0.00126177309497657\\
-19.8347600033485	-0.00100072003819507\\
-19.7105496526212	-0.000866551240997705\\
-19.5863393018938	-0.000871070071131516\\
-19.4621289511665	-0.000988039344259177\\
-19.3379186004391	-0.00115829515641403\\
-19.2137082497118	-0.0013130674652801\\
-19.0894978989844	-0.00140472305985305\\
-18.9652875482571	-0.00141655239505435\\
-18.8410771975298	-0.00135569806038887\\
-18.7168668468024	-0.00124205769072795\\
-18.5926564960751	-0.00109812219704305\\
-18.4684461453477	-0.000943613190215167\\
-18.3442357946204	-0.000793269344778228\\
-18.220025443893	-0.000656470932621776\\
-18.0958150931657	-0.000538337997709033\\
-17.9716047424383	-0.000440352506664758\\
-17.847394391711	-0.000361597450516051\\
-17.7231840409837	-0.000299438075548103\\
-17.5989736902563	-0.000250131674757436\\
-17.474763339529	-0.000209452717892592\\
-17.3505529888016	-0.000173198950071248\\
-17.2263426380743	-0.000137754099978965\\
-17.1021322873469	-0.000100497796899085\\
-16.9779219366196	-5.93177154733042e-05\\
-16.8537115858923	-1.21639378465216e-05\\
-16.7295012351649	4.33827561198103e-05\\
-16.6052908844376	0.000108825816302259\\
-16.4810805337102	0.000181027602294773\\
-16.3568701829829	0.000247961858521551\\
-16.2326598322555	0.000290812124717824\\
-16.1084494815282	0.000296562026743423\\
-15.9842391308008	0.00026880579766657\\
-15.8600287800735	0.000224930805567623\\
-15.7358184293462	0.000185043548534246\\
-15.6116080786188	0.000164797692191019\\
-15.4873977278915	0.000174122389049003\\
-15.3631873771641	0.000219945211924857\\
-15.2389770264368	0.000310048147925538\\
-15.1147666757094	0.000455992687408046\\
-14.9905563249821	0.00067376186581987\\
-14.8663459742548	0.000977091929958762\\
-14.7421356235274	0.0013633006445627\\
-14.6179252728001	0.0018021996613024\\
-14.4937149220727	0.00225196970084985\\
-14.3695045713454	0.0026998441262299\\
-14.245294220618	0.00319114338858624\\
-14.1210838698907	0.00383122509847309\\
-13.9968735191633	0.00478405825285964\\
-13.872663168436	0.0062955793782486\\
-13.7484528177087	0.00875886256532269\\
-13.6242424669813	0.0128244993916826\\
-13.500032116254	0.0195533721872866\\
-13.3758217655266	0.0305457566281623\\
-13.2516114147993	0.047866941265035\\
-13.1274010640719	0.0735824921097367\\
-13.0031907133446	0.109268015728995\\
-12.8789803626173	0.15646848551398\\
-12.7547700118899	0.218602663671794\\
-12.6305596611626	0.303595309852779\\
-12.5063493104352	0.426978708766471\\
-12.3821389597079	0.616956723306209\\
-12.2579286089805	0.924236463777742\\
-12.1337182582532	1.44164216563332\\
-12.0095079075259	2.34263488051468\\
-11.8852975567985	3.95696690311567\\
-11.7610872060712	6.91818909309005\\
-11.6368768553438	12.4410191249262\\
-11.5126665046165	22.7876263184681\\
-11.3884561538891	41.7864342559927\\
-11.2642458031618	74.4577341348773\\
-11.1400354524344	123.661967837021\\
-11.0158251017071	184.31200223462\\
-10.8916147509798	243.040770813379\\
-10.7674044002524	286.698398591359\\
-10.6431940495251	308.766384892246\\
-10.5189836987977	308.187080485572\\
-10.3947733480704	286.633674909522\\
-10.270562997343	248.22966904022\\
-10.1463526466157	200.119852877437\\
-10.0221422958884	151.097749970381\\
-9.89793194516101	108.354232637113\\
-9.77372159443367	75.2177728713943\\
-9.64951124370632	51.5077592315383\\
-9.52530089297898	35.288658199252\\
-9.40109054225163	24.3486553803829\\
-9.27688019152429	16.8881719384869\\
-9.15266984079695	11.6688469353213\\
-9.02845949006961	7.93677074156072\\
-8.90424913934226	5.27089966350783\\
-8.78003878861492	3.4203444170108\\
-8.65582843788757	2.19043158829289\\
-8.53161808716023	1.40576888821412\\
-8.40740773643289	0.918416331925987\\
-8.28319738570554	0.618323914895092\\
-8.1589870349782	0.431650981571506\\
-8.03477668425086	0.311903028807641\\
-7.91056633352351	0.230584646334618\\
-7.78635598279617	0.170474627242603\\
-7.66214563206882	0.122061824569847\\
-7.53793528134148	0.0818761958447408\\
-7.41372493061413	0.0502135169292851\\
-7.28951457988679	0.0277138402617268\\
-7.16530422915945	0.0133524715512652\\
-7.04109387843211	0.00490127798651855\\
-6.91688352770476	0.000133411453965069\\
-6.79267317697742	-0.00256089251880215\\
-6.66846282625007	-0.00414407154772515\\
-6.54425247552273	-0.00511283339896469\\
-6.42004212479539	-0.00567921489326047\\
-6.29583177406804	-0.00589869860213356\\
-6.1716214233407	-0.0057552095397578\\
-6.04741107261336	-0.0052210235800307\\
-5.92320072188602	-0.00430099582740221\\
-5.79899037115867	-0.00306113352567787\\
-5.67478002043133	-0.00163625094304556\\
-5.55056966970398	-0.000209743916082137\\
-5.42635931897664	0.00103948189094993\\
-5.3021489682493	0.00201283161199958\\
-5.17793861752195	0.0027476873097717\\
-5.05372826679461	0.00340002425053124\\
-4.92951791606726	0.00415799740336067\\
-4.80530756533992	0.00515282562785474\\
-4.68109721461258	0.00641696078555183\\
-4.55688686388523	0.00788410572828677\\
-4.43267651315789	0.00940219029448388\\
-4.30846616243054	0.0107374836951452\\
-4.1842558117032	0.0115656015553773\\
-4.06004546097586	0.0114907536932062\\
-3.93583511024851	0.0102052420412212\\
-3.81162475952117	0.00780605426676463\\
-3.68741440879383	0.00490758892728126\\
-3.56320405806648	0.00224961595540762\\
-3.43899370733914	0.000240209394745248\\
-3.3147833566118	-0.00108911272209428\\
-3.19057300588445	-0.00187492527825791\\
-3.06636265515711	-0.00225701813550829\\
-2.94215230442976	-0.00232654380188567\\
-2.81794195370242	-0.00213853257203712\\
-2.69373160297508	-0.00174344867484304\\
-2.56952125224774	-0.0012166038864818\\
-2.44531090152039	-0.000676896057149034\\
-2.32110055079305	-0.000288085774571633\\
-2.1968902000657	-0.000229958999450608\\
-2.07267984933836	-0.000628399757584426\\
-1.94846949861102	-0.00147502745787675\\
-1.82425914788367	-0.00262006728728861\\
-1.70004879715633	-0.00386756014576088\\
-1.57583844642899	-0.00507874325878305\\
-1.45162809570165	-0.00619739848899459\\
-1.3274177449743	-0.00721600099448514\\
-1.20320739424695	-0.00813825654417034\\
-1.07899704351961	-0.00896290393386348\\
-0.954786692792264	-0.00968525885213957\\
-0.830576342064923	-0.0103046880567307\\
-0.706365991337581	-0.0108280907290718\\
-0.582155640610236	-0.0112656182199255\\
-0.457945289882891	-0.0116237640694796\\
-0.333734939155546	-0.0119082199619885\\
-0.209524588428208	-0.0121470518908246\\
-0.0853142377008631	-0.0124319025716036\\
0.0388961130264822	-0.0129592862416521\\
0.163106463753827	-0.0140440667888965\\
0.287316814481173	-0.0160763265575433\\
0.41152716520851	-0.0194141546497459\\
0.535737515935855	-0.0242652943077104\\
0.659947866663201	-0.0306640518654461\\
0.784158217390546	-0.0385905812024481\\
0.908368568117883	-0.0481400604260188\\
1.03257891884523	-0.0596177962988221\\
1.15678926957257	-0.0735382533924824\\
1.28099962029992	-0.0905843407896128\\
1.40520997102726	-0.111576091330434\\
1.5294203217546	-0.137462541602393\\
1.65363067248195	-0.169331234129259\\
1.77784102320929	-0.208424909550457\\
1.90205137393664	-0.256148598509915\\
2.02626172466398	-0.314031396676071\\
2.15047207539132	-0.383575930865921\\
2.27468242611867	-0.465900846056364\\
2.39889277684601	-0.561090396165918\\
2.52310312757336	-0.667250818391942\\
2.6473134783007	-0.77946264014799\\
2.77152382902804	-0.889091109499739\\
2.89573417975539	-0.984141412183383\\
3.01994453048273	-1.05122848951654\\
3.14415488121007	-1.07901150504034\\
3.26836523193742	-1.06187224730254\\
3.39257558266476	-1.00205294478667\\
3.51678593339211	-0.909059511106396\\
3.64099628411945	-0.796572847838445\\
3.76520663484679	-0.678413776028322\\
3.88941698557413	-0.565393668706666\\
4.01362733630148	-0.46403987686275\\
};
\addplot [color=mycolor1,only marks,mark=asterisk,mark options={solid},forget plot]
  table[row sep=crcr]{%
-10.6431940495251	308.766384892246\\
};
\end{axis}
\end{tikzpicture}%
\end{document}
\caption{Tikhonov regularization, L-curve and curvature for the matrix $\mathbf{A}_{5}$ and vector $\mathbf{b}_{err5}$.}
\label{fig:A5Tikh}
\end{figure}
\begin{figure}
% This file was created by matlab2tikz.
% Minimal pgfplots version: 1.3
%
%The latest updates can be retrieved from
%  http://www.mathworks.com/matlabcentral/fileexchange/22022-matlab2tikz
%where you can also make suggestions and rate matlab2tikz.
%
\documentclass[tikz]{standalone}
\usepackage{pgfplots}
\usepackage{grffile}
\pgfplotsset{compat=newest}
\usetikzlibrary{plotmarks}
\usepackage{amsmath}

\begin{document}
\definecolor{mycolor1}{rgb}{0.00000,0.44700,0.74100}%
\definecolor{mycolor2}{rgb}{0.85000,0.32500,0.09800}%
%
\begin{tikzpicture}

\begin{axis}[%
width=2in,
height=2in,
scale only axis,
xmode=log,
xmin=1e-05,
xmax=10,
xminorticks=true,
xlabel={$\|\mathbf{Ax} - \mathbf{b}\|$},
ymode=log,
ymin=1,
ymax=100,
yminorticks=true,
ylabel={$\|\mathbf{Lx}\|$}
]
\addplot [color=mycolor1,solid,forget plot]
  table[row sep=crcr]{%
5.66830450624692e-05	28.6848005629667\\
5.80610399133609e-05	27.437410993754\\
5.94401650470529e-05	26.2296730165903\\
6.08183946416106e-05	25.0620561754006\\
6.21938470374061e-05	23.9348312752615\\
6.35647832808861e-05	22.8480863054564\\
6.49296016213285e-05	21.8017447602702\\
6.62868291530273e-05	20.7955850692487\\
6.76351119476439e-05	19.8292599177824\\
6.89732049805943e-05	18.9023144136506\\
7.02999630271125e-05	18.0142023078326\\
7.16143334442141e-05	17.1642997719588\\
7.2915351447859e-05	16.3519165319768\\
7.42021381831405e-05	15.5763044243951\\
7.54739015322745e-05	14.8366636539015\\
7.67299393910082e-05	14.1321471771722\\
7.79696449086945e-05	13.4618637168912\\
7.91925130657033e-05	12.8248799310774\\
8.03981479051632e-05	12.2202222397078\\
8.15862696969899e-05	11.6468787583269\\
8.27567213134359e-05	11.1038017194138\\
8.39094731446306e-05	10.5899106847974\\
8.50446258862504e-05	10.1040967695963\\
8.61624105977991e-05	9.6452280095918\\
8.72631855352674e-05	9.21215590790047\\
8.83474293191873e-05	8.80372309274132\\
8.94157302499918e-05	8.41877190874055\\
9.0468771725704e-05	8.05615365644142\\
9.15073140355684e-05	7.71473809903005\\
9.25321730812473e-05	7.39342278413208\\
9.35441968875941e-05	7.09114169356715\\
9.45442410198119e-05	6.80687274348678\\
9.55331442319056e-05	6.53964371427937\\
9.65117058060013e-05	6.2885362905681\\
9.74806659697595e-05	6.05268802708884\\
9.84406906984617e-05	5.83129221213767\\
9.93923619008944e-05	5.62359576002203\\
0.000100336173628289	5.42889541072136\\
0.000101272534494354	5.2465326338288\\
0.000102201776014654	5.07588771339703\\
0.000103124166102156	4.91637352374608\\
0.000104039926532712	4.76742949195064\\
0.000104949252883758	4.62851618408326\\
0.000105852335247245	4.4991108574579\\
0.000106749377970019	4.37870420182185\\
0.000107640616742738	4.26679836257133\\
0.000108526331590509	4.16290621289974\\
0.000109406854599911	4.06655173215508\\
0.000110282571659615	3.97727126436822\\
0.000111153917842753	3.89461537957937\\
0.000112021366562169	3.81815104245191\\
0.000112885412990153	3.74746380489466\\
0.000113746552650568	3.68215977616067\\
0.000114605256370216	3.62186717758271\\
0.000115461943061853	3.56623735178429\\
0.000116316951948974	3.51494516058541\\
0.000117170515969607	3.46768876597093\\
0.000118022738031347	3.42418884010291\\
0.000118873571736979	3.38418729072542\\
0.00011972280791262	3.34744561610442\\
0.000120570068012363	3.31374301865677\\
0.00012141480501316	3.28287440932273\\
0.000122256311985695	3.25464842689792\\
0.000123093737988116	3.22888557989741\\
0.000123926110500488	3.20541659544486\\
0.000124752363152126	3.18408103280008\\
0.000125571367191253	3.16472619117378\\
0.000126381964948197	3.14720631497743\\
0.00012718300347205	3.13138207677738\\
0.000127973366625105	3.11712030052652\\
0.000128752004104991	3.10429387598707\\
0.000129517956186711	3.09278180974868\\
0.000130270373310633	3.08246935834755\\
0.000131008530090327	3.07324819366752\\
0.000131731833608046	3.06501655872395\\
0.000132439826262864	3.05767938169283\\
0.000133132183676925	3.05114832635181\\
0.000133808708391955	3.04534176689256\\
0.000134469320185151	3.0401846835955\\
0.000135114043935324	3.03560848269871\\
0.000135742995962133	3.03155074879243\\
0.000136356369708053	3.02795494129535\\
0.000136954421554517	3.02477004822663\\
0.000137537457453106	3.02195021086347\\
0.000138105820881355	3.01945433227167\\
0.000138659882510331	3.01724568140745\\
0.000139200031750576	3.0152915027674\\
0.000139726670222816	3.01356263962324\\
0.000140240207009421	3.01203317688774\\
0.000140741055434473	3.01068010775084\\
0.000141229630997701	3.00948302649515\\
0.000141706350078988	3.00842384841617\\
0.000142171628976409	3.00748655657252\\
0.000142625882925701	3.00665697419084\\
0.00014306952479783	3.00592256094044\\
0.000143502963324675	3.00527223094867\\
0.000143926600791487	3.00469619030899\\
0.00014434083029858	3.00418579189009\\
0.000144746032784064	3.00373340543375\\
0.000145142574106809	3.00333230117944\\
0.000145530802505094	3.00297654552986\\
0.000145911046747406	3.0026609075369\\
0.000146283615275949	3.00238077521637\\
0.000146648796465904	3.00213208087677\\
0.000147006860110044	3.00191123476875\\
0.000147358060011664	3.00171506643061\\
0.000147702637488367	3.00154077313163\\
0.000148040825461501	3.00138587481318\\
0.000148372852748353	3.00124817491073\\
0.000148698948108106	3.00112572642355\\
0.000149019343675208	3.00101680259153\\
0.000149334277414709	3.00091987154911\\
0.000149643994394814	3.00083357435605\\
0.000149948746779142	3.00075670585385\\
0.000150248792593718	3.00068819786116\\
0.000150544393467295	3.00062710429529\\
0.000150835811661061	3.00057258788475\\
0.000151123306795499	3.00052390821256\\
0.000151407132721339	3.00048041089795\\
0.000151687535005929	3.00044151778051\\
0.000151964749445092	3.00040671801515\\
0.00015223900194214	3.0003755600168\\
0.000152510510010597	3.00034764421285\\
0.000152779485989214	3.00032261657006\\
0.000153046141953333	3.00030016286323\\
0.000153310696193403	3.00028000364912\\
0.000153573380952072	3.00026188990175\\
0.000153834451088788	3.00024559925789\\
0.000154094193207463	3.00023093281465\\
0.000154352934763861	3.00021771241646\\
0.000154611052662074	3.00020577836617\\
0.000154868980835379	3.00019498749554\\
0.000155127216382948	3.00018521153246\\
0.000155386323917653	3.00017633570698\\
0.000155646937839777	3.00016825754393\\
0.000155909762472485	3.00016088579669\\
0.000156175570112314	3.00015413948404\\
0.000156445197258309	3.00014794699948\\
0.000156719539546753	3.00014224526973\\
0.000156999546085209	3.00013697894588\\
0.000157286214195337	3.00013209961733\\
0.000157580585736592	3.0001275650438\\
0.000157883746444929	3.00012333840574\\
0.0001581968298363	3.00011938757697\\
0.000158521027355906	3.00011568442618\\
0.000158857606541956	3.00011220415538\\
0.000159207938923838	3.0001089246838\\
0.000159573539395707	3.00010582608536\\
0.00015995611872295	3.00010289008599\\
0.0001603576508324	3.00010009962512\\
0.000160780456529305	3.00009743848298\\
0.000161227305404863	3.00009489097221\\
0.000161701537913063	3.00009244169017\\
0.000162207209904884	3.00009007532551\\
0.000162749262384749	3.00008777651143\\
0.000163333719802228	3.00008552971659\\
0.000163967920764843	3.00008331916499\\
0.000164660785668436	3.00008112877597\\
0.000165423126217006	3.00007894211669\\
0.000166268002109813	3.00007674236052\\
0.000167211130141714	3.00007451224579\\
0.000168271350523777	3.00007223403095\\
0.000169471154140242	3.00006988944281\\
0.000170837272560502	3.00006745961577\\
0.000172401329750005	3.00006492502027\\
0.000174200550358056	3.00006226537935\\
0.000176278514144559	3.00005945957256\\
0.000178685939587499	3.00005648552642\\
0.000181481472491097	3.00005332009088\\
0.000184732448164201	3.00004993890139\\
0.000188515589949071	3.00004631622582\\
0.000192917604299047	3.00004242479566\\
0.000198035635461551	3.00003823562045\\
0.000203977552977231	3.0000337177845\\
0.000210862063558474	3.00002883822427\\
0.000218818664914995	3.00002356148486\\
0.000227987489303435	3.00001784945399\\
0.000238519113848894	3.00001166107154\\
0.000250574436655982	3.00000495201337\\
0.00026432472629102	2.99999767434812\\
0.000279951945034705	2.9999897761664\\
0.000297649423866396	2.99998120118201\\
0.000317622935270453	2.99997188830496\\
0.00034009217557917	2.99996177118617\\
0.000365292639678181	2.99995077773258\\
0.000393477852365606	2.99993882959086\\
0.000424921914832854	2.99992584159539\\
0.000459922330040477	2.99991172117509\\
0.000498803084230528	2.99989636771062\\
0.000541917978583492	2.99987967183244\\
0.000589654221696342	2.99986151464843\\
0.000642436306450276	2.99984176688998\\
0.000700730202549626	2.99982028796617\\
0.000765047897723708	2.99979692491837\\
0.00083595231693417	2.999771511271\\
0.000914062641520121	2.99974386577849\\
0.00100006004117187	2.9997137910734\\
0.0010946938237344	2.99968107222431\\
0.00119878800430606	2.99964547521484\\
0.00131324829860864	2.99960674535468\\
0.00143906955891889	2.9995646056304\\
0.00157734369577604	2.99951875499604\\
0.00172926816586876	2.99946886659171\\
0.00189615515569363	2.99941458586277\\
0.00207944164976812	2.99935552853214\\
0.00228070063884521	2.99929127835678\\
0.00250165379348439	2.99922138457538\\
0.00274418599728653	2.99914535893192\\
0.00301036219706914	2.99906267213831\\
0.00330244707939141	2.99897274962262\\
0.00362292811911281	2.99887496639779\\
0.00397454256164662	2.99876864088171\\
0.00436030889174566	2.99865302750334\\
0.00478356330534589	2.99852730794339\\
0.00524800163479409	2.99839058088048\\
0.00575772708292962	2.99824185014518\\
0.0063173040014093	2.99808001122274\\
0.00693181781108728	2.99790383608807\\
0.00760694101998751	2.99771195639987\\
0.0083490051627222	2.9975028451208\\
0.00916507838389948	2.99727479666204\\
0.0100630483363283	2.99702590566968\\
0.0110517100790314	2.99675404457376\\
0.0121408587517985	2.99645684000662\\
0.0133413869727949	2.99613164816584\\
0.0146653871453059	2.99577552915157\\
0.0161262591480482	2.99538522025207\\
0.0177388241932234	2.99495710809217\\
0.0195194459347798	2.99448719950195\\
0.0214861601658233	2.99397109091507\\
0.02365881463511	2.99340393607101\\
0.0260592206253905	2.99278041177648\\
0.0287113179715902	2.99209468147846\\
0.0316413551672899	2.99134035641205\\
0.0348780861372188	2.99051045410653\\
0.0384529851705215	2.9895973540565\\
0.0424004814467357	2.98859275038524\\
0.0467582145736541	2.98748760133624\\
0.0515673126202381	2.9862720754203\\
0.0568726942893097	2.98493549401243\\
0.0627233971474374	2.98346627013004\\
0.0691729342196584	2.98185184302792\\
0.0762796817615915	2.98007860811463\\
0.084107301626507	2.97813184153167\\
0.0927252023219804	2.97599561854618\\
0.102209043552364	2.97365272470239\\
0.112641289699807	2.97108455847484\\
0.124111818211899	2.96827102399291\\
0.136718589113894	2.96519041229657\\
0.150568381698174	2.9618192695778\\
0.165777603695645	2.95813225100741\\
0.182473176732425	2.95410195908905\\
0.200793499469426	2.949698766062\\
0.220889486411073	2.94489062071819\\
0.242925675932375	2.93964284111186\\
0.267081395707049	2.93391789599562\\
0.293551967655208	2.92767517935312\\
0.322549928143392	2.92087078401734\\
0.354306232960625	2.91345728193431\\
0.389071411135541	2.90538352000563\\
0.427116627523233	2.89659444146723\\
0.46873461175288	2.88703094330591\\
0.514240410873899	2.8766297801891\\
0.56397192488321	2.86532352475913\\
0.618290187959317	2.8530405929623\\
0.677579363059902	2.83970534146295\\
0.742246422689473	2.82523824231234\\
0.812720493076894	2.80955613813405\\
0.889451841632917	2.7925725794091\\
0.972910487366126	2.77419824425926\\
1.06358441008976	2.7543414406781\\
1.16197732621141	2.73290869164102\\
1.26860598653025	2.70980540505377\\
1.38399693517014	2.68493663307774\\
1.50868264962002	2.65820792885029\\
1.64319696171995	2.62952631266595\\
1.78806964111834	2.59880136374848\\
1.94382000996052	2.56594645706492\\
2.11094945483759	2.53088016627256\\
2.28993271420178	2.49352785282493\\
2.48120785117196	2.45382345651933\\
2.6851648764776	2.4117114935974\\
2.90213306580767	2.36714925457898\\
3.13236711881777	2.32010917556089\\
3.37603242891656	2.27058133466962\\
3.63318986566364	2.21857600132021\\
3.90378060414722	2.16412614207571\\
4.1876116550856	2.10728976574166\\
4.48434284207768	2.04815197445178\\
4.79347602591589	1.9868265792263\\
5.11434738018668	1.92345713958828\\
5.44612347120586	1.85821729831659\\
5.78780178670923	1.79131030442365\\
6.13821619425856	1.72296764920197\\
6.49604759905761	1.6534467800853\\
6.8598398228337	1.58302790280854\\
7.2280204551117	1.51200993106273\\
7.59892615285026	1.44070569127511\\
7.97083160318665	1.36943653485033\\
8.34198113682757	1.29852654776429\\
};
\addplot [color=mycolor1,only marks,mark=asterisk,mark options={solid},forget plot]
  table[row sep=crcr]{%
0.000131008530090327	3.07324819366752\\
};
\end{axis}
\end{tikzpicture}%
\end{document}
% This file was created by matlab2tikz.
% Minimal pgfplots version: 1.3
%
%The latest updates can be retrieved from
%  http://www.mathworks.com/matlabcentral/fileexchange/22022-matlab2tikz
%where you can also make suggestions and rate matlab2tikz.
%
\documentclass[tikz]{standalone}
\usepackage{pgfplots}
\usepackage{grffile}
\pgfplotsset{compat=newest}
\usetikzlibrary{plotmarks}
\usepackage{amsmath}

\begin{document}
\definecolor{mycolor1}{rgb}{0.00000,0.44700,0.74100}%
\definecolor{mycolor2}{rgb}{0.85000,0.32500,0.09800}%
%
\begin{tikzpicture}

\begin{axis}[%
width=2in,
height=2in,
at={(1.5in,0.48125in)},
scale only axis,
xmin=-14,
xmax=2,
xlabel={$\lambda$},
ymin=-2,
ymax=10,
ylabel={curvature $\kappa$}
]
\addplot [color=mycolor1,solid,forget plot]
  table[row sep=crcr]{%
-13.4334336820871	-0.553353389219459\\
-13.3826250073059	-0.539144179129227\\
-13.3318163325246	-0.523890448952891\\
-13.2810076577434	-0.507810481400978\\
-13.2301989829621	-0.491094758376784\\
-13.1793903081809	-0.473904935866607\\
-13.1285816333996	-0.456373998903415\\
-13.0777729586183	-0.438607966961896\\
-13.0269642838371	-0.420688414216151\\
-12.9761556090558	-0.402675735568651\\
-12.9253469342746	-0.384612629870979\\
-12.8745382594933	-0.366527335990671\\
-12.8237295847121	-0.348436923518377\\
-12.7729209099308	-0.330349640434022\\
-12.7221122351496	-0.312267037727438\\
-12.6713035603683	-0.294185439547397\\
-12.620494885587	-0.276096768756489\\
-12.5696862108058	-0.2579890399712\\
-12.5188775360245	-0.239846489491995\\
-12.4680688612433	-0.221649132737198\\
-12.417260186462	-0.203372286584923\\
-12.3664515116808	-0.184985646167043\\
-12.3156428368995	-0.166451766958436\\
-12.2648341621182	-0.147724667303734\\
-12.214025487337	-0.128747189944612\\
-12.1632168125557	-0.109448499588126\\
-12.1124081377745	-0.0897405018526104\\
-12.0615994629932	-0.069513796201092\\
-12.010790788212	-0.0486328997794817\\
-11.9599821134307	-0.0269310386856314\\
-11.9091734386494	-0.00420441815155189\\
-11.8583647638682	0.0197941887880605\\
-11.8075560890869	0.0453614997688203\\
-11.7567474143057	0.0728506494750886\\
-11.7059387395244	0.102677880708185\\
-11.6551300647432	0.135329618109324\\
-11.6043213899619	0.171369830014795\\
-11.5535127151806	0.211447840854087\\
-11.5027040403994	0.256306616130949\\
-11.4518953656181	0.306791354644001\\
-11.4010866908369	0.3638584312973\\
-11.3502780160556	0.428584073561401\\
-11.2994693412744	0.502172403552592\\
-11.2486606664931	0.585961589778618\\
-11.1978519917118	0.681427748602038\\
-11.1470433169306	0.790184332530579\\
-11.0962346421493	0.913977586840608\\
-11.0454259673681	1.05467370619503\\
-10.9946172925868	1.21423946127786\\
-10.9438086178056	1.39471156870595\\
-10.8929999430243	1.5981554056724\\
-10.842191268243	1.82660834146201\\
-10.7913825934618	2.08200737763355\\
-10.7405739186805	2.36609549872943\\
-10.6897652438993	2.68030671805696\\
-10.638956569118	3.02562292916661\\
-10.5881478943368	3.40240731428872\\
-10.5373392195555	3.81020999825282\\
-10.4865305447742	4.24756113192096\\
-10.435721869993	4.71175942428521\\
-10.3849131952117	5.19868579139346\\
-10.3341045204305	5.70266735477351\\
-10.2832958456492	6.21643665644691\\
-10.232487170868	6.73121446998822\\
-10.1816784960867	7.23694977800106\\
-10.1308698213054	7.72272479789417\\
-10.0800611465242	8.17730642679272\\
-10.0292524717429	8.5898024651881\\
-9.97844379696168	8.95034591776004\\
-9.92763512218042	9.25072819362358\\
-9.87682644739916	9.48489314167159\\
-9.82601777261791	9.64924587433486\\
-9.77520909783665	9.74272659657442\\
-9.72440042305539	9.76668139685268\\
-9.67359174827413	9.72455383472135\\
-9.62278307349288	9.62146518544728\\
-9.57197439871162	9.463736671695\\
-9.52116572393036	9.25842300246216\\
-9.47035704914911	9.01289019709702\\
-9.41954837436785	8.73446392457975\\
-9.36873969958659	8.43016100690852\\
-9.31793102480534	8.10650180016572\\
-9.26712235002408	7.7693879737907\\
-9.21631367524282	7.42404406654804\\
-9.16550500046156	7.07499431464829\\
-9.11469632568031	6.72607946235536\\
-9.06388765089905	6.38048973355412\\
-9.01307897611779	6.04081604393718\\
-8.96227030133654	5.70910624091257\\
-8.91146162655528	5.38693259503715\\
-8.86065295177402	5.07545390481753\\
-8.80984427699277	4.77548299215041\\
-8.75903560221151	4.48754658978133\\
-8.70822692743025	4.21194551455492\\
-8.657418252649	3.94880329450905\\
-8.60660957786774	3.69811149447712\\
-8.55580090308648	3.45976395085426\\
-8.50499222830523	3.23358524482764\\
-8.45418355352397	3.01934850599578\\
-8.40337487874271	2.8167872152762\\
-8.35256620396146	2.6256043853517\\
-8.3017575291802	2.44547052664623\\
-8.25094885439894	2.27602890085632\\
-8.20014017961768	2.11689234624509\\
-8.14933150483643	1.96764530233192\\
-8.09852283005517	1.82784598842045\\
-8.04771415527391	1.69703164638552\\
-7.99690548049266	1.57472345953771\\
-7.9460968057114	1.46043672689682\\
-7.89528813093014	1.35368778442286\\
-7.84447945614889	1.25400410384861\\
-7.79367078136763	1.16093197053167\\
-7.74286210658637	1.07404312852401\\
-7.69205343180511	0.992939508186773\\
-7.64124475702386	0.91725551914026\\
-7.5904360822426	0.84665814023797\\
-7.53962740746134	0.780844616385999\\
-7.48881873268009	0.719539194400502\\
-7.43801005789883	0.662488005562325\\
-7.38720138311757	0.609453894909309\\
-7.33639270833632	0.560211949628504\\
-7.28558403355506	0.514544614418549\\
-7.2347753587738	0.472238762889847\\
-7.18396668399255	0.433084235222102\\
-7.13315800921129	0.396872405385872\\
-7.08234933443003	0.363397454241482\\
-7.03154065964878	0.332457338163478\\
-6.98073198486752	0.303855968850689\\
-6.92992331008626	0.277405829847981\\
-6.879114635305	0.252930258819842\\
-6.82830596052375	0.230266017858392\\
-6.77749728574249	0.209265426879899\\
-6.72668861096123	0.18979739260536\\
-6.67587993617998	0.171748473112151\\
-6.62507126139872	0.155022403503632\\
-6.57426258661746	0.1395388125847\\
-6.52345391183621	0.125231261630174\\
-6.47264523705495	0.11204403943335\\
-6.42183656227369	0.0999289763277343\\
-6.37102788749243	0.0888417797588125\\
-6.32021921271118	0.0787389490659472\\
-6.26941053792992	0.0695751446493514\\
-6.21860186314866	0.0613015432960606\\
-6.16779318836741	0.0538652192911123\\
-6.11698451358615	0.0472092909090969\\
-6.06617583880489	0.0412739106443864\\
-6.01536716402364	0.0359976376035249\\
-5.96455848924238	0.0313190583946929\\
-5.91374981446112	0.027178357272408\\
-5.86294113967987	0.0235187197287401\\
-5.81213246489861	0.020287446159635\\
-5.76132379011735	0.0174366983463495\\
-5.7105151153361	0.0149239163769617\\
-5.65970644055484	0.0127118817862591\\
-5.60889776577358	0.0107684634237106\\
-5.55808909099232	0.00906611105310979\\
-5.50728041621107	0.0075811402675792\\
-5.45647174142981	0.00629290738344581\\
-5.40566306664855	0.00518296635364988\\
-5.3548543918673	0.00423431896955415\\
-5.30404571708604	0.00343083866897401\\
-5.25323704230478	0.00275692221864844\\
-5.20242836752353	0.0021973743887887\\
-5.15161969274227	0.00173748267165355\\
-5.10081101796101	0.00136321508261857\\
-5.05000234317975	0.00106146063919477\\
-4.9991936683985	0.000820248917254187\\
-4.94838499361724	0.000628904363525933\\
-4.89757631883598	0.000478117835764924\\
-4.84676764405473	0.000359938086827539\\
-4.79595896927347	0.000267698793927374\\
-4.74515029449221	0.000195901238585892\\
-4.69434161971096	0.000140073251521836\\
-4.6435329449297	9.66209734618039e-05\\
-4.59272427014844	6.26852252520796e-05\\
-4.54191559536719	3.60101127968258e-05\\
-4.49110692058593	1.48275116334431e-05\\
-4.44029824580467	-2.2413420343116e-06\\
-4.38948957102342	-1.6267805925667e-05\\
-4.33868089624216	-2.80827815180467e-05\\
-4.2878722214609	-3.83306585761161e-05\\
-4.23706354667964	-4.75122410291857e-05\\
-4.18625487189839	-5.60186301098618e-05\\
-4.13544619711713	-6.41577294118681e-05\\
-4.08463752233587	-7.21749043065518e-05\\
-4.03382884755462	-8.0268997320295e-05\\
-3.98302017277336	-8.86047272982792e-05\\
-3.9322114979921	-9.73223122734463e-05\\
-3.88140282321085	-0.000106544917194559\\
-3.83059414842959	-0.000116384491145077\\
-3.77978547364833	-0.000126946333250733\\
-3.72897679886708	-0.000138332707133797\\
-3.67816812408582	-0.000150645707345388\\
-3.62735944930456	-0.000163989541232288\\
-3.5765507745233	-0.000178472332798149\\
-3.52574209974205	-0.000194207554432315\\
-3.47493342496079	-0.000211315143290541\\
-3.42412475017953	-0.000229922400042691\\
-3.37331607539828	-0.00025016473379283\\
-3.32250740061702	-0.000272186362960384\\
-3.27169872583576	-0.000296141074919888\\
-3.22089005105451	-0.000322193173643077\\
-3.17008137627325	-0.000350518747581574\\
-3.11927270149199	-0.000381307396671857\\
-3.06846402671074	-0.000414764560000866\\
-3.01765535192948	-0.000451114565494307\\
-2.96684667714822	-0.000490604514802304\\
-2.91603800236696	-0.000533509084018687\\
-2.86522932758571	-0.000580136280605663\\
-2.81442065280445	-0.000630834149038627\\
-2.76361197802319	-0.000685998367091691\\
-2.71280330324194	-0.00074608058716143\\
-2.66199462846068	-0.00081159733096838\\
-2.61118595367942	-0.000883139180156811\\
-2.56037727889817	-0.000961379956848636\\
-2.50956860411691	-0.00104708560365782\\
-2.45875992933565	-0.00114112249285855\\
-2.4079512545544	-0.00124446500256553\\
-2.35714257977314	-0.00135820230657908\\
-2.30633390499188	-0.00148354452329803\\
-2.25552523021063	-0.00162182849624526\\
-2.20471655542937	-0.00177452365740804\\
-2.15390788064811	-0.00194323848104835\\
-2.10309920586685	-0.00212972806266534\\
-2.0522905310856	-0.00233590327836675\\
-2.00148185630434	-0.0025638418403706\\
-1.95067318152308	-0.00281580140032229\\
-1.89986450674183	-0.00309423468158384\\
-1.84905583196057	-0.00340180648335286\\
-1.79824715717931	-0.00374141233802021\\
-1.74743848239805	-0.00411619860708634\\
-1.6966298076168	-0.00452958383730679\\
-1.64582113283554	-0.00498528135030294\\
-1.59501245805428	-0.00548732314030353\\
-1.54420378327303	-0.00604008534043612\\
-1.49339510849177	-0.0066483156737113\\
-1.44258643371051	-0.00731716348947166\\
-1.39177775892926	-0.00805221316504709\\
-1.340969084148	-0.0088595218692187\\
-1.29016040936674	-0.00974566291894544\\
-1.23935173458549	-0.0107177761900381\\
-1.18854305980423	-0.0117836273271665\\
-1.13773438502297	-0.0129516777401553\\
-1.08692571024172	-0.0142311675895215\\
-1.03611703546046	-0.0156322140980953\\
-0.985308360679203	-0.0171659275607032\\
-0.934499685897946	-0.0188445472264905\\
-0.883691011116686	-0.020681598900905\\
-0.832882336335431	-0.0226920754543358\\
-0.782073661554174	-0.0248926406074954\\
-0.731264986772916	-0.0273018552998441\\
-0.680456311991659	-0.0299404248541863\\
-0.629647637210404	-0.0328314640985676\\
-0.578838962429147	-0.0360007769010608\\
-0.528030287647889	-0.0394771463526897\\
-0.477221612866632	-0.0432926324086034\\
-0.426412938085377	-0.0474828752317974\\
-0.375604263304119	-0.052087404805693\\
-0.324795588522862	-0.0571499605007273\\
-0.273986913741605	-0.0627188277475868\\
-0.223178238960349	-0.068847202481552\\
-0.172369564179092	-0.075593596967798\\
-0.121560889397835	-0.0830223025608338\\
-0.0707522146165774	-0.0912039255404957\\
-0.0199435398353222	-0.100216011105026\\
0.0308651349459353	-0.110143767742155\\
0.0816738097271946	-0.121080899386065\\
0.132482484508452	-0.133130545582723\\
0.183291159289707	-0.146406319833265\\
0.234099834070964	-0.161033422308671\\
0.284908508852222	-0.177149784008648\\
0.335717183633479	-0.194907173455491\\
0.386525858414734	-0.214472162400989\\
0.437334533195992	-0.236026801700196\\
0.488143207977249	-0.25976880061366\\
0.538951882758506	-0.285910930438865\\
0.589760557539762	-0.314679285353749\\
0.640569232321019	-0.346309929317206\\
0.691377907102276	-0.381043339842902\\
0.742186581883533	-0.419115933450224\\
0.792995256664789	-0.460747837218487\\
0.843803931446046	-0.506125981896118\\
0.894612606227303	-0.555381578906287\\
0.945421281008561	-0.608561176109287\\
0.996229955789818	-0.66559086489703\\
1.04703863057108	-0.726233961103818\\
1.09784730535233	-0.790043739507185\\
1.14865598013359	-0.856314665364494\\
1.19946465491485	-0.924038017294307\\
1.2502733296961	-0.991870579634787\\
1.30108200447736	-1.05812758555352\\
1.35189067925862	-1.1208122776793\\
1.40269935403987	-1.17769297672861\\
1.45350802882113	-1.22643316865332\\
1.50431670360239	-1.26477042304826\\
1.55512537838364	-1.29072716842276\\
1.6059340531649	-1.3028237422238\\
1.65674272794616	-1.30025639389508\\
};
\addplot [color=mycolor1,only marks,mark=asterisk,mark options={solid},forget plot]
  table[row sep=crcr]{%
-9.72440042305539	9.76668139685268\\
};
\end{axis}
\end{tikzpicture}%
\end{document}
\caption{Tikhonov regularization, L-curve and curvature for the matrix $\mathbf{A}_{6}$ and vector $\mathbf{b}_{err6}$.}
\label{fig:A6Tikh}
\end{figure}

Figure~\ref{fig:A2Tikh} shows a more dubious case. For the given matrix $\mathbf{A}_2$ the L-curve displays two edges. The selection criterion now chooses the edge at the bottom as it is curvier then the previous one. However maybe the emphasis should have been to reduce the residual instead of the solution norm. In that case the previous edge should have been selected. This example shows that selection by curvature is not free of flaws, it tends to oversmooth the solution in this case. Figures~\ref{fig:A4Tikh}, \ref{fig:A5Tikh} and \ref{fig:A6Tikh} show more examples where the L-curve criterion performs very well. However there is an absentee. Matrix $\mathbf{A}_{3}$ is missing. Here the L-curve's edged is not found exactly by the curvature based selection criterion, this example will be solved using GCV, which as explained in the next section.

\subsection{Generalized Cross validation (GCV)}
\begin{figure}
\centering
% This file was created by matlab2tikz.
% Minimal pgfplots version: 1.3
%
%The latest updates can be retrieved from
%  http://www.mathworks.com/matlabcentral/fileexchange/22022-matlab2tikz
%where you can also make suggestions and rate matlab2tikz.
%
\documentclass[tikz]{standalone}
\usepackage{pgfplots}
\usepackage{grffile}
\pgfplotsset{compat=newest}
\usetikzlibrary{plotmarks}
\usepackage{amsmath}

\begin{document}
\definecolor{mycolor1}{rgb}{0.00000,0.44700,0.74100}%
\definecolor{mycolor2}{rgb}{0.85000,0.32500,0.09800}%
%
\begin{tikzpicture}

\begin{axis}[%
width=2in,
height=2in,
scale only axis,
xmode=log,
xmin=1e-20,
xmax=100000,
xlabel={$\lambda$},
xminorticks=true,
ymode=log,
ymin=1e-09,
ymax=0.01,
yminorticks=true,
ylabel={GCV function value}
]
\addplot [color=mycolor1,solid,forget plot]
  table[row sep=crcr]{%
2.99330373246548	0.00533806328545414\\
2.87967510916424	0.004999194097946\\
2.77035993521106	0.0046725259968239\\
2.66519446801416	0.00435881520025981\\
2.56402118080449	0.00405866278101949\\
2.46668852667718	0.00377251298839123\\
2.37305071159036	0.00350065535063742\\
2.28296747598101	0.00324323028796049\\
2.19630388467097	0.00300023783794606\\
2.11293012474826	0.00277154899573035\\
2.03272131112108	0.00255691910363861\\
1.9555572994531	0.00235600269099817\\
1.88132250619992	0.00216836916300219\\
1.80990573547714	0.00199351876456119\\
1.74120001250064	0.00183089829616156\\
1.67510242334968	0.00167991612805559\\
1.61151396081266	0.00153995614050783\\
1.55033937608482	0.00141039030537142\\
1.49148703609555	0.001290589712609\\
1.43486878625173	0.0011799339299961\\
1.38039981839145	0.00107781866162554\\
1.32799854375037	0.000983661738446427\\
1.27758647075031	0.000896907530336707\\
1.22908808742718	0.000817029913366427\\
1.18243074832205	0.000743533957886744\\
1.13754456566587	0.000675956523353247\\
1.09436230469499	0.000613865955254965\\
1.05281928294062	0.00055686107937827\\
1.01285327334125	0.000504569680312316\\
0.974404411033138	0.000456646636166824\\
0.937415103679035	0.000412771861563968\\
0.901829945200947	0.000372648187731106\\
0.867595632787682	0.000335999283549929\\
0.834660887052864	0.000302567696179594\\
0.802976375223826	0.000272113065656833\\
0.77249463724632	0.000244410545747286\\
0.743170014694372	0.000219249444078492\\
0.714958582378772	0.00019643207874516\\
0.687818082551783	0.000175772836383055\\
0.661707861609492	0.000157097408145033\\
0.636588809197005	0.000140242174850209\\
0.612423299625261	0.000125053710415847\\
0.589175135511724	0.000111388373010271\\
0.566809493560523	9.91119556159213e-05\\
0.54529287240083	8.80993712901361e-05\\
0.524593042405346	7.82343528179479e-05\\
0.504678997413717	6.94091511920627e-05\\
0.485520908288583	6.15242220407937e-05\\
0.467090078234675	5.44878934583889e-05\\
0.449358899814048	4.8216012469359e-05\\
0.432300813593051	4.26315704592126e-05\\
0.415890268359098	3.7664310281238e-05\\
0.400102682847653	3.32503194139421e-05\\
0.384914408922085	2.93316145513685e-05\\
0.370302696151259	2.58557234432113e-05\\
0.356245657731791	2.27752697648333e-05\\
0.342722237703926	2.00475663968545e-05\\
0.329712179411932	1.7634221835854e-05\\
0.317195995161773	1.55007636397414e-05\\
0.305154937030591	1.36162819189112e-05\\
0.293570968784308	1.19530949878845e-05\\
0.28242673886124	1.04864384462785e-05\\
0.271705554381298	9.19417820138021e-06\\
0.261391356141811	8.05654730142309e-06\\
0.251468694562535	7.05590593018032e-06\\
0.241922706543817	6.17652352045344e-06\\
0.232739093203235	5.40438166913641e-06\\
0.223904098457386	4.72699636684825e-06\\
0.215404488416723	4.13325797440413e-06\\
0.207227531562585	3.61328736897299e-06\\
0.199360979676728	3.15830672721727e-06\\
0.191793049494783	2.76052349492059e-06\\
0.184512405056175	2.41302619857521e-06\\
0.177508140724047	2.10969087260285e-06\\
0.170769764849768	1.84509699737275e-06\\
0.16428718405755	1.61445196145419e-06\\
0.158050688125638	1.41352317225149e-06\\
0.152050935441424	1.23857703978339e-06\\
0.146278939008691	1.08632414775858e-06\\
0.140726052986052	9.53870004175029e-07\\
0.135383959736379	8.38670831063353e-07\\
0.130244657367874	7.38493910742093e-07\\
0.12530044774807	6.51382055320049e-07\\
0.120543924972843	5.7562180846848e-07\\
0.115967964273155	5.09715024942833e-07\\
0.111565711342895	4.5235350507266e-07\\
0.107330572071854	4.02396389420144e-07\\
0.103256202668448	3.58850043816416e-07\\
0.0993365001573815	3.20850187644081e-07\\
0.0955655932380394	2.87646039019662e-07\\
0.0919378334898964	2.58586269788292e-07\\
0.0884477869117796	2.33106581220923e-07\\
0.0850902257823085	2.10718728160088e-07\\
0.0818601208293201	1.91000835192181e-07\\
0.0787526336965502	1.73588863282377e-07\\
0.0757631096962861	1.58169099210616e-07\\
0.072887070837136	1.44471553091827e-07\\
0.0701202091164705	1.32264161240473e-07\\
0.0674583800674897	1.21347702636939e-07\\
0.0648975965512486	1.11551347266852e-07\\
0.0624340227843446	1.02728763642407e-07\\
0.0600639685933188	9.4754720901336e-08\\
0.0577838838871649	8.75221280567391e-08\\
0.0555903533396688	8.09394593002318e-08\\
0.0534800912736094	7.49285198180643e-08\\
0.0514499367391614	6.94225114538577e-08\\
0.0494968487791262	6.43643618366364e-08\\
0.0476179018738988	5.97052843856754e-08\\
0.0458102815593482	5.54035399936446e-08\\
0.0440712802110476	5.14233742569271e-08\\
0.0423982929885382	4.77341069312409e-08\\
0.0407888139335537	4.43093528902181e-08\\
0.039240432216359	4.11263562852292e-08\\
0.037750828524582	3.81654218620912e-08\\
0.0363177715891282	3.54094294878774e-08\\
0.0349391148419753	3.28434198730377e-08\\
0.0336127932008409	3.04542412317778e-08\\
0.0323368199759069	2.82302481966674e-08\\
0.0311092838939685	2.6161045685655e-08\\
0.0299283462355481	2.42372716088995e-08\\
0.0287922380806876	2.24504133034895e-08\\
0.0276992576592938	2.07926534068368e-08\\
0.0266477678020652	1.92567415397027e-08\\
0.0256361934881864	1.78358886869115e-08\\
0.0246630194861122	1.65236815601409e-08\\
0.0237267880839116	1.53140145249485e-08\\
0.0228260969057683	1.42010368961926e-08\\
0.0219595968113703	1.31791135725738e-08\\
0.0211259898750401	1.22427971109689e-08\\
0.0203240274415788	1.13868094501031e-08\\
0.0195525082559127	1.06060315937182e-08\\
0.0188102766637396	9.89549966549543e-09\\
0.0180962208804812	9.25040585805418e-09\\
0.0174092713259465	8.66610292027196e-09\\
0.0167483990222141	8.13811096200117e-09\\
0.0161126140523319	7.66212550195001e-09\\
0.0155009640775254	7.23402584014533e-09\\
0.0149125329106954	6.84988299689492e-09\\
0.0143464391440662	6.50596662063632e-09\\
0.0138018348289296	6.19875042251006e-09\\
0.0132779042055075	5.92491584119707e-09\\
0.0127738624810297	5.68135377344936e-09\\
0.0122889546541975	5.46516432085685e-09\\
0.0118224543842707	5.27365459957286e-09\\
0.0113736629030869	5.10433473638571e-09\\
0.0109419079683795	4.95491223163481e-09\\
0.0105265428568304	4.82328490829882e-09\\
0.0101269453953466	4.70753268886512e-09\\
0.00974251702911052	4.60590844963363e-09\\
0.00937268192500803	4.51682819833868e-09\\
0.00901688610909133	4.43886080814822e-09\\
0.00867459663678436	4.37071752151965e-09\\
0.00834530079458812	4.31124141347773e-09\\
0.00802850533208999	4.25939697750097e-09\\
0.00772373572312664	4.21425997008001e-09\\
0.00743053545499395	4.17500762336261e-09\\
0.00714846534463914	4.1409093102048e-09\\
0.00687710288081094	4.11131772292544e-09\\
0.00661604159118235	4.08566060664237e-09\\
0.00636489043349793	4.0634330704122e-09\\
0.00612327320983386	4.04419048446926e-09\\
0.00589082800309312	4.02754195967517e-09\\
0.00566720663489187	4.01314439554475e-09\\
0.00545207414402499	4.0006970757528e-09\\
0.0052451082847296	3.98993678452399e-09\\
0.00504599904399483	3.98063341354075e-09\\
0.00485444817719514	3.97258602659787e-09\\
0.00467016876135131	3.96561934809644e-09\\
0.00449288476535011	3.95958064114828e-09\\
0.00432233063647882	3.95433694160198e-09\\
0.00415825090265531	3.94977261528568e-09\\
0.00400039978975782	3.9457872071884e-09\\
0.0038485408534813	3.94229355303094e-09\\
0.00370244662516874	3.93921612551738e-09\\
0.00356189827108717	3.93648958950648e-09\\
0.00342668526463783	3.9340575423542e-09\\
0.00329660507100953	3.93187141757032e-09\\
0.00317146284380288	3.92988953186961e-09\\
0.00305107113317099	3.92807625744477e-09\\
0.00293524960503932	3.92640130299788e-09\\
0.00282382477098434	3.92483908865033e-09\\
0.00271662972836613	3.92336820129581e-09\\
0.00261350391032587	3.92197091834048e-09\\
0.00251429284527363	3.92063278904752e-09\\
0.00241884792550625	3.91934226388777e-09\\
0.00232702618460862	3.91809036341923e-09\\
0.00223869008330519	3.91687037932168e-09\\
0.00215370730344056	3.91567760126372e-09\\
0.00207195054978088	3.91450906425647e-09\\
0.00199329735933905	3.91336331225966e-09\\
0.00191762991793806	3.9122401746796e-09\\
0.00184483488373781	3.91114055348291e-09\\
0.00177480321746097	3.9100662194585e-09\\
0.0017074300190637	3.90901961716112e-09\\
0.00164261437060639	3.90800367870206e-09\\
0.00158025918508932	3.90702164733635e-09\\
0.0015202710610265	3.90607691222555e-09\\
0.00146256014254017	3.90517285617114e-09\\
0.00140703998476613	3.90431271819962e-09\\
0.0013536274243685	3.90349947295359e-09\\
0.00130224245496978	3.90273572855296e-09\\
0.00125280810730979	3.90202364426289e-09\\
0.0012052503339538	3.90136486879265e-09\\
0.0011594978983773	3.9007604994606e-09\\
0.00111548226826121	3.90021106177298e-09\\
0.00107313751283773	3.89971650838974e-09\\
0.00103240020413293	3.89927623579964e-09\\
0.000993209321958432	3.89888911652635e-09\\
0.000955506162509545	3.89855354432684e-09\\
0.000919234250433192	3.89826748955093e-09\\
0.000884339254233779	3.89802856177635e-09\\
0.00085076890489036	3.8978340768059e-09\\
0.00081847291756315	3.89768112534609e-09\\
0.000787402916272154	3.89756664090106e-09\\
0.000757512361435043	3.8974874648092e-09\\
0.000728756480155786	3.89744040668085e-09\\
0.000701092199159566	3.89742229900394e-09\\
0.000674478080273568	3.89743004499123e-09\\
0.000648874258356965	3.89746065922482e-09\\
0.000624242381587151	3.89751130091649e-09\\
0.000600545554012754	3.89757929995919e-09\\
0.000577748280287397	3.89766217611667e-09\\
0.000555816412501416	3.89775765193209e-09\\
0.000534717099031898	3.89786366000308e-09\\
0.000514418735334412	3.89797834537935e-09\\
0.000494890916602743	3.89810006384233e-09\\
0.000476104392225699	3.89822737681298e-09\\
0.000458031021972785	3.89835904361919e-09\\
0.000440643733843104	3.89849401177741e-09\\
0.000423916483514361	3.89863140588029e-09\\
0.000407824215331218	3.89877051561095e-09\\
0.000392342824774562	3.89891078331292e-09\\
0.000377449122355485	3.89905179148117e-09\\
0.000363120798879876	3.89919325046762e-09\\
0.0003493363920316	3.89933498660638e-09\\
0.000336075254224219	3.89947693093416e-09\\
0.000323317521673083	3.89961910861301e-09\\
0.000311044084641479	3.89976162910298e-09\\
0.000299236558816262	3.89990467713224e-09\\
0.000287877257770095	3.90004850444413e-09\\
0.000276949166469046	3.90019342228758e-09\\
0.000266435915785868	3.90033979460625e-09\\
0.000256321757980767	3.90048803186101e-09\\
0.000246591543112956	3.90063858537521e-09\\
0.000237230696347642	3.90079194216221e-09\\
0.000228225196124458	3.90094862008255e-09\\
0.000219561553154649	3.90110916327634e-09\\
0.000211226790215541	3.90127413772344e-09\\
0.000203208422712033	3.90144412687079e-09\\
0.000195494439975986	3.90161972718593e-09\\
0.000188073287275514	3.90180154356616e-09\\
0.000180933848507216	3.90199018448975e-09\\
0.000174065429545424	3.90218625685111e-09\\
0.000167457742223532	3.90239036039375e-09\\
0.000161100888923408	3.90260308168566e-09\\
0.000154985347749812	3.90282498763316e-09\\
0.000149101958267593	3.90305661848795e-09\\
0.00014344190778033	3.90329848040303e-09\\
0.000137996718129844	3.90355103757972e-09\\
0.000132758232996808	3.90381470409159e-09\\
0.000127718605683442	3.90408983552551e-09\\
0.000122870287359992	3.90437672061154e-09\\
0.000118206015757376	3.90467557304203e-09\\
0.000113718804289073	3.90498652372326e-09\\
0.000109401931585952	3.90530961374298e-09\\
0.000105248931428374	3.90564478831517e-09\\
0.000101253583060474	3.90599189203846e-09\\
9.74099018721298e-05	3.90635066573054e-09\\
9.37121304346413e-05	3.90672074510787e-09\\
9.01547298767154e-05	3.90710166155827e-09\\
8.67323715878194e-05	3.90749284515712e-09\\
8.34399292364855e-05	3.90789363004977e-09\\
8.02724710916065e-05	3.9083032622041e-09\\
7.72252526352239e-05	3.9087209094779e-09\\
7.42937094557406e-05	3.90914567382752e-09\\
7.14734504109144e-05	3.90957660541124e-09\\
6.87602510503896e-05	3.91001271825935e-09\\
6.61500472879175e-05	3.91045300709411e-09\\
6.36389293137829e-05	3.91089646486836e-09\\
6.12231357383231e-05	3.91134210053527e-09\\
5.88990479577622e-05	3.91178895654433e-09\\
5.66631847339251e-05	3.91223612565209e-09\\
5.45121969797098e-05	3.91268276655211e-09\\
5.24428627425096e-05	3.91312811799897e-09\\
5.04520823780663e-05	3.9135715110969e-09\\
4.85368739075318e-05	3.91401237953764e-09\\
4.6694368550778e-05	3.91445026763097e-09\\
4.49218064292671e-05	3.91488483607352e-09\\
4.32165324320447e-05	3.91531586543647e-09\\
4.15759922386638e-05	3.91574325747825e-09\\
3.99977284930828e-05	3.91616703436939e-09\\
3.84793771228054e-05	3.91658733601353e-09\\
3.70186637977489e-05	3.91700441565036e-09\\
3.56134005235389e-05	3.91741863394423e-09\\
3.42614823641236e-05	3.91783045177186e-09\\
3.29608842888029e-05	3.91824042189818e-09\\
3.1709658138945e-05	3.9186491797554e-09\\
3.05059297098513e-05	3.91905743348321e-09\\
2.93478959433948e-05	3.91946595337512e-09\\
2.82338222272307e-05	3.91987556087869e-09\\
2.71620397965284e-05	3.92028711725119e-09\\
2.61309432343393e-05	3.92070151196276e-09\\
2.51389880668511e-05	3.92111965093397e-09\\
2.41846884499292e-05	3.92154244465792e-09\\
2.32666149434791e-05	3.92197079630298e-09\\
2.23833923702958e-05	3.92240558984047e-09\\
2.15336977561935e-05	3.92284767828217e-09\\
2.07162583483303e-05	3.9232978721194e-09\\
1.99298497087584e-05	3.92375692806356e-09\\
1.91732938803456e-05	3.92422553822218e-09\\
1.84454576223193e-05	3.92470431984197e-09\\
1.77452507127922e-05	3.92519380581262e-09\\
1.70716243157246e-05	3.92569443609914e-09\\
1.64235694098785e-05	3.92620655031887e-09\\
1.58001152774107e-05	3.92673038168031e-09\\
1.52003280498398e-05	3.92726605248409e-09\\
1.46233093092097e-05	3.92781357142346e-09\\
1.40681947423545e-05	3.92837283283797e-09\\
1.35341528462484e-05	3.92894361810753e-09\\
1.3020383682502e-05	3.92952559927504e-09\\
1.25261176791392e-05	3.93011834495978e-09\\
1.20506144778594e-05	3.9307213285567e-09\\
1.15931618250592e-05	3.93133393861786e-09\\
1.11530745049512e-05	3.93195549126587e-09\\
1.07296933131836e-05	3.93258524439141e-09\\
1.03223840694212e-05	3.93322241329697e-09\\
9.93053666741072e-06	3.93386618744564e-09\\
9.55356416110549e-06	3.93451574781384e-09\\
9.19090188548259e-06	3.93517028442091e-09\\
8.84200661073411e-06	3.93582901352544e-09\\
8.50635572856633e-06	3.93649119396447e-09\\
8.18344646938752e-06	3.93715614220074e-09\\
7.87279514921228e-06	3.93782324561091e-09\\
7.57393644515392e-06	3.93849197365166e-09\\
7.28642269841996e-06	3.93916188657698e-09\\
7.00982324376642e-06	3.93983264149781e-09\\
6.74372376440683e-06	3.94050399561527e-09\\
6.48772567140934e-06	3.9411758065804e-09\\
6.24144550665268e-06	3.94184803000841e-09\\
6.00451436844624e-06	3.94252071425398e-09\\
5.77657735895437e-06	3.94319399262229e-09\\
5.55729305259683e-06	3.94386807327365e-09\\
5.34633298462938e-06	3.94454322710551e-09\\
5.1433811591382e-06	3.9452197739873e-09\\
4.94813357571137e-06	3.94589806768587e-09\\
4.76029777407836e-06	3.94657847994473e-09\\
4.57959239603544e-06	3.94726138410464e-09\\
4.40574676400073e-06	3.94794713870775e-09\\
4.2385004755678e-06	3.94863607154462e-09\\
4.07760301345033e-06	3.94932846455915e-09\\
3.9228133702337e-06	3.95002454001596e-09\\
3.77389968737126e-06	3.95072444836276e-09\\
3.63063890788472e-06	3.95142825809382e-09\\
3.49281644224838e-06	3.952135947975e-09\\
3.36022584695663e-06	3.95284740185967e-09\\
3.23266851529339e-06	3.95356240629384e-09\\
3.1099533798402e-06	3.95428065104264e-09\\
2.9918966262774e-06	3.95500173255003e-09\\
2.8783214180497e-06	3.95572516030043e-09\\
2.76905763148365e-06	3.95645036590366e-09\\
2.66394160096038e-06	3.95717671469853e-09\\
2.56281587376172e-06	3.95790351952561e-09\\
2.46552897422271e-06	3.95863005627015e-09\\
2.37193517683701e-06	3.95935558071406e-09\\
2.28189428797546e-06	3.96007934617377e-09\\
2.19527143589087e-06	3.96080062139628e-09\\
2.11193686869433e-06	3.96151870814343e-09\\
2.03176576000059e-06	3.96223295795427e-09\\
1.95463802195133e-06	3.9629427875631e-09\\
1.88043812533621e-06	3.9636476925438e-09\\
1.80905492654231e-06	3.96434725879016e-09\\
1.74038150107272e-06	3.96504117153533e-09\\
1.67431498338494e-06	3.96572922168883e-09\\
1.61075641280915e-06	3.96641130936345e-09\\
1.5496105853156e-06	3.96708744453587e-09\\
1.49078591090896e-06	3.96775774489674e-09\\
1.43419427643626e-06	3.96842243095269e-09\\
1.3797509136026e-06	3.96908181858923e-09\\
1.32737427199725e-06	3.96973630927886e-09\\
1.27698589693974e-06	3.97038637820497e-09\\
1.22851031196299e-06	3.97103256059095e-09\\
1.18187490575756e-06	3.97167543654049e-09\\
1.1370098234076e-06	3.97231561471717e-09\\
1.09384786175549e-06	3.97295371520493e-09\\
1.05232436873866e-06	3.97359035189567e-09\\
1.01237714654761e-06	3.97422611474415e-09\\
9.73946358460126e-07	3.97486155225066e-09\\
9.36974439212251e-07	3.97549715451631e-09\\
9.01406008771532e-07	3.97613333721774e-09\\
8.67187789383616e-07	3.97677042683816e-09\\
8.34268525767778e-07	3.97740864749525e-09\\
8.0259890834193e-07	3.97804810966589e-09\\
7.72131499362068e-07	3.97868880110115e-09\\
7.42820661865544e-07	3.97933058018508e-09\\
7.14622491311708e-07	3.97997317195015e-09\\
6.87494749817541e-07	3.98061616690195e-09\\
6.61396802889765e-07	3.98125902274516e-09\\
6.36289558558665e-07	3.98190106904138e-09\\
6.1213540882244e-07	3.98254151472412e-09\\
5.88898173314387e-07	3.98317945833766e-09\\
5.66543045108532e-07	3.98381390076203e-09\\
5.45036538582531e-07	3.98444376014658e-09\\
5.24346439259735e-07	3.98506788861455e-09\\
5.04441755555314e-07	3.9856850903703e-09\\
4.85292672354124e-07	3.98629414069244e-09\\
4.66870506350824e-07	3.9868938052977e-09\\
4.491476630853e-07	3.98748285959072e-09\\
4.32097595609082e-07	3.98806010733405e-09\\
4.15694764720818e-07	3.98862439821884e-09\\
3.9991460071125e-07	3.98917464400178e-09\\
3.84733466560372e-07	3.98970983287676e-09\\
3.70128622531677e-07	3.99022904178588e-09\\
3.56078192110432e-07	3.99073144659257e-09\\
3.42561129234966e-07	3.99121632996612e-09\\
3.29557186771892e-07	3.991683087052e-09\\
3.17046886188037e-07	3.9921312289307e-09\\
3.05011488373656e-07	3.99256038414956e-09\\
2.93432965573228e-07	3.99297029840185e-09\\
2.82293974381772e-07	3.9933608326821e-09\\
2.71577829766265e-07	3.99373196013644e-09\\
2.61268480073221e-07	3.99408376195453e-09\\
2.5135048298501e-07	3.99441642253164e-09\\
2.41808982388889e-07	3.99473022420592e-09\\
2.32629686124126e-07	3.99502554176792e-09\\
2.23798844573839e-07	3.99530283707218e-09\\
2.15303230069531e-07	3.99556265380744e-09\\
2.07130117077433e-07	3.99580561263701e-09\\
1.99267263136999e-07	3.99603240685565e-09\\
1.91702890522993e-07	3.99624379848215e-09\\
1.84425668603701e-07	3.99644061494825e-09\\
1.77424696868838e-07	3.99662374624601e-09\\
1.70689488601738e-07	3.99679414255271e-09\\
1.64209955171352e-07	3.99695281205977e-09\\
1.57976390920553e-07	3.99710081905743e-09\\
1.51979458628081e-07	3.99723928189292e-09\\
1.46210175522369e-07	3.99736937067356e-09\\
1.40659899826306e-07	3.99749230452448e-09\\
1.3532031781276e-07	3.99760934804993e-09\\
1.30183431351497e-07	3.9977218067633e-09\\
1.25241545928824e-07	3.99783102134351e-09\\
1.20487259122022e-07	3.99793836041083e-09\\
1.15913449511296e-07	3.99804521170165e-09\\
1.11513266012638e-07	3.99815297137802e-09\\
1.07280117615631e-07	3.99826303184489e-09\\
1.03207663510808e-07	3.99837676754563e-09\\
9.92898035917915e-08	3.99849551932903e-09\\
9.55206693179727e-08	3.99862057751573e-09\\
9.18946149240626e-08	3.998753163936e-09\\
8.84062089633288e-08	3.99889441353356e-09\\
8.5050226171863e-08	3.99904535597369e-09\\
8.18216396416857e-08	3.99920689792774e-09\\
7.87156132909694e-08	3.99937980685138e-09\\
7.57274946200991e-08	3.99956469655807e-09\\
7.28528077427182e-08	3.99976201585928e-09\\
7.00872466813233e-08	3.99997203982847e-09\\
6.74266689173642e-08	4.00019486571594e-09\\
6.48670891861894e-08	4.00043041201795e-09\\
6.24046735075392e-08	4.00067842253733e-09\\
6.00357334426483e-08	4.00093847435051e-09\\
5.77567205693541e-08	4.00120998996654e-09\\
5.55642211669348e-08	4.00149225296672e-09\\
5.34549511027159e-08	4.00178442762476e-09\\
5.14257509127861e-08	4.00208557977654e-09\\
4.9473581069454e-08	4.00239470124584e-09\\
4.75955174283567e-08	4.00271073364184e-09\\
4.57887468483997e-08	4.00303259357241e-09\\
4.40505629779687e-08	4.00335919705141e-09\\
4.23783622011004e-08	4.00368948265812e-09\\
4.07696397375414e-08	4.00402243296098e-09\\
3.92219858908506e-08	4.00435709401303e-09\\
3.77330824389289e-08	4.00469259187085e-09\\
3.63006991615673e-08	4.00502814655137e-09\\
3.49226904998122e-08	4.00536308314275e-09\\
3.35969923421447e-08	4.00569683988949e-09\\
3.23216189326591e-08	4.00602897362455e-09\\
3.10946598966109e-08	4.0063591624719e-09\\
2.99142773788763e-08	4.00668720649787e-09\\
2.87787032910394e-08	4.00701302547342e-09\\
2.76862366629829e-08	4.00733665620341e-09\\
2.66352410950137e-08	4.00765824691667e-09\\
2.56241423067092e-08	4.00797805114969e-09\\
2.46514257788118e-08	4.0082964207198e-09\\
2.3715634484638e-08	4.00861379785049e-09\\
2.28153667076071e-08	4.00893070619439e-09\\
2.19492739416175e-08	4.00924774236158e-09\\
2.11160588711263e-08	4.00956556661147e-09\\
2.03144734279084e-08	4.00988489372805e-09\\
1.95433169215816e-08	4.01020648290956e-09\\
1.88014342410994e-08	4.01053112913487e-09\\
1.80877141245159e-08	4.0108596526981e-09\\
1.74010874944338e-08	4.01119288994096e-09\\
1.67405258566382e-08	4.01153168330306e-09\\
1.61050397595218e-08	4.01187687154471e-09\\
1.549367731199e-08	4.01222927957012e-09\\
1.49055227576291e-08	4.01258970794818e-09\\
1.43396951029996e-08	4.0129589230008e-09\\
1.37953467980011e-08	4.01333764602915e-09\\
1.32716624663316e-08	4.01372654303977e-09\\
1.276785768414e-08	4.01412621350163e-09\\
1.2283177805042e-08	4.01453718054687e-09\\
1.18168968297393e-08	4.0149598800855e-09\\
1.13683163185493e-08	4.01539465140236e-09\\
1.09367643452166e-08	4.01584172715805e-09\\
1.05215944904376e-08	4.01630122516979e-09\\
1.01221848735935e-08	4.0167731408146e-09\\
9.7379372212379e-09	4.01725734056145e-09\\
9.36827597094717e-09	4.01775355703216e-09\\
9.01264740918808e-09	4.0182613863811e-09\\
8.67051884191367e-09	4.01878028702008e-09\\
8.34137779664372e-09	4.01930958093169e-09\\
8.02473125483504e-09	4.01984845710513e-09\\
7.72010491339184e-09	4.02039597794294e-09\\
7.42704247420952e-09	4.02095108740366e-09\\
7.1451049606886e-09	4.02151262239143e-09\\
6.87387006019382e-09	4.02207932565848e-09\\
6.61293149147462e-09	4.02264986144242e-09\\
6.36189839609848e-09	4.02322283293089e-09\\
6.12039475298654e-09	4.02379680001394e-09\\
5.88805881517338e-09	4.02437029986117e-09\\
5.66454256794852e-09	4.02494186621001e-09\\
5.44951120756695e-09	4.02551004996568e-09\\
5.2426426397486e-09	4.026073439053e-09\\
5.04362699721494e-09	4.02663067629581e-09\\
4.85216617554069e-09	4.02718047785989e-09\\
4.66797338662465e-09	4.02772164855527e-09\\
4.49077272911167e-09	4.02825309502177e-09\\
4.32029877512123e-09	4.02877383780927e-09\\
4.15629617266475e-09	4.02928301990646e-09\\
3.9985192631551e-09	4.02977991343415e-09\\
3.84673171343605e-09	4.03026392408822e-09\\
3.70070616178016e-09	4.03073459235219e-09\\
3.5602238773248e-09	4.0311915938343e-09\\
3.42507443243655e-09	4.03163473652419e-09\\
3.29505538751273e-09	4.03206395725509e-09\\
3.16997198774829e-09	4.03247931581174e-09\\
3.04963687141359e-09	4.03288098903172e-09\\
2.9338697892064e-09	4.03326926302465e-09\\
2.82249733425742e-09	4.03364452519377e-09\\
2.71535268238512e-09	4.03400725606975e-09\\
2.61227534221068e-09	4.03435802051844e-09\\
2.51311091475892e-09	4.03469745967655e-09\\
2.41771086218484e-09	4.03502628219565e-09\\
2.32593228527971e-09	4.03534525685251e-09\\
2.23763770942303e-09	4.03565520446297e-09\\
2.15269487866017e-09	4.03595699122646e-09\\
2.0709765575968e-09	4.03625152194809e-09\\
1.99236034081381e-09	4.03653973421382e-09\\
1.91672846951681e-09	4.03682259265793e-09\\
1.84396765514596e-09	4.03710108397323e-09\\
1.77396890968164e-09	4.03737621234815e-09\\
1.70662738239189e-09	4.03764899507876e-09\\
1.64184220277711e-09	4.03792045858088e-09\\
1.57951632947664e-09	4.03819163466232e-09\\
1.51955640491115e-09	4.03846355713846e-09\\
1.46187261544271e-09	4.03873725754098e-09\\
1.40637855684355e-09	4.03901376189561e-09\\
1.35299110487157e-09	4.0392940869227e-09\\
1.30163029075909e-09	4.03957923600113e-09\\
1.25221918142795e-09	4.03987019454432e-09\\
1.20468376425201e-09	4.04016792517331e-09\\
1.15895283619394e-09	4.04047336384023e-09\\
1.11495789715068e-09	4.04078741208365e-09\\
1.07263304734744e-09	4.04111093249623e-09\\
1.03191488862684e-09	4.04144474184121e-09\\
9.92742429485132e-10	4.04178960343401e-09\\
9.550569937134e-10	4.04214622045197e-09\\
9.18802132506752e-10	4.04251522641367e-09\\
8.83923539910007e-10	4.04289717909303e-09\\
8.50368971473076e-10	4.04329255010055e-09\\
8.18088165994314e-10	4.04370171830241e-09\\
7.87032770234524e-10	4.04412496087934e-09\\
7.57156266488931e-10	4.04456244507019e-09\\
7.28413902908543e-10	4.04501422269722e-09\\
7.0076262646664e-10	4.04548022313734e-09\\
6.74161018469855e-10	4.04596024874252e-09\\
6.4856923251735e-10	4.04645397178221e-09\\
6.23948934815124e-10	4.04696093405147e-09\\
6.00263246756023e-10	4.04748054352999e-09\\
5.77476689679494e-10	4.04801208333538e-09\\
5.55555131728273e-10	4.0485547118895e-09\\
5.34465736722501e-10	4.04910747267011e-09\\
5.14176914974552e-10	4.04966930316205e-09\\
4.94658275971049e-10	4.05023904793472e-09\\
4.75880582851059e-10	4.05081547530301e-09\\
4.57815708612381e-10	4.0513972879171e-09\\
4.40436593980249e-10	4.05198314777831e-09\\
4.23717206875407e-10	4.05257169029374e-09\\
4.0763250342079e-10	4.05316154788601e-09\\
3.92158390428458e-10	4.05375136679759e-09\\
3.77271689310522e-10	4.05433983356536e-09\\
3.62950101360082e-10	4.05492568543995e-09\\
3.49172174350108e-10	4.05550773732856e-09\\
3.35917270400277e-10	4.05608489207912e-09\\
3.23165535063597e-10	4.05665616126864e-09\\
3.10897867586553e-10	4.05722067142714e-09\\
2.99095892298179e-10	4.05777768103754e-09\\
2.8774193108526e-10	4.05832658220156e-09\\
2.76818976912371e-10	4.0588669108844e-09\\
2.66310668347138e-10	4.05939834560039e-09\\
2.56201265052541e-10	4.05992070907533e-09\\
2.46475624209546e-10	4.06043397107754e-09\\
2.37119177834766e-10	4.0609382347966e-09\\
2.28117910959154e-10	4.06143374126783e-09\\
2.19458340635064e-10	4.06192084897665e-09\\
2.11127495740217e-10	4.06240005036438e-09\\
2.03112897548326e-10	4.06287192750366e-09\\
1.95402541037284e-10	4.06333716395088e-09\\
1.87984876906908e-10	4.06379653581886e-09\\
1.80848794279306e-10	4.06425088475686e-09\\
1.73983604055954e-10	4.06470111490656e-09\\
1.67379022906555e-10	4.06514818649138e-09\\
1.61025157865698e-10	4.06559308304148e-09\\
1.54912491514237e-10	4.06603682121248e-09\\
1.49031867723203e-10	4.06648042612454e-09\\
1.43374477938889e-10	4.06692491479237e-09\\
1.37931847988566e-10	4.0673712977704e-09\\
1.32695825387068e-10	4.06782054789723e-09\\
1.2765856712523e-10	4.06827361085265e-09\\
1.22812527921884e-10	4.06873135955117e-09\\
1.1815044892183e-10	4.06919461469743e-09\\
1.13665346822835e-10	4.06966412173014e-09\\
1.09350503415381e-10	4.07014051268581e-09\\
1.05199455519499e-10	4.07062435857701e-09\\
1.01205985303607e-10	4.07111607559814e-09\\
9.73641109708534e-11	4.07161599419524e-09\\
9.36680777990192e-11	4.0721242971125e-09\\
9.01123495205499e-11	4.07264103585469e-09\\
8.66916000298099e-11	4.07316612638302e-09\\
8.34007054051422e-11	4.07369932356114e-09\\
8.02347362337701e-11	4.07424026261596e-09\\
7.71889502280609e-11	4.07478841495878e-09\\
7.42587851220766e-11	4.07534312387114e-09\\
7.14398518377843e-11	4.07590359191778e-09\\
6.87279279106771e-11	4.07646887592637e-09\\
6.61189511649714e-11	4.07703795146946e-09\\
6.36090136288931e-11	4.07760968313812e-09\\
6.11943556809517e-11	4.07818282102584e-09\\
5.88713604184209e-11	4.07875608709182e-09\\
5.66365482396027e-11	4.07932810052228e-09\\
5.44865716317495e-11	4.07989749692148e-09\\
5.24182101568447e-11	4.08046288653014e-09\\
5.04283656277261e-11	4.08102285972166e-09\\
4.85140574673273e-11	4.08157604976077e-09\\
4.66724182440906e-11	4.08212114355148e-09\\
4.49006893768541e-11	4.08265690462472e-09\\
4.31962170027908e-11	4.08318211709672e-09\\
4.15564480022002e-11	4.08369570132978e-09\\
3.99789261742066e-11	4.08419668224024e-09\\
3.84612885576272e-11	4.08468415308367e-09\\
3.70012618915077e-11	4.08515738102034e-09\\
3.55966592100156e-11	4.08561566748942e-09\\
3.42453765665982e-11	4.08605855642752e-09\\
3.29453898824904e-11	4.08648560667343e-09\\
3.16947519148605e-11	4.08689651954221e-09\\
3.04915893400443e-11	4.08729112785206e-09\\
2.93340999475055e-11	4.08766937475691e-09\\
2.82205499403131e-11	4.08803128672362e-09\\
2.71492713380978e-11	4.08837700399761e-09\\
2.61186594785925e-11	4.08870671307066e-09\\
2.51271706140189e-11	4.0890207358617e-09\\
2.41733195987146e-11	4.08931937900746e-09\\
2.32556776645429e-11	4.08960310024841e-09\\
2.23728702807488e-11	4.08987233308564e-09\\
2.15235750950561e-11	4.09012760740066e-09\\
2.07065199529248e-11	4.09036944767281e-09\\
1.99204809919963e-11	4.09059842434697e-09\\
1.9164280808878e-11	4.09081514482788e-09\\
1.84367866955166e-11	4.0910201692779e-09\\
1.77369089425214e-11	4.09121414948197e-09\\
1.70635992068942e-11	4.09139771918796e-09\\
1.64158489417227e-11	4.09157146423944e-09\\
1.57926878854831e-11	4.09173600590122e-09\\
1.51931826086914e-11	4.09189198286486e-09\\
1.46164351157238e-11	4.09204000174503e-09\\
1.40615814997148e-11	4.09218064292598e-09\\
1.35277906485155e-11	4.09231453438156e-09\\
1.30142629997754e-11	4.09244224511214e-09\\
1.25202293432821e-11	4.09256431050144e-09\\
1.20449496687664e-11	4.09268138189935e-09\\
1.15877120574441e-11	4.0927939580626e-09\\
1.11478316156375e-11	4.09290260201068e-09\\
1.07246494488763e-11	4.09300784950341e-09\\
1.03175316749441e-11	4.09311014287387e-09\\
9.92586847438905e-12	4.0932101690078e-09\\
9.54907317707886e-12	4.0933083501039e-09\\
9.18658138343102e-12	4.09340522885785e-09\\
8.83785011900157e-12	4.0935012516345e-09\\
8.50235702116695e-12	4.09359703074831e-09\\
8.17959955667968e-12	4.09369306058769e-09\\
7.86909426892683e-12	4.09378979457913e-09\\
7.57037605376279e-12	4.09388785863898e-09\\
7.28299746283271e-12	4.09398766162084e-09\\
7.00652803334167e-12	4.09408975662155e-09\\
6.74055364326722e-12	4.0941948983573e-09\\
6.48467589104801e-12	4.09430344879166e-09\\
6.23851149882058e-12	4.09441592297383e-09\\
6.00169173830931e-12	4.09453312272818e-09\\
5.77386187851064e-12	4.09465543764364e-09\\
5.55468065434319e-12	4.09478372260078e-09\\
5.34381975546905e-12	4.09491852712073e-09\\
5.14096333451915e-12	4.09506056940032e-09\\
4.94580753398755e-12	4.09521062686402e-09\\
4.75806003108479e-12	4.0953690872658e-09\\
4.57743959986932e-12	4.09553708409208e-09\\
4.40367569000063e-12	4.09571543112953e-09\\
4.23650802148347e-12	4.09590465218488e-09\\
4.07568619479591e-12	4.09610574541393e-09\\
3.92096931581715e-12	4.09631978618188e-09\\
3.77212563499372e-12	4.09654723219633e-09\\
3.62893220020298e-12	4.09678915762877e-09\\
3.4911745227945e-12	4.09704720453129e-09\\
3.35864625630857e-12	4.097321334824e-09\\
3.23114888739111e-12	4.09761319809703e-09\\
3.10849143844148e-12	4.09792370778674e-09\\
2.99049018154836e-12	4.09825352545729e-09\\
2.87696836328459e-12	4.09860450985965e-09\\
2.76775593994927e-12	4.09897705876177e-09\\
2.66268932286018e-12	4.0993729581576e-09\\
2.56161113331532e-12	4.09979187645685e-09\\
2.46436996685608e-12	4.10023664073107e-09\\
2.37082016647945e-12	4.10070668592892e-09\\
2.28082160445914e-12	4.10120438202257e-09\\
2.19423947244913e-12	4.10172953967611e-09\\
2.11094407955484e-12	4.10228342196219e-09\\
2.03081065807004e-12	4.10286628299256e-09\\
1.95371917658783e-12	4.10348113101608e-09\\
1.8795541602064e-12	4.10412587115796e-09\\
1.80820451755972e-12	4.10480240687307e-09\\
1.73956337441447e-12	4.10551156633033e-09\\
1.67352791358364e-12	4.1062512939209e-09\\
1.60999922091733e-12	4.10702506579006e-09\\
1.54888213713974e-12	4.10783087824393e-09\\
1.49008511531058e-12	4.10866991503564e-09\\
1.43352008369752e-12	4.10953986816782e-09\\
1.37910231385394e-12	4.11044156525707e-09\\
1.32675029370472e-12	4.11137572153867e-09\\
1.27638560544971e-12	4.11233915113027e-09\\
1.22793280810217e-12	4.11333126657934e-09\\
1.18131932448614e-12	4.11435446813828e-09\\
1.13647533252348e-12	4.11540247611977e-09\\
1.09333366064773e-12	4.1164768914389e-09\\
1.05182968718829e-12	4.11757394685788e-09\\
1.01190124357387e-12	4.11869528285549e-09\\
9.73488521210612e-13	4.11983480344937e-09\\
9.3653398189508e-13	4.1209947524833e-09\\
9.00982271628138e-13	4.1221674976343e-09\\
8.66780137700484e-13	4.12335438313562e-09\\
8.33876348925719e-13	4.12455261720454e-09\\
8.02221618901429e-13	4.12576058395383e-09\\
7.71768532183372e-13	4.12697389039922e-09\\
7.42471473262129e-13	4.12819223826542e-09\\
7.14286558235904e-13	4.1294088199626e-09\\
6.87171569077063e-13	4.13062401784998e-09\\
6.61085890393973e-13	4.13183277893908e-09\\
6.35990448593463e-13	4.13303301937964e-09\\
6.1184765335267e-13	4.13422498985329e-09\\
5.88621341312733e-13	4.13539610371113e-09\\
5.66276721909874e-13	4.13656558525392e-09\\
5.44780325262831e-13	4.13770081266581e-09\\
5.24099952038484e-13	4.1388240275516e-09\\
5.04204625220672e-13	4.13991977590717e-09\\
4.8506454370988e-13	4.14099267237775e-09\\
4.6665103768435e-13	4.14203984357551e-09\\
4.48936525655698e-13	4.14305355909463e-09\\
4.31894473154772e-13	4.14404022227683e-09\\
4.15499352985804e-13	4.14499253822563e-09\\
3.99726606989379e-13	4.14590971119584e-09\\
3.8455260925689e-13	4.14679871777446e-09\\
3.69954630741436e-13	4.14765050032097e-09\\
3.55910805212095e-13	4.14847318367317e-09\\
3.4240009650063e-13	4.14926047460561e-09\\
3.29402266991516e-13	4.15000841556139e-09\\
3.16897847308144e-13	4.15072160765457e-09\\
3.04868107149737e-13	4.15140564731685e-09\\
2.93295027235342e-13	4.15205634079352e-09\\
2.82161272312851e-13	4.15266962432793e-09\\
2.71450165192615e-13	4.15325221453513e-09\\
2.61145661766789e-13	4.15380317025229e-09\\
2.51232326976934e-13	4.15432527919572e-09\\
2.41695311693944e-13	4.15481637522943e-09\\
2.32520330475608e-13	4.15528091433707e-09\\
2.23693640168531e-13	4.15571642422057e-09\\
2.15202019322337e-13	4.1561351451287e-09\\
2.07032748385337e-13	4.15652543226084e-09\\
1.99173590651977e-13	4.1568881355035e-09\\
1.91612773933553e-13	4.15723036534895e-09\\
1.84338972924703e-13	4.15754849876866e-09\\
1.77341292239307e-13	4.1578416312249e-09\\
1.70609250090341e-13	4.15811967254565e-09\\
1.6413276258927e-13	4.15838679984184e-09\\
1.57902128641446e-13	4.1586278761579e-09\\
1.51908015414893e-13	4.15885772548939e-09\\
1.46141444360708e-13	4.15907262015249e-09\\
1.40593777764146e-13	4.15927277790083e-09\\
1.35256705806233e-13	4.15945301518882e-09\\
1.30122234116531e-13	4.15963044371039e-09\\
1.25182671798422e-13	4.1597903967906e-09\\
1.2043061990895e-13	4.1599454304859e-09\\
1.15858960375991e-13	4.16008292117236e-09\\
1.11460845336127e-13	4.16020588756118e-09\\
1.07229686877274e-13	4.16033787645641e-09\\
1.03159147170683e-13	4.16045002380962e-09\\
9.92431289775412e-14	4.16055535450525e-09\\
9.54757665159518e-14	4.16065193874346e-09\\
9.18514166746139e-14	4.16073640870929e-09\\
8.83646505600344e-14	4.16083667207808e-09\\
8.50102453646215e-14	4.16091292884103e-09\\
8.17831765434676e-14	4.16099424118128e-09\\
7.86786102881139e-14	4.16106036556894e-09\\
7.56918962860137e-14	4.16112204937571e-09\\
7.28185607548562e-14	4.16118493583668e-09\\
7.00542997413115e-14	4.16124898029719e-09\\
6.73949726741647e-14	4.16129388514079e-09\\
6.48365961621752e-14	4.16134722974218e-09\\
6.23753380273792e-14	4.16140212365225e-09\\
6.00075115648896e-14	4.16144627869358e-09\\
5.77295700206026e-14	4.16149680838726e-09\\
5.55381012785347e-14	4.16153705101016e-09\\
5.34298227498313e-14	4.16157153406354e-09\\
5.1401576455797e-14	4.16161363189274e-09\\
4.94503242975755e-14	4.16165381426772e-09\\
4.75731435053995e-14	4.16169364423719e-09\\
4.57672222605886e-14	4.16172199863635e-09\\
4.40298554837435e-14	4.16176071763132e-09\\
4.23584407828203e-14	4.16180142013077e-09\\
4.07504745550248e-14	4.16183042092245e-09\\
3.9203548236677e-14	4.16186595483245e-09\\
3.77153446954386e-14	4.16189489142825e-09\\
3.62836347594928e-14	4.16193432298831e-09\\
3.49062738784806e-14	4.1619732109702e-09\\
3.35811989111899e-14	4.16201766508518e-09\\
3.23064250351889e-14	4.16204306608882e-09\\
3.10800427737703e-14	4.16208463636153e-09\\
2.99002151357584e-14	4.16212255086473e-09\\
2.87651748638884e-14	4.16216634129341e-09\\
2.76732217876429e-14	4.16221841288011e-09\\
2.66227202765751e-14	4.16226111954049e-09\\
2.56120967903079e-14	4.1623096845715e-09\\
2.46398375215354e-14	4.16235882312738e-09\\
2.37044861285003e-14	4.16241681226278e-09\\
2.28046415535476e-14	4.16248226828445e-09\\
2.19389559244873e-14	4.16253126742359e-09\\
2.11061325356251e-14	4.16261484777864e-09\\
2.03049239054333e-14	4.16268801292938e-09\\
1.9534129907956e-14	4.16275327033517e-09\\
1.87925959751465e-14	4.16283651010741e-09\\
1.80792113674464e-14	4.16293435437266e-09\\
1.73929075100148e-14	4.16302408281008e-09\\
1.67326563921168e-14	4.16312236610058e-09\\
1.60974690272706e-14	4.16325233553371e-09\\
1.54863939718514e-14	4.16336776331093e-09\\
1.48985158999283e-14	4.16348566861122e-09\\
1.43329542322034e-14	4.16361924811428e-09\\
1.37888618169964e-14	4.16374935134241e-09\\
1.32654236613016e-14	4.16394365922881e-09\\
1.27618557100132e-14	4.16409342746671e-09\\
1.22774036714948e-14	4.16428830894415e-09\\
1.1811341887729e-14	4.16447556429462e-09\\
1.13629722473593e-14	4.16470168235806e-09\\
1.09316231399922e-14	4.16497181676833e-09\\
1.05166484501961e-14	4.16517031117779e-09\\
1.01174265896886e-14	4.1654614048518e-09\\
9.73335956626271e-15	4.16577871869239e-09\\
9.36387208805767e-15	4.16613650012985e-09\\
9.00841070183257e-15	4.16638983225434e-09\\
8.6664429639518e-15	4.16674627925562e-09\\
8.33745664284048e-15	4.1671807088994e-09\\
8.02095895171598e-15	4.16766392412434e-09\\
7.71647581044499e-15	4.16814005761694e-09\\
7.42355113542182e-15	4.16866270043225e-09\\
7.14174615640301e-15	4.16909963193848e-09\\
6.87063875927609e-15	4.16974201781734e-09\\
6.60982285377703e-15	4.17042985284679e-09\\
6.35890776520996e-15	4.17109377308985e-09\\
6.11751764925763e-15	4.17188563331479e-09\\
5.88529092900643e-15	4.17263034473213e-09\\
5.66187975334218e-15	4.17356685903704e-09\\
5.44694947590606e-15	4.17471435736961e-09\\
5.24017815382949e-15	4.17554404082558e-09\\
5.04125606549786e-15	4.17672019480179e-09\\
4.84988524662017e-15	4.17769580141562e-09\\
4.66577904391e-15	4.17906104219783e-09\\
4.48866168570906e-15	4.18049621595531e-09\\
4.31826786891051e-15	4.18211398221758e-09\\
4.15434236156278e-15	4.18372598457965e-09\\
3.99663962055908e-15	4.18542490381975e-09\\
3.84492342383981e-15	4.18729844312204e-09\\
3.69896651655669e-15	4.18918031590849e-09\\
3.55855027066924e-15	4.19148868050186e-09\\
3.42346435746278e-15	4.1942891771851e-09\\
3.29350643249841e-15	4.19678767155636e-09\\
3.16848183252227e-15	4.19956947150507e-09\\
3.04820328388066e-15	4.20211115589073e-09\\
2.93249062200374e-15	4.20580834394368e-09\\
2.82117052153816e-15	4.21026575686331e-09\\
2.71407623672383e-15	4.21273238542848e-09\\
2.61104735162652e-15	4.21764165184082e-09\\
2.5119295398516e-15	4.22208354796933e-09\\
2.41657433337948e-15	4.2269769171588e-09\\
2.32483890017609e-15	4.23316100609196e-09\\
2.23658583024571e-15	4.23897878537541e-09\\
2.15168292980514e-15	4.24382243548951e-09\\
2.07000302327152e-15	4.25147339510577e-09\\
1.9914237627666e-15	4.25630100652752e-09\\
1.91582744485262e-15	4.26459969131099e-09\\
1.84310083422497e-15	4.27307722875778e-09\\
1.77313499409759e-15	4.28074201963469e-09\\
1.70582512302728e-15	4.29305719029342e-09\\
1.64107039793207e-15	4.30173321227998e-09\\
1.578773823069e-15	4.3140161512397e-09\\
1.51884208474467e-15	4.3247876348034e-09\\
1.46118541154119e-15	4.3372726112728e-09\\
1.40571743984808e-15	4.35628295770124e-09\\
1.35235508449869e-15	4.36418675017108e-09\\
1.30101841431739e-15	4.38063875112469e-09\\
1.25163053239113e-15	4.39717189448821e-09\\
1.20411746088594e-15	4.4144817954348e-09\\
1.15840803023596e-15	4.44065966722614e-09\\
1.11443377253896e-15	4.46328052626243e-09\\
1.07212881899865e-15	4.48429632887928e-09\\
1.03142980126012e-15	4.50382824445953e-09\\
9.92275756490829e-16	4.53802448771025e-09\\
9.54608036064615e-16	4.5635727442439e-09\\
9.18370217712326e-16	4.58904829605485e-09\\
8.83508021007161e-16	4.63055798502413e-09\\
8.49969226058361e-16	4.67047314858777e-09\\
8.17703595291284e-16	4.70863624607376e-09\\
7.86662798196866e-16	4.73481755578589e-09\\
7.56800338937585e-16	4.76917847032924e-09\\
7.2807148670161e-16	4.85795301331768e-09\\
7.00433208700787e-16	4.86544675375761e-09\\
6.73844105712034e-16	4.9377374562111e-09\\
6.48264350065706e-16	5.00850291753481e-09\\
6.23655625987923e-16	5.09175545628871e-09\\
5.99981072207605e-16	5.11821867024193e-09\\
5.77205226742166e-16	5.2017843742846e-09\\
5.55293973779218e-16	5.44659706159577e-09\\
5.3421449257467e-16	5.365527689128e-09\\
5.13935208290737e-16	5.47252560466885e-09\\
4.94425744700147e-16	5.60708337257503e-09\\
4.75656878685775e-16	5.7802401025223e-09\\
4.57600496467485e-16	5.97319667794856e-09\\
4.40229551490666e-16	5.95124804707521e-09\\
4.2351802391334e-16	6.18686419537308e-09\\
4.07440881631193e-16	6.26528656487606e-09\\
3.91974042782112e-16	6.43269012632083e-09\\
3.77094339674113e-16	6.81403498235104e-09\\
3.62779484082573e-16	7.1128608484434e-09\\
3.49008033864829e-16	7.08801589280278e-09\\
3.35759360842106e-16	7.55719241108875e-09\\
3.23013619900684e-16	8.32014685421849e-09\\
3.10751719266019e-16	8.13655760877818e-09\\
2.9895529190527e-16	8.64393750389357e-09\\
2.87606668015426e-16	8.78259120766206e-09\\
2.76688848555811e-16	1.0152549064906e-08\\
2.6618547978531e-16	9.68853564927524e-09\\
2.56080828766194e-16	1.04054733185419e-08\\
2.46359759797835e-16	1.0805653017474e-08\\
2.37007711745032e-16	1.20633993238286e-08\\
2.28010676226961e-16	1.32718982775177e-08\\
2.19355176634103e-16	1.56987303741952e-08\\
2.11028247941703e-16	1.4001453296199e-08\\
2.03017417289535e-16	1.65589026055504e-08\\
1.95310685298865e-16	1.74595979149739e-08\\
1.87896508098661e-16	2.08782796232283e-08\\
1.80763780034083e-16	2.01212345272221e-08\\
1.7390181703139e-16	2.1910957654642e-08\\
1.67300340594321e-16	2.18940087978512e-08\\
1.60949462407997e-16	2.46581814861985e-08\\
1.54839669527261e-16	3.14566904023342e-08\\
1.48961810127303e-16	2.98401488307738e-08\\
1.43307079795181e-16	3.48033194966184e-08\\
1.37867008341746e-16	4.55085884500651e-08\\
1.32633447114188e-16	4.77636187370504e-08\\
1.27598556790222e-16	5.24090565050805e-08\\
1.22754795635602e-16	6.98295021148675e-08\\
1.18094908207402e-16	1.1051761634074e-07\\
1.13611914486133e-16	7.27978030434514e-08\\
1.09299099420405e-16	8.22453743298338e-08\\
1.0515000286849e-16	1.0791530323645e-07\\
1.01158409921715e-16	1.25952367855896e-07\\
9.73183415951772e-17	1.10486865861201e-07\\
9.36240458718655e-17	1.57472678411506e-07\\
9.00699890867393e-17	1.39458276058244e-07\\
8.66508476378854e-17	1.79854566136734e-07\\
8.33615000123209e-17	2.07376969455414e-07\\
8.01970191145115e-17	2.13513161814593e-07\\
7.71526648861026e-17	2.79616889084464e-07\\
7.42238772058062e-17	4.20387966013751e-07\\
7.14062690588282e-17	4.18211316673892e-07\\
6.86956199655765e-17	6.91526549776868e-07\\
6.60878696598356e-17	4.75885993806269e-07\\
6.35791120069081e-17	5.2230697576437e-07\\
6.11655891526432e-17	7.30002596264703e-07\\
5.88436858945671e-17	6.19188737954165e-07\\
5.66099242666875e-17	8.15535783590551e-07\\
5.44609583298723e-17	1.18166742970651e-06\\
5.23935691599821e-17	1.65770398310264e-06\\
5.04046600262664e-17	1.04557938709238e-06\\
4.84912517527818e-17	1.70012817593465e-06\\
};
\addplot [color=mycolor1,only marks,mark=asterisk,mark options={solid},forget plot]
  table[row sep=crcr]{%
0.000702695221786386	3.89742261986992e-09\\
};
\end{axis}
\end{tikzpicture}%
\end{document}
% This file was created by matlab2tikz.
% Minimal pgfplots version: 1.3
%
%The latest updates can be retrieved from
%  http://www.mathworks.com/matlabcentral/fileexchange/22022-matlab2tikz
%where you can also make suggestions and rate matlab2tikz.
%
\documentclass[tikz]{standalone}
\usepackage{pgfplots}
\usepackage{grffile}
\pgfplotsset{compat=newest}
\usetikzlibrary{plotmarks}
\usepackage{amsmath}

\begin{document}
\definecolor{mycolor1}{rgb}{0.00000,0.44700,0.74100}%
\definecolor{mycolor2}{rgb}{0.85000,0.32500,0.09800}%
%
\begin{tikzpicture}

\begin{axis}[%
width=2in,
height=2in,
scale only axis,
xmode=log,
xmin=1e-20,
xmax=1,
xminorticks=true,
xlabel={$\lambda$},
ymode=log,
ymin=1e-09,
ymax=0.01,
yminorticks=true,
ylabel={GCV function value}
]
\addplot [color=mycolor1,solid,forget plot]
  table[row sep=crcr]{%
0.446980426276884	6.41099299208483e-08\\
0.427826773308999	5.90297794521629e-08\\
0.409493877583373	5.41938824541029e-08\\
0.391946568657486	4.96167491430793e-08\\
0.375151183184905	4.53090511324951e-08\\
0.359075500334391	4.12776085793335e-08\\
0.343688679976369	3.75254905511503e-08\\
0.328961203518195	3.40522173138384e-08\\
0.314864817274691	3.08540488539489e-08\\
0.301372478265331	2.79243409296999e-08\\
0.28845830233408	2.52539483958922e-08\\
0.276097514492356	2.28316554272807e-08\\
0.264266401389864	2.06446134061032e-08\\
0.252942265822107	1.86787693865611e-08\\
0.242103383187309	1.69192708968656e-08\\
0.231728959809209	1.53508360726402e-08\\
0.22179909304577	1.3958081441711e-08\\
0.212294733107291	1.27258028545197e-08\\
0.203197646510645	1.16392078923588e-08\\
0.194490381099553	1.06841004682063e-08\\
0.186156232563786	9.84702020413948e-09\\
0.178179212393048	9.11534051913797e-09\\
0.170544017204091	8.47733022525526e-09\\
0.163235999382182	7.92218386920421e-09\\
0.156241138980632	7.44002614568422e-09\\
0.149546016824459	7.02189552832603e-09\\
0.143137788766591	6.6597118910759e-09\\
0.137004161047226	6.34623239627504e-09\\
0.131133366709066	6.07499936376003e-09\\
0.125514143023195	5.84028325412942e-09\\
0.120135709882281	5.63702333322449e-09\\
0.114987749119654	5.4607680575256e-09\\
0.110060384714592	5.30761674618672e-09\\
0.10534416384583	5.17416369217193e-09\\
0.10083003875695	5.0574455142631e-09\\
0.09650934939886	4.95489226151189e-09\\
0.0923738068160692	4.86428254737974e-09\\
0.0884154772448736	4.78370280625225e-09\\
0.0846267668929596	4.71151062350076e-09\\
0.081000407371217	4.64630198505307e-09\\
0.0775294417498175	4.58688221718447e-09\\
0.0742072112118075	4.53224033627619e-09\\
0.0710273422786095	4.48152649674045e-09\\
0.0679837345829273	4.43403220916697e-09\\
0.0650705491655965	4.38917299681799e-09\\
0.062282197273929	4.34647316445803e-09\\
0.0596133296400627	4.30555236735341e-09\\
0.0570588262187461	4.26611368881204e-09\\
0.0546137863648719	4.22793296087654e-09\\
0.0522735194319146	4.19084909391228e-09\\
0.0500335357732373	4.15475521601703e-09\\
0.0478895381290022	4.11959046144128e-09\\
0.0458374133821639	4.08533228732042e-09\\
0.0438732246677266	4.05198923845808e-09\\
0.0419932038201311	4.01959411886646e-09\\
0.040193744144279	3.98819756429153e-09\\
0.0384713934963279	3.95786204003532e-09\\
0.0368228476609827	3.92865631123264e-09\\
0.0352449440125781	3.90065044695988e-09\\
0.0337346554477915	3.87391142438191e-09\\
0.0322890845783461	3.84849939458315e-09\\
0.0309054581725643	3.82446465860738e-09\\
0.0295811218351067	3.8018453821484e-09\\
0.0283135349146912	3.78066605251087e-09\\
0.027100265630022	3.76093665448533e-09\\
0.0259389864045792	3.74265251532317e-09\\
0.0248274694013174	3.72579474549923e-09\\
0.0237635822487085	3.7103311833807e-09\\
0.0227452839499289	3.69621773958447e-09\\
0.0217706209673422	3.68340003127231e-09\\
0.0208377234747671	3.67181519774987e-09\\
0.0199448017703407	3.66139379574201e-09\\
0.0190901428430934	3.65206168440735e-09\\
0.0182721070866519	3.64374182507719e-09\\
0.017489125153763	3.63635593734972e-09\\
0.0167396949456054	3.62982597013784e-09\\
0.0160223787301124	3.62407536238252e-09\\
0.015335800383779	3.61903008251978e-09\\
0.0146786427516601	3.61461944786086e-09\\
0.014049645120496	3.61077673453962e-09\\
0.0134476008001185	3.60743959557971e-09\\
0.0128713548084953	3.60455030910941e-09\\
0.0123198016559737	3.60205588110754e-09\\
0.0117918832244729	3.59990802767566e-09\\
0.0112865867375536	3.59806306109445e-09\\
0.0108029428174749	3.59648170222058e-09\\
0.0103400236255064	3.59512883945535e-09\\
0.00989694108193215	3.59397325185607e-09\\
0.00947284516232805	3.59298731119559e-09\\
0.00906692226684686	3.59214667507326e-09\\
0.00867839365938074	3.59142998066334e-09\\
0.00830651397360789	3.59081854642363e-09\\
0.00795056978305666	3.59029608712033e-09\\
0.00760987823244437	3.58984844586148e-09\\
0.00728378572766474	3.58946334546409e-09\\
0.00697166668191105	3.58913016038668e-09\\
0.0066729223155294	3.5888397096102e-09\\
0.00638697950729974	3.58858407021363e-09\\
0.00611328969494088	3.5883564109328e-09\\
0.00585132782273016	3.58815084467967e-09\\
0.00560059133421896	3.5879622988068e-09\\
0.00536059920811159	3.58778640180034e-09\\
0.00513089103545778	3.58761938505358e-09\\
0.00491102613638878	3.58745799839281e-09\\
0.00470058271470227	3.58729943808182e-09\\
0.0044991570486744	3.58714128610804e-09\\
0.00430636271654645	3.58698145964231e-09\\
0.00412182985520036	3.5868181696562e-09\\
0.00394520445060095	3.58664988776945e-09\\
0.00377614765864346	3.58647532048035e-09\\
0.00361433515510367	3.58629338999856e-09\\
0.00345945651344342	3.58610322095103e-09\\
0.00331121460927795	3.58590413226331e-09\\
0.00316932505036243	3.58569563353078e-09\\
0.00303351563100441	3.58547742518483e-09\\
0.00290352580985524	3.58524940173014e-09\\
0.00277910621007882	3.58501165728142e-09\\
0.0026600181409387	3.58476449256504e-09\\
0.00254603313988575	3.58450842247941e-09\\
0.00243693253426797	3.58424418323304e-09\\
0.00233250702182148	3.58397273801222e-09\\
0.00223255626913808	3.58369528008178e-09\\
0.00213688852733889	3.58341323220479e-09\\
0.0020453202642169	3.58312824129507e-09\\
0.00195767581214256	3.5828421673022e-09\\
0.00187378703105716	3.58255706548359e-09\\
0.00179349298590728	3.58227516145021e-09\\
0.00171663963790156	3.58199881868258e-09\\
0.00164307954899755	3.58173049859619e-09\\
0.00157267159905159	3.58147271367659e-09\\
0.0015052807150892	3.58122797467946e-09\\
0.00144077761217658	3.58099873336792e-09\\
0.00137903854539598	3.58078732270026e-09\\
0.00131994507244939	3.58059589674212e-09\\
0.00126338382643477	3.58042637281949e-09\\
0.00120924629835926	3.58028037851727e-09\\
0.00115742862897182	3.58015920603958e-09\\
0.0011078314095162	3.58006377617739e-09\\
0.00106035949102183	3.57999461368546e-09\\
0.00101492180176684	3.57995183528517e-09\\
0.000971431172562998	3.57993515082705e-09\\
0.000929804169527454	3.57994387742279e-09\\
0.000889960934020339	3.57997696565225e-09\\
0.000851825029441286	3.58003303632449e-09\\
0.00081532329459089	3.58011042577158e-09\\
0.000780385703315802	3.58020723731427e-09\\
0.000746945230168215	3.58032139637659e-09\\
0.000714937721821986	3.58045070673955e-09\\
0.000684301773998746	3.58059290559577e-09\\
0.000654978613667872	3.58074571536588e-09\\
0.00062691198629433	3.58090689062676e-09\\
0.000600048047918088	3.5810742589392e-09\\
0.000574335261858056	3.58124575481124e-09\\
0.000549724299842386	3.58141944645803e-09\\
0.000526167947375462	3.58159355539585e-09\\
0.000503621013160024	3.58176646921836e-09\\
0.000482040242400666	3.58193674814071e-09\\
0.000461384233822389	3.58210312605866e-09\\
0.000441613360245002	3.58226450696286e-09\\
0.000422689692561006	3.58241995757986e-09\\
0.000404576926971133	3.58256869709348e-09\\
0.000387240315337902	3.58271008474522e-09\\
0.000370646598523641	3.58284360603172e-09\\
0.000354763942585032	3.58296885812114e-09\\
0.000339561877701809	3.5830855350085e-09\\
0.000325011239722429	3.58319341282866e-09\\
0.000311084114214591	3.58329233564977e-09\\
0.000297753782913245	3.58338220198386e-09\\
0.000284994672463378	3.58346295217577e-09\\
0.000272782305359236	3.58353455676818e-09\\
0.000261093252985847	3.58359700588947e-09\\
0.000249905090672786	3.58365029967038e-09\\
0.000239196354673932	3.58369443966587e-09\\
0.000228946500990699	3.58372942123749e-09\\
0.000219135865959735	3.58375522683825e-09\\
0.000209745628529495	3.58377182013551e-09\\
0.000200757774153304	3.58377914090547e-09\\
0.00019215506022964	3.58377710063606e-09\\
0.000183920983023357	3.58376557878144e-09\\
0.000176039746004357	3.58374441961955e-09\\
0.000168496229543005	3.58371342967608e-09\\
0.000161275961904116	3.58367237569054e-09\\
0.000154365091483898	3.58362098311394e-09\\
0.000147750360236569	3.58355893514311e-09\\
0.000141419078239674	3.58348587231224e-09\\
0.000135359099349317	3.58340139267856e-09\\
0.000129558797898586	3.58330505265606e-09\\
0.000124007046394487	3.58319636856794e-09\\
0.000118693194170584	3.58307481900513e-09\\
0.000113607046954409	3.58293984809429e-09\\
0.00010873884731042	3.58279086979367e-09\\
0.000104079255921017	3.58262727334789e-09\\
9.96193336696748e-05	3.58244843004314e-09\\
9.53505244918455e-05	3.58225370141006e-09\\
9.12646389607166e-05	3.58204244902215e-09\\
8.73538385763391e-05	3.58181404603169e-09\\
8.36106207279868e-05	3.58156789056864e-09\\
8.00278043008951e-05	3.58130342110241e-09\\
7.65985158997704e-05	3.58102013382577e-09\\
7.33161766626371e-05	3.58071760206455e-09\\
7.01744896397285e-05	3.58039549764424e-09\\
6.71674277132066e-05	3.58005361405172e-09\\
6.42892220345372e-05	3.57969189111817e-09\\
6.1534350957337e-05	3.57931044081758e-09\\
5.88975294444622e-05	3.578909573626e-09\\
5.63736989290155e-05	3.5784898247236e-09\\
5.39580176098217e-05	3.5780519791524e-09\\
5.16458511627539e-05	3.57759709487528e-09\\
4.94327638500902e-05	3.57712652252994e-09\\
4.7314510010846e-05	3.57664192054956e-09\\
4.52870259157553e-05	3.57614526424821e-09\\
4.33464219712761e-05	3.57563884746062e-09\\
4.14889752576631e-05	3.57512527540295e-09\\
3.97111223867946e-05	3.57460744759859e-09\\
3.80094526660478e-05	3.57408853000035e-09\\
3.63807015551128e-05	3.5735719158433e-09\\
3.48217444031878e-05	3.5730611752702e-09\\
3.33295904545452e-05	3.57255999436935e-09\\
3.19013771109644e-05	3.57207210491356e-09\\
3.05343644400281e-05	3.57160120675475e-09\\
2.92259299187434e-05	3.57115088544514e-09\\
2.79735634024062e-05	3.57072452817994e-09\\
2.67748623090547e-05	3.57032524151469e-09\\
2.56275270102763e-05	3.56995577447114e-09\\
2.45293564195224e-05	3.56961845056314e-09\\
2.34782437694709e-05	3.56931511194387e-09\\
2.24721725703323e-05	3.56904707830645e-09\\
2.15092127413488e-05	3.56881512240882e-09\\
2.0587516908062e-05	3.5686194631809e-09\\
1.97053168582477e-05	3.56845977640705e-09\\
1.88609201497187e-05	3.56833522201684e-09\\
1.80527068634865e-05	3.56824448616273e-09\\
1.72791264960546e-05	3.56818583555539e-09\\
1.65386949848802e-05	3.56815718104603e-09\\
1.5829991861299e-05	3.56815614717631e-09\\
1.51516575254506e-05	3.56818014439983e-09\\
1.45023906379763e-05	3.56822644085682e-09\\
1.38809456234861e-05	3.56829223094851e-09\\
1.32861302810047e-05	3.56837469843763e-09\\
1.27168034968139e-05	3.56847107235788e-09\\
1.21718730553009e-05	3.56857867458802e-09\\
1.16502935436156e-05	3.56869495850225e-09\\
1.11510643461157e-05	3.5688175386e-09\\
1.06732277247515e-05	3.56894421143199e-09\\
1.02158669817098e-05	3.56907296846482e-09\\
9.77810470078945e-06	3.56920200174561e-09\\
9.3591010641368e-06	3.56932970337407e-09\\
8.95805224111118e-06	3.56945465983885e-09\\
8.57418884618897e-06	3.56957564227688e-09\\
8.20677446294867e-06	3.56969159364969e-09\\
7.85510423130492e-06	3.56980161374795e-09\\
7.51850349528125e-06	3.56990494281647e-09\\
7.19632650872741e-06	3.57000094447322e-09\\
6.88795519649824e-06	3.57008908846929e-09\\
6.59279796871769e-06	3.57016893372195e-09\\
6.31028858585273e-06	3.57024011193958e-09\\
6.03988507242068e-06	3.57030231206807e-09\\
5.78106867724506e-06	3.57035526570088e-09\\
5.53334287826595e-06	3.57039873353215e-09\\
5.29623242999553e-06	3.57043249287708e-09\\
5.069282451791e-06	3.57045632624186e-09\\
4.85205755519644e-06	3.57047001089644e-09\\
4.64414100867889e-06	3.57047330938282e-09\\
4.44513393815662e-06	3.57046596087772e-09\\
4.25465456178573e-06	3.57044767332168e-09\\
4.07233745753699e-06	3.57041811622965e-09\\
3.89783286215799e-06	3.57037691409612e-09\\
3.73080600017557e-06	3.57032364031684e-09\\
3.57093644165134e-06	3.57025781156158e-09\\
3.41791748745804e-06	3.5701788825353e-09\\
3.27145558089777e-06	3.5700862410909e-09\\
3.13126974453287e-06	3.56997920365673e-09\\
2.99709104114941e-06	3.56985701097139e-09\\
2.86866205781901e-06	3.5697188241274e-09\\
2.74573641206923e-06	3.56956372094669e-09\\
2.62807827921514e-06	3.56939069273386e-09\\
2.51546193994543e-06	3.56919864146906e-09\\
2.40767134729475e-06	3.56898637753202e-09\\
2.30449971217211e-06	3.5687526180727e-09\\
2.20574910664964e-06	3.56849598616612e-09\\
2.11123008425107e-06	3.56821501092572e-09\\
2.02076131651122e-06	3.56790812877539e-09\\
1.9341692451094e-06	3.56757368611012e-09\\
1.85128774890932e-06	3.56720994361406e-09\\
1.77195782526667e-06	3.56681508252708e-09\\
1.69602728499317e-06	3.56638721318821e-09\\
1.62335046039168e-06	3.56592438620912e-09\\
1.55378792580244e-06	3.56542460664628e-09\\
1.48720623012418e-06	3.56488585156674e-09\\
1.42347764079703e-06	3.56430609137995e-09\\
1.36247989875613e-06	3.56368331532105e-09\\
1.30409598388571e-06	3.56301556141466e-09\\
1.24821389052379e-06	3.56230095120436e-09\\
1.19472641258672e-06	3.56153772942017e-09\\
1.14353093790149e-06	3.56072430866721e-09\\
1.09452925135104e-06	3.55985931901961e-09\\
1.04762734645512e-06	3.55894166220025e-09\\
1.00273524502508e-06	3.55797056979538e-09\\
9.59766824546685e-07	3.55694566462184e-09\\
9.18639652959829e-07	3.55586702404193e-09\\
8.79274830518072e-07	3.55473524363524e-09\\
8.41596838424732e-07	3.55355149927057e-09\\
8.05533393955084e-07	3.552317605208e-09\\
7.71015311786762e-07	3.55103606549972e-09\\
7.37976371272303e-07	3.54971011574396e-09\\
7.06353189399253e-07	3.54834375194514e-09\\
6.76085099194042e-07	3.54694174329091e-09\\
6.47114033336454e-07	3.54550962578585e-09\\
6.1938441276131e-07	3.54405367400132e-09\\
5.92843040033731e-07	3.54258084888918e-09\\
5.67438997293396e-07	3.54109872039957e-09\\
5.43123548572002e-07	3.53961536479241e-09\\
5.19850046296559e-07	3.53813923781047e-09\\
4.97573841799104e-07	3.53667902624688e-09\\
4.76252199661216e-07	3.53524348194816e-09\\
4.55844215728935e-07	3.53384124359811e-09\\
4.36310738640879e-07	3.53248065271213e-09\\
4.17614294718943e-07	3.53116957130287e-09\\
3.99719016077547e-07	3.52991520862416e-09\\
3.82590571813479e-07	3.52872396480493e-09\\
3.66196102144329e-07	3.52760129806808e-09\\
3.50504155369194e-07	3.52655162142949e-09\\
3.35484627530665e-07	3.52557823296744e-09\\
3.21108704662397e-07	3.52468328210309e-09\\
3.07348807511418e-07	3.52386777223046e-09\\
2.94178538629174e-07	3.52313159812232e-09\\
2.81572631729769e-07	3.52247361479889e-09\\
2.69506903218282e-07	3.52189173301788e-09\\
2.57958205796139e-07	3.5213830356062e-09\\
2.46904384054565e-07	3.52094390821086e-09\\
2.36324231970901e-07	3.52057017803325e-09\\
2.26197452226259e-07	3.52025725421085e-09\\
2.16504617266463e-07	3.52000026442044e-09\\
2.07227132031577e-07	3.51979418305549e-09\\
1.98347198282523e-07	3.51963394726844e-09\\
1.89847780456333e-07	3.51951455853877e-09\\
1.81712572984561e-07	3.51943116818116e-09\\
1.73925969012127e-07	3.51937914635927e-09\\
1.66473030456608e-07	3.51935413503564e-09\\
1.59339459350516e-07	3.51935208561294e-09\\
1.52511570411594e-07	3.51936928280437e-09\\
1.45976264788522e-07	3.51940235656483e-09\\
1.39721004931627e-07	3.51944828363361e-09\\
1.33733790540439e-07	3.51950438094569e-09\\
1.28003135541904e-07	3.51956829240234e-09\\
1.22518046055117e-07	3.51963797095376e-09\\
1.17267999300297e-07	3.51971165717377e-09\\
1.12242923411527e-07	3.51978785594952e-09\\
1.07433178114552e-07	3.51986531204194e-09\\
1.02829536232551e-07	3.51994298566202e-09\\
9.84231659844124e-08	3.52002002841895e-09\\
9.42056140415492e-08	3.52009576047044e-09\\
9.01687893107487e-08	3.52016964894979e-09\\
8.63049474119481e-08	3.52024128809827e-09\\
8.2606675821156e-08	3.52031038102914e-09\\
7.90668796500171e-08	3.52037672340393e-09\\
7.56787680347409e-08	3.52044018880056e-09\\
7.24358411082795e-08	3.52050071579234e-09\\
6.93318775307662e-08	3.52055829668973e-09\\
6.63609225542869e-08	3.52061296773959e-09\\
6.35172765990922e-08	3.52066480068071e-09\\
6.07954843193326e-08	3.52071389569183e-09\\
5.81903241373392e-08	3.52076037524954e-09\\
5.56967982263749e-08	3.5208043791803e-09\\
5.33101229226347e-08	3.52084606044838e-09\\
5.10257195481057e-08	3.52088558183463e-09\\
4.88392056266758e-08	3.52092311320188e-09\\
4.67463864766467e-08	3.52095882940199e-09\\
4.47432471635135e-08	3.52099290867288e-09\\
4.28259447975808e-08	3.5210255314475e-09\\
4.09908011616345e-08	3.5210568796185e-09\\
3.92342956545249e-08	3.52108713599137e-09\\
3.75530585371291e-08	3.52111648417644e-09\\
3.59438644677287e-08	3.52114510857322e-09\\
3.44036263144073e-08	3.5211731947075e-09\\
3.29293892325921e-08	3.52120092957833e-09\\
3.15183249963816e-08	3.52122850232848e-09\\
3.01677265727814e-08	3.52125610493491e-09\\
2.88750029284419e-08	3.52128393313689e-09\\
2.76376740589325e-08	3.52131218731488e-09\\
2.64533662310179e-08	3.5213410736585e-09\\
2.5319807428809e-08	3.52137080529369e-09\\
2.42348229950507e-08	3.52140160356946e-09\\
2.3196331459187e-08	3.52143369935505e-09\\
2.22023405441976e-08	3.52146733457521e-09\\
2.12509433445473e-08	3.52150276349699e-09\\
2.03403146679137e-08	3.52154025440023e-09\\
1.94687075336777e-08	3.52158009118795e-09\\
1.86344498214568e-08	3.52162257480661e-09\\
1.78359410632532e-08	3.52166802499172e-09\\
1.70716493730629e-08	3.52171678174299e-09\\
1.6340108508053e-08	3.52176920682401e-09\\
1.56399150556737e-08	3.52182568513565e-09\\
1.49697257413031e-08	3.52188662582405e-09\\
1.43282548512653e-08	3.52195246322435e-09\\
1.37142717662732e-08	3.52202365753037e-09\\
1.31265986005679e-08	3.52210069476605e-09\\
1.25641079422225e-08	3.522184086408e-09\\
1.20257206902776e-08	3.522274368346e-09\\
1.15104039845577e-08	3.52237209876927e-09\\
1.10171692241975e-08	3.52247785532144e-09\\
1.0545070171077e-08	3.52259223101059e-09\\
1.00932011345262e-08	3.52271582879415e-09\\
9.66069523381808e-09	3.52284925478369e-09\\
9.24672273511538e-09	3.52299311001837e-09\\
8.85048945968121e-09	3.52314798046651e-09\\
8.47123526030015e-09	3.52331442562582e-09\\
8.10823256298608e-09	3.52349296559187e-09\\
7.76078497118007e-09	3.52368406631489e-09\\
7.42822592975961e-09	3.52388812470293e-09\\
7.10991744629701e-09	3.52410545201923e-09\\
6.80524886711335e-09	3.52433625779527e-09\\
6.51363570577991e-09	3.5245806335092e-09\\
6.23451852181971e-09	3.52483853769415e-09\\
5.9673618474583e-09	3.52510978285615e-09\\
5.71165316036422e-09	3.525394024768e-09\\
5.46690190040913e-09	3.52569075485994e-09\\
5.23263852856062e-09	3.5259992972851e-09\\
5.00841362610294e-09	3.52631880917946e-09\\
4.79379703245691e-09	3.52664828687707e-09\\
4.58837701994549e-09	3.52698657659029e-09\\
4.39175950392161e-09	3.52733238944723e-09\\
4.20356728674293e-09	3.5276843216994e-09\\
4.02343933414316e-09	3.52804087737414e-09\\
3.85103008261192e-09	3.52840049561696e-09\\
3.68600877645401e-09	3.52876157743497e-09\\
3.52805883325663e-09	3.52912251631358e-09\\
3.37687723654702e-09	3.52948172561203e-09\\
3.23217395447556e-09	3.52983766639816e-09\\
3.09367138340881e-09	3.53018887222157e-09\\
2.96110381536551e-09	3.53053397217824e-09\\
2.83421692827338e-09	3.53087170764059e-09\\
2.71276729806916e-09	3.53120094771008e-09\\
2.59652193170571e-09	3.53152069990002e-09\\
2.48525782017036e-09	3.53183011488401e-09\\
2.37876151065686e-09	3.5321284898112e-09\\
2.27682869707039e-09	3.53241526736277e-09\\
2.17926382807992e-09	3.53269003031241e-09\\
2.08587973196595e-09	3.53295249585117e-09\\
1.99649725754397e-09	3.53320250548594e-09\\
1.91094493047486e-09	3.53344001565657e-09\\
1.82905862430274e-09	3.53366508570328e-09\\
1.75068124558927e-09	3.53387786591139e-09\\
1.67566243254033e-09	3.53407858562274e-09\\
1.6038582665469e-09	3.53426754172018e-09\\
1.53513099608679e-09	3.53444508626348e-09\\
1.46934877245744e-09	3.53461161655806e-09\\
1.406385396833e-09	3.53476756459102e-09\\
1.3461200781603e-09	3.53491338862801e-09\\
1.28843720142911e-09	3.53504956427041e-09\\
1.23322610587255e-09	3.53517657850778e-09\\
1.1803808726717e-09	3.53529492215602e-09\\
1.12980012175739e-09	3.53540508631649e-09\\
1.08138681731929e-09	3.53550755604554e-09\\
1.03504808164913e-09	3.53560280831332e-09\\
9.90695016961016e-10	3.53569130851266e-09\\
9.48242534846892e-10	3.53577350808011e-09\\
9.07609193040121e-10	3.53584984317723e-09\\
8.68717039173893e-10	3.53592073342523e-09\\
8.31491461234789e-10	3.53598658092667e-09\\
7.95861044424582e-10	3.53604776998743e-09\\
7.61757434155697e-10	3.53610466794898e-09\\
7.29115204917482e-10	3.53615762375646e-09\\
6.97871734761709e-10	3.53620696954798e-09\\
6.67967085166515e-10	3.53625302143122e-09\\
6.39343886048342e-10	3.53629607933777e-09\\
6.11947225701246e-10	3.53633642830621e-09\\
5.85724545452422e-10	3.536374339935e-09\\
5.60625538831901e-10	3.53641007295942e-09\\
5.36602055062909e-10	3.5364438739737e-09\\
5.13608006687821e-10	3.53647597883553e-09\\
4.91599281152417e-10	3.53650661446972e-09\\
4.70533656178892e-10	3.53653599928014e-09\\
4.50370718765214e-10	3.53656434454809e-09\\
4.31071787655462e-10	3.53659185548094e-09\\
4.12599839132411e-10	3.53661873218482e-09\\
3.94919435989987e-10	3.53664517261629e-09\\
3.77996659549349e-10	3.53667137132881e-09\\
3.61799044588146e-10	3.53669752164712e-09\\
3.4629551705815e-10	3.5367238188467e-09\\
3.31456334471758e-10	3.53675045768351e-09\\
3.17253028843008e-10	3.53677763706902e-09\\
3.03658352073639e-10	3.5368055595491e-09\\
2.90646223679419e-10	3.53683443285062e-09\\
2.78191680756474e-10	3.53686447296239e-09\\
2.66270830091615e-10	3.53689590180613e-09\\
2.54860802324793e-10	3.53692895219807e-09\\
2.43939708075753e-10	3.53696386926613e-09\\
2.33486595950714e-10	3.5370009076456e-09\\
2.23481412348509e-10	3.53704033871012e-09\\
2.13904962989081e-10	3.53708244824313e-09\\
2.04738876090539e-10	3.53712753835207e-09\\
1.95965567124111e-10	3.53717592833231e-09\\
1.87568205079392e-10	3.53722796153795e-09\\
1.79530680175171e-10	3.53728399871327e-09\\
1.71837572953887e-10	3.53734442016094e-09\\
1.6447412470042e-10	3.53740963686921e-09\\
1.57426209128483e-10	3.537480075853e-09\\
1.50680305280271e-10	3.53755619002996e-09\\
1.44223471587412e-10	3.5376384579616e-09\\
1.3804332104342e-10	3.53772738364244e-09\\
1.3212799744005e-10	3.53782348905609e-09\\
1.26466152621949e-10	3.53792731609274e-09\\
1.21046924715974e-10	3.53803942863284e-09\\
1.1585991729341e-10	3.5381604013965e-09\\
1.10895179425109e-10	3.53829081884542e-09\\
1.06143186591302e-10	3.53843127146706e-09\\
1.01594822409431e-10	3.5385823387084e-09\\
9.72413611449809e-11	3.53874459154333e-09\\
9.30744509717331e-11	3.53891857465329e-09\\
8.90860979493466e-11	3.5391048076333e-09\\
8.52686506875113e-11	3.53930373996047e-09\\
8.16147856672639e-11	3.53951577820421e-09\\
7.81174931913052e-11	3.53974124296064e-09\\
7.47700639363597e-11	3.53998037480943e-09\\
7.15660760817853e-11	3.54023327656394e-09\\
6.84993829897377e-11	3.54049995749211e-09\\
6.55641014132536e-11	3.54078027176873e-09\\
6.27546002096312e-11	3.54107393554901e-09\\
6.00654895374583e-11	3.54138050684705e-09\\
5.74916105165589e-11	3.54169936721843e-09\\
5.50280253310261e-11	3.54202975567851e-09\\
5.26700077563472e-11	3.54237072502911e-09\\
5.0413034092458e-11	3.54272120319627e-09\\
4.8252774485325e-11	3.54307993308545e-09\\
4.61850846204068e-11	3.54344557590602e-09\\
4.42059977720632e-11	3.54381667655606e-09\\
4.23117171936544e-11	3.54419166162477e-09\\
4.0498608833736e-11	3.54456894808877e-09\\
3.87631943643718e-11	3.544946900718e-09\\
3.71021445081933e-11	3.54532390560099e-09\\
3.55122726514019e-11	3.54569835041405e-09\\
3.39905287304624e-11	3.54606871027134e-09\\
3.25339933807581e-11	3.54643351252763e-09\\
3.11398723359843e-11	3.54679140263916e-09\\
2.98054910675344e-11	3.54714114821304e-09\\
2.85282896535931e-11	3.54748164089417e-09\\
2.73058178680978e-11	3.54781191037298e-09\\
2.61357304801417e-11	3.54813114731876e-09\\
2.5015782754805e-11	3.5484386754428e-09\\
2.39438261467796e-11	3.5487339762243e-09\\
2.29178041785276e-11	3.54901665951152e-09\\
2.19357484950656e-11	3.54928649539249e-09\\
2.09957750878073e-11	3.54954335655324e-09\\
2.0096080680217e-11	3.54978724388052e-09\\
1.92349392683444e-11	3.55001823787482e-09\\
1.84106988096001e-11	3.55023650960195e-09\\
1.76217780534217e-11	3.5504423349683e-09\\
1.68666635077498e-11	3.55063606181363e-09\\
1.61439065354938e-11	3.55081804323027e-09\\
1.54521205754186e-11	3.55098873302569e-09\\
1.47899784821178e-11	3.55114858456192e-09\\
1.41562099799743e-11	3.55129804931722e-09\\
1.35495992262207e-11	3.55143767035143e-09\\
1.29689824784256e-11	3.55156793835266e-09\\
1.24132458619312e-11	3.55168936448897e-09\\
1.18813232329587e-11	3.551802471857e-09\\
1.13721941332823e-11	3.55190776477678e-09\\
1.08848818325477e-11	3.55200574861795e-09\\
1.04184514544803e-11	3.55209688703972e-09\\
9.97200818338669e-12	3.55218167041131e-09\\
9.54469554751041e-12	3.55226056450096e-09\\
9.13569377594766e-12	3.55233394959543e-09\\
8.74421822597136e-12	3.55240231815531e-09\\
8.36951787774638e-12	3.55246605620838e-09\\
8.01087389354751e-12	3.55252551468125e-09\\
7.6675982387173e-12	3.55258107090933e-09\\
7.33903236171716e-12	3.5526331082256e-09\\
7.02454593074013e-12	3.55268194682429e-09\\
6.72353562446106e-12	3.55272789328988e-09\\
6.43542397460471e-12	3.5527712611014e-09\\
6.15965825811127e-12	3.55281236018826e-09\\
5.89570943677399e-12	3.55285144319933e-09\\
5.6430711423143e-12	3.5528888108214e-09\\
5.40125870494815e-12	3.55292469671035e-09\\
5.16980822357908e-12	3.55295937661159e-09\\
4.9482756758348e-12	3.55299309839442e-09\\
4.7362360662398e-12	3.55302609301841e-09\\
4.53328261088971e-12	3.55305860386302e-09\\
4.33902595706359e-12	3.55309084056639e-09\\
4.15309343627631e-12	3.55312308536969e-09\\
3.97512834933902e-12	3.55315552948928e-09\\
3.8047892820554e-12	3.553188426361e-09\\
3.64174945024121e-12	3.55322201173945e-09\\
3.48569607281055e-12	3.55325652622593e-09\\
3.33632977172609e-12	3.55329222577735e-09\\
3.19336399766226e-12	3.55332936948988e-09\\
3.05652448027929e-12	3.55336817420458e-09\\
2.92554870205395e-12	3.55340901047374e-09\\
2.80018539465695e-12	3.55345206872987e-09\\
2.6801940569115e-12	3.55349765888961e-09\\
2.56534449340765e-12	3.55354611226356e-09\\
2.45541637288774e-12	3.55359781568999e-09\\
2.35019880555556e-12	3.55365303750432e-09\\
2.24948993849824e-12	3.55371215510922e-09\\
2.15309656844482e-12	3.55377556933513e-09\\
2.06083377111868e-12	3.5538436184365e-09\\
1.97252454647256e-12	3.55391684820112e-09\\
1.88799947912573e-12	3.55399555100706e-09\\
1.80709641335184e-12	3.55408033013243e-09\\
1.72966014199393e-12	3.55417163319598e-09\\
1.65554210870983e-12	3.55426991649755e-09\\
1.58460012297664e-12	3.55437572461926e-09\\
1.51669808730773e-12	3.55448961406606e-09\\
1.45170573615866e-12	3.55461224786563e-09\\
1.38949838602149e-12	3.5547441413298e-09\\
1.32995669622766e-12	3.5548858952053e-09\\
1.27296644000091e-12	3.5550380808667e-09\\
1.21841828532079e-12	3.55520132843459e-09\\
1.16620758517639e-12	3.55537633576091e-09\\
1.11623417680807e-12	3.55556351225424e-09\\
1.0684021895518e-12	3.55576380905042e-09\\
1.02261986091774e-12	3.55597736131296e-09\\
9.78799360549895e-13	3.55620515195368e-09\\
9.36856621729499e-13	3.55644723683268e-09\\
8.96711180098561e-13	3.5567043552308e-09\\
8.58286019294336e-13	3.55697661499775e-09\\
8.21507423198572e-13	3.55726472070613e-09\\
7.86304834518011e-13	3.55756814171145e-09\\
7.52610719424931e-13	3.5578877577317e-09\\
7.20360437997967e-13	3.55822280909027e-09\\
6.89492120214718e-13	3.55857392044073e-09\\
6.59946547258234e-13	3.55893974183874e-09\\
6.31667037909637e-13	3.55932052232186e-09\\
6.04599339808965e-13	3.55971562814528e-09\\
5.78691525375633e-13	3.56012389553489e-09\\
5.53893892188816e-13	3.56054457596204e-09\\
5.30158867636661e-13	3.56097645266114e-09\\
5.07440917651378e-13	3.56141826212107e-09\\
4.85696459355168e-13	3.56186838396761e-09\\
4.64883777449371e-13	3.56232513505731e-09\\
4.44962944186419e-13	3.56278801386024e-09\\
4.25895742771128e-13	3.56325403915971e-09\\
4.07645594044291e-13	3.56372144269079e-09\\
3.90177486308015e-13	3.56418905149655e-09\\
3.73457908158087e-13	3.56465505394392e-09\\
3.57454784194573e-13	3.56511693442193e-09\\
3.42137413487313e-13	3.56557396168697e-09\\
3.27476410678197e-13	3.56602364243782e-09\\
3.13443649607329e-13	3.56646467439681e-09\\
3.00012209354848e-13	3.56689651127628e-09\\
2.8715632259494e-13	3.56731566259116e-09\\
2.74851326162926e-13	3.56772301796763e-09\\
2.63073613740624e-13	3.56811764523526e-09\\
2.51800590569193e-13	3.56849726999928e-09\\
2.41010630102594e-13	3.56886269113447e-09\\
2.30683032518497e-13	3.56921277044566e-09\\
2.20797985007039e-13	3.56954731182204e-09\\
2.11336523761278e-13	3.56986596360511e-09\\
2.02280497596376e-13	3.57016858087818e-09\\
1.93612533127766e-13	3.57045563504986e-09\\
1.85316001441465e-13	3.57072670024381e-09\\
1.77374986192608e-13	3.57098474902994e-09\\
1.69774253070993e-13	3.57122434676408e-09\\
1.62499220575075e-13	3.57145219831098e-09\\
1.55535932038321e-13	3.57166507401443e-09\\
1.48871028854277e-13	3.57186467556532e-09\\
1.4249172484896e-13	3.57205115277707e-09\\
1.36385781751439e-13	3.57222533027612e-09\\
1.30541485715526e-13	3.57238787189903e-09\\
1.24947624847537e-13	3.57253881648802e-09\\
1.19593467697021e-13	3.57268050499674e-09\\
1.14468742669183e-13	3.57281095775327e-09\\
1.09563618319516e-13	3.57293242102088e-09\\
1.04868684492821e-13	3.57304527068278e-09\\
1.00374934270457e-13	3.57315023212035e-09\\
9.60737466911612e-14	3.57324646571108e-09\\
9.19568702123086e-14	3.5733365716153e-09\\
8.80164068798757e-14	3.57342049726261e-09\\
8.42447971767404e-14	3.5734974724259e-09\\
8.06348055202524e-14	3.57356853967011e-09\\
7.71795063812449e-14	3.57363469290572e-09\\
7.387227099787e-14	3.5736966177741e-09\\
7.07067546587579e-14	3.5737532658144e-09\\
6.76768845311109e-14	3.57380625557263e-09\\
6.47768480103765e-14	3.57385628877033e-09\\
6.20010815691507e-14	3.57390056910096e-09\\
5.93442600839223e-14	3.5739434255943e-09\\
5.68012866191757e-14	3.57398269999541e-09\\
5.43672826492588e-14	3.57402137214753e-09\\
5.20375786992568e-14	3.5740544211038e-09\\
4.98077053869139e-14	3.57408677303993e-09\\
4.767338484842e-14	3.57411900459527e-09\\
4.56305225316138e-14	3.57414925528323e-09\\
4.36751993408563e-14	3.57417580018379e-09\\
4.1803664118507e-14	3.57420139795472e-09\\
4.00123264485755e-14	3.57422703476739e-09\\
3.82977497687479e-14	3.57425366620873e-09\\
3.66566447775708e-14	3.57427776745451e-09\\
3.5085863124144e-14	3.57430050160686e-09\\
3.35823913682192e-14	3.5743243440863e-09\\
3.21433451991144e-14	3.57434842615437e-09\\
3.07659639023564e-14	3.57437251476941e-09\\
2.94476050634323e-14	3.57439448533395e-09\\
2.81857394984939e-14	3.57441794542088e-09\\
2.69779464022858e-14	3.57444230223042e-09\\
2.58219087039919e-14	3.57446923045133e-09\\
2.47154086220884e-14	3.57449543614219e-09\\
2.36563234096775e-14	3.57452496881955e-09\\
2.26426212821389e-14	3.57455232301014e-09\\
2.16723575192853e-14	3.5745819544271e-09\\
2.07436707345464e-14	3.57461577113751e-09\\
1.98547793040222e-14	3.57464687473916e-09\\
1.90039779485561e-14	3.5746869267225e-09\\
1.81896344622701e-14	3.57472298701166e-09\\
1.74101865812858e-14	3.57476996499632e-09\\
1.66641389866255e-14	3.57481670166359e-09\\
1.59500604355422e-14	3.57486093645094e-09\\
1.52665810157747e-14	3.57491272777598e-09\\
1.46123895174626e-14	3.57497712733564e-09\\
1.39862309176772e-14	3.57503329572357e-09\\
1.33869039727431e-14	3.57510217999569e-09\\
1.28132589137322e-14	3.57516804576304e-09\\
1.22641952407085e-14	3.57525389285946e-09\\
1.17386596114919e-14	3.57534336379613e-09\\
1.12356438208914e-14	3.5754398767749e-09\\
1.075418286653e-14	3.57553244328868e-09\\
1.02933530975525e-14	3.57563361040539e-09\\
9.85227044266167e-15	3.57575086026712e-09\\
9.43008871408732e-15	3.57586969651675e-09\\
9.02599798423036e-15	3.57599699769787e-09\\
8.6392230318711e-15	3.57615226661412e-09\\
8.26902185495849e-15	3.57631601854746e-09\\
7.9146842471286e-15	3.57648625131573e-09\\
7.575530435221e-15	3.57664381048514e-09\\
7.25090977517899e-15	3.57685031284494e-09\\
6.94019950383216e-15	3.57709506163953e-09\\
6.64280354416668e-15	3.57728069262466e-09\\
6.35815136179129e-15	3.57755571983404e-09\\
6.08569687040471e-15	3.57781516931736e-09\\
5.82491738416551e-15	3.57807778904993e-09\\
5.57531261495399e-15	3.57843173177173e-09\\
5.33640371260275e-15	3.57874134711835e-09\\
5.10773234625436e-15	3.57901613844488e-09\\
4.88885982508408e-15	3.5794310654303e-09\\
4.67936625670064e-15	3.57980258094989e-09\\
4.47884974161067e-15	3.58017015780738e-09\\
4.28692560220111e-15	3.58060087618924e-09\\
4.10322564476083e-15	3.58106469996426e-09\\
3.92739745312546e-15	3.58151666626732e-09\\
3.75910371259035e-15	3.58196787964028e-09\\
3.59802156279474e-15	3.58251642508333e-09\\
3.44384197833559e-15	3.58298328232164e-09\\
3.29626917592297e-15	3.58351715146049e-09\\
3.1550200469393e-15	3.58414160916706e-09\\
3.0198236143144e-15	3.58468283877417e-09\\
2.89042051267394e-15	3.58527292281542e-09\\
2.76656249076423e-15	3.58593752359628e-09\\
2.64801193519866e-15	3.58655706800071e-09\\
2.53454141461216e-15	3.5870844482225e-09\\
2.42593324334934e-15	3.58787936145135e-09\\
2.32197906384891e-15	3.58851655269407e-09\\
2.22247944692361e-15	3.58922082465885e-09\\
2.12724350916856e-15	3.58997762552305e-09\\
2.0360885467641e-15	3.59074040148445e-09\\
1.94883968497067e-15	3.59153394868369e-09\\
1.86532954264321e-15	3.59238068395599e-09\\
1.78539791112161e-15	3.59324298857721e-09\\
1.70889144688099e-15	3.59407833398479e-09\\
1.63566337735236e-15	3.59496620878892e-09\\
1.56557321934916e-15	3.59595411623834e-09\\
1.49848650955965e-15	3.59706693633574e-09\\
1.43427454658796e-15	3.59807268201679e-09\\
1.372814144049e-15	3.5992546418764e-09\\
1.31398739424356e-15	3.60018949802798e-09\\
1.25768144196026e-15	3.60144051393315e-09\\
1.20378826797028e-15	3.60306783591256e-09\\
1.15220448179967e-15	3.60432402627113e-09\\
1.1028311233816e-15	3.60572287027996e-09\\
1.05557347320801e-15	3.60736201121411e-09\\
1.01034087061659e-15	3.60923353172512e-09\\
9.67046539864225e-16	3.61093500498839e-09\\
9.25607423653624e-16	3.61302805492102e-09\\
8.859440237933e-16	3.61521658616474e-09\\
8.47980248685628e-16	3.61732134029158e-09\\
8.11643267350153e-16	3.62002361391742e-09\\
7.76863369702207e-16	3.62282058356661e-09\\
7.4357383281873e-16	3.62568149560492e-09\\
7.11710792934769e-16	3.62885959012373e-09\\
6.81213122925109e-16	3.63219192914995e-09\\
6.52022315035922e-16	3.63600371279828e-09\\
6.24082368641537e-16	3.6397809706968e-09\\
5.9733968281097e-16	3.64499880280498e-09\\
5.71742953478085e-16	3.6487212640483e-09\\
5.47243075018154e-16	3.65403864127827e-09\\
5.23793046041979e-16	3.65930991046534e-09\\
5.01347879226855e-16	3.66552131811054e-09\\
4.79864515011378e-16	3.67122991784393e-09\\
4.59301738988528e-16	3.67818663787687e-09\\
4.39620102838575e-16	3.68561496152907e-09\\
4.20781848650102e-16	3.69389654135104e-09\\
4.02750836483954e-16	3.70301258410976e-09\\
3.85492475041165e-16	3.71403799298346e-09\\
3.68973655301851e-16	3.72429872330123e-09\\
3.5316268700776e-16	3.73751883925171e-09\\
3.38029237866606e-16	3.74836856900232e-09\\
3.23544275361594e-16	3.76147925240761e-09\\
3.09680011054452e-16	3.77681126397859e-09\\
2.9640984727517e-16	3.794262760161e-09\\
2.83708326096131e-16	3.80705269844932e-09\\
2.71551080492765e-16	3.82853254487478e-09\\
2.59914787597043e-16	3.84673226065869e-09\\
2.48777123954092e-16	3.86906598211309e-09\\
2.38116722696134e-16	3.89502662567459e-09\\
2.2791313255157e-16	3.91761092169202e-09\\
2.18146778610576e-16	3.95075668301448e-09\\
2.08798924771937e-16	3.98001644360165e-09\\
1.99851637799081e-16	4.01146610977161e-09\\
1.91287752916355e-16	4.04613759452616e-09\\
1.83090840879548e-16	4.08474781340948e-09\\
1.75245176457471e-16	4.12767757924113e-09\\
1.67735708264153e-16	4.17465859886011e-09\\
1.6054802988375e-16	4.23215221677067e-09\\
1.53668352232796e-16	4.29396198080072e-09\\
1.47083477106765e-16	4.35468205484265e-09\\
1.40780771860187e-16	4.44481942986951e-09\\
1.34748145171764e-16	4.4916079715703e-09\\
1.28974023847965e-16	4.58750149680286e-09\\
1.23447330620631e-16	4.66333479212214e-09\\
1.18157462895965e-16	4.77384752291611e-09\\
1.13094272414167e-16	4.85661004617343e-09\\
1.08248045780667e-16	4.9864715953911e-09\\
1.03609485831623e-16	5.15443042567228e-09\\
9.91696937979324e-17	5.2567141822741e-09\\
9.4920152233532e-17	5.44650483708616e-09\\
9.08527086752466e-17	5.60798783247383e-09\\
8.69595600028256e-17	5.84571821631844e-09\\
8.32332374691799e-17	6.15138306253079e-09\\
7.96665923720847e-17	6.5111363427398e-09\\
7.62527823398666e-17	6.54406537192496e-09\\
7.29852582047751e-17	7.00795310251535e-09\\
6.98577514388305e-17	7.20221709774393e-09\\
6.6864262128077e-17	7.81156854470096e-09\\
6.39990474621409e-17	8.44219117062138e-09\\
6.12566107170345e-17	8.62782530142502e-09\\
5.86316907100539e-17	8.99785912761129e-09\\
5.61192517065506e-17	1.01179994168795e-08\\
5.37144737592075e-17	1.03596844501131e-08\\
5.14127434612926e-17	1.1337993320055e-08\\
4.92096450961429e-17	1.19996923215469e-08\\
4.71009521659061e-17	1.28965162630107e-08\\
4.50826192832852e-17	1.68288382625841e-08\\
4.31507744107304e-17	1.56038947317475e-08\\
4.13017114321949e-17	1.88392903735227e-08\\
3.95318830431955e-17	1.84683251506295e-08\\
3.78378939455446e-17	2.04846661450413e-08\\
3.62164943336975e-17	2.30545273816748e-08\\
3.46645736602155e-17	2.50969959613006e-08\\
3.31791546683866e-17	2.81830812682921e-08\\
3.17573876805579e-17	3.15541632264872e-08\\
3.03965451312173e-17	3.56791745953265e-08\\
2.90940163343405e-17	3.96186023494106e-08\\
2.78473024749628e-17	5.52262202388843e-08\\
2.66540118153697e-17	5.07085006446605e-08\\
2.5511855106706e-17	6.01360727345703e-08\\
2.44186411972047e-17	6.60426036613248e-08\\
2.33722728286067e-17	8.28981958444279e-08\\
2.23707426127118e-17	8.57593256082581e-08\\
2.14121291803368e-17	9.91664316083795e-08\\
2.04945934952963e-17	1.16904225202764e-07\\
1.96163753263342e-17	1.33128468058806e-07\\
1.87757898702372e-17	1.55393961145824e-07\\
1.79712245196516e-17	1.78980677587088e-07\\
1.72011357694025e-17	2.47207351043003e-07\\
1.64640462553831e-17	3.01491653797368e-07\\
1.57585419203287e-17	2.81933104970076e-07\\
1.5083269301042e-17	3.54179521022342e-07\\
1.44369329318644e-17	5.2153805891216e-07\\
1.38182928594102e-17	4.43134018262624e-07\\
1.32261622637994e-17	6.25670731784021e-07\\
1.26594051818219e-17	6.25397132642527e-07\\
1.21169343276679e-17	6.78652001741138e-07\\
1.15977090070428e-17	8.98779296330048e-07\\
1.11007331206629e-17	1.25536750512629e-06\\
1.06250532533066e-17	1.08837655265994e-06\\
1.01697568447497e-17	1.54318295846371e-06\\
9.73397043907952e-18	1.57411416821239e-06\\
9.31685800902816e-18	1.83909536967216e-06\\
8.91761935210888e-18	2.02392471949143e-06\\
8.5354885554816e-18	2.36443731194928e-06\\
8.16973252659952e-18	2.74022243854513e-06\\
7.81964958681996e-18	3.22581666236429e-06\\
7.48456812528051e-18	4.99355042313308e-06\\
7.16384531045793e-18	7.28334486130956e-06\\
6.85686585693903e-18	5.11219197893756e-06\\
6.56304084503622e-18	7.82819744616663e-06\\
6.28180659098405e-18	6.82110552948138e-06\\
6.01262356554989e-18	1.15001529173053e-05\\
5.75497535898227e-18	1.04149667590832e-05\\
5.50836769031359e-18	1.08291765101649e-05\\
5.27232745911473e-18	1.49958414258121e-05\\
5.04640183788321e-18	1.41540198865313e-05\\
4.83015740332396e-18	2.27647266336118e-05\\
4.6231793048552e-18	2.05326774709241e-05\\
4.42507046874556e-18	2.90490670087043e-05\\
4.23545083635416e-18	2.64796430990246e-05\\
4.05395663501342e-18	2.64915547325192e-05\\
3.88023968015556e-18	3.73272609290244e-05\\
3.71396670734337e-18	3.32403456060332e-05\\
3.55481873292476e-18	3.76718740785845e-05\\
3.4024904420837e-18	4.22226279955647e-05\\
3.25668960311394e-18	9.17370242092668e-05\\
3.1171365067922e-18	5.51981108942914e-05\\
2.98356342977419e-18	6.01487796624591e-05\\
2.85571412098551e-18	0.000104362484637902\\
2.73334331002084e-18	8.35936195707367e-05\\
2.61621623660891e-18	7.82560478757849e-05\\
2.50410820024066e-18	0.000117010463824433\\
2.39680412909607e-18	9.3103915259696e-05\\
2.29409816744337e-18	0.000129770383796661\\
2.19579328071832e-18	0.000106960686137226\\
2.10170087752651e-18	0.000187632564228857\\
2.0116404478434e-18	0.000128491520767904\\
1.92543921671771e-18	0.000134568875483988\\
1.84293181281425e-18	0.000135103188691818\\
1.76395995115995e-18	0.000144616284028216\\
1.68837212948465e-18	0.000158307150366073\\
1.61602333757411e-18	0.000162459465501675\\
1.54677477907744e-18	0.000251007500418369\\
1.48049360523567e-18	0.000202872890329271\\
1.41705266002013e-18	0.000205350419664544\\
1.35633023619204e-18	0.000272181205249711\\
1.29820984181536e-18	0.000202992283160761\\
1.24257997677464e-18	0.000214930952592777\\
1.18933391886954e-18	0.000228004764819253\\
1.13836951907534e-18	0.000269454284266202\\
1.08958900557679e-18	0.000243970917104967\\
1.04289879619946e-18	0.000253283909758588\\
9.98209318878463e-19	0.000432075400299865\\
9.55434839820484e-19	0.000254721152617228\\
9.14493299029138e-19	0.000249537855368224\\
8.75306152878334e-19	0.000247448660229675\\
8.37798223431559e-19	0.000255117873160962\\
8.01897554217962e-19	0.000261404213356805\\
7.67535272188698e-19	0.000280097059223049\\
7.34645455588532e-19	0.00029447775504006\\
7.03165007489312e-19	0.000271748708095661\\
6.73033534742745e-19	0.00026848671305015\\
6.44193232120129e-19	0.000276189528608749\\
6.16588771416863e-19	0.000330953698754617\\
5.90167195308974e-19	0.000309147432295575\\
5.64877815757987e-19	0.000342448035421022\\
5.40672116769318e-19	0.000306935541809309\\
5.17503661317547e-19	0.000282145666186846\\
4.95328002260056e-19	0.000321547085799283\\
4.74102597068182e-19	0.000322560851737055\\
4.53786726212147e-19	0.000296602189207126\\
4.34341415043384e-19	0.000304925558773441\\
4.15729359024255e-19	0.000355482477009757\\
3.97914852161756e-19	0.000308424720911864\\
3.80863718507973e-19	0.000309163213380543\\
3.64543246595759e-19	0.000298415813927092\\
3.48922126683998e-19	0.000484085097864378\\
3.33970390691915e-19	0.000311349928032748\\
3.19659354707311e-19	0.000307374000784943\\
3.05961563958398e-19	0.000327094975231244\\
2.9285074014362e-19	0.000605490547330731\\
2.80301731018499e-19	0.000356988074921982\\
2.68290462142714e-19	0.000439413319402075\\
2.56793890694878e-19	0.000356971437084952\\
2.45789961266445e-19	0.000318868307963188\\
2.35257563549836e-19	0.000335743593094621\\
2.25176491839747e-19	0.00037579897498099\\
2.1552740626983e-19	0.000325575117824629\\
2.06291795710446e-19	0.000345841928283335\\
1.97451942256301e-19	0.000329056971640837\\
1.88990887235806e-19	0.000366920856966117\\
1.80892398677014e-19	0.0004932951301212\\
1.73140940167641e-19	0.000353791038221965\\
1.65721641049496e-19	0.00041557383733997\\
1.58620267890118e-19	0.000390217792413908\\
1.51823197176873e-19	0.000383739460952828\\
1.45317389181157e-19	0.00040834745419795\\
1.39090362942539e-19	0.000431140051490612\\
1.33130172324867e-19	0.000386444638988706\\
1.27425383098402e-19	0.000497470466758793\\
1.21965051004006e-19	0.000421786105024308\\
1.16738700757308e-19	0.000400598868790188\\
1.11736305952571e-19	0.000403685548581348\\
1.06948269827689e-19	0.00053995367141807\\
1.02365406853447e-19	0.000654067983611431\\
9.79789251116873e-20	0.000440713525720218\\
9.3780409428602e-20	0.000462050497753278\\
8.97618052307783e-20	0.000472026117979457\\
8.59154030930341e-20	0.000621782441546694\\
8.22338239484022e-20	0.000516475127249801\\
7.87100049318752e-20	0.000551331316573642\\
7.53371858307725e-20	0.000562608470853313\\
7.21088961157195e-20	0.000710821459689415\\
6.90189425273666e-20	0.000611521492382993\\
6.60613971950337e-20	0.000657423937972023\\
6.32305862644819e-20	0.000809897848643636\\
6.05210790130042e-20	0.000790724741779337\\
5.79276774309431e-20	0.000922137261138313\\
5.5445406249653e-20	0.000843039142487648\\
5.30695033967808e-20	0.00101947640862982\\
5.07954108605446e-20	0.000987064635955382\\
4.86187659454911e-20	0.00113576147577164\\
4.65353929029549e-20	0.00135400780777905\\
4.45412949201604e-20	0.00128086956065063\\
};
\addplot [color=mycolor1,only marks,mark=asterisk,mark options={solid},forget plot]
  table[row sep=crcr]{%
0.000975298695663297	3.5799356034256e-09\\
};
\end{axis}
\end{tikzpicture}%
\end{document}
\caption{Plot of the value of the GCV-function for different values of $\lambda$ for the $\mathbf{A}_1, \mathbf{b}_{err1}$ and the $\mathbf{A}_3, \mathbf{b}_{err3}$ pair.}
\label{fig:GCVA1}
\end{figure}
An other way to find a suitable regularization parameter is generalized cross validation. When this method is used the function
\begin{equation}
G = \frac{\| \mathbf{Ax}_{reg} - \mathbf{b} \|^2_2}{(\text{trace}(\mathbf{I}_m - \mathbf{AA}^I))^2}
\end{equation}
is minimized. With $\mathbf{x}_{reg}$ defined as in equation~\ref{eq:tikh}. The trace of the denominator can be simplified to:
\begin{eqnarray}
\text{trace}(\mathbf{I}_m) - \text{trace}(\mathbf{AA}^I) & \text{assuming symmetry} \\
= n - \text{trace}(\mathbf{A}^T \mathbf{A}^I) \\
= n - \text{trace}(\mathbf{V}\mathbf{\Sigma}^T \mathbf{U}^T \mathbf{U}\mathbf{F}\mathbf{\Sigma}^{-1} \mathbf{V}^T) & \text{svd} \\
= n - \text{trace}(\mathbf{V}^T \mathbf{V}\mathbf{\Sigma}^T \mathbf{F}\mathbf{\Sigma}^{-1} ) & \text{cyclic rotation} \\
= n - \text{trace}(\mathbf{\Sigma}^{-1}\mathbf{\Sigma} \mathbf{F} ) & \text{cyclic rotation} \\
= n - \text{trace}(\mathbf{F} ) & \text{F is diagonal}  \\
= n - \text{sum}(\mathbf{F})
\end{eqnarray}
With $\mathbf{F}$ being a matrix with the filter factors on the diagonal. The reasoning described above is used to compute the GCV function efficiently.
\begin{figure}
\centering
% This file was created by matlab2tikz.
% Minimal pgfplots version: 1.3
%
%The latest updates can be retrieved from
%  http://www.mathworks.com/matlabcentral/fileexchange/22022-matlab2tikz
%where you can also make suggestions and rate matlab2tikz.
%
\documentclass[tikz]{standalone}
\usepackage{pgfplots}
\usepackage{grffile}
\pgfplotsset{compat=newest}
\usetikzlibrary{plotmarks}
\usepackage{amsmath}

\begin{document}
\definecolor{mycolor1}{rgb}{0.00000,0.44700,0.74100}%
\definecolor{mycolor2}{rgb}{0.85000,0.32500,0.09800}%
%
\begin{tikzpicture}

\begin{axis}[%
width=2in,
height=2in,
at={(0.771875in,0.483542in)},
scale only axis,
xmin=1,
xmax=6,
xlabel={problem no.},
ymode=log,
ymin=1e-05,
ymax=0.1,
yminorticks=true,
ylabel={$\lambda$},
legend style={legend cell align=left,align=left,draw=white!15!black}
]
\addplot [color=mycolor1,only marks,mark=o,mark options={solid}]
  table[row sep=crcr]{%
1	0.000700429712987355\\
2	0.0925364518699232\\
3	0.0622026861633196\\
4	0.000577706308768941\\
5	2.38626985222061e-05\\
6	5.98062478687114e-05\\
};
\addlegendentry{L-curve};

\addplot [color=mycolor2,only marks,mark=asterisk,mark options={solid}]
  table[row sep=crcr]{%
1	0.000702695221786386\\
2	0.00124023200120442\\
3	0.000975298695663297\\
4	0.000726454189092671\\
5	4.02258506472218e-05\\
6	0.00211845786003602\\
};
\addlegendentry{GCV};

\end{axis}
\end{tikzpicture}%
\end{document}
\caption{Plot of the lambdas selected by the L-curve curvature criterion and generalized cross validation.}
\label{fig:comparisonLGCV}
\end{figure}
A comparison of the $\lambda$ values selected by the L-curve curvature criterion and GCV is given in figure~\ref{fig:comparisonLGCV}. For the first problem both methods select almost the same value. In two an three the value selected by the L-curve curvature criterion is significantly larger then what is chosen by the GCV criterion and what a human would have selected. In these six examples this generally happens when the value of the maximum of the $\kappa$ curve is not very large. In these cases GCV should be preferred.

\subsection{Picard condition}
The Picard condition states that the Fourier coefficients $| \mathbf{u}_i^T \mathbf{b}|$ have to decay faster then the singular values. If this condition is satisfied, then one has every reason to believe that $\mathbf{x}_{reg}$ approximates the true $\mathbf{x}$ well.Figures~\ref{fig:picard12},\ref{fig:picard34},\ref{fig:picard56} show that regularization is necessary as the Fourier coefficients violate the Picard condition.  Additionally all solutions where the singular values have been filtered using Tikhonov regularization adhere to the Picard condition, which indicated that regularization was successful.

\begin{figure}
% This file was created by matlab2tikz.
% Minimal pgfplots version: 1.3
%
%The latest updates can be retrieved from
%  http://www.mathworks.com/matlabcentral/fileexchange/22022-matlab2tikz
%where you can also make suggestions and rate matlab2tikz.
%
\documentclass[tikz]{standalone}
\usepackage{pgfplots}
\usepackage{grffile}
\pgfplotsset{compat=newest}
\usetikzlibrary{plotmarks}
\usepackage{amsmath}

\begin{document}
\definecolor{mycolor1}{rgb}{0.00000,0.44700,0.74100}%
\definecolor{mycolor2}{rgb}{0.85000,0.32500,0.09800}%
\definecolor{mycolor3}{rgb}{0.92900,0.69400,0.12500}%
%
\begin{tikzpicture}

\begin{axis}[%
width=2in,
height=2in,
at={(0.758333in,0.48125in)},
scale only axis,
xmode=log,
xmin=1,
xmax=1000,
xminorticks=true,
ymode=log,
ymin=1e-40,
ymax=10000000000,
yminorticks=true,
legend style={legend cell align=left,align=left,draw=white!15!black}
]
\addplot [color=mycolor1,solid]
  table[row sep=crcr]{%
1	2.99330373246548\\
2	1.8567344516343\\
3	1.03399980133487\\
4	0.39339284526325\\
5	0.0590154348312629\\
6	0.0345422003109507\\
7	0.0244920540054261\\
8	0.00435530209466828\\
9	0.00132744485517824\\
10	7.83854781200961e-05\\
11	1.02935045792844e-05\\
12	2.44953945278807e-06\\
13	5.19531948847239e-07\\
14	6.64792819542601e-08\\
15	6.12818542013426e-09\\
16	4.3471444009698e-10\\
17	5.80439614777215e-11\\
18	1.09990793825876e-12\\
19	7.71318525291645e-13\\
20	6.92085341883737e-13\\
21	1.23837155270691e-15\\
22	8.4962925207976e-16\\
23	8.3613909371963e-16\\
24	7.64720589843762e-16\\
25	5.82117001508659e-16\\
26	4.4149566578133e-16\\
27	4.0791894793164e-16\\
28	3.99653465472219e-16\\
29	3.03819335858694e-16\\
30	2.96345892144409e-16\\
31	2.38253549885617e-16\\
32	1.87538253595704e-16\\
33	1.87538253595704e-16\\
34	1.87538253595704e-16\\
35	1.87538253595704e-16\\
36	1.87538253595704e-16\\
37	1.87538253595704e-16\\
38	1.87538253595704e-16\\
39	1.87538253595704e-16\\
40	1.87538253595704e-16\\
41	1.87538253595704e-16\\
42	1.87538253595704e-16\\
43	1.87538253595704e-16\\
44	1.87538253595704e-16\\
45	1.87538253595704e-16\\
46	1.87538253595704e-16\\
47	1.87538253595704e-16\\
48	1.87538253595704e-16\\
49	1.87538253595704e-16\\
50	1.87538253595704e-16\\
51	1.87538253595704e-16\\
52	1.87538253595704e-16\\
53	1.87538253595704e-16\\
54	1.87538253595704e-16\\
55	1.87538253595704e-16\\
56	1.87538253595704e-16\\
57	1.87538253595704e-16\\
58	1.87538253595704e-16\\
59	1.87538253595704e-16\\
60	1.87538253595704e-16\\
61	1.87538253595704e-16\\
62	1.87538253595704e-16\\
63	1.87538253595704e-16\\
64	1.87538253595704e-16\\
65	1.87538253595704e-16\\
66	1.87538253595704e-16\\
67	1.87538253595704e-16\\
68	1.87538253595704e-16\\
69	1.87538253595704e-16\\
70	1.87538253595704e-16\\
71	1.87538253595704e-16\\
72	1.87538253595704e-16\\
73	1.87538253595704e-16\\
74	1.87538253595704e-16\\
75	1.87538253595704e-16\\
76	1.87538253595704e-16\\
77	1.87538253595704e-16\\
78	1.87538253595704e-16\\
79	1.87538253595704e-16\\
80	1.87538253595704e-16\\
81	1.87538253595704e-16\\
82	1.87538253595704e-16\\
83	1.87538253595704e-16\\
84	1.87538253595704e-16\\
85	1.87538253595704e-16\\
86	1.87538253595704e-16\\
87	1.87538253595704e-16\\
88	1.87538253595704e-16\\
89	1.87538253595704e-16\\
90	1.87538253595704e-16\\
91	1.87538253595704e-16\\
92	1.87538253595704e-16\\
93	1.87538253595704e-16\\
94	1.87538253595704e-16\\
95	1.87538253595704e-16\\
96	1.87538253595704e-16\\
97	1.87538253595704e-16\\
98	1.87538253595704e-16\\
99	1.87538253595704e-16\\
100	1.87538253595704e-16\\
101	1.87538253595704e-16\\
102	1.87538253595704e-16\\
103	1.87538253595704e-16\\
104	1.87538253595704e-16\\
105	1.87538253595704e-16\\
106	1.87538253595704e-16\\
107	1.87538253595704e-16\\
108	1.87538253595704e-16\\
109	1.87538253595704e-16\\
110	1.87538253595704e-16\\
111	1.87538253595704e-16\\
112	1.87538253595704e-16\\
113	1.87538253595704e-16\\
114	1.87538253595704e-16\\
115	1.87538253595704e-16\\
116	1.87538253595704e-16\\
117	1.87538253595704e-16\\
118	1.87538253595704e-16\\
119	1.87538253595704e-16\\
120	1.87538253595704e-16\\
121	1.87538253595704e-16\\
122	1.87538253595704e-16\\
123	1.87538253595704e-16\\
124	1.87538253595704e-16\\
125	1.87538253595704e-16\\
126	1.87538253595704e-16\\
127	1.87538253595704e-16\\
128	1.87538253595704e-16\\
129	1.87538253595704e-16\\
130	1.87538253595704e-16\\
131	1.87538253595704e-16\\
132	1.87538253595704e-16\\
133	1.87538253595704e-16\\
134	1.87538253595704e-16\\
135	1.87538253595704e-16\\
136	1.87538253595704e-16\\
137	1.87538253595704e-16\\
138	1.87538253595704e-16\\
139	1.87538253595704e-16\\
140	1.87538253595704e-16\\
141	1.87538253595704e-16\\
142	1.87538253595704e-16\\
143	1.87538253595704e-16\\
144	1.87538253595704e-16\\
145	1.87538253595704e-16\\
146	1.87538253595704e-16\\
147	1.87538253595704e-16\\
148	1.87538253595704e-16\\
149	1.87538253595704e-16\\
150	1.87538253595704e-16\\
151	1.87538253595704e-16\\
152	1.87538253595704e-16\\
153	1.87538253595704e-16\\
154	1.87538253595704e-16\\
155	1.87538253595704e-16\\
156	1.87538253595704e-16\\
157	1.87538253595704e-16\\
158	1.87538253595704e-16\\
159	1.87538253595704e-16\\
160	1.87538253595704e-16\\
161	1.87538253595704e-16\\
162	1.87538253595704e-16\\
163	1.87538253595704e-16\\
164	1.87538253595704e-16\\
165	1.87538253595704e-16\\
166	1.87538253595704e-16\\
167	1.87538253595704e-16\\
168	1.87538253595704e-16\\
169	1.87538253595704e-16\\
170	1.87538253595704e-16\\
171	1.87538253595704e-16\\
172	1.87538253595704e-16\\
173	1.87538253595704e-16\\
174	1.87538253595704e-16\\
175	1.87538253595704e-16\\
176	1.87538253595704e-16\\
177	1.87538253595704e-16\\
178	1.87538253595704e-16\\
179	1.87538253595704e-16\\
180	1.87538253595704e-16\\
181	1.87538253595704e-16\\
182	1.87538253595704e-16\\
183	1.87538253595704e-16\\
184	1.87538253595704e-16\\
185	1.87538253595704e-16\\
186	1.87538253595704e-16\\
187	1.87538253595704e-16\\
188	1.87538253595704e-16\\
189	1.87538253595704e-16\\
190	1.87538253595704e-16\\
191	1.87538253595704e-16\\
192	1.87538253595704e-16\\
193	1.87538253595704e-16\\
194	1.87538253595704e-16\\
195	1.87538253595704e-16\\
196	1.87538253595704e-16\\
197	1.87538253595704e-16\\
198	1.87538253595704e-16\\
199	1.87538253595704e-16\\
200	1.87538253595704e-16\\
201	1.87538253595704e-16\\
202	1.87538253595704e-16\\
203	1.87538253595704e-16\\
204	1.87538253595704e-16\\
205	1.87538253595704e-16\\
206	1.87538253595704e-16\\
207	1.87538253595704e-16\\
208	1.87538253595704e-16\\
209	1.87538253595704e-16\\
210	1.87538253595704e-16\\
211	1.87538253595704e-16\\
212	1.87538253595704e-16\\
213	1.87538253595704e-16\\
214	1.87538253595704e-16\\
215	1.87538253595704e-16\\
216	1.87538253595704e-16\\
217	1.87538253595704e-16\\
218	1.87538253595704e-16\\
219	1.87538253595704e-16\\
220	1.87538253595704e-16\\
221	1.87538253595704e-16\\
222	1.87538253595704e-16\\
223	1.87538253595704e-16\\
224	1.87538253595704e-16\\
225	1.87538253595704e-16\\
226	1.87538253595704e-16\\
227	1.87538253595704e-16\\
228	1.87538253595704e-16\\
229	1.87538253595704e-16\\
230	1.87538253595704e-16\\
231	1.87538253595704e-16\\
232	1.87538253595704e-16\\
233	1.87538253595704e-16\\
234	1.87538253595704e-16\\
235	1.87538253595704e-16\\
236	1.87538253595704e-16\\
237	1.87538253595704e-16\\
238	1.87538253595704e-16\\
239	1.87538253595704e-16\\
240	1.87538253595704e-16\\
241	1.87538253595704e-16\\
242	1.87538253595704e-16\\
243	1.87538253595704e-16\\
244	1.87538253595704e-16\\
245	1.87538253595704e-16\\
246	1.87538253595704e-16\\
247	1.87538253595704e-16\\
248	1.87538253595704e-16\\
249	1.87538253595704e-16\\
250	1.87538253595704e-16\\
251	1.87538253595704e-16\\
252	1.87538253595704e-16\\
253	1.87538253595704e-16\\
254	1.87538253595704e-16\\
255	1.87538253595704e-16\\
256	1.87538253595704e-16\\
257	1.87538253595704e-16\\
258	1.87538253595704e-16\\
259	1.87538253595704e-16\\
260	1.87538253595704e-16\\
261	1.87538253595704e-16\\
262	1.87538253595704e-16\\
263	1.87538253595704e-16\\
264	1.87538253595704e-16\\
265	1.87538253595704e-16\\
266	1.87538253595704e-16\\
267	1.87538253595704e-16\\
268	1.87538253595704e-16\\
269	1.87538253595704e-16\\
270	1.87538253595704e-16\\
271	1.87538253595704e-16\\
272	1.87538253595704e-16\\
273	1.87538253595704e-16\\
274	1.87538253595704e-16\\
275	1.87538253595704e-16\\
276	1.87538253595704e-16\\
277	1.87538253595704e-16\\
278	1.87538253595704e-16\\
279	1.87538253595704e-16\\
280	1.87538253595704e-16\\
281	1.87538253595704e-16\\
282	1.87538253595704e-16\\
283	1.87538253595704e-16\\
284	1.87538253595704e-16\\
285	1.87538253595704e-16\\
286	1.87538253595704e-16\\
287	1.87538253595704e-16\\
288	1.87538253595704e-16\\
289	1.87538253595704e-16\\
290	1.87538253595704e-16\\
291	1.87538253595704e-16\\
292	1.87538253595704e-16\\
293	1.87538253595704e-16\\
294	1.87538253595704e-16\\
295	1.87538253595704e-16\\
296	1.87538253595704e-16\\
297	1.47942036885314e-16\\
298	1.35165317483904e-16\\
299	1.15761547770891e-16\\
300	4.84912517527819e-17\\
};
\addlegendentry{$\sigma$};

\addplot [color=mycolor2,solid]
  table[row sep=crcr]{%
1	38.5145443798111\\
2	6.25506884618799\\
3	10.2627118115302\\
4	1.51480049515919\\
5	0.0878468265880232\\
6	0.0587676981563168\\
7	0.0421159672208311\\
8	0.000434254963010194\\
9	0.00248731414358421\\
10	0.000788886568186674\\
11	0.000368034476574283\\
12	6.29321933546118e-05\\
13	5.69521825739619e-05\\
14	0.00123671743654791\\
15	4.47858397732252e-05\\
16	0.000191803255596576\\
17	0.00031598752731099\\
18	0.00014821056102092\\
19	8.96892873727927e-05\\
20	0.00123804520782178\\
21	0.00218801319190959\\
22	0.000621560074542719\\
23	9.97605944612623e-05\\
24	0.000419089761735714\\
25	0.000129263243272559\\
26	0.00141879853237775\\
27	0.000627653257489369\\
28	0.00045241888185793\\
29	0.000754988442707827\\
30	0.00047987010290762\\
31	9.43420832111297e-06\\
32	0.00189929309623915\\
33	0.000599167406702961\\
34	0.000447615985426753\\
35	0.000299966897096549\\
36	0.000267812051711447\\
37	0.000342870921637872\\
38	4.42488502445659e-05\\
39	0.00235670270004526\\
40	0.00031689624673259\\
41	0.0002982729607659\\
42	0.000482917961481366\\
43	0.00111042884138968\\
44	0.000800523059875491\\
45	0.00188783234407142\\
46	0.000671845960845352\\
47	1.336282588138e-05\\
48	3.2163939284402e-05\\
49	0.00111818409743947\\
50	0.000743651291775893\\
51	0.000971061640016035\\
52	0.000407310564881191\\
53	0.00054853624641249\\
54	0.00129369714008498\\
55	0.00116513229169102\\
56	0.00145270397841046\\
57	0.00137900656778567\\
58	0.000179045705811176\\
59	0.00147341759711317\\
60	0.000150845125554078\\
61	0.00106213967339253\\
62	0.000258850611321693\\
63	0.000305868390803266\\
64	0.0001350892865849\\
65	0.000637911011201475\\
66	0.000999670129173902\\
67	0.000738369900509017\\
68	0.00287081035058301\\
69	0.0015224032275466\\
70	0.00197219761479706\\
71	0.0010471140611682\\
72	0.00031398332074617\\
73	0.000826589371382136\\
74	0.00120466234085229\\
75	0.00147858533113443\\
76	0.00126830810566314\\
77	0.000461035554519679\\
78	0.00173349013155322\\
79	0.00113336790990881\\
80	0.000313599278587984\\
81	0.00182151514305495\\
82	0.00126961114767044\\
83	0.00059221437218826\\
84	0.00105072978968113\\
85	0.000629591030829824\\
86	0.00202469767152058\\
87	0.000544018361861465\\
88	0.000284128529808081\\
89	0.000949567100304805\\
90	0.000342373447643429\\
91	0.000103001312856676\\
92	0.000207695098265903\\
93	0.00136969257627806\\
94	0.000181121884974544\\
95	0.00132771317492564\\
96	0.000748593769053674\\
97	0.000718280292009155\\
98	0.00161741941729687\\
99	0.000897615209496161\\
100	0.00118424913664583\\
101	0.000689385493766848\\
102	0.000384365599070402\\
103	0.00099866529500714\\
104	1.95238957796384e-05\\
105	0.000628024595704657\\
106	0.00136271782323627\\
107	2.32696832936918e-05\\
108	0.000796308427046588\\
109	2.76593744219511e-06\\
110	6.41444871742891e-05\\
111	6.90627986886794e-05\\
112	0.000363821714061352\\
113	0.00170752489962489\\
114	0.00140271901207112\\
115	0.000589185578568019\\
116	0.00142589515980163\\
117	0.00127241396759115\\
118	0.000384556965816962\\
119	0.00112672994909419\\
120	0.00080692258383408\\
121	0.00106900838193658\\
122	0.00174452244395407\\
123	0.00129735456525792\\
124	0.00059464395475084\\
125	0.00104045204386116\\
126	0.000437519193950076\\
127	0.000707315744659542\\
128	0.001201347146359\\
129	0.00126964062082857\\
130	0.000831392141211085\\
131	0.00224667713145177\\
132	0.00252395394105276\\
133	0.00067736847538491\\
134	0.00180417083893314\\
135	0.000451517327805795\\
136	0.00120751127162855\\
137	3.76273784155057e-05\\
138	0.00187690899478199\\
139	0.00224992288517856\\
140	0.0010413151540013\\
141	0.000160813754850109\\
142	0.00028772324375459\\
143	0.000868782469164997\\
144	0.00191869855061499\\
145	0.00160018663879491\\
146	0.00129031213025166\\
147	0.00115325096107044\\
148	0.000126149737804226\\
149	0.00116159968026208\\
150	0.00133229380958189\\
151	0.000558322437599679\\
152	0.0011664244684999\\
153	0.000767724751862708\\
154	6.96829152810885e-05\\
155	0.000307679812652673\\
156	0.000727262094944336\\
157	0.000318438534197441\\
158	0.00208602979091865\\
159	3.51201509545263e-05\\
160	0.000177709749130672\\
161	0.0003406891105458\\
162	0.0013726637561926\\
163	0.0019658994589844\\
164	0.000409167944240285\\
165	0.00129488359474283\\
166	0.0013302597992646\\
167	0.00155391924181014\\
168	6.97547995777703e-05\\
169	0.000506511065668458\\
170	5.63477349868613e-05\\
171	0.000601447024594748\\
172	0.000673308481366597\\
173	0.000985058882155387\\
174	0.000637549180263708\\
175	0.00182585236672941\\
176	6.4891152308566e-05\\
177	0.000282990157396523\\
178	0.000414585471454787\\
179	0.000588550406206423\\
180	0.00135819898736184\\
181	0.000787160899760386\\
182	0.000230348955918114\\
183	0.00221487018110983\\
184	0.000354056802502001\\
185	0.0018269175857806\\
186	0.00095358905590115\\
187	9.65937222870972e-05\\
188	0.000845284732420959\\
189	6.02647446150684e-05\\
190	0.000434050417316367\\
191	0.000770095033782781\\
192	0.000301638285020278\\
193	0.000148106187837893\\
194	0.000879916985892404\\
195	0.000745990329181545\\
196	0.000750284036663551\\
197	0.00125852617128319\\
198	0.000813036329828543\\
199	0.00125559906137562\\
200	0.00231501480114105\\
201	0.00132098860312513\\
202	0.000580347519311797\\
203	0.00175131503152059\\
204	2.86661001681621e-05\\
205	0.000584210162602128\\
206	0.000109622804627563\\
207	8.20994548185183e-05\\
208	0.00117700262943465\\
209	0.000660816602519376\\
210	0.000208469140467238\\
211	0.000191535331024645\\
212	0.000440306717767291\\
213	0.000667594440414643\\
214	0.000170828426457104\\
215	0.00167901239791868\\
216	0.00113735699423434\\
217	0.000746539349857467\\
218	0.00100098796648929\\
219	0.00153082238676094\\
220	0.00117757121724127\\
221	0.00159584539284729\\
222	0.000682488585122654\\
223	0.000955877488380441\\
224	0.000136734951170614\\
225	0.000355157080711044\\
226	0.00104508697162187\\
227	0.00107126054551379\\
228	0.000151818862459185\\
229	0.0020558026965023\\
230	0.00098246662217516\\
231	0.00161242918510973\\
232	0.000258894483033845\\
233	0.000615754635662658\\
234	0.0016380826550757\\
235	0.000310115825337576\\
236	0.000669055274705621\\
237	0.0016815600900103\\
238	0.00010794537771508\\
239	0.000457876821889136\\
240	0.000305994923511577\\
241	0.00135740993228894\\
242	0.000176242475108612\\
243	0.000561710063481824\\
244	0.00024763708889286\\
245	0.000508912693857333\\
246	0.000627776049335171\\
247	0.0016795893162378\\
248	5.65597820800789e-08\\
249	0.00118725246340512\\
250	0.00153208467856236\\
251	0.00146288522829419\\
252	0.00042614116270201\\
253	0.00116311387356571\\
254	0.000517084783924954\\
255	0.00191949110231693\\
256	0.00156362698589874\\
257	0.00071150687913768\\
258	0.00200716781194082\\
259	0.00160790232961094\\
260	0.00179790914013089\\
261	0.00157113710767498\\
262	0.000235709123397765\\
263	6.76191076571797e-05\\
264	0.000835587392222448\\
265	0.000739897742340068\\
266	0.000306818105020007\\
267	0.00162008635520894\\
268	0.000990625494077579\\
269	0.00126584413272812\\
270	1.87980401240573e-05\\
271	0.000586397962689701\\
272	0.000365728627884088\\
273	0.00206720525745746\\
274	0.00189155026623455\\
275	0.000920174704380883\\
276	0.00193502326527539\\
277	0.0020488213198768\\
278	0.000626170069388088\\
279	0.000819985681222546\\
280	0.000259511448163385\\
281	0.00224768330243785\\
282	0.000691327530916219\\
283	0.00125606543791745\\
284	9.01547865641283e-05\\
285	0.00117355276408895\\
286	0.000634180257229944\\
287	0.00061393509572838\\
288	0.000548113971368197\\
289	0.000422273303106041\\
290	0.000957537485823057\\
291	0.00037321546056085\\
292	0.000625596990484922\\
293	0.000545135250951678\\
294	0.00230832899305267\\
295	0.00119971384444317\\
296	9.88014911385005e-05\\
297	0.00115470922645367\\
298	0.000552002393178708\\
299	0.00214023342786418\\
300	0.00013949393688617\\
};
\addlegendentry{$\|u_,^T b\|$};

\addplot [color=mycolor3,solid]
  table[row sep=crcr]{%
1	38.514542270929\\
2	6.25506795604134\\
3	10.2627071022976\\
4	1.51479569306948\\
5	0.0878344539358222\\
6	0.0587435441447198\\
7	0.0420815504248328\\
8	0.000423306639196687\\
9	0.0019456199741985\\
10	9.75778733550386e-06\\
11	7.94679703290133e-08\\
12	7.69674878946669e-10\\
13	3.13332562763751e-11\\
14	1.11407389487914e-11\\
15	3.42827296888984e-15\\
16	7.38813777309379e-17\\
17	2.1699758910885e-18\\
18	3.65479230641195e-22\\
19	1.08762448425762e-22\\
20	1.20872271484336e-21\\
21	6.83947471130623e-27\\
22	9.14561470039891e-28\\
23	1.42163126163365e-28\\
24	4.99555214152703e-28\\
25	8.92825310588293e-29\\
26	5.63695408424185e-28\\
27	2.12881745632975e-28\\
28	1.47291866731395e-28\\
29	1.42050249722105e-28\\
30	8.58998477706782e-29\\
31	1.0915788192895e-30\\
32	1.36157824079251e-28\\
33	4.29535233489354e-29\\
34	3.20890012812974e-29\\
35	2.15042323300808e-29\\
36	1.91990937551121e-29\\
37	2.45799654211385e-29\\
38	3.17214187706504e-30\\
39	1.68948916981393e-28\\
40	2.27178751396475e-29\\
41	2.13827962623855e-29\\
42	3.46197535147504e-29\\
43	7.96051831798702e-29\\
44	5.73884452977296e-29\\
45	1.3533621782992e-28\\
46	4.81637533070959e-29\\
47	9.57963412375278e-31\\
48	2.30579050468664e-30\\
49	8.01611472861954e-29\\
50	5.33113830416235e-29\\
51	6.96141318121585e-29\\
52	2.91995586929331e-29\\
53	3.93238420589829e-29\\
54	9.27434464750792e-29\\
55	8.35268015850123e-29\\
56	1.04142437585608e-28\\
57	9.88591671462063e-29\\
58	1.28355511649631e-29\\
59	1.05627369667259e-28\\
60	1.08138886561872e-29\\
61	7.61433962371508e-29\\
62	1.8556659879805e-29\\
63	2.19273026521691e-29\\
64	9.68437328306204e-30\\
65	4.57310013990813e-29\\
66	7.16650367733939e-29\\
67	5.29327670479468e-29\\
68	2.05804618283373e-28\\
69	1.09139085086237e-28\\
70	1.41384253129117e-28\\
71	7.50662298587595e-29\\
72	2.25090513068675e-29\\
73	5.92571048867629e-29\\
74	8.63606588185604e-29\\
75	1.05997837722664e-28\\
76	9.09233399895749e-29\\
77	3.3051032539842e-29\\
78	1.24271627608467e-28\\
79	8.12496548321147e-29\\
80	2.24815198296269e-29\\
81	1.30582024910664e-28\\
82	9.10167533573503e-29\\
83	4.24550694494359e-29\\
84	7.53254366803345e-29\\
85	4.5134552948811e-29\\
86	1.45147913146429e-28\\
87	3.89999608574771e-29\\
88	2.03688006101626e-29\\
89	6.80732165302986e-29\\
90	2.45443021647933e-29\\
91	7.38402864925258e-30\\
92	1.48893884297566e-29\\
93	9.81914593449719e-29\\
94	1.29843897184612e-29\\
95	9.5182011274164e-29\\
96	5.36657027372021e-29\\
97	5.14925694368053e-29\\
98	1.15950670762078e-28\\
99	6.43488538064212e-29\\
100	8.48972630568249e-29\\
101	4.94211393539195e-29\\
102	2.75546642716772e-29\\
103	7.15930015335092e-29\\
104	1.39964241021653e-30\\
105	4.50222572749647e-29\\
106	9.76914484726809e-29\\
107	1.66817299057949e-30\\
108	5.70863038134202e-29\\
109	1.98286417446685e-31\\
110	4.59843392136209e-30\\
111	4.95102120511286e-30\\
112	2.60819001750548e-29\\
113	1.22410214281551e-28\\
114	1.00559081089872e-28\\
115	4.22379392183131e-29\\
116	1.02220548639023e-28\\
117	9.12176838310577e-29\\
118	2.7568383102072e-29\\
119	8.07737881517021e-29\\
120	5.78472187535692e-29\\
121	7.66358049188103e-29\\
122	1.25062519574635e-28\\
123	9.30056424753987e-29\\
124	4.26292430278665e-29\\
125	7.45886386001615e-29\\
126	3.13651755798757e-29\\
127	5.07065354581121e-29\\
128	8.61229968854796e-29\\
129	9.10188662491489e-29\\
130	5.96014091390241e-29\\
131	1.61061328676827e-28\\
132	1.8093893847693e-28\\
133	4.85596551110219e-29\\
134	1.29338634559768e-28\\
135	3.23686833853302e-29\\
136	8.65648949189e-29\\
137	2.69745727034501e-30\\
138	1.34553137285875e-28\\
139	1.6129401249264e-28\\
140	7.46505138308348e-29\\
141	1.15285265795425e-29\\
142	2.06265009250997e-29\\
143	6.22818725734634e-29\\
144	1.37548975581025e-28\\
145	1.14715275536268e-28\\
146	9.25007795722492e-29\\
147	8.26750446192426e-29\\
148	9.04350878846328e-30\\
149	8.32735533176866e-29\\
150	9.55103909482367e-29\\
151	4.00254012341362e-29\\
152	8.36194360408148e-29\\
153	5.50371777334339e-29\\
154	4.99547654805994e-30\\
155	2.20571611021627e-29\\
156	5.21364630762047e-29\\
157	2.28284396995828e-29\\
158	1.49544732120747e-28\\
159	2.51771743122214e-30\\
160	1.27397781876297e-29\\
161	2.44235542534742e-29\\
162	9.84044593253252e-29\\
163	1.40932746622402e-28\\
164	2.93327117763407e-29\\
165	9.28285018490686e-29\\
166	9.53645754236969e-29\\
167	1.11398426698133e-28\\
168	5.00062983878063e-30\\
169	3.63111121237693e-29\\
170	4.03949501145744e-30\\
171	4.31169461575867e-29\\
172	4.82685994797317e-29\\
173	7.0617575691653e-29\\
174	4.57050622150577e-29\\
175	1.30892954771556e-28\\
176	4.65196136286044e-30\\
177	2.02871921891752e-29\\
178	2.97210871771194e-29\\
179	4.21924045470486e-29\\
180	9.73674990721525e-29\\
181	5.64305296133465e-29\\
182	1.65134136900146e-29\\
183	1.58781130240041e-28\\
184	2.53818665084171e-29\\
185	1.30969318924317e-28\\
186	6.83615452372582e-29\\
187	6.92467690864998e-30\\
188	6.05973507311603e-29\\
189	4.32030028001809e-30\\
190	3.11165035452678e-29\\
191	5.52071001268101e-29\\
192	2.16240519320733e-29\\
193	1.06175378136964e-29\\
194	6.30800914333148e-29\\
195	5.34790655568997e-29\\
196	5.37868757990624e-29\\
197	9.02220859792726e-29\\
198	5.82855051629731e-29\\
199	9.00122453198211e-29\\
200	1.65960366337822e-28\\
201	9.4699935566222e-29\\
202	4.16043503741949e-29\\
203	1.25549471243482e-28\\
204	2.0550350188059e-30\\
205	4.18812581915106e-29\\
206	7.85871468557577e-30\\
207	5.88560193700613e-30\\
208	8.43777704856226e-29\\
209	4.73730731147639e-29\\
210	1.49448784971254e-29\\
211	1.3730915969927e-29\\
212	3.15650094960657e-29\\
213	4.78589673991495e-29\\
214	1.22464652155661e-29\\
215	1.20366190534453e-28\\
216	8.15356270408742e-29\\
217	5.3518424127104e-29\\
218	7.17595107972698e-29\\
219	1.09742643537156e-28\\
220	8.44185317976238e-29\\
221	1.14404057323865e-28\\
222	4.89267090441381e-29\\
223	6.85255999519077e-29\\
224	9.80234881270261e-30\\
225	2.5460744006126e-29\\
226	7.49209104752409e-29\\
227	7.67972595635993e-29\\
228	1.08836945743187e-29\\
229	1.4737779147745e-28\\
230	7.04317399830289e-29\\
231	1.15592927579802e-28\\
232	1.85598049852867e-29\\
233	4.41426399771587e-29\\
234	1.17431991101663e-28\\
235	2.2231795647615e-29\\
236	4.79636926881133e-29\\
237	1.20548831229744e-28\\
238	7.73846215634923e-30\\
239	3.28245871520803e-29\\
240	2.19363736156566e-29\\
241	9.7310932751776e-29\\
242	1.26345912426886e-29\\
243	4.02682556781244e-29\\
244	1.77527772052549e-29\\
245	3.64832816899514e-29\\
246	4.50044393126372e-29\\
247	1.20407549049977e-28\\
248	4.05469642682424e-33\\
249	8.51125676021871e-29\\
250	1.09833135576243e-28\\
251	1.04872317999126e-28\\
252	3.05494994843057e-29\\
253	8.33821038441578e-29\\
254	3.70691280788362e-29\\
255	1.3760579257014e-28\\
256	1.12094362104035e-28\\
257	5.10069923766288e-29\\
258	1.43891220568828e-28\\
259	1.15268403262944e-28\\
260	1.28889741608186e-28\\
261	1.12632752856767e-28\\
262	1.6897677046853e-29\\
263	4.84752489384556e-30\\
264	5.99021611663332e-29\\
265	5.30422960192847e-29\\
266	2.19953864152981e-29\\
267	1.16141860033308e-28\\
268	7.10166387789796e-29\\
269	9.07467010105501e-29\\
270	1.34760677291789e-30\\
271	4.20380987023172e-29\\
272	2.62186043190641e-29\\
273	1.48195226075428e-28\\
274	1.35602750779775e-28\\
275	6.5966114324052e-29\\
276	1.38719272904438e-28\\
277	1.46877305769178e-28\\
278	4.48893087224228e-29\\
279	5.87836950244927e-29\\
280	1.86040344039076e-29\\
281	1.61133459751575e-28\\
282	4.95603614429368e-29\\
283	9.00456792407525e-29\\
284	6.46307807529771e-30\\
285	8.41304541762977e-29\\
286	4.5463548553621e-29\\
287	4.40121995524705e-29\\
288	3.92935697163989e-29\\
289	3.02722177169592e-29\\
290	6.86446029852757e-29\\
291	2.67553255068266e-29\\
292	4.48482254496046e-29\\
293	3.90800291673876e-29\\
294	1.65481069549235e-28\\
295	8.6005907619294e-29\\
296	7.08294895394363e-30\\
297	5.15141693714858e-29\\
298	2.05561783085453e-29\\
299	5.84602502990372e-29\\
300	6.68579412936497e-31\\
};
\addlegendentry{$\|f u_,^T b\|$};

\end{axis}
\end{tikzpicture}%
\end{document}
% This file was created by matlab2tikz.
% Minimal pgfplots version: 1.3
%
%The latest updates can be retrieved from
%  http://www.mathworks.com/matlabcentral/fileexchange/22022-matlab2tikz
%where you can also make suggestions and rate matlab2tikz.
%
\documentclass[tikz]{standalone}
\usepackage{pgfplots}
\usepackage{grffile}
\pgfplotsset{compat=newest}
\usetikzlibrary{plotmarks}
\usepackage{amsmath}

\begin{document}
\definecolor{mycolor1}{rgb}{0.00000,0.44700,0.74100}%
\definecolor{mycolor2}{rgb}{0.85000,0.32500,0.09800}%
\definecolor{mycolor3}{rgb}{0.92900,0.69400,0.12500}%
%
\begin{tikzpicture}

\begin{axis}[%
width=2in,
height=2in,
at={(0.758333in,0.48125in)},
scale only axis,
xmode=log,
xmin=1,
xmax=1000,
xminorticks=true,
ymode=log,
ymin=1e-30,
ymax=1.00306269402821,
yminorticks=true,
legend style={legend cell align=left,align=left,draw=white!15!black}
]
\addplot [color=mycolor1,solid]
  table[row sep=crcr]{%
1	0.536205864055256\\
2	0.0128489103543282\\
3	0.000246321004479307\\
4	4.5527658856149e-06\\
5	8.30999540825155e-08\\
6	1.50796320010826e-09\\
7	4.41779030515203e-11\\
8	4.361141249733e-11\\
9	2.68522542781873e-11\\
10	1.43981260920365e-11\\
11	1.42575796018979e-11\\
12	1.28295733853418e-11\\
13	1.26594636330019e-11\\
14	8.30162322784996e-12\\
15	8.27470324422607e-12\\
16	7.48482207320233e-12\\
17	7.39830765193137e-12\\
18	6.88318432937019e-12\\
19	6.88215193543087e-12\\
20	6.65361061909142e-12\\
21	6.5559038098579e-12\\
22	6.55310738411251e-12\\
23	6.54274742416027e-12\\
24	6.17180091499206e-12\\
25	6.16463013279266e-12\\
26	5.78707200938571e-12\\
27	5.72233851248293e-12\\
28	5.35857495279949e-12\\
29	5.21707548933374e-12\\
30	5.10724517456588e-12\\
31	5.06030358876717e-12\\
32	5.03717315401278e-12\\
33	5.0346962460891e-12\\
34	4.93537319021544e-12\\
35	4.8841679958691e-12\\
36	4.86610161184406e-12\\
37	4.83690400767012e-12\\
38	4.75865919978054e-12\\
39	4.74025358214874e-12\\
40	4.58303147135831e-12\\
41	4.58093326306111e-12\\
42	4.4734151600971e-12\\
43	4.45860658456971e-12\\
44	4.41642414432769e-12\\
45	4.39363693158798e-12\\
46	4.34087878178962e-12\\
47	4.2895661739437e-12\\
48	4.24523255069558e-12\\
49	4.17830339575981e-12\\
50	4.1678057509157e-12\\
51	4.06189706090399e-12\\
52	4.00543157803481e-12\\
53	3.99620525040862e-12\\
54	3.97501237208706e-12\\
55	3.97480806864914e-12\\
56	3.91683037172815e-12\\
57	3.89086188010497e-12\\
58	3.88575679110746e-12\\
59	3.70084333296939e-12\\
60	3.5740517968995e-12\\
61	3.5720338504338e-12\\
62	3.44157536489474e-12\\
63	3.37299416084085e-12\\
64	3.32915673864825e-12\\
65	3.32612212820083e-12\\
66	3.22013322487734e-12\\
67	3.2062521650705e-12\\
68	3.17364826514853e-12\\
69	3.16064433019653e-12\\
70	3.13233426343277e-12\\
71	3.12940420008906e-12\\
72	3.0076645942228e-12\\
73	2.98743630249128e-12\\
74	2.98681638262325e-12\\
75	2.97987340231994e-12\\
76	2.92848577080368e-12\\
77	2.8798557639102e-12\\
78	2.84009862286563e-12\\
79	2.83354820136565e-12\\
80	2.80012457938812e-12\\
81	2.7998958908849e-12\\
82	2.79193880584921e-12\\
83	2.78676640944178e-12\\
84	2.7759939501516e-12\\
85	2.77468814979306e-12\\
86	2.7466104959238e-12\\
87	2.71757135545931e-12\\
88	2.71656140546955e-12\\
89	2.71081824086109e-12\\
90	2.70991104927929e-12\\
91	2.67323054608835e-12\\
92	2.57142536946519e-12\\
93	2.55907869033864e-12\\
94	2.47418753417212e-12\\
95	2.46881160913307e-12\\
96	2.3900520655707e-12\\
97	2.34198704164785e-12\\
98	2.29635579486974e-12\\
99	2.29013200830342e-12\\
100	2.24214596991682e-12\\
101	2.24080127321005e-12\\
102	2.23421643504819e-12\\
103	2.23419347549873e-12\\
104	2.19974547202693e-12\\
105	2.17043999744548e-12\\
106	2.15670364158121e-12\\
107	2.14611584708509e-12\\
108	2.12343766575414e-12\\
109	2.08820872366465e-12\\
110	2.07884853838017e-12\\
111	2.07678095135627e-12\\
112	2.03988794556486e-12\\
113	2.03906700258972e-12\\
114	2.03188836587899e-12\\
115	2.02277687238593e-12\\
116	2.00909629728866e-12\\
117	2.00030459052271e-12\\
118	1.99445471485064e-12\\
119	1.9860005036029e-12\\
120	1.97937278103031e-12\\
121	1.95046068835204e-12\\
122	1.87357803278577e-12\\
123	1.87082024698877e-12\\
124	1.8699952254854e-12\\
125	1.8425998268903e-12\\
126	1.8380057930775e-12\\
127	1.81205636927122e-12\\
128	1.79146502179152e-12\\
129	1.78335767131868e-12\\
130	1.75422829482963e-12\\
131	1.73251730236237e-12\\
132	1.72575977906109e-12\\
133	1.72466704267919e-12\\
134	1.71693107454721e-12\\
135	1.69915757144353e-12\\
136	1.67575557067438e-12\\
137	1.67150431409567e-12\\
138	1.656101966807e-12\\
139	1.65454388966572e-12\\
140	1.63698111451644e-12\\
141	1.6248718267306e-12\\
142	1.6037442420301e-12\\
143	1.58715214640023e-12\\
144	1.55791235054729e-12\\
145	1.55499467301294e-12\\
146	1.54484806543184e-12\\
147	1.53379746269893e-12\\
148	1.51779911485272e-12\\
149	1.50952225855598e-12\\
150	1.50616269227195e-12\\
151	1.49581429903512e-12\\
152	1.47475212765797e-12\\
153	1.46684391072078e-12\\
154	1.4473738628954e-12\\
155	1.43607474679927e-12\\
156	1.431575744067e-12\\
157	1.41753561188507e-12\\
158	1.4028478831867e-12\\
159	1.39275514651843e-12\\
160	1.37434504132033e-12\\
161	1.34838786326108e-12\\
162	1.33409097670716e-12\\
163	1.33053833939709e-12\\
164	1.32589730364162e-12\\
165	1.3199249074326e-12\\
166	1.31343274879047e-12\\
167	1.29935228193453e-12\\
168	1.27087391158071e-12\\
169	1.24471406796872e-12\\
170	1.2425424712875e-12\\
171	1.24180152803099e-12\\
172	1.23729952351309e-12\\
173	1.21307968664717e-12\\
174	1.21273673544603e-12\\
175	1.20975179221567e-12\\
176	1.20748509548202e-12\\
177	1.16605306735764e-12\\
178	1.15216407396824e-12\\
179	1.15157948561871e-12\\
180	1.13931412989482e-12\\
181	1.13803415003071e-12\\
182	1.1331268672274e-12\\
183	1.13127791207621e-12\\
184	1.09419555282983e-12\\
185	1.08566659069285e-12\\
186	1.08461377723108e-12\\
187	1.08093370416653e-12\\
188	1.07793314691648e-12\\
189	1.05573169932796e-12\\
190	1.0537713771496e-12\\
191	1.05290148352758e-12\\
192	1.0375721554539e-12\\
193	1.01170229699993e-12\\
194	1.00745890255617e-12\\
195	1.00217077359428e-12\\
196	9.91292999592537e-13\\
197	9.86980363434334e-13\\
198	9.69922356435801e-13\\
199	9.34970292695366e-13\\
200	9.33411982829871e-13\\
201	9.23062991125281e-13\\
202	9.15003199811027e-13\\
203	9.04247152823742e-13\\
204	8.97732659714442e-13\\
205	8.91485722442326e-13\\
206	8.84739176450774e-13\\
207	8.82921877396405e-13\\
208	8.81981339245974e-13\\
209	8.80726196603391e-13\\
210	8.75021587655936e-13\\
211	8.74567467677427e-13\\
212	8.45318530773659e-13\\
213	8.4130764937703e-13\\
214	7.9132579604315e-13\\
215	7.86139405902674e-13\\
216	7.66165538141336e-13\\
217	7.59189055206972e-13\\
218	7.53555934026106e-13\\
219	7.4998655544059e-13\\
220	7.40712561226423e-13\\
221	7.38964946180797e-13\\
222	7.33144009285606e-13\\
223	7.27017545014912e-13\\
224	7.2206057566083e-13\\
225	7.01927184497181e-13\\
226	6.99911031569918e-13\\
227	6.90629841312224e-13\\
228	6.84155242443531e-13\\
229	6.79779534226397e-13\\
230	6.70793107504898e-13\\
231	6.67173876996793e-13\\
232	6.64138855657572e-13\\
233	6.59360536691586e-13\\
234	6.58998327800041e-13\\
235	6.55311516769742e-13\\
236	6.48300217303725e-13\\
237	6.46167663397732e-13\\
238	6.31955969410722e-13\\
239	6.16695595893269e-13\\
240	6.10003643342699e-13\\
241	6.07881358791686e-13\\
242	5.98515463323529e-13\\
243	5.97156414363192e-13\\
244	5.94621799939614e-13\\
245	5.91086538767881e-13\\
246	5.88525897478942e-13\\
247	5.84046426955061e-13\\
248	5.82582223288204e-13\\
249	5.70346774753978e-13\\
250	5.69827736414645e-13\\
251	5.53893646833146e-13\\
252	5.51974874054012e-13\\
253	5.51499032709644e-13\\
254	5.27377513749702e-13\\
255	5.27019280505028e-13\\
256	5.23550752670261e-13\\
257	4.95727784704037e-13\\
258	4.90784348704905e-13\\
259	4.43833658733527e-13\\
260	4.34411676052559e-13\\
261	3.97085784286878e-13\\
262	3.77810251701056e-13\\
263	3.71988544032703e-13\\
264	3.65268461748413e-13\\
265	3.62914496612364e-13\\
266	3.57768847022263e-13\\
267	3.55830097051104e-13\\
268	3.49343474209231e-13\\
269	3.45637896026195e-13\\
270	3.30335034534662e-13\\
271	3.29289059581083e-13\\
272	3.29102052220645e-13\\
273	3.07167066881905e-13\\
274	3.05115763274179e-13\\
275	2.95512703023722e-13\\
276	2.67876436314007e-13\\
277	2.65097838133931e-13\\
278	2.64921201953736e-13\\
279	2.58505449244614e-13\\
280	2.53418060154238e-13\\
281	2.44933830991752e-13\\
282	2.08853578902867e-13\\
283	1.96253100284313e-13\\
284	1.92918614072582e-13\\
285	1.70402767464273e-13\\
286	1.69373092669225e-13\\
287	1.28903094500653e-13\\
288	1.17047155937048e-13\\
289	1.11515966397114e-13\\
290	1.0895908870728e-13\\
291	1.02629815835018e-13\\
292	1.02480776156657e-13\\
293	9.87332310332459e-14\\
294	9.62438296629637e-14\\
295	9.05307120788325e-14\\
296	8.9983785457711e-14\\
297	6.8609462344955e-14\\
298	6.18394372638126e-14\\
299	1.5639700828303e-14\\
300	6.40276593151801e-15\\
};
\addlegendentry{$\sigma$};

\addplot [color=mycolor2,solid]
  table[row sep=crcr]{%
1	1.00306269402821\\
2	0.156820989285278\\
3	0.00790024590117674\\
4	0.0031016778928327\\
5	0.0173889760316406\\
6	0.00948842854319111\\
7	0.00881089532586302\\
8	0.00376309049435355\\
9	0.00466052297792859\\
10	0.0146314790606381\\
11	0.0229365826524925\\
12	0.00484687825599551\\
13	0.000515554536408774\\
14	0.0140215781622701\\
15	0.0101910112139421\\
16	0.00796780904896596\\
17	0.0117912254337581\\
18	0.0117282525128102\\
19	0.0114688933835067\\
20	0.0160936634858542\\
21	0.0101223087822388\\
22	0.0070567079586362\\
23	0.0058808359402986\\
24	0.00619062242227081\\
25	0.00449281227874192\\
26	0.00402311780615037\\
27	0.00664481262024154\\
28	0.00652408354611199\\
29	0.0171222868972294\\
30	0.00727716787413992\\
31	0.0089737525768023\\
32	0.0062137243069922\\
33	0.00434999028676903\\
34	0.019868575173457\\
35	0.0145245128004446\\
36	0.00065279798303325\\
37	0.0222115316373837\\
38	0.00628238983121866\\
39	0.00910319973438118\\
40	0.0039485061691802\\
41	0.00558737465058413\\
42	0.00115096359797042\\
43	0.00535955324354315\\
44	0.00887342624541675\\
45	0.0107748129452414\\
46	0.00283409330689128\\
47	0.00491846314164334\\
48	0.00231728089747968\\
49	0.00937525515542735\\
50	0.000388087794237602\\
51	0.00922174469736291\\
52	0.00933310745639782\\
53	0.00697699423598163\\
54	0.00212899207705302\\
55	0.00279114236154396\\
56	0.0033171713857913\\
57	0.00749225962934308\\
58	0.0105098867669545\\
59	0.0110565543757462\\
60	0.0162867014082241\\
61	0.0106874923979469\\
62	0.0150438406421202\\
63	0.00644036072012375\\
64	0.0128045532941873\\
65	0.0076606296234891\\
66	0.00766680146061181\\
67	0.0203541869480934\\
68	0.00921085513829187\\
69	0.0084620493242931\\
70	0.000829304878047873\\
71	0.0160006315961418\\
72	0.0140588862219675\\
73	0.0120239204799012\\
74	0.0167857768893241\\
75	0.0108966012280214\\
76	0.00290882767719631\\
77	0.00808143630770248\\
78	0.00541455937113022\\
79	0.00123678135062135\\
80	0.000192328081502919\\
81	0.00225922512480464\\
82	0.00121649228779227\\
83	0.00908047632725867\\
84	0.00339520481801497\\
85	6.88135928507217e-05\\
86	0.0114716719317967\\
87	0.00739778266573604\\
88	0.00733967312662228\\
89	0.00424371094648958\\
90	0.00252861462188757\\
91	0.0010253460093292\\
92	0.013702340497209\\
93	0.00703576427444349\\
94	0.00553653668490208\\
95	0.0125554265479132\\
96	0.0214097404967043\\
97	0.00074662713121737\\
98	0.00852829409756244\\
99	0.00123440559264974\\
100	0.0251484522038404\\
101	0.0123885525649897\\
102	0.00445901504014811\\
103	0.00651180675035347\\
104	0.0106014619462272\\
105	0.00463101155082546\\
106	0.00100887118560111\\
107	0.00972205542170057\\
108	0.00122937005149197\\
109	0.00095119454320597\\
110	0.00296309246576237\\
111	0.0211925658335149\\
112	0.00329260607569097\\
113	0.00815562059169033\\
114	0.00295960885848368\\
115	0.0135207572731812\\
116	0.0102026339732442\\
117	0.00246906508325804\\
118	0.00561449319221097\\
119	0.0088243561858611\\
120	0.000494598064962781\\
121	0.0116877537266813\\
122	0.00179215264772436\\
123	0.00319189620803154\\
124	0.010075453916187\\
125	0.00925165227374716\\
126	0.00857636591756871\\
127	0.00166491949394228\\
128	0.000158756310788415\\
129	0.0143222436699629\\
130	0.00243011044708843\\
131	0.0020137489042754\\
132	0.000280180034288282\\
133	0.00617462326153074\\
134	0.00166815749102051\\
135	0.00227574631818167\\
136	0.00259868335974357\\
137	0.00664051320772901\\
138	0.0112894319807761\\
139	0.00367067145303739\\
140	0.0108890516113278\\
141	0.00239306755533344\\
142	0.00770028952010158\\
143	0.010432165295499\\
144	0.0101560094201521\\
145	0.011918453690872\\
146	0.00688267117841078\\
147	0.0032058990774836\\
148	0.00891354575739114\\
149	0.0063877228249306\\
150	0.0125297329376051\\
151	0.000547789104236267\\
152	0.0129134550189426\\
153	0.000401889214045711\\
154	0.00829470922637865\\
155	0.0143915671285048\\
156	8.77431361832053e-05\\
157	0.00901752540125578\\
158	0.0070351034206788\\
159	0.000346998667534346\\
160	0.0070950440895971\\
161	0.00173118124753617\\
162	0.00407473655784481\\
163	0.00158703195345857\\
164	0.0109709299173663\\
165	0.00296735590181467\\
166	0.00721917589569445\\
167	0.0140282739752901\\
168	0.00919162665245243\\
169	0.0133774624618373\\
170	0.00861519172593263\\
171	0.000721080922632553\\
172	0.00255220708718285\\
173	0.00121028713931431\\
174	0.00566608650075773\\
175	0.00939310867165578\\
176	0.0193311550118975\\
177	0.0124715320639233\\
178	0.00262819531417334\\
179	0.0102995405036402\\
180	0.0026354268986626\\
181	0.0301191719549659\\
182	0.0132424358524105\\
183	0.0165051891702156\\
184	0.00120378532712262\\
185	0.00807154451102855\\
186	0.0048472086582099\\
187	0.00326328363234558\\
188	0.00729406862108727\\
189	0.0136489762698538\\
190	0.0101264992675956\\
191	0.00726666305430895\\
192	0.0120596888212267\\
193	0.00145156869791851\\
194	0.00423867391842035\\
195	0.00920960667229173\\
196	0.00934173231162008\\
197	0.0118395332640035\\
198	0.00488425608548445\\
199	0.00588259429222901\\
200	0.00943001802686355\\
201	0.00327843180396539\\
202	0.00530716080292846\\
203	0.00522468285148786\\
204	0.000289827683860809\\
205	0.00738889004281921\\
206	0.00920650344545399\\
207	0.00282814006977782\\
208	0.00158802323748025\\
209	0.00534407099954689\\
210	0.0119539239647982\\
211	0.00726544914074244\\
212	0.00514328564861651\\
213	0.00979423793940451\\
214	0.00232308951384133\\
215	0.00947197918949033\\
216	0.0110677865336573\\
217	0.00485606038590195\\
218	0.00537218315705119\\
219	0.0123125829684139\\
220	0.00196242049464673\\
221	0.00148466473750587\\
222	0.0138735029616988\\
223	0.00302559978796105\\
224	0.0109007295360574\\
225	0.0120009487641596\\
226	0.00776378191340297\\
227	0.00653312593248536\\
228	0.000831053682190042\\
229	0.00576597745268721\\
230	0.00180778445739414\\
231	0.0127919326661271\\
232	0.011601938284956\\
233	0.000845263735266274\\
234	0.00806192560289779\\
235	0.00039906715420455\\
236	0.0123788709293152\\
237	0.00569203777758009\\
238	0.00528370214495426\\
239	9.62286104747717e-07\\
240	0.00877575277322976\\
241	0.0112084599995816\\
242	0.0262545773964034\\
243	0.00684573258106729\\
244	0.0122923650756999\\
245	0.0011764552963562\\
246	0.000935059968312691\\
247	0.00776267175684594\\
248	0.0202683287041338\\
249	0.00342348104482683\\
250	0.0146265029014942\\
251	0.00287366731732774\\
252	0.00877577985953854\\
253	0.00711455846451261\\
254	0.0142496958802818\\
255	4.16470816799403e-05\\
256	0.0249214186905218\\
257	0.0195448817892545\\
258	0.000699532836924782\\
259	0.00199168968412178\\
260	0.00526949553379331\\
261	0.00145544146199788\\
262	0.0127835972717431\\
263	0.0201980734149067\\
264	0.00994498501625647\\
265	0.0158609252480078\\
266	0.00940822014937121\\
267	0.015354687171957\\
268	0.00686238307670088\\
269	0.010434726385836\\
270	0.00954811167770356\\
271	0.000928133393388052\\
272	0.0129636071197241\\
273	0.00843162570145561\\
274	0.00337520921683276\\
275	0.00703640545316179\\
276	0.0101721501376682\\
277	0.00383281035353564\\
278	0.0148521015696118\\
279	0.00925093685143093\\
280	0.0132312565694606\\
281	0.0173677953476787\\
282	0.00428428230131704\\
283	0.00240817737698039\\
284	0.0115933462879782\\
285	0.00437541796027568\\
286	0.0013127012688668\\
287	0.0099027605897643\\
288	0.0103215212082462\\
289	0.0145441298698442\\
290	0.0185191704220305\\
291	0.00156281102769245\\
292	0.0152670370531336\\
293	0.00528072542610267\\
294	0.0136188942083718\\
295	0.0108624758686947\\
296	0.00912671504440261\\
297	0.0124599289733725\\
298	0.0190510280143014\\
299	0.00423016252736553\\
300	0.00774735479414734\\
};
\addlegendentry{$\|u_,^T b\|$};

\addplot [color=mycolor3,solid]
  table[row sep=crcr]{%
1	0.97405286973334\\
2	0.00296631727053202\\
3	5.59776599075278e-08\\
4	7.50795470270897e-12\\
5	1.40232891793562e-14\\
6	2.51970728268452e-18\\
7	2.00818884713363e-21\\
8	8.35832553110325e-22\\
9	3.92437471724676e-22\\
10	3.54221150003582e-22\\
11	5.44495927337946e-22\\
12	9.31667305196358e-23\\
13	9.6489384708658e-24\\
14	1.12848878766444e-22\\
15	8.14885233609044e-23\\
16	5.21286157627771e-23\\
17	7.53699158189515e-23\\
18	6.48912920106625e-23\\
19	6.34372490042526e-23\\
20	8.32039592637334e-23\\
21	5.08064722227779e-23\\
22	3.53892229901215e-23\\
23	2.93990764494038e-23\\
24	2.75380032438514e-23\\
25	1.99391497921919e-23\\
26	1.57345684495275e-23\\
27	2.54099694541528e-23\\
28	2.18772357500813e-23\\
29	5.44239833176967e-23\\
30	2.21671632950286e-23\\
31	2.68349969313684e-23\\
32	1.84119613245196e-23\\
33	1.28768355138042e-23\\
34	5.65172513691833e-23\\
35	4.04629063724393e-23\\
36	1.80515901328401e-24\\
37	6.06858944838134e-23\\
38	1.66137779287492e-23\\
39	2.38875459408453e-23\\
40	9.68529411010666e-24\\
41	1.3692729734188e-23\\
42	2.68976483840406e-24\\
43	1.24423154902768e-23\\
44	2.02119031339101e-23\\
45	2.42902723911866e-23\\
46	6.23654089721252e-24\\
47	1.05689167852873e-23\\
48	4.87703605261416e-24\\
49	1.91142540229767e-23\\
50	7.87261904499751e-25\\
51	1.77682736490507e-23\\
52	1.74863519103881e-23\\
53	1.30118260792169e-23\\
54	3.92848706992418e-24\\
55	5.14977941604473e-24\\
56	5.9430823942759e-24\\
57	1.32458172989815e-23\\
58	1.85320549705785e-23\\
59	1.76846113819017e-23\\
60	2.42956841021442e-23\\
61	1.59250668423107e-23\\
62	2.08088273261684e-23\\
63	8.55688478997249e-24\\
64	1.65732327838474e-23\\
65	9.89726416511032e-24\\
66	9.2840230516932e-24\\
67	2.44356234401863e-23\\
68	1.08340753088377e-23\\
69	9.87190846964842e-24\\
70	9.5022124367813e-25\\
71	1.82993119892936e-23\\
72	1.48519690795717e-23\\
73	1.25319229776636e-23\\
74	1.74877046244104e-23\\
75	1.12995461101945e-23\\
76	2.91325528891502e-24\\
77	7.82716234355215e-24\\
78	5.10040045766863e-24\\
79	1.15965405491877e-24\\
80	1.76105013710186e-25\\
81	2.06831932116654e-24\\
82	1.10737690959289e-24\\
83	8.23538824424738e-24\\
84	3.05546506487924e-24\\
85	6.18695426529455e-26\\
86	1.01063697348014e-23\\
87	6.38025143885157e-24\\
88	6.32543045167976e-24\\
89	3.64184038000811e-24\\
90	2.16853796483746e-24\\
91	8.55692230857732e-25\\
92	1.05807613107537e-23\\
93	5.38087464540494e-24\\
94	3.95801762832994e-24\\
95	8.93679275826663e-24\\
96	1.42823752834839e-23\\
97	4.78241275900695e-25\\
98	5.25188052838389e-24\\
99	7.56054717545059e-25\\
100	1.4764316287173e-23\\
101	7.26443038769503e-24\\
102	2.59934386037881e-24\\
103	3.79592283559823e-24\\
104	5.99080159872548e-24\\
105	2.5476850924771e-24\\
106	5.48013143854437e-25\\
107	5.22924185630162e-24\\
108	6.47345265158126e-25\\
109	4.84385880199975e-25\\
110	1.49542698953039e-24\\
111	1.06742959027863e-23\\
112	1.60002479108255e-24\\
113	3.95999210867327e-24\\
114	1.42694856567746e-24\\
115	6.4605767269165e-24\\
116	4.80936929368134e-24\\
117	1.15371652842282e-24\\
118	2.6081540102665e-24\\
119	4.06458351645325e-24\\
120	2.26298624394791e-25\\
121	5.1925390151676e-24\\
122	7.34670976399789e-25\\
123	1.3046295077164e-24\\
124	4.11452712461814e-24\\
125	3.66822246917097e-24\\
126	3.38354040785076e-24\\
127	6.38426578291852e-25\\
128	5.95006866434368e-26\\
129	5.31939548369728e-24\\
130	8.7331827692314e-25\\
131	7.05886362538746e-25\\
132	9.74478324573298e-26\\
133	2.14484200702126e-24\\
134	5.74271264262402e-25\\
135	7.67300484220269e-25\\
136	8.52214702551273e-25\\
137	2.16666097905176e-24\\
138	3.61593446336794e-24\\
139	1.17348194451423e-24\\
140	3.40762430974836e-24\\
141	7.37848993780196e-25\\
142	2.31287194389905e-24\\
143	3.06892231206704e-24\\
144	2.87861410399352e-24\\
145	3.36551893204807e-24\\
146	1.91823972939235e-24\\
147	8.8076528733549e-25\\
148	2.39802331961798e-24\\
149	1.69980623902024e-24\\
150	3.31940236898425e-24\\
151	1.43134084334113e-25\\
152	3.27985640485068e-24\\
153	1.00983046538132e-25\\
154	2.02925643188256e-24\\
155	3.46606331418276e-24\\
156	2.09998469348048e-26\\
157	2.116068119357e-24\\
158	1.61683596823186e-24\\
159	7.86052701078849e-26\\
160	1.56502389075291e-24\\
161	3.67575470758877e-25\\
162	8.46924590212461e-25\\
163	3.28106452190931e-25\\
164	2.25235856331159e-24\\
165	6.03729503976793e-25\\
166	1.45437907771861e-24\\
167	2.76587392134801e-24\\
168	1.73369069553069e-24\\
169	2.42040059188899e-24\\
170	1.55332289943242e-24\\
171	1.29856202704626e-25\\
172	4.56288911673143e-25\\
173	2.07989491260118e-25\\
174	9.73174184909199e-25\\
175	1.60537393919601e-24\\
176	3.291513754651e-24\\
177	1.98029892829183e-24\\
178	4.07437134196141e-25\\
179	1.59507093911524e-24\\
180	3.99495834033616e-25\\
181	4.55541482043424e-24\\
182	1.98563429619823e-24\\
183	2.46679713339553e-24\\
184	1.68311290339473e-25\\
185	1.11102519164283e-24\\
186	6.65911112148417e-25\\
187	4.45273914202028e-25\\
188	9.89755245008414e-25\\
189	1.77656669327601e-24\\
190	1.31318660751436e-24\\
191	9.40772928168008e-25\\
192	1.51616697251168e-24\\
193	1.7350715144572e-25\\
194	5.0241083724812e-25\\
195	1.08018679138032e-24\\
196	1.07202721388361e-24\\
197	1.34687074503922e-24\\
198	5.36595071287274e-25\\
199	6.00535706466296e-25\\
200	9.59474816381538e-25\\
201	3.26214412977393e-25\\
202	5.18897747723854e-25\\
203	4.98894305025862e-25\\
204	2.72777302295636e-26\\
205	6.85776030392408e-25\\
206	8.41588030557767e-25\\
207	2.57465939958839e-25\\
208	1.44261339233164e-25\\
209	4.84092508064907e-25\\
210	1.0688637237004e-24\\
211	6.48968200814157e-25\\
212	4.29196070716971e-25\\
213	8.09570434130446e-25\\
214	1.69883384437317e-25\\
215	6.83619088650936e-25\\
216	7.58717873303175e-25\\
217	3.26857360275364e-25\\
218	3.56251016783963e-25\\
219	8.08780067838441e-25\\
220	1.25737784382988e-25\\
221	9.46782849065344e-26\\
222	8.70841315092179e-25\\
223	1.86756436240746e-25\\
224	6.63708140195352e-25\\
225	6.90516434708891e-25\\
226	4.44153734574543e-25\\
227	3.6390332407775e-25\\
228	4.54268590881348e-26\\
229	3.11159757175225e-25\\
230	9.49944307517123e-26\\
231	6.64949317615099e-25\\
232	5.97616587636632e-25\\
233	4.29153295036697e-26\\
234	4.08866684543754e-25\\
235	2.00131706455264e-26\\
236	6.07585893884953e-25\\
237	2.77544437856675e-25\\
238	2.46425860540591e-25\\
239	4.27385931319547e-29\\
240	3.81349824788156e-25\\
241	4.8367975604762e-25\\
242	1.0983230454859e-24\\
243	2.85082426522589e-25\\
244	5.07564708875173e-25\\
245	4.8001176343212e-26\\
246	3.78220386746403e-26\\
247	3.0922904458923e-25\\
248	8.03353535738863e-25\\
249	1.30052953959699e-25\\
250	5.54628247329748e-25\\
251	1.02958823128334e-25\\
252	3.12247271385109e-25\\
253	2.52703802011566e-25\\
254	4.62831731137411e-25\\
255	1.35086483206333e-27\\
256	7.97746029876272e-25\\
257	5.60910905163443e-25\\
258	1.96772239273341e-26\\
259	4.58180346438801e-26\\
260	1.16130510034678e-25\\
261	2.68001815013968e-26\\
262	2.130957901436e-25\\
263	3.26394919728959e-25\\
264	1.54954002227324e-25\\
265	2.439559779161e-25\\
266	1.40632865957865e-25\\
267	2.27039093617409e-25\\
268	9.78035325369295e-26\\
269	1.45578783088233e-25\\
270	1.21674921887234e-25\\
271	1.17527450569351e-26\\
272	1.63968851611957e-25\\
273	9.29041113953168e-26\\
274	3.66947811841815e-26\\
275	7.17591818674178e-26\\
276	8.52424847069438e-26\\
277	3.14560376938236e-26\\
278	1.21729449371752e-25\\
279	7.21937222925696e-26\\
280	9.92317222024361e-26\\
281	1.21679259202318e-25\\
282	2.1824094654428e-26\\
283	1.08316804185585e-26\\
284	5.03884951310222e-26\\
285	1.48370358497215e-26\\
286	4.39773497630219e-27\\
287	1.92157473371959e-26\\
288	1.65135236830535e-26\\
289	2.11220546482662e-26\\
290	2.56757280012765e-26\\
291	1.92232971676632e-27\\
292	1.87246576005926e-26\\
293	6.01166261275316e-27\\
294	1.47320081782999e-26\\
295	1.03966879906498e-26\\
296	8.63012970555019e-27\\
297	6.84948489204113e-27\\
298	8.50792762437123e-27\\
299	1.20833748821505e-28\\
300	3.70905274601049e-29\\
};
\addlegendentry{$\|f u_,^T b\|$};

\end{axis}
\end{tikzpicture}%
\end{document}
\caption{Plot of Fourier coefficients and singular values for the first $\mathbf{A}_1, \mathbf{b}_{err1}$ and second $\mathbf{A}_2, \mathbf{b}_{err2}$ value pair .}
\label{fig:picard12}
\end{figure}
\begin{figure}
% This file was created by matlab2tikz.
% Minimal pgfplots version: 1.3
%
%The latest updates can be retrieved from
%  http://www.mathworks.com/matlabcentral/fileexchange/22022-matlab2tikz
%where you can also make suggestions and rate matlab2tikz.
%
\documentclass[tikz]{standalone}
\usepackage{pgfplots}
\usepackage{grffile}
\pgfplotsset{compat=newest}
\usetikzlibrary{plotmarks}
\usepackage{amsmath}

\begin{document}
\definecolor{mycolor1}{rgb}{0.00000,0.44700,0.74100}%
\definecolor{mycolor2}{rgb}{0.85000,0.32500,0.09800}%
\definecolor{mycolor3}{rgb}{0.92900,0.69400,0.12500}%
%
\begin{tikzpicture}

\begin{axis}[%
width=2in,
height=2in,
at={(0.758333in,0.48125in)},
scale only axis,
xmode=log,
xmin=1,
xmax=1000,
xminorticks=true,
ymode=log,
ymin=1e-40,
ymax=1,
yminorticks=true,
legend style={legend cell align=left,align=left,draw=white!15!black}
]
\addplot [color=mycolor1,solid]
  table[row sep=crcr]{%
1	0.446980426276884\\
2	0.0336898438481169\\
3	0.00110796542625124\\
4	2.35822671541614e-05\\
5	3.72802626805478e-07\\
6	4.69361029065491e-09\\
7	4.91199467307743e-11\\
8	4.39904777127957e-13\\
9	3.43812927000423e-15\\
10	4.03123992891142e-16\\
11	2.0545898614428e-16\\
12	1.65443371869834e-16\\
13	1.28749251110295e-16\\
14	1.16211061362324e-16\\
15	5.94109566154083e-17\\
16	4.41397998369132e-17\\
17	4.41397998369132e-17\\
18	4.41397998369132e-17\\
19	4.41397998369132e-17\\
20	4.41397998369132e-17\\
21	4.41397998369132e-17\\
22	4.41397998369132e-17\\
23	4.41397998369132e-17\\
24	4.41397998369132e-17\\
25	4.41397998369132e-17\\
26	4.41397998369132e-17\\
27	4.41397998369132e-17\\
28	4.41397998369132e-17\\
29	4.41397998369132e-17\\
30	4.41397998369132e-17\\
31	4.41397998369132e-17\\
32	4.41397998369132e-17\\
33	4.41397998369132e-17\\
34	4.41397998369132e-17\\
35	4.41397998369132e-17\\
36	4.41397998369132e-17\\
37	4.41397998369132e-17\\
38	4.41397998369132e-17\\
39	4.41397998369132e-17\\
40	4.41397998369132e-17\\
41	4.41397998369132e-17\\
42	4.41397998369132e-17\\
43	4.41397998369132e-17\\
44	4.41397998369132e-17\\
45	4.41397998369132e-17\\
46	4.41397998369132e-17\\
47	4.41397998369132e-17\\
48	4.41397998369132e-17\\
49	4.41397998369132e-17\\
50	4.41397998369132e-17\\
51	4.41397998369132e-17\\
52	4.41397998369132e-17\\
53	4.41397998369132e-17\\
54	4.41397998369132e-17\\
55	4.41397998369132e-17\\
56	4.41397998369132e-17\\
57	4.41397998369132e-17\\
58	4.41397998369132e-17\\
59	4.41397998369132e-17\\
60	4.41397998369132e-17\\
61	4.41397998369132e-17\\
62	4.41397998369132e-17\\
63	4.41397998369132e-17\\
64	4.41397998369132e-17\\
65	4.41397998369132e-17\\
66	4.41397998369132e-17\\
67	4.41397998369132e-17\\
68	4.41397998369132e-17\\
69	4.41397998369132e-17\\
70	4.41397998369132e-17\\
71	4.41397998369132e-17\\
72	4.41397998369132e-17\\
73	4.41397998369132e-17\\
74	4.41397998369132e-17\\
75	4.41397998369132e-17\\
76	4.41397998369132e-17\\
77	4.41397998369132e-17\\
78	4.41397998369132e-17\\
79	4.41397998369132e-17\\
80	4.41397998369132e-17\\
81	4.41397998369132e-17\\
82	4.41397998369132e-17\\
83	4.41397998369132e-17\\
84	4.41397998369132e-17\\
85	4.41397998369132e-17\\
86	4.41397998369132e-17\\
87	4.41397998369132e-17\\
88	4.41397998369132e-17\\
89	4.41397998369132e-17\\
90	4.41397998369132e-17\\
91	4.41397998369132e-17\\
92	4.41397998369132e-17\\
93	4.41397998369132e-17\\
94	4.41397998369132e-17\\
95	4.41397998369132e-17\\
96	4.41397998369132e-17\\
97	4.41397998369132e-17\\
98	4.41397998369132e-17\\
99	4.41397998369132e-17\\
100	4.41397998369132e-17\\
101	4.41397998369132e-17\\
102	4.41397998369132e-17\\
103	4.41397998369132e-17\\
104	4.41397998369132e-17\\
105	4.41397998369132e-17\\
106	4.41397998369132e-17\\
107	4.41397998369132e-17\\
108	4.41397998369132e-17\\
109	4.41397998369132e-17\\
110	4.41397998369132e-17\\
111	4.41397998369132e-17\\
112	4.41397998369132e-17\\
113	4.41397998369132e-17\\
114	4.41397998369132e-17\\
115	4.41397998369132e-17\\
116	4.41397998369132e-17\\
117	4.41397998369132e-17\\
118	4.41397998369132e-17\\
119	4.41397998369132e-17\\
120	4.41397998369132e-17\\
121	4.41397998369132e-17\\
122	4.41397998369132e-17\\
123	4.41397998369132e-17\\
124	4.41397998369132e-17\\
125	4.41397998369132e-17\\
126	4.41397998369132e-17\\
127	4.41397998369132e-17\\
128	4.41397998369132e-17\\
129	4.41397998369132e-17\\
130	4.41397998369132e-17\\
131	4.41397998369132e-17\\
132	4.41397998369132e-17\\
133	4.41397998369132e-17\\
134	4.41397998369132e-17\\
135	4.41397998369132e-17\\
136	4.41397998369132e-17\\
137	4.41397998369132e-17\\
138	4.41397998369132e-17\\
139	4.41397998369132e-17\\
140	4.41397998369132e-17\\
141	4.41397998369132e-17\\
142	4.41397998369132e-17\\
143	4.41397998369132e-17\\
144	4.41397998369132e-17\\
145	4.41397998369132e-17\\
146	4.41397998369132e-17\\
147	4.41397998369132e-17\\
148	4.41397998369132e-17\\
149	4.41397998369132e-17\\
150	4.41397998369132e-17\\
151	4.41397998369132e-17\\
152	4.41397998369132e-17\\
153	4.41397998369132e-17\\
154	4.41397998369132e-17\\
155	4.41397998369132e-17\\
156	4.41397998369132e-17\\
157	4.41397998369132e-17\\
158	4.41397998369132e-17\\
159	4.41397998369132e-17\\
160	4.41397998369132e-17\\
161	4.41397998369132e-17\\
162	4.41397998369132e-17\\
163	4.41397998369132e-17\\
164	4.41397998369132e-17\\
165	4.41397998369132e-17\\
166	4.41397998369132e-17\\
167	4.41397998369132e-17\\
168	4.41397998369132e-17\\
169	4.41397998369132e-17\\
170	4.41397998369132e-17\\
171	4.41397998369132e-17\\
172	4.41397998369132e-17\\
173	4.41397998369132e-17\\
174	4.41397998369132e-17\\
175	4.41397998369132e-17\\
176	4.41397998369132e-17\\
177	4.41397998369132e-17\\
178	4.41397998369132e-17\\
179	4.41397998369132e-17\\
180	4.41397998369132e-17\\
181	4.41397998369132e-17\\
182	4.41397998369132e-17\\
183	4.41397998369132e-17\\
184	4.41397998369132e-17\\
185	4.41397998369132e-17\\
186	4.41397998369132e-17\\
187	4.41397998369132e-17\\
188	4.41397998369132e-17\\
189	4.41397998369132e-17\\
190	4.41397998369132e-17\\
191	4.41397998369132e-17\\
192	4.41397998369132e-17\\
193	4.41397998369132e-17\\
194	4.41397998369132e-17\\
195	4.41397998369132e-17\\
196	4.41397998369132e-17\\
197	4.41397998369132e-17\\
198	4.41397998369132e-17\\
199	4.41397998369132e-17\\
200	4.41397998369132e-17\\
201	4.41397998369132e-17\\
202	4.41397998369132e-17\\
203	4.41397998369132e-17\\
204	4.41397998369132e-17\\
205	4.41397998369132e-17\\
206	4.41397998369132e-17\\
207	4.41397998369132e-17\\
208	4.41397998369132e-17\\
209	4.41397998369132e-17\\
210	4.41397998369132e-17\\
211	4.41397998369132e-17\\
212	4.41397998369132e-17\\
213	4.41397998369132e-17\\
214	4.41397998369132e-17\\
215	4.41397998369132e-17\\
216	4.41397998369132e-17\\
217	4.41397998369132e-17\\
218	4.41397998369132e-17\\
219	4.41397998369132e-17\\
220	4.41397998369132e-17\\
221	4.41397998369132e-17\\
222	4.41397998369132e-17\\
223	4.41397998369132e-17\\
224	4.41397998369132e-17\\
225	4.41397998369132e-17\\
226	4.41397998369132e-17\\
227	4.41397998369132e-17\\
228	4.41397998369132e-17\\
229	4.41397998369132e-17\\
230	4.41397998369132e-17\\
231	4.41397998369132e-17\\
232	4.41397998369132e-17\\
233	4.41397998369132e-17\\
234	4.41397998369132e-17\\
235	4.41397998369132e-17\\
236	4.41397998369132e-17\\
237	4.41397998369132e-17\\
238	4.41397998369132e-17\\
239	4.41397998369132e-17\\
240	4.41397998369132e-17\\
241	4.41397998369132e-17\\
242	4.41397998369132e-17\\
243	4.41397998369132e-17\\
244	4.41397998369132e-17\\
245	4.41397998369132e-17\\
246	4.41397998369132e-17\\
247	4.41397998369132e-17\\
248	4.41397998369132e-17\\
249	4.41397998369132e-17\\
250	4.41397998369132e-17\\
251	4.41397998369132e-17\\
252	4.41397998369132e-17\\
253	4.41397998369132e-17\\
254	4.41397998369132e-17\\
255	4.41397998369132e-17\\
256	4.41397998369132e-17\\
257	4.41397998369132e-17\\
258	4.41397998369132e-17\\
259	4.41397998369132e-17\\
260	4.41397998369132e-17\\
261	4.41397998369132e-17\\
262	4.41397998369132e-17\\
263	4.41397998369132e-17\\
264	4.41397998369132e-17\\
265	4.41397998369132e-17\\
266	4.41397998369132e-17\\
267	4.41397998369132e-17\\
268	4.41397998369132e-17\\
269	4.41397998369132e-17\\
270	4.41397998369132e-17\\
271	4.41397998369132e-17\\
272	4.41397998369132e-17\\
273	4.41397998369132e-17\\
274	4.41397998369132e-17\\
275	4.41397998369132e-17\\
276	4.41397998369132e-17\\
277	4.41397998369132e-17\\
278	4.41397998369132e-17\\
279	4.41397998369132e-17\\
280	4.41397998369132e-17\\
281	4.41397998369132e-17\\
282	4.41397998369132e-17\\
283	4.41397998369132e-17\\
284	4.41397998369132e-17\\
285	4.41397998369132e-17\\
286	4.41397998369132e-17\\
287	4.41397998369132e-17\\
288	4.41397998369132e-17\\
289	4.41397998369132e-17\\
290	4.41397998369132e-17\\
291	4.41397998369132e-17\\
292	4.41397998369132e-17\\
293	4.41397998369132e-17\\
294	4.41397998369132e-17\\
295	4.41397998369132e-17\\
296	4.41397998369132e-17\\
297	4.41397998369132e-17\\
298	4.41397998369132e-17\\
299	2.98427479735278e-17\\
300	4.45412949201604e-20\\
};
\addlegendentry{$\sigma$};

\addplot [color=mycolor2,solid]
  table[row sep=crcr]{%
1	0.146013581972142\\
2	0.0101448324602603\\
3	0.00156957472683076\\
4	0.00180870805449516\\
5	0.00255713174147866\\
6	0.000859920603978897\\
7	0.000816655196863779\\
8	0.000516859948234312\\
9	0.000777387980615141\\
10	0.000638045726378672\\
11	0.000108271004378891\\
12	0.000898407515396247\\
13	0.000184635435254296\\
14	0.000705880708005417\\
15	0.00182755470293057\\
16	0.00124700319507047\\
17	0.000122411886355495\\
18	0.0010831694274768\\
19	0.00129951792015128\\
20	0.00102029410847988\\
21	0.000784140288791479\\
22	0.0012675945361936\\
23	0.00049205840383077\\
24	0.00124748770419211\\
25	0.000645543721506041\\
26	0.000637117701095237\\
27	0.000907974826941688\\
28	0.000205269398928789\\
29	0.000470112613867324\\
30	0.00103257161587428\\
31	0.000622690288220238\\
32	0.000880368567070502\\
33	0.000849693508237935\\
34	0.0011748600382815\\
35	0.000915850852006695\\
36	0.000206039343224133\\
37	0.0005605965450797\\
38	0.000459670125639168\\
39	0.000665631014169459\\
40	0.000451858741585831\\
41	0.000570506067417279\\
42	0.000579094806112814\\
43	0.00122944474270078\\
44	0.00125435470469593\\
45	0.000331322248613612\\
46	0.00091006271544282\\
47	0.00140936775377338\\
48	0.000190461964313623\\
49	0.00159377761623401\\
50	0.000568304105287123\\
51	0.000778623981873019\\
52	0.000128874082982671\\
53	0.00187127740201529\\
54	0.000791520559918393\\
55	0.00131409003733549\\
56	0.000526448805230442\\
57	0.00113853684011905\\
58	0.00192014682254853\\
59	0.000728974063615139\\
60	0.00069264950942778\\
61	0.000274519979291279\\
62	0.00047653013188245\\
63	1.9662557799532e-05\\
64	0.000503593774791833\\
65	0.0011432561768416\\
66	0.00102595266316782\\
67	0.000462113350374838\\
68	0.00110525702878369\\
69	0.00107956452181369\\
70	0.000856590312852381\\
71	0.00117383774663899\\
72	0.00109011800158444\\
73	0.000361063833471673\\
74	0.000677452982804176\\
75	0.00138520959647584\\
76	0.000995232387958703\\
77	0.00181979970302597\\
78	0.000601859725860168\\
79	0.0012076014785146\\
80	0.000826129649146026\\
81	0.000647021692292431\\
82	0.00249840369604963\\
83	0.000180736491425922\\
84	0.000442283191184899\\
85	0.000587925918042416\\
86	0.000319395655028652\\
87	0.00055580574211048\\
88	0.00078388753473098\\
89	0.000792798575078961\\
90	0.000756124855570662\\
91	0.00044426950661572\\
92	0.00166205148781133\\
93	0.00127305629714733\\
94	0.000515456446720668\\
95	0.00188528003175945\\
96	0.000117176550513467\\
97	0.00038325136917682\\
98	0.000436476116806729\\
99	0.000474662533062789\\
100	0.000196029008841863\\
101	0.000492545912001864\\
102	0.000130155194773111\\
103	0.000141489007448776\\
104	0.000159508020911067\\
105	0.000380998737294314\\
106	0.00187800387110677\\
107	0.000645483274778964\\
108	0.000825852002085839\\
109	0.000640329024356165\\
110	0.000435852314380756\\
111	0.000879454599391562\\
112	0.00183619608725309\\
113	0.00156423205251027\\
114	0.00161567762104876\\
115	0.00022586319414027\\
116	0.00108113357698459\\
117	9.26213618182041e-05\\
118	0.000591039589189864\\
119	0.000195650522483164\\
120	0.000786207173706546\\
121	0.000808303568912096\\
122	0.00130702673135344\\
123	0.000849603233834659\\
124	0.00110495904759481\\
125	0.000861976807690033\\
126	0.00195914109182814\\
127	0.000224059662449161\\
128	0.000994935708071768\\
129	0.00208294077369196\\
130	8.52208831583507e-07\\
131	0.000946186542605884\\
132	0.000516973019801818\\
133	0.000429711649011231\\
134	0.000942799357088193\\
135	0.000805665365245305\\
136	0.00131567517988952\\
137	0.000114179168783249\\
138	0.000913912125762109\\
139	0.00100768452634018\\
140	0.000160535843617689\\
141	0.000123027505549362\\
142	0.00294976121016805\\
143	0.000445366610250153\\
144	0.00102986831799211\\
145	0.0030118120413289\\
146	0.00139340971825546\\
147	0.000202298997246262\\
148	0.00237216513447496\\
149	0.000608430331133581\\
150	0.0019334881009935\\
151	4.54522952558836e-05\\
152	0.000190879974141108\\
153	0.000489795599557545\\
154	0.000300750544656492\\
155	0.000181765289284899\\
156	0.000323287166865315\\
157	0.000321005769033644\\
158	0.00117054802991817\\
159	0.000657953991249401\\
160	0.00140765920851539\\
161	0.000654199033128368\\
162	0.000741313281077539\\
163	0.000279318593304869\\
164	0.000774459076276331\\
165	0.00101623336610026\\
166	0.000722574187018705\\
167	0.00207504773896754\\
168	0.000812684637459036\\
169	0.00139765541793192\\
170	0.000738473641391409\\
171	0.000975292507184776\\
172	1.8665490390942e-05\\
173	0.000175251399392656\\
174	7.9649654389512e-05\\
175	0.00113131240115712\\
176	0.00106885492305668\\
177	0.000991434224295511\\
178	4.69124535904227e-05\\
179	0.00136570486010608\\
180	0.000751886637897296\\
181	0.000365742184158325\\
182	0.00257542786339268\\
183	0.00113216728576933\\
184	0.000224738171264012\\
185	0.000756841063666246\\
186	0.000199230499697252\\
187	0.000152446766574583\\
188	0.00278236778849366\\
189	0.000580426713434144\\
190	0.000287796140940269\\
191	0.000399052189874412\\
192	0.000225726783183137\\
193	0.000833351912118526\\
194	0.000534654145848606\\
195	0.0016258855501807\\
196	0.000747289414884279\\
197	0.000266402630615068\\
198	0.00170947026422449\\
199	0.000318611071702274\\
200	0.000525833084145956\\
201	0.000319006006344377\\
202	0.000716620477714405\\
203	2.97705229190121e-05\\
204	0.00275048963658379\\
205	0.000442657180866217\\
206	0.00212657515520636\\
207	0.000929938637183062\\
208	7.22434139057114e-05\\
209	0.000460532337294117\\
210	0.000385488932547805\\
211	0.0016977921577356\\
212	0.00133554214947295\\
213	0.000578539625879178\\
214	6.7193484247067e-05\\
215	0.000652363244444612\\
216	0.00169491641026077\\
217	0.000217047035014948\\
218	0.00111593652607311\\
219	0.00110958365252263\\
220	5.20973152408353e-05\\
221	0.000149225574990825\\
222	0.000371264315716042\\
223	0.00167783002304724\\
224	0.00127553388373117\\
225	0.000364658149516193\\
226	0.000908479117940749\\
227	0.00012516990975989\\
228	0.00156774619497072\\
229	0.00204156807502407\\
230	0.000166489745619427\\
231	0.000576944217845828\\
232	0.00152036274907739\\
233	0.00196004489402193\\
234	0.0017033073686116\\
235	0.000483215583838979\\
236	0.000397177516246953\\
237	0.00104222348838839\\
238	1.08387542375283e-05\\
239	0.000177783056306497\\
240	0.00115616808693504\\
241	0.000204471508632853\\
242	9.46150564637112e-05\\
243	0.000704740292546052\\
244	0.00147517836259805\\
245	0.00167451393059616\\
246	0.000456011811599841\\
247	0.000593086409113898\\
248	2.94722352652786e-05\\
249	3.93014075589651e-05\\
250	0.000799172575384625\\
251	0.000510287752769635\\
252	0.000597316569877486\\
253	0.000629026200813166\\
254	0.000287673772588319\\
255	0.000790070239365513\\
256	0.00126840590384599\\
257	0.000421761873360393\\
258	0.0012795923279067\\
259	0.000615545845281515\\
260	0.00047655779421038\\
261	0.000806413829597792\\
262	0.00103263574218232\\
263	0.00150022036945672\\
264	0.0012517720211169\\
265	2.16818958704928e-05\\
266	0.000524429159950761\\
267	0.000974216849571494\\
268	0.000197282045764998\\
269	0.00120732877582739\\
270	0.000183339815856906\\
271	0.000256637332786083\\
272	0.00128278260151709\\
273	0.00148806916763664\\
274	0.000661929030766782\\
275	0.000750459825927602\\
276	0.00100142942058316\\
277	0.00198201160950656\\
278	0.000108574686484995\\
279	0.00211185201131878\\
280	0.00193233191965886\\
281	0.0016061244075453\\
282	0.000155797068781869\\
283	6.45455626669621e-05\\
284	0.000865949629549766\\
285	0.00153646386034214\\
286	0.000868383156885472\\
287	0.000660946744721706\\
288	7.16178297249054e-05\\
289	0.00114967825933909\\
290	0.00171582181792608\\
291	0.000948584835744366\\
292	0.000888443763501019\\
293	0.000742931094460139\\
294	0.000254173498869189\\
295	0.000296614957919722\\
296	0.00188418881564204\\
297	0.000669800106328669\\
298	0.00101790181542502\\
299	8.79853716698707e-05\\
300	0.000522691841682172\\
};
\addlegendentry{$\|u_,^T b\|$};

\addplot [color=mycolor3,solid]
  table[row sep=crcr]{%
1	0.14323959540127\\
2	0.00230096479812431\\
3	4.97826907510555e-07\\
4	2.59968806762118e-10\\
5	9.18528743638041e-14\\
6	4.89614341383289e-18\\
7	5.09256082674407e-22\\
8	2.58506791147674e-26\\
9	2.37500181350184e-30\\
10	2.67985205432469e-32\\
11	1.18125638121305e-33\\
12	6.35556028350384e-33\\
13	7.91017694248069e-34\\
14	2.46381428291112e-33\\
15	1.66719038313854e-33\\
16	6.27928226307924e-34\\
17	6.16404825441263e-35\\
18	5.45429763191615e-34\\
19	6.54372006328212e-34\\
20	5.13768908037151e-34\\
21	3.94853696176447e-34\\
22	6.38296992290126e-34\\
23	2.47775917478625e-34\\
24	6.28172200785759e-34\\
25	3.25063420568539e-34\\
26	3.20820499562155e-34\\
27	4.57210554766475e-34\\
28	1.03363367547237e-34\\
29	2.36725119035488e-34\\
30	5.1995124459581e-34\\
31	3.13555578499705e-34\\
32	4.43309427756989e-34\\
33	4.2786300760284e-34\\
34	5.91600553161752e-34\\
35	4.61176525719057e-34\\
36	1.03751072853559e-34\\
37	2.82288285721957e-34\\
38	2.31466806035755e-34\\
39	3.35178373042864e-34\\
40	2.27533385052543e-34\\
41	2.87278223846889e-34\\
42	2.91603081615198e-34\\
43	6.19086670891867e-34\\
44	6.31630077608742e-34\\
45	1.66837256496798e-34\\
46	4.58261910631823e-34\\
47	7.09686869561295e-34\\
48	9.59070866084001e-35\\
49	8.0254642141036e-34\\
50	2.86169426226921e-34\\
51	3.92075960856462e-34\\
52	6.48945204505229e-35\\
53	9.42281386785988e-34\\
54	3.98570030325962e-34\\
55	6.61709793218574e-34\\
56	2.65093197689517e-34\\
57	5.73310013501431e-34\\
58	9.66889574381206e-34\\
59	3.67074753777567e-34\\
60	3.48783531291161e-34\\
61	1.38234484373327e-34\\
62	2.39956659035117e-34\\
63	9.90107731282836e-36\\
64	2.53584551374687e-34\\
65	5.75686434627905e-34\\
66	5.16618272194925e-34\\
67	2.32697091395731e-34\\
68	5.565519707103e-34\\
69	5.43614513617281e-34\\
70	4.31335892280169e-34\\
71	5.91085778395827e-34\\
72	5.48928716387587e-34\\
73	1.8181362600518e-34\\
74	3.41131323144043e-34\\
75	6.97522033959685e-34\\
76	5.01149083342798e-34\\
77	9.16359801060655e-34\\
78	3.03066352708253e-34\\
79	6.08087499284752e-34\\
80	4.15997430751841e-34\\
81	3.25807652482379e-34\\
82	1.25807071518605e-33\\
83	9.10098265496294e-35\\
84	2.22711618434025e-34\\
85	2.96049986380321e-34\\
86	1.60831622521437e-34\\
87	2.79875877780304e-34\\
88	3.94726421916422e-34\\
89	3.99213574621714e-34\\
90	3.80746530003067e-34\\
91	2.23711827198756e-34\\
92	8.36925716709836e-34\\
93	6.41047260999744e-34\\
94	2.59557997612029e-34\\
95	9.49332408382124e-34\\
96	5.90042301572806e-35\\
97	1.92986155471464e-34\\
98	2.19787467213931e-34\\
99	2.39016229997789e-34\\
100	9.87103708423264e-35\\
101	2.48021402127257e-34\\
102	6.55396240536684e-35\\
103	7.12467633127107e-35\\
104	8.03202341810466e-35\\
105	1.91851843107151e-34\\
106	9.45668499042492e-34\\
107	3.25032982630392e-34\\
108	4.15857621626463e-34\\
109	3.22437560791275e-34\\
110	2.19473351618674e-34\\
111	4.42849199502716e-34\\
112	9.24616197280249e-34\\
113	7.87668758307608e-34\\
114	8.13574164750409e-34\\
115	1.13733369285173e-34\\
116	5.44404611055954e-34\\
117	4.66394694693961e-35\\
118	2.9761787490591e-34\\
119	9.85197841069907e-35\\
120	3.9589447568995e-34\\
121	4.07021111374661e-34\\
122	6.58153066808644e-34\\
123	4.27817549943906e-34\\
124	5.56401922338212e-34\\
125	4.34048260751071e-34\\
126	9.86525131404409e-34\\
127	1.12825201238482e-34\\
128	5.00999690241062e-34\\
129	1.04886443811792e-33\\
130	4.29129598205677e-37\\
131	4.76452057062583e-34\\
132	2.60321667704198e-34\\
133	2.16381220717102e-34\\
134	4.74746440426869e-34\\
135	4.05692644410282e-34\\
136	6.62507991456023e-34\\
137	5.74948991460492e-35\\
138	4.60200280480174e-34\\
139	5.07419355302431e-34\\
140	8.08377941132778e-35\\
141	6.19504774743834e-35\\
142	1.48535170724086e-33\\
143	2.24264273529264e-34\\
144	5.18589999451416e-34\\
145	1.51659739169924e-33\\
146	7.01651203752469e-34\\
147	1.0186762233399e-34\\
148	1.194503311049e-33\\
149	3.06374979768245e-34\\
150	9.73607572653983e-34\\
151	2.28874947991125e-35\\
152	9.61175753790705e-35\\
153	2.46636482808864e-34\\
154	1.51442880670903e-34\\
155	9.15278775196423e-35\\
156	1.6279119258101e-34\\
157	1.61642395128359e-34\\
158	5.89429242154595e-34\\
159	3.31312609583256e-34\\
160	7.08826531915285e-34\\
161	3.29421801121719e-34\\
162	3.73288164429457e-34\\
163	1.40650825564915e-34\\
164	3.89978723420019e-34\\
165	5.11724121969233e-34\\
166	3.63852097111029e-34\\
167	1.04488990195451e-33\\
168	4.09227197624392e-34\\
169	7.03789125032716e-34\\
170	3.71858264395107e-34\\
171	4.91108360097934e-34\\
172	9.39900420518916e-36\\
173	8.82478094792637e-35\\
174	4.01075686129415e-35\\
175	5.6967214860953e-34\\
176	5.38221697160562e-34\\
177	4.99236518738566e-34\\
178	2.36227572561451e-35\\
179	6.87700427598388e-34\\
180	3.78612376284084e-34\\
181	1.84169408620917e-34\\
182	1.29685622028641e-33\\
183	5.70102625596569e-34\\
184	1.13166864225656e-34\\
185	3.81107176456024e-34\\
186	1.00322480965469e-34\\
187	7.67645408768548e-35\\
188	1.4010607809761e-33\\
189	2.92273763298451e-34\\
190	1.44919693095633e-34\\
191	2.00942655786837e-34\\
192	1.13664679573151e-34\\
193	4.19634199924667e-34\\
194	2.69224995427473e-34\\
195	8.18714365560992e-34\\
196	3.76297445493274e-34\\
197	1.34146994961333e-34\\
198	8.60803432728942e-34\\
199	1.60436545733765e-34\\
200	2.64783151452259e-34\\
201	1.60635414999138e-34\\
202	3.6085410790123e-34\\
203	1.49909412635767e-35\\
204	1.38500854352725e-33\\
205	2.228999408683e-34\\
206	1.07083652279155e-33\\
207	4.68270427317203e-34\\
208	3.63781575986078e-35\\
209	2.31900972553799e-34\\
210	1.94112880089611e-34\\
211	8.54922924384462e-34\\
212	6.72512000284511e-34\\
213	2.91323520712096e-34\\
214	3.38352664608253e-35\\
215	3.28497390072394e-34\\
216	8.53474842279851e-34\\
217	1.09293994001857e-34\\
218	5.61929629578625e-34\\
219	5.58730641287114e-34\\
220	2.62335933732177e-35\\
221	7.51425104556136e-35\\
222	1.86950076936911e-34\\
223	8.44870995185134e-34\\
224	6.42294849261961e-34\\
225	1.83623543179046e-34\\
226	4.57464490405001e-34\\
227	6.30292847150317e-35\\
228	7.89438304088169e-34\\
229	1.02803122341984e-33\\
230	8.38358803558285e-35\\
231	2.90520153294451e-34\\
232	7.65578378745736e-34\\
233	9.86980240830526e-34\\
234	8.57700107792469e-34\\
235	2.43323117120972e-34\\
236	1.99998664231361e-34\\
237	5.24811443200188e-34\\
238	5.45785267484838e-36\\
239	8.95226248460668e-35\\
240	5.82188224547342e-34\\
241	1.02961590037528e-34\\
242	4.76433940363136e-35\\
243	3.548718428754e-34\\
244	7.42825789361083e-34\\
245	8.43201177449856e-34\\
246	2.29624662683538e-34\\
247	2.98648550696919e-34\\
248	1.48407385711722e-35\\
249	1.97902164464867e-35\\
250	4.02423201286858e-34\\
251	2.56955302736962e-34\\
252	3.00778647360487e-34\\
253	3.16746026104204e-34\\
254	1.44858074534832e-34\\
255	3.97839721682048e-34\\
256	6.38705556320169e-34\\
257	2.12378112670779e-34\\
258	6.44338478069631e-34\\
259	3.0995799559037e-34\\
260	2.39970588395187e-34\\
261	4.06069533495384e-34\\
262	5.19983535386264e-34\\
263	7.55435687244359e-34\\
264	6.30329567774203e-34\\
265	1.09179146218457e-35\\
266	2.64076205685604e-34\\
267	4.90566712907378e-34\\
268	9.9341337351274e-35\\
269	6.07950179897468e-34\\
270	9.23207300813232e-35\\
271	1.29229681060803e-34\\
272	6.45944939751152e-34\\
273	7.49316951834095e-34\\
274	3.33314239990876e-34\\
275	3.77893905382848e-34\\
276	5.04269597965077e-34\\
277	9.98041576315954e-34\\
278	5.46727631300163e-35\\
279	1.06342268643288e-33\\
280	9.73025377758556e-34\\
281	8.08763645872446e-34\\
282	7.84515849284941e-35\\
283	3.25019060430084e-35\\
284	4.36048774457597e-34\\
285	7.73686090320196e-34\\
286	4.37274176693632e-34\\
287	3.32819610036078e-34\\
288	3.60631448010378e-35\\
289	5.78920273071774e-34\\
290	8.64001756410791e-34\\
291	4.77659717125176e-34\\
292	4.47375691413528e-34\\
293	3.74102814056538e-34\\
294	1.27989017951465e-34\\
295	1.49360406740902e-34\\
296	9.48782926709719e-34\\
297	3.37277718622086e-34\\
298	5.12564269315545e-34\\
299	2.02520884774264e-35\\
300	2.68011308987601e-40\\
};
\addlegendentry{$\|f u_,^T b\|$};

\end{axis}
\end{tikzpicture}%
\end{document}
% This file was created by matlab2tikz.
% Minimal pgfplots version: 1.3
%
%The latest updates can be retrieved from
%  http://www.mathworks.com/matlabcentral/fileexchange/22022-matlab2tikz
%where you can also make suggestions and rate matlab2tikz.
%
\documentclass[tikz]{standalone}
\usepackage{pgfplots}
\usepackage{grffile}
\pgfplotsset{compat=newest}
\usetikzlibrary{plotmarks}
\usepackage{amsmath}

\begin{document}
\definecolor{mycolor1}{rgb}{0.00000,0.44700,0.74100}%
\definecolor{mycolor2}{rgb}{0.85000,0.32500,0.09800}%
\definecolor{mycolor3}{rgb}{0.92900,0.69400,0.12500}%
%
\begin{tikzpicture}

\begin{axis}[%
width=2in,
height=2in,
at={(0.758333in,0.48125in)},
scale only axis,
xmode=log,
xmin=1,
xmax=1000,
xminorticks=true,
ymode=log,
ymin=1e-40,
ymax=10000000000,
yminorticks=true,
legend style={legend cell align=left,align=left,draw=white!15!black}
]
\addplot [color=mycolor1,solid]
  table[row sep=crcr]{%
1	3.22867412569157\\
2	0.63135901789012\\
3	0.0715967516856907\\
4	0.00477654603445335\\
5	0.000236640876439147\\
6	9.34452523296071e-06\\
7	3.06908981503878e-07\\
8	8.63002956397216e-09\\
9	2.12171104754993e-10\\
10	4.6339069302837e-12\\
11	2.7307247574405e-13\\
12	1.43171261103148e-13\\
13	1.072038343321e-13\\
14	9.07937267004917e-14\\
15	5.74755447780305e-14\\
16	4.89678142417896e-14\\
17	4.11508474866177e-14\\
18	3.43543395947347e-14\\
19	3.08030619017049e-14\\
20	2.61384249510657e-14\\
21	2.12078828989212e-14\\
22	2.05457929235225e-14\\
23	1.97041860373373e-14\\
24	1.91019073347752e-14\\
25	1.81078117300885e-14\\
26	1.63631714508596e-14\\
27	1.41515444259893e-14\\
28	1.39120620583692e-14\\
29	1.35637106612877e-14\\
30	1.29576994399367e-14\\
31	1.15836723488219e-14\\
32	1.07730591873455e-14\\
33	1.05978502614299e-14\\
34	1.04119433353942e-14\\
35	1.03048945460862e-14\\
36	1.02935291616766e-14\\
37	1.00947351994327e-14\\
38	9.74756301164901e-15\\
39	9.235325065615e-15\\
40	8.97828141075779e-15\\
41	8.89164642365144e-15\\
42	8.8880623605589e-15\\
43	8.76176761339043e-15\\
44	8.6266706846148e-15\\
45	8.37157880270035e-15\\
46	8.21313610885143e-15\\
47	8.11091104335507e-15\\
48	8.0011142476464e-15\\
49	7.92976730843321e-15\\
50	7.90314068632101e-15\\
51	7.83815975942301e-15\\
52	7.717817075265e-15\\
53	7.46267629476145e-15\\
54	7.4242603502506e-15\\
55	7.21983472306773e-15\\
56	7.17340132379639e-15\\
57	6.9890608328923e-15\\
58	6.78367313194093e-15\\
59	6.63178088364025e-15\\
60	6.60735438072375e-15\\
61	6.52719056698693e-15\\
62	6.43915075191362e-15\\
63	6.33791393721237e-15\\
64	6.26324705054878e-15\\
65	6.1817753270134e-15\\
66	6.17881757541671e-15\\
67	6.02361785723123e-15\\
68	5.88247661678009e-15\\
69	5.76289897096969e-15\\
70	5.6381916725129e-15\\
71	5.58570548793141e-15\\
72	5.56972999488292e-15\\
73	5.48218933301531e-15\\
74	5.43293277471745e-15\\
75	5.30041853289255e-15\\
76	5.25311821492227e-15\\
77	5.24814372479017e-15\\
78	5.17730379896095e-15\\
79	5.02598173597062e-15\\
80	4.94449215854586e-15\\
81	4.90477412852739e-15\\
82	4.87564463634205e-15\\
83	4.87332003560299e-15\\
84	4.87265542973667e-15\\
85	4.79307819707925e-15\\
86	4.65339064383984e-15\\
87	4.62781704587007e-15\\
88	4.57988161831833e-15\\
89	4.57987589236974e-15\\
90	4.50033866187458e-15\\
91	4.41875284568619e-15\\
92	4.40284572538464e-15\\
93	4.25602516436353e-15\\
94	4.23438418829312e-15\\
95	4.16211724784193e-15\\
96	4.14627823910946e-15\\
97	4.07786580579887e-15\\
98	4.06306311033728e-15\\
99	3.89567073518618e-15\\
100	3.8698110676429e-15\\
101	3.86795150454194e-15\\
102	3.82865042029204e-15\\
103	3.73825615582403e-15\\
104	3.65858575359573e-15\\
105	3.63783164897366e-15\\
106	3.58722512749387e-15\\
107	3.57175094523823e-15\\
108	3.54886405771301e-15\\
109	3.47451547264846e-15\\
110	3.43275164356748e-15\\
111	3.4139795261489e-15\\
112	3.37665611875974e-15\\
113	3.32771782475889e-15\\
114	3.26058856628562e-15\\
115	3.22892149016548e-15\\
116	3.19991249934822e-15\\
117	3.17998343701716e-15\\
118	3.11779538093535e-15\\
119	3.09003375768003e-15\\
120	3.06082309801158e-15\\
121	3.04729517634055e-15\\
122	2.96305106025211e-15\\
123	2.95735832485709e-15\\
124	2.92291848766655e-15\\
125	2.86763868268342e-15\\
126	2.86763868268342e-15\\
127	2.86153489721341e-15\\
128	2.85731459598141e-15\\
129	2.80769667052955e-15\\
130	2.80370841939114e-15\\
131	2.70156279491695e-15\\
132	2.65825140502276e-15\\
133	2.636514808367e-15\\
134	2.62499171382705e-15\\
135	2.58495019948386e-15\\
136	2.49901535770141e-15\\
137	2.44350384167028e-15\\
138	2.4318407181556e-15\\
139	2.38737252708011e-15\\
140	2.38036126803894e-15\\
141	2.33425448412753e-15\\
142	2.3227848949799e-15\\
143	2.28863011940915e-15\\
144	2.27483736457906e-15\\
145	2.24518155913264e-15\\
146	2.23059178866288e-15\\
147	2.17136776178926e-15\\
148	2.14981734056811e-15\\
149	2.11028579947696e-15\\
150	2.08556016762178e-15\\
151	2.05376899530518e-15\\
152	2.05370521587227e-15\\
153	2.01397123603994e-15\\
154	1.97027312679364e-15\\
155	1.96460961428456e-15\\
156	1.92449175974846e-15\\
157	1.91029480594618e-15\\
158	1.86601117038855e-15\\
159	1.85009683466833e-15\\
160	1.84524427209449e-15\\
161	1.81463311882967e-15\\
162	1.78467440670738e-15\\
163	1.70270079248612e-15\\
164	1.69727301248652e-15\\
165	1.68198635890044e-15\\
166	1.66178735767625e-15\\
167	1.61398710301802e-15\\
168	1.58416326853956e-15\\
169	1.54316266735293e-15\\
170	1.50041792836263e-15\\
171	1.4877984789168e-15\\
172	1.48290293890221e-15\\
173	1.4660102362651e-15\\
174	1.43616896572654e-15\\
175	1.40184159403023e-15\\
176	1.38039082582761e-15\\
177	1.3539796630327e-15\\
178	1.33533567837832e-15\\
179	1.31068531339294e-15\\
180	1.27917675211952e-15\\
181	1.26426552888414e-15\\
182	1.22870227102962e-15\\
183	1.20737766090725e-15\\
184	1.20234627496191e-15\\
185	1.18126363630714e-15\\
186	1.17679409230074e-15\\
187	1.1705294804389e-15\\
188	1.15552171742388e-15\\
189	1.12844865716061e-15\\
190	1.05943725209341e-15\\
191	1.04295310401473e-15\\
192	1.03518570602746e-15\\
193	1.02236151438466e-15\\
194	1.00076996766795e-15\\
195	9.78585779456933e-16\\
196	9.63710947862694e-16\\
197	9.50815541359294e-16\\
198	9.37077059948624e-16\\
199	9.08201942890717e-16\\
200	8.71233812183292e-16\\
201	8.54262015077478e-16\\
202	8.45090413498785e-16\\
203	8.19312708206517e-16\\
204	8.1486144736072e-16\\
205	8.01892978275033e-16\\
206	7.84914792358734e-16\\
207	7.79711813091671e-16\\
208	7.63145716181268e-16\\
209	7.42530889665463e-16\\
210	7.17715327516936e-16\\
211	6.91292717108393e-16\\
212	6.85222869156858e-16\\
213	6.72838979746748e-16\\
214	6.65301259952188e-16\\
215	6.60855391982341e-16\\
216	6.43163326929007e-16\\
217	6.27452749815761e-16\\
218	6.09986625914371e-16\\
219	5.86837870608084e-16\\
220	5.83634418934718e-16\\
221	5.68326153607562e-16\\
222	5.49119173613888e-16\\
223	5.45154043304325e-16\\
224	5.24667420396916e-16\\
225	5.10563005235944e-16\\
226	5.02986656514544e-16\\
227	4.85505122645056e-16\\
228	4.66713466098897e-16\\
229	4.53408887150346e-16\\
230	4.34168517461949e-16\\
231	4.00604707863515e-16\\
232	3.84345401957198e-16\\
233	3.67297704963205e-16\\
234	3.41364915237981e-16\\
235	3.40238523597079e-16\\
236	3.35659618987364e-16\\
237	3.27240936608662e-16\\
238	3.18146297180742e-16\\
239	3.18146297180742e-16\\
240	3.18146297180742e-16\\
241	3.18146297180742e-16\\
242	3.18146297180742e-16\\
243	3.18146297180742e-16\\
244	3.18146297180742e-16\\
245	3.18146297180742e-16\\
246	3.18146297180742e-16\\
247	3.18146297180742e-16\\
248	3.18146297180742e-16\\
249	3.18146297180742e-16\\
250	3.18146297180742e-16\\
251	3.18146297180742e-16\\
252	3.18146297180742e-16\\
253	3.18146297180742e-16\\
254	3.18146297180742e-16\\
255	3.18146297180742e-16\\
256	3.18146297180742e-16\\
257	3.18146297180742e-16\\
258	3.18146297180742e-16\\
259	3.18146297180742e-16\\
260	3.18146297180742e-16\\
261	3.18146297180742e-16\\
262	3.18146297180742e-16\\
263	3.18146297180742e-16\\
264	3.18146297180742e-16\\
265	3.18146297180742e-16\\
266	3.18146297180742e-16\\
267	3.18146297180742e-16\\
268	3.18146297180742e-16\\
269	3.18146297180742e-16\\
270	3.18146297180742e-16\\
271	3.18146297180742e-16\\
272	3.18146297180742e-16\\
273	3.18146297180742e-16\\
274	3.18146297180742e-16\\
275	3.18146297180742e-16\\
276	3.18146297180742e-16\\
277	3.18146297180742e-16\\
278	3.18146297180742e-16\\
279	3.18146297180742e-16\\
280	3.18146297180742e-16\\
281	3.18146297180742e-16\\
282	3.18146297180742e-16\\
283	3.18146297180742e-16\\
284	3.18146297180742e-16\\
285	3.0174077697214e-16\\
286	2.69827721955834e-16\\
287	2.60255924198561e-16\\
288	2.33685649853635e-16\\
289	2.08736626506453e-16\\
290	1.89264553370077e-16\\
291	1.84326228755923e-16\\
292	1.6803222802398e-16\\
293	1.62224039328469e-16\\
294	1.38384729446136e-16\\
295	1.24106347642519e-16\\
296	9.89501111748281e-17\\
297	8.10831418644399e-17\\
298	5.82151078481272e-17\\
299	2.76302096188864e-17\\
300	8.85566055732899e-18\\
};
\addlegendentry{$\sigma$};

\addplot [color=mycolor2,solid]
  table[row sep=crcr]{%
1	2.85509245473343\\
2	0.489748853377841\\
3	0.027072915783981\\
4	0.000695555138117098\\
5	2.37308436588264e-05\\
6	4.70905927812612e-06\\
7	6.22521046830873e-05\\
8	0.000116395888335188\\
9	5.99545509849461e-05\\
10	6.10107987160241e-05\\
11	1.86040720399072e-05\\
12	3.95802692758074e-05\\
13	3.17943059280484e-05\\
14	1.09330839789534e-05\\
15	6.76859715603578e-05\\
16	6.57388134373688e-06\\
17	2.02445753589124e-05\\
18	1.41176000753029e-05\\
19	0.000196252827479843\\
20	3.45435517945733e-05\\
21	4.03543503536413e-05\\
22	1.02325921520902e-05\\
23	1.69338318593282e-05\\
24	8.83424052494641e-06\\
25	1.96723620681308e-05\\
26	9.13937380977795e-05\\
27	2.41081197361465e-05\\
28	2.72858442119453e-05\\
29	3.02769427945049e-05\\
30	1.49555043832267e-05\\
31	4.23114843418441e-05\\
32	6.89460932283836e-05\\
33	4.14464356343625e-06\\
34	3.01320887785561e-05\\
35	2.85790226810835e-05\\
36	5.20173362317244e-05\\
37	2.54629234006654e-05\\
38	7.37070809971804e-05\\
39	7.11500603421955e-05\\
40	5.05809745247193e-05\\
41	8.16774387830677e-05\\
42	7.24911562477448e-05\\
43	6.21059117406619e-05\\
44	5.3542203291625e-05\\
45	8.31075433190137e-05\\
46	1.1362780110713e-05\\
47	3.49514636288038e-05\\
48	5.80554192430825e-05\\
49	6.75494024781065e-05\\
50	2.60531181129064e-05\\
51	8.48339147477006e-05\\
52	2.18690661849819e-05\\
53	1.93525329536259e-05\\
54	2.30364890254769e-05\\
55	3.9371930326923e-05\\
56	0.000146034997303784\\
57	1.50819598159629e-05\\
58	4.90235724604632e-05\\
59	5.07550108894139e-06\\
60	1.32558752888728e-05\\
61	4.82214842386364e-05\\
62	3.31247888429309e-05\\
63	1.49352120721868e-05\\
64	5.20532801795426e-06\\
65	8.31188739899638e-05\\
66	1.41963984858773e-05\\
67	3.98499501637062e-05\\
68	5.3612904506728e-05\\
69	2.66414349120586e-05\\
70	9.89060154180099e-06\\
71	6.23345646331638e-05\\
72	3.63223070353746e-05\\
73	9.4957207157631e-06\\
74	4.12212797082645e-05\\
75	5.56836142881351e-05\\
76	7.03601209227633e-05\\
77	9.35285602545605e-06\\
78	8.58997759686786e-06\\
79	7.50341742648125e-06\\
80	0.000123131201711403\\
81	2.9323524888332e-05\\
82	1.31266066503593e-05\\
83	2.72074790093335e-05\\
84	6.66713899036969e-06\\
85	1.49046777232764e-06\\
86	0.000114231170021517\\
87	3.33602628163447e-05\\
88	2.60307305620378e-06\\
89	8.68525941873297e-05\\
90	1.1549167866845e-06\\
91	1.25240983152169e-05\\
92	1.34902089778538e-05\\
93	6.8097971053227e-06\\
94	4.33571577891784e-05\\
95	3.91097822445563e-05\\
96	7.036397776463e-05\\
97	1.5918540643034e-05\\
98	3.47155177547742e-06\\
99	0.000145151514749441\\
100	0.000127148026024579\\
101	4.97372583127681e-05\\
102	3.60282032534076e-06\\
103	4.70178243572386e-05\\
104	2.73365414582838e-05\\
105	2.84952684910661e-05\\
106	2.19344249152709e-07\\
107	6.39167688365194e-05\\
108	2.81401020815961e-05\\
109	5.97314806423527e-05\\
110	4.80204739748968e-05\\
111	3.13019642187023e-05\\
112	7.88572760098832e-05\\
113	1.06261846334194e-05\\
114	3.76562463700603e-05\\
115	1.99960166003682e-06\\
116	4.58180120062324e-06\\
117	4.31754054690398e-05\\
118	6.18496019998138e-06\\
119	7.68690459023447e-05\\
120	1.06853495878551e-05\\
121	5.11036588144928e-05\\
122	1.27451184373378e-05\\
123	5.07379596560928e-05\\
124	6.61425125143139e-06\\
125	8.71225650794782e-06\\
126	4.1782757246811e-05\\
127	3.14877009365e-06\\
128	4.09230869258537e-05\\
129	6.87771037324036e-05\\
130	5.00466391996079e-05\\
131	3.31041368456242e-05\\
132	1.67965268112052e-05\\
133	2.38228329880394e-05\\
134	1.05359295352256e-05\\
135	9.11422843576926e-05\\
136	1.88904352828034e-05\\
137	5.42546561773868e-05\\
138	5.00534203570074e-05\\
139	0.000130498072239194\\
140	9.02114592617881e-06\\
141	7.35524828705258e-05\\
142	1.94514189027772e-05\\
143	1.43963153866333e-05\\
144	0.000118689938359336\\
145	5.0873699927477e-05\\
146	3.56009156033538e-05\\
147	2.71788362560116e-05\\
148	0.000100978374447058\\
149	1.79268895190571e-05\\
150	2.42565135998019e-06\\
151	0.000106433847226049\\
152	6.00068375257307e-05\\
153	1.67826555982076e-05\\
154	2.6612584902358e-05\\
155	4.13667661920136e-05\\
156	5.50778482705863e-05\\
157	1.1720899067813e-05\\
158	4.21400953842521e-05\\
159	7.65316096475038e-05\\
160	1.910333180484e-05\\
161	6.30987294771984e-05\\
162	3.73448073563709e-06\\
163	4.56805469556676e-05\\
164	1.01774860679499e-05\\
165	2.00131331938164e-05\\
166	4.72683860733639e-05\\
167	3.57667139484963e-05\\
168	3.42159526059704e-05\\
169	1.55851865003553e-05\\
170	1.79703783077731e-05\\
171	0.000132580864759438\\
172	1.19866561362966e-05\\
173	6.38097273134158e-05\\
174	4.33907741471638e-05\\
175	7.65012239899859e-05\\
176	3.53239281452415e-05\\
177	1.3876493593093e-05\\
178	3.3395188296953e-05\\
179	5.71541352234403e-05\\
180	1.38258271928371e-06\\
181	6.19243664852498e-07\\
182	2.20147962082667e-05\\
183	8.83710952244085e-05\\
184	7.45979505235063e-05\\
185	6.74723858513915e-06\\
186	0.000116770203055443\\
187	6.04258226885243e-05\\
188	3.51283508566655e-05\\
189	1.23611063564955e-05\\
190	3.65529507623252e-06\\
191	1.21955585559441e-06\\
192	1.81094706121904e-05\\
193	5.15285394188658e-05\\
194	4.20001803417558e-05\\
195	2.97434562740069e-05\\
196	0.000133111349059897\\
197	4.5879730763837e-06\\
198	8.58787419757839e-05\\
199	1.04431756278722e-06\\
200	3.17640756854302e-05\\
201	9.76704270823081e-05\\
202	1.19928789640775e-05\\
203	2.24308283770566e-05\\
204	4.42076285609577e-05\\
205	7.16124064110563e-05\\
206	1.51875730132964e-05\\
207	6.51170639257154e-05\\
208	1.75763557277997e-06\\
209	1.65078786278361e-05\\
210	2.38348654983678e-06\\
211	9.6614298241704e-06\\
212	1.01619409441055e-05\\
213	0.000112973535795213\\
214	2.30935145741629e-05\\
215	5.09772947154818e-06\\
216	8.94382351885048e-05\\
217	2.81408747003289e-05\\
218	8.06994579471212e-05\\
219	1.18028816024213e-05\\
220	1.43778871245955e-05\\
221	5.1572169849981e-05\\
222	2.69526980767182e-05\\
223	0.000116229662239151\\
224	1.29672663181576e-05\\
225	6.02173963885622e-05\\
226	1.83248094945757e-05\\
227	1.51457114684062e-05\\
228	3.30328912021491e-05\\
229	2.70757215152101e-05\\
230	2.47798525629672e-05\\
231	1.29269261709974e-05\\
232	3.71464760713716e-05\\
233	2.542715718859e-05\\
234	0.000103521328764136\\
235	5.46820751247991e-05\\
236	3.0317543705638e-05\\
237	4.04143821137598e-05\\
238	3.44479657964081e-05\\
239	8.20988553175278e-05\\
240	1.72170573340125e-05\\
241	3.68768991796664e-05\\
242	5.53597790977842e-05\\
243	4.46494849750956e-05\\
244	2.93426920007049e-05\\
245	6.33328801957211e-05\\
246	8.24635280676222e-05\\
247	5.44442656045738e-05\\
248	6.34580706339628e-05\\
249	3.30656176052019e-05\\
250	1.47252333820985e-05\\
251	5.56436314094744e-06\\
252	2.14390096148487e-05\\
253	1.4861470193625e-05\\
254	6.85461168210477e-05\\
255	3.71658457804236e-05\\
256	5.71152308507725e-05\\
257	5.4212864391159e-05\\
258	7.05579862655889e-05\\
259	2.19699894572767e-06\\
260	6.44132000786717e-05\\
261	3.76590667087139e-05\\
262	3.43048783275034e-05\\
263	2.98604508049138e-06\\
264	6.37835151124505e-05\\
265	2.69818255266181e-05\\
266	4.82549637374156e-06\\
267	5.27700738452361e-05\\
268	9.44890876274085e-05\\
269	3.02623524016227e-05\\
270	5.69978595932191e-05\\
271	2.69313890270788e-05\\
272	1.92213349278145e-05\\
273	1.49580509365421e-05\\
274	7.5763153999946e-05\\
275	4.55984619714828e-06\\
276	3.2273405207027e-05\\
277	1.6558679318485e-05\\
278	9.11615135314632e-05\\
279	4.86268312174104e-05\\
280	2.80254831841464e-06\\
281	2.58657290360953e-05\\
282	4.37507519393532e-06\\
283	5.9599986330823e-06\\
284	8.34904027939131e-06\\
285	5.80984305624116e-05\\
286	3.70187396926819e-05\\
287	3.64834452498894e-05\\
288	6.48312159359093e-05\\
289	7.33731183688493e-05\\
290	2.49382697115677e-05\\
291	1.0398877202885e-05\\
292	5.11467081743665e-05\\
293	1.50904422062739e-05\\
294	4.24191628339635e-05\\
295	2.58062094959932e-05\\
296	1.38491796780327e-05\\
297	2.95181682829423e-05\\
298	6.39297407900828e-05\\
299	8.59312152765718e-05\\
300	3.32283027177499e-06\\
};
\addlegendentry{$\|u_,^T b\|$};

\addplot [color=mycolor3,solid]
  table[row sep=crcr]{%
1	2.85509236332507\\
2	0.48974844333002\\
3	0.0270711532628192\\
4	0.000685527215222904\\
5	3.40968175574323e-06\\
6	1.23174499485972e-09\\
7	1.75694793551583e-11\\
8	2.59745472089326e-14\\
9	8.08686903319601e-18\\
10	3.92542879576543e-21\\
11	4.15670924737959e-24\\
12	2.43095111123649e-24\\
13	1.09485288263593e-24\\
14	2.70047492110851e-25\\
15	6.69962723636351e-25\\
16	4.72312107352603e-26\\
17	1.02719292332656e-25\\
18	4.99240561852888e-26\\
19	5.57942545207839e-25\\
20	7.07150085483884e-26\\
21	5.43839531637315e-26\\
22	1.29424726920136e-26\\
23	1.96996278220596e-26\\
24	9.65847292650228e-27\\
25	1.93274353544862e-26\\
26	7.33224861342257e-26\\
27	1.4466277693007e-26\\
28	1.58236324756817e-26\\
29	1.66899422385631e-26\\
30	7.52389364883797e-27\\
31	1.70112634418012e-26\\
32	2.3975823539865e-26\\
33	1.39478903653645e-27\\
34	9.7876540296601e-27\\
35	9.09327355645474e-27\\
36	1.65143880967763e-26\\
37	7.77470475312252e-27\\
38	2.09839428334484e-26\\
39	1.81829952400254e-26\\
40	1.22168544337558e-26\\
41	1.93487208812666e-26\\
42	1.71587241545037e-26\\
43	1.42857215859482e-26\\
44	1.19390138865329e-26\\
45	1.74518295851861e-26\\
46	2.29661619009361e-27\\
47	6.88954312786709e-27\\
48	1.11360068303251e-26\\
49	1.27270625445491e-26\\
50	4.87578912042036e-27\\
51	1.56164929778911e-26\\
52	3.90305777745334e-27\\
53	3.22933279378333e-27\\
54	3.80459519462493e-27\\
55	6.1493186612816e-27\\
56	2.25160904039913e-26\\
57	2.20740082267184e-27\\
58	6.75959323460896e-27\\
59	6.68844317904018e-28\\
60	1.73400118588599e-27\\
61	6.15572114749243e-27\\
62	4.11524867426456e-27\\
63	1.79758670956383e-27\\
64	6.11833097609117e-28\\
65	9.51725847825506e-27\\
66	1.62395767933339e-27\\
67	4.33239839588875e-27\\
68	5.55872941958518e-27\\
69	2.65109569184818e-27\\
70	9.42080809436218e-28\\
71	5.82734561901049e-27\\
72	3.37619474333775e-27\\
73	8.55109581878823e-28\\
74	3.64565789378158e-27\\
75	4.68741627622478e-27\\
76	5.81763814874942e-27\\
77	7.71865213717568e-28\\
78	6.89898356096848e-28\\
79	5.67919393370743e-28\\
80	9.01980778209812e-27\\
81	2.113683603614e-27\\
82	9.34980040096512e-28\\
83	1.93608289495521e-27\\
84	4.74303895020805e-28\\
85	1.02597646429359e-28\\
86	7.41155659552136e-27\\
87	2.14075814845631e-27\\
88	1.63598998710193e-28\\
89	5.45853366722587e-27\\
90	7.00852887413677e-29\\
91	7.32709241799118e-28\\
92	7.83558440637476e-28\\
93	3.69596905437715e-28\\
94	2.3293089911603e-27\\
95	2.03001757517088e-27\\
96	3.62454125264096e-27\\
97	7.93149199096489e-28\\
98	1.71718293001357e-28\\
99	6.60042404370483e-27\\
100	5.70525324164554e-27\\
101	2.22961389414141e-27\\
102	1.58241289174736e-28\\
103	1.96873119615621e-27\\
104	1.09636667265125e-27\\
105	1.12990967376538e-27\\
106	8.45725251807903e-30\\
107	2.44322198873786e-27\\
108	1.06191602684986e-27\\
109	2.16061586086628e-27\\
110	1.69549685838643e-27\\
111	1.093148579587e-27\\
112	2.69402250025453e-27\\
113	3.52578741442693e-28\\
114	1.19954020521202e-27\\
115	6.24660779927032e-29\\
116	1.40571806784184e-28\\
117	1.3081932992389e-27\\
118	1.80143214237185e-28\\
119	2.19919489226097e-27\\
120	2.99951456391038e-28\\
121	1.42189269470226e-27\\
122	3.35280214801933e-28\\
123	1.32961735554111e-27\\
124	1.69316731469475e-28\\
125	2.14667003112659e-28\\
126	1.02951276420207e-27\\
127	7.72546847607677e-29\\
128	1.00108345793788e-27\\
129	1.62453860735587e-27\\
130	1.17876271288381e-27\\
131	7.23932679045077e-28\\
132	3.55629165975096e-28\\
133	4.96180412189519e-28\\
134	2.17527671287653e-28\\
135	1.82477805785267e-27\\
136	3.53480609363155e-28\\
137	9.70619131973081e-28\\
138	8.86930911599725e-28\\
139	2.22859042093326e-27\\
140	1.53155725769604e-28\\
141	1.20082444620232e-27\\
142	3.14452500627548e-28\\
143	2.25937516258521e-28\\
144	1.84034984880563e-27\\
145	7.6839058791871e-28\\
146	5.3074650786625e-28\\
147	3.83957723345995e-28\\
148	1.39835447927113e-27\\
149	2.392066640812e-28\\
150	3.16125556158317e-29\\
151	1.34514383883459e-27\\
152	7.58337846572282e-28\\
153	2.03963750441338e-28\\
154	3.09546364188071e-28\\
155	4.78398579333044e-28\\
156	6.11216200179217e-28\\
157	1.28158523515472e-28\\
158	4.39652728012049e-28\\
159	7.84902202004015e-28\\
160	1.94895867374065e-28\\
161	6.22564080495048e-28\\
162	3.56396962770561e-29\\
163	3.96820181273063e-28\\
164	8.78475608764382e-29\\
165	1.69646851489773e-28\\
166	3.91117686129483e-28\\
167	2.79167553762363e-28\\
168	2.57284897273354e-28\\
169	1.11204202407965e-28\\
170	1.2121810464021e-28\\
171	8.7933598809846e-28\\
172	7.89785721972728e-29\\
173	4.10909965210652e-28\\
174	2.68160158536985e-28\\
175	4.5045565860222e-28\\
176	2.01678174093462e-28\\
177	7.62236605672073e-29\\
178	1.78422895088088e-28\\
179	2.94191624003388e-28\\
180	6.77856897795757e-30\\
181	2.96567765706868e-30\\
182	9.9584964081205e-29\\
183	3.85995542466584e-28\\
184	3.23125922242002e-28\\
185	2.82101578815766e-29\\
186	4.84527882575956e-28\\
187	2.48069324333185e-28\\
188	1.40539940647803e-28\\
189	4.71635764824543e-29\\
190	1.22930231698302e-29\\
191	3.97481607293879e-30\\
192	5.81471010447819e-29\\
193	1.61377356579787e-28\\
194	1.26039149176189e-28\\
195	8.53443969055511e-29\\
196	3.70420019098986e-28\\
197	1.2427941027309e-29\\
198	2.2595505726282e-28\\
199	2.58097153393638e-30\\
200	7.22422802226749e-29\\
201	2.13565480873429e-28\\
202	2.56634819921069e-29\\
203	4.5115988200048e-29\\
204	8.79529762384021e-29\\
205	1.3797705417543e-28\\
206	2.80362023567512e-29\\
207	1.18617514074105e-28\\
208	3.06711143136915e-30\\
209	2.72713160914981e-29\\
210	3.6787724658843e-30\\
211	1.38341075076831e-29\\
212	1.42963820079935e-29\\
213	1.53244461751014e-28\\
214	3.06275673875901e-29\\
215	6.67076189911043e-30\\
216	1.10854045457281e-28\\
217	3.31959747726203e-29\\
218	8.99698555881087e-29\\
219	1.21789569484503e-29\\
220	1.46744773075911e-29\\
221	4.991102262225e-29\\
222	2.43512490506302e-29\\
223	1.03500209848099e-28\\
224	1.06955323175351e-29\\
225	4.70334070736308e-29\\
226	1.38911492015136e-29\\
227	1.06970269953077e-29\\
228	2.15592192944224e-29\\
229	1.66780731698524e-29\\
230	1.39959104403548e-29\\
231	6.21603062592893e-30\\
232	1.64417022684806e-29\\
233	1.02782698762306e-29\\
234	3.61454290119657e-29\\
235	1.8966960383868e-29\\
236	1.02347676181942e-29\\
237	1.29675215055991e-29\\
238	1.04472769270045e-29\\
239	2.48986974139656e-29\\
240	5.22153809890624e-30\\
241	1.11839166415824e-29\\
242	1.67893496606051e-29\\
243	1.35411634155634e-29\\
244	8.89896462761737e-30\\
245	1.92074081209792e-29\\
246	2.50092942843634e-29\\
247	1.65116954429095e-29\\
248	1.92453755058951e-29\\
249	1.00280424663053e-29\\
250	4.46582511914561e-30\\
251	1.68754355479527e-30\\
252	6.50195926838973e-30\\
253	4.50714261541209e-30\\
254	2.07884630672336e-29\\
255	1.12715475099163e-29\\
256	1.73217378632393e-29\\
257	1.64415167690141e-29\\
258	2.13986168670914e-29\\
259	6.66299326070012e-31\\
260	1.95350443318976e-29\\
261	1.14211300906259e-29\\
262	1.04038817836925e-29\\
263	9.05598898261869e-31\\
264	1.93440753423651e-29\\
265	8.18296804341659e-30\\
266	1.46346223241178e-30\\
267	1.60039515302093e-29\\
268	2.86563703313374e-29\\
269	9.17787650714819e-30\\
270	1.7286141856273e-29\\
271	8.16767181138473e-30\\
272	5.82938946480289e-30\\
273	4.53643333674902e-30\\
274	2.29772247040183e-29\\
275	1.38289663451569e-30\\
276	9.78778263913216e-30\\
277	5.02186716648675e-30\\
278	2.76471935259283e-29\\
279	1.47474011909079e-29\\
280	8.49948544344827e-31\\
281	7.84448160903854e-30\\
282	1.32685983252459e-30\\
283	1.8075307137765e-30\\
284	2.53207218065881e-30\\
285	1.58495957159642e-29\\
286	8.0757067879634e-30\\
287	7.40428064064225e-30\\
288	1.06080127925685e-29\\
289	9.57899788322352e-30\\
290	2.6766443253312e-30\\
291	1.05863562500125e-30\\
292	4.32701737418936e-30\\
293	1.18992110640484e-30\\
294	2.43402038291422e-30\\
295	1.19096222795021e-30\\
296	4.06295864938765e-31\\
297	5.81482061219838e-31\\
298	6.49172532423126e-31\\
299	1.96564503100089e-31\\
300	7.80792909434806e-34\\
};
\addlegendentry{$\|f u_,^T b\|$};

\end{axis}
\end{tikzpicture}%
\end{document}
\caption{Plot of Fourier coefficients and singular values for the third $\mathbf{A}_3,\mathbf{b}_{err3}$ and fourth $\mathbf{A}_4, \mathbf{b}_{err4}$ value pair .}
\label{fig:picard34}
\end{figure}
\begin{figure}
% This file was created by matlab2tikz.
% Minimal pgfplots version: 1.3
%
%The latest updates can be retrieved from
%  http://www.mathworks.com/matlabcentral/fileexchange/22022-matlab2tikz
%where you can also make suggestions and rate matlab2tikz.
%
\documentclass[tikz]{standalone}
\usepackage{pgfplots}
\usepackage{grffile}
\pgfplotsset{compat=newest}
\usetikzlibrary{plotmarks}
\usepackage{amsmath}

\begin{document}
\definecolor{mycolor1}{rgb}{0.00000,0.44700,0.74100}%
\definecolor{mycolor2}{rgb}{0.85000,0.32500,0.09800}%
\definecolor{mycolor3}{rgb}{0.92900,0.69400,0.12500}%
%
\begin{tikzpicture}

\begin{axis}[%
width=2in,
height=2in,
at={(0.758333in,0.48125in)},
scale only axis,
xmode=log,
xmin=1,
xmax=1000,
xminorticks=true,
ymode=log,
ymin=1e-30,
ymax=100000,
yminorticks=true,
legend style={legend cell align=left,align=left,draw=white!15!black}
]
\addplot [color=mycolor1,solid]
  table[row sep=crcr]{%
1	70.9551535801115\\
2	36.1309518145902\\
3	20.5344829695318\\
4	11.3088480243225\\
5	5.84819418375812\\
6	2.84330565042282\\
7	1.30096609395132\\
8	0.560511963329427\\
9	0.227548612917364\\
10	0.0870697746135456\\
11	0.031390210338866\\
12	0.0106505265331544\\
13	0.00339502848104143\\
14	0.00101444677672621\\
15	0.000283369138386563\\
16	7.37684279342944e-05\\
17	1.7836212029178e-05\\
18	3.99086690045318e-06\\
19	8.23248990479976e-07\\
20	1.55981815658176e-07\\
21	2.7049858397896e-08\\
22	4.28004206306239e-09\\
23	6.16342010754258e-10\\
24	8.06316588935238e-11\\
25	9.57392520944952e-12\\
26	1.0319776498624e-12\\
27	1.01502864487273e-13\\
28	9.07745604166991e-15\\
29	5.26946882467235e-15\\
30	5.26946882467235e-15\\
31	5.26946882467235e-15\\
32	5.26946882467235e-15\\
33	5.26946882467235e-15\\
34	5.26946882467235e-15\\
35	5.26946882467235e-15\\
36	5.26946882467235e-15\\
37	5.26946882467235e-15\\
38	5.26946882467235e-15\\
39	5.26946882467235e-15\\
40	5.26946882467235e-15\\
41	5.26946882467235e-15\\
42	5.26946882467235e-15\\
43	5.26946882467235e-15\\
44	5.26946882467235e-15\\
45	5.26946882467235e-15\\
46	5.26946882467235e-15\\
47	5.26946882467235e-15\\
48	5.26946882467235e-15\\
49	5.26946882467235e-15\\
50	5.26946882467235e-15\\
51	5.26946882467235e-15\\
52	5.26946882467235e-15\\
53	5.26946882467235e-15\\
54	5.26946882467235e-15\\
55	5.26946882467235e-15\\
56	5.26946882467235e-15\\
57	5.26946882467235e-15\\
58	5.26946882467235e-15\\
59	5.26946882467235e-15\\
60	5.26946882467235e-15\\
61	5.26946882467235e-15\\
62	5.26946882467235e-15\\
63	5.26946882467235e-15\\
64	5.26946882467235e-15\\
65	5.26946882467235e-15\\
66	5.26946882467235e-15\\
67	5.26946882467235e-15\\
68	5.26946882467235e-15\\
69	5.26946882467235e-15\\
70	5.26946882467235e-15\\
71	5.26946882467235e-15\\
72	5.26946882467235e-15\\
73	5.26946882467235e-15\\
74	5.26946882467235e-15\\
75	5.26946882467235e-15\\
76	5.26946882467235e-15\\
77	5.26946882467235e-15\\
78	5.26946882467235e-15\\
79	5.26946882467235e-15\\
80	5.26946882467235e-15\\
81	5.26946882467235e-15\\
82	5.26946882467235e-15\\
83	5.26946882467235e-15\\
84	5.26946882467235e-15\\
85	5.26946882467235e-15\\
86	5.26946882467235e-15\\
87	5.26946882467235e-15\\
88	5.26946882467235e-15\\
89	5.26946882467235e-15\\
90	5.26946882467235e-15\\
91	5.26946882467235e-15\\
92	5.26946882467235e-15\\
93	5.26946882467235e-15\\
94	5.26946882467235e-15\\
95	5.26946882467235e-15\\
96	5.26946882467235e-15\\
97	5.26946882467235e-15\\
98	5.26946882467235e-15\\
99	5.26946882467235e-15\\
100	5.26946882467235e-15\\
101	5.26946882467235e-15\\
102	5.26946882467235e-15\\
103	5.26946882467235e-15\\
104	5.26946882467235e-15\\
105	5.26946882467235e-15\\
106	5.26946882467235e-15\\
107	5.26946882467235e-15\\
108	5.26946882467235e-15\\
109	5.26946882467235e-15\\
110	5.26946882467235e-15\\
111	5.26946882467235e-15\\
112	5.26946882467235e-15\\
113	5.26946882467235e-15\\
114	5.26946882467235e-15\\
115	5.26946882467235e-15\\
116	5.26946882467235e-15\\
117	5.26946882467235e-15\\
118	5.26946882467235e-15\\
119	5.26946882467235e-15\\
120	5.26946882467235e-15\\
121	5.26946882467235e-15\\
122	5.26946882467235e-15\\
123	5.26946882467235e-15\\
124	5.26946882467235e-15\\
125	5.26946882467235e-15\\
126	5.26946882467235e-15\\
127	5.26946882467235e-15\\
128	5.26946882467235e-15\\
129	5.26946882467235e-15\\
130	5.26946882467235e-15\\
131	5.26946882467235e-15\\
132	5.26946882467235e-15\\
133	5.26946882467235e-15\\
134	5.26946882467235e-15\\
135	5.26946882467235e-15\\
136	5.26946882467235e-15\\
137	5.26946882467235e-15\\
138	5.26946882467235e-15\\
139	5.26946882467235e-15\\
140	5.26946882467235e-15\\
141	5.26946882467235e-15\\
142	5.26946882467235e-15\\
143	5.26946882467235e-15\\
144	5.26946882467235e-15\\
145	5.26946882467235e-15\\
146	5.26946882467235e-15\\
147	5.26946882467235e-15\\
148	5.26946882467235e-15\\
149	5.26946882467235e-15\\
150	5.26946882467235e-15\\
151	5.26946882467235e-15\\
152	5.26946882467235e-15\\
153	5.26946882467235e-15\\
154	5.26946882467235e-15\\
155	5.26946882467235e-15\\
156	5.26946882467235e-15\\
157	5.26946882467235e-15\\
158	5.26946882467235e-15\\
159	5.26946882467235e-15\\
160	5.26946882467235e-15\\
161	5.26946882467235e-15\\
162	5.26946882467235e-15\\
163	5.26946882467235e-15\\
164	5.26946882467235e-15\\
165	5.26946882467235e-15\\
166	5.26946882467235e-15\\
167	5.26946882467235e-15\\
168	5.26946882467235e-15\\
169	5.26946882467235e-15\\
170	5.26946882467235e-15\\
171	5.26946882467235e-15\\
172	5.26946882467235e-15\\
173	5.26946882467235e-15\\
174	5.26946882467235e-15\\
175	5.26946882467235e-15\\
176	5.26946882467235e-15\\
177	5.26946882467235e-15\\
178	5.26946882467235e-15\\
179	5.26946882467235e-15\\
180	5.26946882467235e-15\\
181	5.26946882467235e-15\\
182	5.26946882467235e-15\\
183	5.26946882467235e-15\\
184	5.26946882467235e-15\\
185	5.26946882467235e-15\\
186	5.26946882467235e-15\\
187	5.26946882467235e-15\\
188	5.26946882467235e-15\\
189	5.26946882467235e-15\\
190	5.26946882467235e-15\\
191	5.26946882467235e-15\\
192	5.26946882467235e-15\\
193	5.26946882467235e-15\\
194	5.26946882467235e-15\\
195	5.26946882467235e-15\\
196	5.26946882467235e-15\\
197	5.26946882467235e-15\\
198	5.26946882467235e-15\\
199	5.26946882467235e-15\\
200	5.26946882467235e-15\\
201	5.26946882467235e-15\\
202	5.26946882467235e-15\\
203	5.26946882467235e-15\\
204	5.26946882467235e-15\\
205	5.26946882467235e-15\\
206	5.26946882467235e-15\\
207	5.26946882467235e-15\\
208	5.26946882467235e-15\\
209	5.26946882467235e-15\\
210	5.26946882467235e-15\\
211	5.26946882467235e-15\\
212	5.26946882467235e-15\\
213	5.26946882467235e-15\\
214	5.26946882467235e-15\\
215	5.26946882467235e-15\\
216	5.26946882467235e-15\\
217	5.26946882467235e-15\\
218	5.26946882467235e-15\\
219	5.26946882467235e-15\\
220	5.26946882467235e-15\\
221	5.26946882467235e-15\\
222	5.26946882467235e-15\\
223	5.26946882467235e-15\\
224	5.26946882467235e-15\\
225	5.26946882467235e-15\\
226	5.26946882467235e-15\\
227	5.26946882467235e-15\\
228	5.26946882467235e-15\\
229	5.26946882467235e-15\\
230	5.26946882467235e-15\\
231	5.26946882467235e-15\\
232	5.26946882467235e-15\\
233	5.26946882467235e-15\\
234	5.26946882467235e-15\\
235	5.26946882467235e-15\\
236	5.26946882467235e-15\\
237	5.26946882467235e-15\\
238	5.26946882467235e-15\\
239	5.26946882467235e-15\\
240	5.26946882467235e-15\\
241	5.26946882467235e-15\\
242	5.26946882467235e-15\\
243	5.26946882467235e-15\\
244	5.26946882467235e-15\\
245	5.26946882467235e-15\\
246	5.26946882467235e-15\\
247	5.26946882467235e-15\\
248	5.26946882467235e-15\\
249	5.26946882467235e-15\\
250	5.26946882467235e-15\\
251	5.26946882467235e-15\\
252	5.26946882467235e-15\\
253	5.26946882467235e-15\\
254	5.26946882467235e-15\\
255	5.26946882467235e-15\\
256	5.26946882467235e-15\\
257	5.26946882467235e-15\\
258	5.26946882467235e-15\\
259	5.26946882467235e-15\\
260	5.26946882467235e-15\\
261	5.26946882467235e-15\\
262	5.26946882467235e-15\\
263	5.26946882467235e-15\\
264	5.26946882467235e-15\\
265	5.26946882467235e-15\\
266	5.26946882467235e-15\\
267	5.26946882467235e-15\\
268	5.26946882467235e-15\\
269	5.26946882467235e-15\\
270	5.26946882467235e-15\\
271	5.26946882467235e-15\\
272	5.26946882467235e-15\\
273	5.26946882467235e-15\\
274	5.26946882467235e-15\\
275	5.26946882467235e-15\\
276	5.26946882467235e-15\\
277	5.26946882467235e-15\\
278	5.26946882467235e-15\\
279	5.26946882467235e-15\\
280	5.26946882467235e-15\\
281	5.26946882467235e-15\\
282	5.26946882467235e-15\\
283	5.26946882467235e-15\\
284	5.26946882467235e-15\\
285	5.26946882467235e-15\\
286	5.26946882467235e-15\\
287	5.26946882467235e-15\\
288	5.26946882467235e-15\\
289	5.26946882467235e-15\\
290	5.26946882467235e-15\\
291	5.26946882467235e-15\\
292	5.26946882467235e-15\\
293	5.26946882467235e-15\\
294	5.26946882467235e-15\\
295	5.26946882467235e-15\\
296	5.26946882467235e-15\\
297	5.26946882467235e-15\\
298	5.26946882467235e-15\\
299	5.26946882467235e-15\\
300	5.26946882467235e-15\\
};
\addlegendentry{$\sigma$};

\addplot [color=mycolor2,solid]
  table[row sep=crcr]{%
1	146.722300275256\\
2	332.05116905194\\
3	272.900539194368\\
4	90.2235213394774\\
5	12.0471607579879\\
6	4.99453612144686\\
7	4.62499499241355\\
8	1.42366759872502\\
9	0.161993676798019\\
10	0.193489787898919\\
11	0.00912570347916475\\
12	0.0281416881601692\\
13	0.0105543818618792\\
14	0.000526816047393019\\
15	0.000613103409899618\\
16	0.000223250565370092\\
17	1.02858925059568e-05\\
18	4.18190230041526e-05\\
19	2.83388404215268e-05\\
20	2.15788465230959e-05\\
21	1.53496255491659e-05\\
22	5.5113784865779e-05\\
23	0.000108127781957235\\
24	2.54503473952061e-05\\
25	2.03876628007693e-05\\
26	1.80128286803027e-05\\
27	8.50211727113637e-05\\
28	4.68580382895212e-06\\
29	4.5042504865922e-06\\
30	4.25427633548026e-05\\
31	5.21854169921454e-06\\
32	3.96807071743055e-05\\
33	1.93796067620156e-05\\
34	7.0273625862427e-05\\
35	2.79644031344617e-05\\
36	1.7805579729302e-05\\
37	4.29506632695364e-05\\
38	6.23637048757786e-05\\
39	9.4006721461426e-05\\
40	3.40667810183604e-05\\
41	3.90616303242997e-05\\
42	2.12681504034862e-05\\
43	4.7844930701757e-05\\
44	2.8426387752134e-05\\
45	2.14082315359576e-05\\
46	3.59760795447528e-05\\
47	5.81033140445442e-05\\
48	1.38467502068806e-05\\
49	4.97660041336445e-05\\
50	5.72125438136339e-05\\
51	1.13293816994542e-07\\
52	4.97477577781069e-05\\
53	6.26182320218049e-06\\
54	2.92655320421886e-05\\
55	2.81219281390577e-05\\
56	0.000112018276187342\\
57	9.66649548672649e-06\\
58	7.8502449163409e-05\\
59	6.90965160750068e-05\\
60	5.46183274394707e-05\\
61	3.7549608613574e-05\\
62	4.62201529050077e-05\\
63	5.88410016604257e-05\\
64	1.29749512112198e-06\\
65	0.000111183417743632\\
66	6.83674463601847e-05\\
67	3.11799925185596e-05\\
68	4.07254089473952e-05\\
69	6.80489923823302e-05\\
70	8.78484786959177e-06\\
71	2.38982324045622e-05\\
72	2.31853297414375e-05\\
73	1.84168159051978e-05\\
74	7.76352012188397e-05\\
75	1.98820320314042e-05\\
76	2.07481633864859e-05\\
77	1.94803248838582e-05\\
78	5.81727538602195e-05\\
79	4.53138221905292e-05\\
80	4.15492132574968e-05\\
81	0.000102137673994918\\
82	6.55701196450309e-07\\
83	1.65551974973255e-05\\
84	1.0683700658376e-05\\
85	4.46872077475291e-06\\
86	5.31152495617526e-05\\
87	5.48623878700027e-05\\
88	5.31577852651566e-05\\
89	5.63433082056974e-06\\
90	2.29680854317849e-05\\
91	4.18417806002935e-06\\
92	1.96664211316033e-05\\
93	7.7142278926523e-05\\
94	3.67972205710387e-05\\
95	2.87757661823207e-05\\
96	3.5990462450286e-05\\
97	3.01603271815054e-05\\
98	2.59074473483167e-05\\
99	4.16596333963071e-06\\
100	1.73609848097556e-05\\
101	1.10615088750876e-05\\
102	2.22705519732358e-05\\
103	1.6500068598102e-05\\
104	1.69566409340405e-05\\
105	2.10461972471876e-05\\
106	3.03767973779401e-05\\
107	9.79769486164628e-06\\
108	5.42047899703135e-05\\
109	0.000109090533966238\\
110	1.87393865846985e-05\\
111	1.62433013173313e-06\\
112	1.60395449313455e-05\\
113	6.98688756717303e-05\\
114	2.280631850482e-06\\
115	4.65196264762824e-05\\
116	8.08563136001084e-06\\
117	1.46865829186993e-05\\
118	6.62418593315728e-05\\
119	5.90560204205559e-05\\
120	2.29414434826936e-06\\
121	4.12339448701005e-05\\
122	0.000133727147121121\\
123	3.76459609130464e-05\\
124	3.53478841876864e-05\\
125	1.11806334146536e-06\\
126	1.84434507608522e-05\\
127	4.04023495441663e-05\\
128	6.69899436473997e-05\\
129	2.64332734722927e-05\\
130	3.72938704629178e-05\\
131	0.000108803361538179\\
132	4.3380719707109e-05\\
133	1.64741480190855e-06\\
134	2.02673539817511e-06\\
135	2.62357158753446e-06\\
136	3.0148725295831e-05\\
137	3.42951146965476e-05\\
138	1.27027592515816e-05\\
139	6.33700459538034e-05\\
140	7.11785431875001e-05\\
141	7.2133333929969e-05\\
142	3.061258594661e-05\\
143	0.000106557201659863\\
144	1.98589168718399e-05\\
145	7.34801562991549e-05\\
146	2.6158288442879e-05\\
147	6.85630790542291e-06\\
148	8.36787914693105e-06\\
149	3.14503565093105e-05\\
150	3.6462172008811e-05\\
151	5.24912454444859e-05\\
152	2.76228107143339e-06\\
153	4.70490702535464e-05\\
154	3.95095339165152e-05\\
155	2.1328637107132e-05\\
156	5.64738585531188e-05\\
157	1.6341573031653e-05\\
158	3.80528403383096e-05\\
159	6.54611056951637e-05\\
160	1.91955031931812e-05\\
161	4.30037600507305e-05\\
162	3.18384066773092e-05\\
163	7.7360527320991e-05\\
164	2.64844367769967e-06\\
165	1.28145797564372e-05\\
166	6.96197429714829e-05\\
167	1.26750459266134e-05\\
168	4.59135532615562e-06\\
169	7.54021436470964e-05\\
170	2.95050637539873e-05\\
171	3.69497487646697e-05\\
172	0.000115903731384037\\
173	4.51556744565096e-05\\
174	7.15214287936305e-05\\
175	7.44130324132897e-05\\
176	6.86442405815058e-05\\
177	1.68917439551386e-05\\
178	3.14424898482102e-05\\
179	4.44656983411562e-05\\
180	4.00714989243767e-05\\
181	1.89320167489626e-05\\
182	8.48275566589507e-06\\
183	0.000114926649981584\\
184	5.65329407784532e-05\\
185	3.03750380314938e-05\\
186	5.1593670827188e-05\\
187	1.1494906345888e-05\\
188	0.00013076891283581\\
189	1.27848866906533e-05\\
190	8.56980109897876e-05\\
191	6.06723744160931e-05\\
192	1.09356161104301e-05\\
193	6.73459197386705e-06\\
194	3.31204481653913e-05\\
195	4.96551511872667e-05\\
196	9.73536756543325e-06\\
197	2.89931405568211e-05\\
198	4.73267463263483e-05\\
199	1.96957352254223e-05\\
200	3.91770576415595e-05\\
201	7.68367979935292e-05\\
202	1.38032656797904e-05\\
203	4.30685764811756e-05\\
204	4.41744985160142e-05\\
205	6.78372445879205e-05\\
206	2.34562490390999e-05\\
207	1.87111695288955e-05\\
208	2.81170598270819e-06\\
209	9.62146352811999e-06\\
210	8.52825395192269e-07\\
211	1.49846537738085e-05\\
212	2.94765705000088e-05\\
213	1.2107935901895e-05\\
214	4.09228596964795e-06\\
215	1.97187134336474e-05\\
216	7.66225340376181e-05\\
217	6.00143274187559e-05\\
218	4.05107697662999e-05\\
219	3.00388664751239e-05\\
220	3.46066089562669e-06\\
221	3.34493284661619e-05\\
222	9.88422140935086e-06\\
223	9.42885603452126e-05\\
224	1.25754486770546e-05\\
225	6.06784837096086e-05\\
226	7.1425682687476e-05\\
227	5.41531296640585e-05\\
228	3.39869647696389e-05\\
229	1.98529116524071e-05\\
230	9.96519092613113e-05\\
231	2.11241077892055e-05\\
232	2.51695408168828e-05\\
233	9.61924953575277e-05\\
234	1.70302148507773e-05\\
235	5.60837523799762e-06\\
236	2.94682105970168e-05\\
237	6.51828755948713e-05\\
238	2.49321079248688e-05\\
239	4.87142952465547e-05\\
240	2.74146441849865e-05\\
241	8.60792095380702e-08\\
242	2.13905192767427e-05\\
243	2.01422825734099e-05\\
244	4.10511836790306e-05\\
245	3.81110480427438e-05\\
246	1.31381866488312e-05\\
247	1.77161778704971e-05\\
248	2.01642105537303e-05\\
249	2.6279413383179e-05\\
250	4.04759951226907e-05\\
251	2.78039144099296e-05\\
252	2.35961825847397e-05\\
253	2.04771958838279e-05\\
254	9.3517800039411e-05\\
255	3.55954253663526e-05\\
256	1.96452724310348e-07\\
257	2.59708891610089e-05\\
258	4.69482421650014e-05\\
259	6.89285472286194e-05\\
260	5.19529656628492e-07\\
261	8.88577278033154e-05\\
262	7.58120101096438e-06\\
263	9.39331111702302e-05\\
264	2.36354282900919e-05\\
265	6.83015599438619e-05\\
266	7.47268863676709e-07\\
267	8.94128596433674e-05\\
268	2.28581699524e-06\\
269	1.55939653829762e-05\\
270	1.32180779852664e-05\\
271	1.26755756841845e-05\\
272	6.04090556972636e-05\\
273	5.99684117457144e-05\\
274	4.99864621676949e-05\\
275	7.11668639890028e-05\\
276	4.04775308204819e-06\\
277	4.67499970234764e-05\\
278	2.58676458422258e-05\\
279	1.42053391751773e-05\\
280	3.33837229735678e-05\\
281	6.5553904921245e-05\\
282	4.50310853814528e-05\\
283	4.32831462315875e-06\\
284	3.35395022803553e-05\\
285	4.93090949271746e-05\\
286	3.16849400063823e-05\\
287	3.27055185955771e-06\\
288	1.68367750426057e-05\\
289	5.18883495765898e-05\\
290	5.38878218581118e-05\\
291	2.87047938094531e-05\\
292	0.000126714493124069\\
293	7.93057658139418e-05\\
294	7.5487406089092e-05\\
295	1.04944031331655e-05\\
296	2.09568525910697e-05\\
297	3.75486561603466e-05\\
298	7.83905499819326e-05\\
299	6.23772940855361e-06\\
300	1.2384670267096e-05\\
};
\addlegendentry{$\|u_,^T b\|$};

\addplot [color=mycolor3,solid]
  table[row sep=crcr]{%
1	146.72230027524\\
2	332.051169051795\\
3	272.900539193999\\
4	90.2235213390757\\
5	12.0471607577874\\
6	4.99453612109507\\
7	4.62499499085752\\
8	1.42366759614468\\
9	0.161993675016514\\
10	0.193489773365682\\
11	0.00912569820545217\\
12	0.0281415468918172\\
13	0.0105538604707786\\
14	0.000526524707922138\\
15	0.000608786242983994\\
16	0.000202102539947013\\
17	3.68680425388845e-06\\
18	1.13785976627034e-06\\
19	3.36890517056449e-08\\
20	9.21973638490307e-10\\
21	1.97236873943125e-11\\
22	1.77303450744999e-12\\
23	7.2134284242869e-14\\
24	2.90580051051305e-16\\
25	3.28177191355647e-18\\
26	3.36886333066156e-20\\
27	1.53831253571852e-21\\
28	6.78068084255007e-25\\
29	2.19642866985809e-25\\
30	2.07453260993624e-24\\
31	2.54474182524465e-25\\
32	1.93496882949506e-24\\
33	9.45016802522461e-25\\
34	3.42678559080323e-24\\
35	1.36364123198418e-24\\
36	8.68261789306121e-25\\
37	2.09442322440106e-24\\
38	3.04107042503646e-24\\
39	4.58409360040741e-24\\
40	1.66121433038287e-24\\
41	1.90478049626762e-24\\
42	1.0371087369838e-24\\
43	2.33308466903958e-24\\
44	1.38616920287778e-24\\
45	1.04393957764379e-24\\
46	1.75431834517073e-24\\
47	2.83331900011035e-24\\
48	6.75215537919823e-25\\
49	2.42676286907363e-24\\
50	2.78988195598617e-24\\
51	5.52459921043391e-27\\
52	2.42587311322456e-24\\
53	3.05348204888152e-25\\
54	1.42708878765667e-24\\
55	1.3713226966301e-24\\
56	5.46239944349086e-24\\
57	4.7137182718606e-25\\
58	3.82805153928713e-24\\
59	3.36938563840834e-24\\
60	2.66337897365566e-24\\
61	1.83104907635863e-24\\
62	2.25385487161886e-24\\
63	2.86929120582948e-24\\
64	6.32703592361054e-26\\
65	5.42168885234804e-24\\
66	3.33383367276975e-24\\
67	1.52044451599936e-24\\
68	1.9859121095514e-24\\
69	3.31830475171674e-24\\
70	4.28379633445621e-25\\
71	1.1653606524931e-24\\
72	1.13059704749942e-24\\
73	8.98067782966828e-25\\
74	3.78576152347476e-24\\
75	9.69516800204777e-25\\
76	1.01175236781096e-24\\
77	9.499281671256e-25\\
78	2.83670512632366e-24\\
79	2.20965904460777e-24\\
80	2.02608366289915e-24\\
81	4.98058703047603e-24\\
82	3.19742636107381e-26\\
83	8.07288816626486e-25\\
84	5.20974277415124e-25\\
85	2.17910314900461e-25\\
86	2.59008368618583e-24\\
87	2.67528020613208e-24\\
88	2.59215787390437e-24\\
89	2.74749500917895e-25\\
90	1.12000346170327e-24\\
91	2.04035025512572e-25\\
92	9.5900286538222e-25\\
93	3.76172492384451e-24\\
94	1.79436002793454e-24\\
95	1.4032061061809e-24\\
96	1.7550197050119e-24\\
97	1.47072210017184e-24\\
98	1.26333693752845e-24\\
99	2.03146812814868e-25\\
100	8.46581798722577e-25\\
101	5.39397516610849e-25\\
102	1.08598931217333e-24\\
103	8.04600539548865e-25\\
104	8.26864589789941e-25\\
105	1.02628553127521e-24\\
106	1.48127793750035e-24\\
107	4.77769562794223e-25\\
108	2.64321345238392e-24\\
109	5.31963258381459e-24\\
110	9.1379744769425e-25\\
111	7.92079677270327e-26\\
112	7.82143809502058e-25\\
113	3.40704856876999e-24\\
114	1.11211514621093e-25\\
115	2.26845824148885e-24\\
116	3.94283412878707e-25\\
117	7.161686922595e-25\\
118	3.23018268012372e-24\\
119	2.87977626542512e-24\\
120	1.11870431058888e-25\\
121	2.01071008379248e-24\\
122	6.52099924096585e-24\\
123	1.83574754890646e-24\\
124	1.72368536109191e-24\\
125	5.45206438908969e-26\\
126	8.99366590399018e-25\\
127	1.97015861365573e-24\\
128	3.26666186335621e-24\\
129	1.28897804234725e-24\\
130	1.81857839781998e-24\\
131	5.3056290606184e-24\\
132	2.11539426616093e-24\\
133	8.03336558683786e-26\\
134	9.88306433513133e-26\\
135	1.27934444735782e-25\\
136	1.47015635177474e-24\\
137	1.67234867214435e-24\\
138	6.1943057381499e-25\\
139	3.09014310446757e-24\\
140	3.47091249730384e-24\\
141	3.51747140342698e-24\\
142	1.4927758048944e-24\\
143	5.19609851936763e-24\\
144	9.68389624667735e-25\\
145	3.58314712980412e-24\\
146	1.2755688184244e-24\\
147	3.34337340062999e-25\\
148	4.08046793302542e-25\\
149	1.53362840166501e-24\\
150	1.77802189888756e-24\\
151	2.55965508138345e-24\\
152	1.34698399726582e-25\\
153	2.29427575503869e-24\\
154	1.92662182812345e-24\\
155	1.04005827818247e-24\\
156	2.7538611025285e-24\\
157	7.9687174649863e-25\\
158	1.85558839989337e-24\\
159	3.19211042561172e-24\\
160	9.36039274509327e-25\\
161	2.0970124073034e-24\\
162	1.55255107366391e-24\\
163	3.77236747236505e-24\\
164	1.29147293821953e-25\\
165	6.24883328736632e-25\\
166	3.39489999517749e-24\\
167	6.18079175447144e-25\\
168	2.23890400852011e-25\\
169	3.67686989595338e-24\\
170	1.43876918433898e-24\\
171	1.80179783068874e-24\\
172	5.65186770701723e-24\\
173	2.20194721295774e-24\\
174	3.48763278764265e-24\\
175	3.62863740360496e-24\\
176	3.34733111675445e-24\\
177	8.23699988444394e-25\\
178	1.5332447965998e-24\\
179	2.16830158574118e-24\\
180	1.95402519052291e-24\\
181	9.23190762026775e-25\\
182	4.13648570544379e-25\\
183	5.60422183145575e-24\\
184	2.7567421563102e-24\\
185	1.48119214507319e-24\\
186	2.51588623217437e-24\\
187	5.6053147909214e-25\\
188	6.37674548316355e-24\\
189	6.2343539264884e-25\\
190	4.17893207662685e-24\\
191	2.95859528866284e-24\\
192	5.33258548468532e-25\\
193	3.28402048978988e-25\\
194	1.61506786042398e-24\\
195	2.42135729534303e-24\\
196	4.74730268397166e-25\\
197	1.41380603423289e-24\\
198	2.30781620311311e-24\\
199	9.60432322412937e-25\\
200	1.91040913236596e-24\\
201	3.74682861454107e-24\\
202	6.73095081852889e-25\\
203	2.10017308069075e-24\\
204	2.15410167238707e-24\\
205	3.30797919508723e-24\\
206	1.14380801113823e-24\\
207	9.12421486136264e-25\\
208	1.37108530077552e-25\\
209	4.69175912967238e-25\\
210	4.15867228671767e-26\\
211	7.30703660801591e-25\\
212	1.43737975453622e-24\\
213	5.90424925139021e-25\\
214	1.99554049629663e-25\\
215	9.61552818921228e-25\\
216	3.73638036158519e-24\\
217	2.92650663675417e-24\\
218	1.97544555939438e-24\\
219	1.46479925526565e-24\\
220	1.68753820875961e-25\\
221	1.63110520367856e-24\\
222	4.8198889825344e-25\\
223	4.5978370414739e-24\\
224	6.13222468639529e-25\\
225	2.95889319969184e-24\\
226	3.4829638758565e-24\\
227	2.6406943170944e-24\\
228	1.65732221326195e-24\\
229	9.680967895234e-25\\
230	4.85937252576884e-24\\
231	1.0300847197726e-24\\
232	1.22735405695742e-24\\
233	4.69067951232716e-24\\
234	8.30452309253511e-25\\
235	2.73483817121485e-25\\
236	1.43697209612268e-24\\
237	3.178542961555e-24\\
238	1.21577600655162e-24\\
239	2.37547789851917e-24\\
240	1.33683307987107e-24\\
241	4.19752011369976e-27\\
242	1.0430758674935e-24\\
243	9.82207518695128e-25\\
244	2.00179801447889e-24\\
245	1.85842680978239e-24\\
246	6.40663523076892e-25\\
247	8.63902243883984e-25\\
248	9.83276803715085e-25\\
249	1.2814752905668e-24\\
250	1.97374982702104e-24\\
251	1.35581524521818e-24\\
252	1.15063165623505e-24\\
253	9.98539052996872e-25\\
254	4.56025209726386e-24\\
255	1.73575632741137e-24\\
256	9.57971575108206e-27\\
257	1.26643057986298e-24\\
258	2.28935902797251e-24\\
259	3.36119489481091e-24\\
260	2.53340671448278e-26\\
261	4.33301082145727e-24\\
262	3.69685640343309e-25\\
263	4.58050410694569e-24\\
264	1.15254541218367e-24\\
265	3.33062082093279e-24\\
266	3.64394201939737e-26\\
267	4.3600809739263e-24\\
268	1.11464360044952e-25\\
269	7.60415806159076e-25\\
270	6.44559301173968e-25\\
271	6.18105008619563e-25\\
272	2.94575495536402e-24\\
273	2.92426762966025e-24\\
274	2.43751316697085e-24\\
275	3.47034297851368e-24\\
276	1.97382471269025e-25\\
277	2.27969190744785e-24\\
278	1.26139607864181e-24\\
279	6.92701578895265e-25\\
280	1.62790605204055e-24\\
281	3.19663563769804e-24\\
282	2.19587181725918e-24\\
283	2.11063624318515e-25\\
284	1.63550239159405e-24\\
285	2.40448239240754e-24\\
286	1.54506750633645e-24\\
287	1.59483445946246e-25\\
288	8.21019513742312e-25\\
289	2.5302557896954e-24\\
290	2.6277569891006e-24\\
291	1.39974524789129e-24\\
292	6.17903792427873e-24\\
293	3.86722404440222e-24\\
294	3.68102758831731e-24\\
295	5.11743474110568e-25\\
296	1.02192877642926e-24\\
297	1.8310026316432e-24\\
298	3.82259494752786e-24\\
299	3.04173307792279e-25\\
300	6.03919451050964e-25\\
};
\addlegendentry{$\|f u_,^T b\|$};

\end{axis}
\end{tikzpicture}%
\end{document}
% This file was created by matlab2tikz.
% Minimal pgfplots version: 1.3
%
%The latest updates can be retrieved from
%  http://www.mathworks.com/matlabcentral/fileexchange/22022-matlab2tikz
%where you can also make suggestions and rate matlab2tikz.
%
\documentclass[tikz]{standalone}
\usepackage{pgfplots}
\usepackage{grffile}
\pgfplotsset{compat=newest}
\usetikzlibrary{plotmarks}
\usepackage{amsmath}

\begin{document}
\definecolor{mycolor1}{rgb}{0.00000,0.44700,0.74100}%
\definecolor{mycolor2}{rgb}{0.85000,0.32500,0.09800}%
\definecolor{mycolor3}{rgb}{0.92900,0.69400,0.12500}%
%
\begin{tikzpicture}

\begin{axis}[%
width=2in,
height=2in,
at={(0.758333in,0.48125in)},
scale only axis,
xmode=log,
xmin=1,
xmax=1000,
xminorticks=true,
ymode=log,
ymin=1e-15,
ymax=100000,
yminorticks=true,
legend style={legend cell align=left,align=left,draw=white!15!black}
]
\addplot [color=mycolor1,solid]
  table[row sep=crcr]{%
1	5.80291325914635\\
2	5.24398633142189\\
3	4.41265376997472\\
4	3.43491948518245\\
5	2.44565012718276\\
6	1.56126453062227\\
7	0.859847540439428\\
8	0.3733558314226\\
9	0.121849840746646\\
10	0.119215305790045\\
11	0.0954961640787028\\
12	0.0393435750918503\\
13	0.0391764973034666\\
14	0.0371479023638504\\
15	0.0308869867646524\\
16	0.0176074208400223\\
17	0.0175121606530242\\
18	0.0174240245553123\\
19	0.011567984856476\\
20	0.00925800883109783\\
21	0.00904526415933576\\
22	0.00657357610817034\\
23	0.00618191918228947\\
24	0.00536941750928665\\
25	0.00507773032597049\\
26	0.0042728286113519\\
27	0.00382302426992483\\
28	0.00345053419308526\\
29	0.00293718879921722\\
30	0.00257039089251196\\
31	0.00256670172067299\\
32	0.00202117032900501\\
33	0.00185491356762632\\
34	0.00182867938644968\\
35	0.00181554110428067\\
36	0.00148522187523704\\
37	0.00132922389063836\\
38	0.0013238685933345\\
39	0.00123071521075591\\
40	0.00099987187667262\\
41	0.000998110327470247\\
42	0.000925667163676346\\
43	0.000867849298387373\\
44	0.000768622445914884\\
45	0.000766356257144352\\
46	0.000732252611400886\\
47	0.000643429303706541\\
48	0.000603463101086622\\
49	0.000588395317569252\\
50	0.000511155340596132\\
51	0.000504422177102059\\
52	0.000469650052227221\\
53	0.000449916704649345\\
54	0.000419841148953739\\
55	0.000407094389234302\\
56	0.000382543288679466\\
57	0.000346702922266728\\
58	0.000334073525419862\\
59	0.000333487032569406\\
60	0.000286333363639469\\
61	0.000277860656146389\\
62	0.000277498923583851\\
63	0.000272520737291836\\
64	0.000242407270030915\\
65	0.000232742550033554\\
66	0.000232231870178852\\
67	0.000217731869940368\\
68	0.000196620814142677\\
69	0.000196508706571953\\
70	0.000184662845631014\\
71	0.000177700190257483\\
72	0.000167332060726475\\
73	0.000166583736818382\\
74	0.000160139178498968\\
75	0.000148827492013223\\
76	0.000143502083879505\\
77	0.000140151050453238\\
78	0.000128479002821029\\
79	0.000127639077181159\\
80	0.000121204964007089\\
81	0.000117346084397946\\
82	0.000112671179105569\\
83	0.00011073554197491\\
84	0.000105754908724099\\
85	9.86144045360324e-05\\
86	9.67402680709878e-05\\
87	9.66184572682738e-05\\
88	8.66882650440434e-05\\
89	8.50303744315653e-05\\
90	8.49592347962177e-05\\
91	8.35686686505744e-05\\
92	7.7098769280915e-05\\
93	7.49300437629744e-05\\
94	7.48019257514232e-05\\
95	7.13757281539831e-05\\
96	6.62900685795648e-05\\
97	6.62422221243725e-05\\
98	6.35533550637135e-05\\
99	6.16033373666505e-05\\
100	5.88255495110435e-05\\
101	5.86928447370937e-05\\
102	5.72121151211203e-05\\
103	5.39307902651237e-05\\
104	5.24094146695952e-05\\
105	5.17154886098673e-05\\
106	4.83620703059389e-05\\
107	4.79316908578732e-05\\
108	4.63586587645992e-05\\
109	4.53118751821771e-05\\
110	4.37964010886163e-05\\
111	4.29337840337269e-05\\
112	4.16564819664303e-05\\
113	3.97076033124348e-05\\
114	3.86215039127612e-05\\
115	3.86143228148279e-05\\
116	3.60095936275083e-05\\
117	3.49431999422961e-05\\
118	3.49277567932593e-05\\
119	3.484786519711e-05\\
120	3.27722009041568e-05\\
121	3.15197634636628e-05\\
122	3.147241205708e-05\\
123	3.08436969623778e-05\\
124	2.85657569710138e-05\\
125	2.85415033542558e-05\\
126	2.82015826489453e-05\\
127	2.73688893023246e-05\\
128	2.59235081727458e-05\\
129	2.59006927224829e-05\\
130	2.58703894996995e-05\\
131	2.44155153953249e-05\\
132	2.35532597560651e-05\\
133	2.35248990132452e-05\\
134	2.24496831768431e-05\\
135	2.19623036074798e-05\\
136	2.14290188559869e-05\\
137	2.12857983390112e-05\\
138	2.07037139949336e-05\\
139	1.99340311936263e-05\\
140	1.95448726117148e-05\\
141	1.91750947706454e-05\\
142	1.82938853071166e-05\\
143	1.81911042714108e-05\\
144	1.76555430697109e-05\\
145	1.72599285627004e-05\\
146	1.68784982249267e-05\\
147	1.66458640850022e-05\\
148	1.62058115873043e-05\\
149	1.555480125622e-05\\
150	1.52632714654318e-05\\
151	1.52587608477172e-05\\
152	1.44445823252664e-05\\
153	1.40522765161644e-05\\
154	1.40487078879052e-05\\
155	1.4005806925792e-05\\
156	1.33613105752287e-05\\
157	1.28694226780742e-05\\
158	1.28555395281982e-05\\
159	1.27143573379767e-05\\
160	1.18731097817948e-05\\
161	1.18353459641706e-05\\
162	1.18342281041128e-05\\
163	1.15334949497683e-05\\
164	1.10334964719736e-05\\
165	1.08913966425552e-05\\
166	1.08754370733499e-05\\
167	1.04840237186045e-05\\
168	1.00267471152559e-05\\
169	1.00158121229961e-05\\
170	9.83317274299059e-06\\
171	9.55276924466675e-06\\
172	9.2314273186855e-06\\
173	9.22310698850645e-06\\
174	9.16520698527827e-06\\
175	8.7251497984196e-06\\
176	8.49743099203932e-06\\
177	8.49551807584123e-06\\
178	8.20423910702998e-06\\
179	7.99227487727527e-06\\
180	7.82921083806178e-06\\
181	7.81062340772052e-06\\
182	7.66113634965329e-06\\
183	7.34024555350838e-06\\
184	7.21535732530547e-06\\
185	7.17233236369028e-06\\
186	6.8911332665264e-06\\
187	6.75389407466256e-06\\
188	6.64304966181557e-06\\
189	6.58278823896935e-06\\
190	6.43914289229584e-06\\
191	6.21920948007317e-06\\
192	6.12421506484769e-06\\
193	6.04214820021176e-06\\
194	5.81762322442484e-06\\
195	5.72751093462573e-06\\
196	5.63016612989317e-06\\
197	5.54915634207551e-06\\
198	5.43400191854584e-06\\
199	5.27265144108831e-06\\
200	5.19092676151477e-06\\
201	5.09933311060402e-06\\
202	4.92346040123619e-06\\
203	4.8503553669644e-06\\
204	4.76614335468791e-06\\
205	4.68927027946347e-06\\
206	4.59624120761783e-06\\
207	4.45703932330304e-06\\
208	4.38912768070659e-06\\
209	4.31428604499421e-06\\
210	4.16876875815649e-06\\
211	4.08996538324689e-06\\
212	4.02535306948148e-06\\
213	3.97109216143523e-06\\
214	3.89169832189398e-06\\
215	3.74698210407305e-06\\
216	3.69397451196597e-06\\
217	3.65531841373834e-06\\
218	3.5316530220354e-06\\
219	3.42778362542132e-06\\
220	3.37842872173977e-06\\
221	3.36247638591971e-06\\
222	3.29748896626226e-06\\
223	3.13679868573996e-06\\
224	3.08291982024803e-06\\
225	3.08210350752092e-06\\
226	2.99597636415588e-06\\
227	2.88034504161119e-06\\
228	2.80647102549396e-06\\
229	2.80118277660148e-06\\
230	2.79473185144397e-06\\
231	2.65071300485738e-06\\
232	2.54250926409505e-06\\
233	2.536997849609e-06\\
234	2.53662101779018e-06\\
235	2.44083705348646e-06\\
236	2.3639500376391e-06\\
237	2.283268236442e-06\\
238	2.27731679111532e-06\\
239	2.24702396681846e-06\\
240	2.15011134185028e-06\\
241	2.06746040480166e-06\\
242	2.06396622135581e-06\\
243	2.02891464267207e-06\\
244	1.97230991490531e-06\\
245	1.9001778949736e-06\\
246	1.8560122869718e-06\\
247	1.79182394989095e-06\\
248	1.76526015921094e-06\\
249	1.74283672112687e-06\\
250	1.70979553094517e-06\\
251	1.60346733776837e-06\\
252	1.55558455755774e-06\\
253	1.55435680409185e-06\\
254	1.54785858257251e-06\\
255	1.46532472892692e-06\\
256	1.42100001262856e-06\\
257	1.35166371794754e-06\\
258	1.3368425037605e-06\\
259	1.33506735275355e-06\\
260	1.27400092761847e-06\\
261	1.21594690850651e-06\\
262	1.20153208427954e-06\\
263	1.12138271515058e-06\\
264	1.10664506651845e-06\\
265	1.09979771233336e-06\\
266	1.07227530234439e-06\\
267	9.94727065371311e-07\\
268	9.66861828400952e-07\\
269	9.21340976323299e-07\\
270	9.07880058458389e-07\\
271	8.91573748021406e-07\\
272	8.48912276696272e-07\\
273	7.9332694723188e-07\\
274	7.86227342302314e-07\\
275	7.04884751422831e-07\\
276	6.9630332189231e-07\\
277	6.95938511936438e-07\\
278	6.81632983709615e-07\\
279	6.09186141088906e-07\\
280	5.88669895516321e-07\\
281	5.33223933614141e-07\\
282	5.21784247985296e-07\\
283	4.98845509999207e-07\\
284	4.74302167797845e-07\\
285	4.37094539188489e-07\\
286	4.31630167640212e-07\\
287	3.54799839341859e-07\\
288	3.4321056083693e-07\\
289	3.12511517671452e-07\\
290	2.98271400657721e-07\\
291	2.74109217076617e-07\\
292	2.50345488155768e-07\\
293	1.97310890347801e-07\\
294	1.94792432754157e-07\\
295	1.16536846215513e-07\\
296	1.13056504879443e-07\\
297	1.03221572887672e-07\\
298	9.9256345962876e-08\\
299	3.87586817520764e-08\\
300	2.70911932465542e-08\\
};
\addlegendentry{$\sigma$};

\addplot [color=mycolor2,solid]
  table[row sep=crcr]{%
1	13.1001099659979\\
2	2.9221621760593e-06\\
3	7.54166164267529\\
4	1.03481288601002e-05\\
5	2.28935484131585\\
6	3.83103219147252e-06\\
7	0.273637904170242\\
8	3.01281362760893e-06\\
9	0.00776851706413927\\
10	3.39477143336446e-06\\
11	0.00283314897827352\\
12	0.000875465031120292\\
13	6.65850518497734e-06\\
14	1.0262582444829e-05\\
15	0.000106405384364617\\
16	8.62478079751821e-06\\
17	0.000194306118680043\\
18	1.26164131013251e-05\\
19	2.47510623297417e-06\\
20	1.9467173640326e-06\\
21	5.5483386318294e-05\\
22	5.2320321013491e-06\\
23	7.82084996781688e-07\\
24	1.8993089796876e-05\\
25	2.35206778653886e-05\\
26	1.88563707011918e-05\\
27	2.02583036561979e-05\\
28	9.99695946857445e-06\\
29	6.49614774504792e-06\\
30	2.98207577534761e-05\\
31	2.60864030792575e-05\\
32	1.11855698305209e-05\\
33	1.02348779954494e-06\\
34	1.83789199529916e-05\\
35	4.10175983581396e-06\\
36	5.23459390105047e-06\\
37	1.00099710394542e-05\\
38	8.29424935665198e-06\\
39	5.98749413345035e-06\\
40	4.32590864381378e-06\\
41	1.56918953554278e-05\\
42	9.58907517432193e-06\\
43	5.4848095108428e-06\\
44	1.54139414405506e-05\\
45	6.20348643615122e-06\\
46	1.0404139298741e-06\\
47	1.95797919069563e-05\\
48	1.84835110212278e-06\\
49	1.30824268412982e-05\\
50	4.45238401534688e-06\\
51	4.40458257553826e-06\\
52	1.50523413870414e-05\\
53	1.51465066823843e-07\\
54	4.8405216012999e-06\\
55	8.31443239818976e-06\\
56	3.94683894748061e-06\\
57	4.00438030191566e-06\\
58	9.9484492984826e-06\\
59	9.96987024020652e-06\\
60	7.51975832974274e-06\\
61	1.23420133042547e-05\\
62	9.88222394794622e-06\\
63	1.61443329974355e-05\\
64	2.49955094263569e-07\\
65	1.39925157269141e-05\\
66	6.03222940025801e-06\\
67	1.25793735854284e-05\\
68	5.94648685132713e-06\\
69	6.93532412959135e-06\\
70	5.9791083466482e-06\\
71	9.8540823659894e-06\\
72	1.32565759779096e-05\\
73	7.41763186529258e-07\\
74	3.39475079880625e-06\\
75	2.10253875187658e-06\\
76	6.37726493152475e-07\\
77	1.39144862931467e-05\\
78	3.82169432639663e-06\\
79	1.28227959313963e-05\\
80	4.40487615207066e-06\\
81	1.48646471534955e-05\\
82	2.83219782560478e-06\\
83	1.64158701481221e-05\\
84	6.8453217686084e-06\\
85	5.2512404286496e-06\\
86	1.59250924899529e-05\\
87	5.68166733840448e-06\\
88	2.17739070436956e-06\\
89	1.75705061852968e-06\\
90	1.24391452691889e-05\\
91	1.10817947206786e-05\\
92	1.28921110312905e-05\\
93	8.28508011738172e-06\\
94	8.59145827007072e-07\\
95	5.30406950114554e-08\\
96	9.43658979945389e-06\\
97	1.33055462940559e-05\\
98	8.97361335390999e-08\\
99	4.13235334589552e-06\\
100	1.68593397750866e-05\\
101	1.05345101792632e-05\\
102	8.86926133723473e-06\\
103	6.74635042728286e-06\\
104	6.56894799855139e-06\\
105	1.07715255520044e-05\\
106	3.31532675657865e-06\\
107	9.93033092144477e-06\\
108	2.01151504812189e-07\\
109	1.01154405889037e-06\\
110	4.33636692898087e-06\\
111	1.84180577857034e-05\\
112	1.04480140804275e-06\\
113	6.65338378742775e-06\\
114	2.37211972930317e-05\\
115	8.89450251210285e-06\\
116	8.99818344515455e-06\\
117	1.03422151671168e-06\\
118	1.81091449097254e-05\\
119	5.62889344437592e-06\\
120	3.06918777672749e-05\\
121	5.95346474185909e-06\\
122	3.62318857851859e-06\\
123	2.57124193051017e-06\\
124	6.85596780001033e-06\\
125	1.02190585175854e-05\\
126	2.24323640270196e-05\\
127	4.80027279631257e-06\\
128	2.10858772152767e-05\\
129	1.02145183780156e-05\\
130	4.60967794113643e-06\\
131	2.55573135543585e-07\\
132	2.19211704166611e-06\\
133	5.13863097850731e-07\\
134	6.36283402488669e-06\\
135	1.06547256142342e-05\\
136	1.32436783749634e-05\\
137	2.97134684446032e-06\\
138	1.04228295559894e-05\\
139	7.2989908712906e-06\\
140	1.13023633353526e-05\\
141	3.13053641048944e-06\\
142	1.26800077700218e-05\\
143	4.71235872628872e-06\\
144	1.60776423692721e-05\\
145	3.27884663045563e-06\\
146	1.3051834567096e-05\\
147	2.50967088000209e-06\\
148	5.51782542465573e-06\\
149	1.88388488281399e-06\\
150	2.37705421193835e-07\\
151	7.76148940551807e-06\\
152	1.34892444694779e-05\\
153	4.62873568160549e-06\\
154	1.87191900027917e-06\\
155	8.2733985073552e-06\\
156	3.0475499986915e-06\\
157	1.75993442465272e-07\\
158	1.12177717359885e-05\\
159	2.15613968734818e-07\\
160	5.54402269237197e-06\\
161	1.02618739020247e-05\\
162	1.64399651224993e-05\\
163	7.98609111829496e-06\\
164	1.01616131129578e-05\\
165	3.19083292042027e-06\\
166	8.33210575194865e-07\\
167	9.94931039843827e-06\\
168	2.73198601360713e-05\\
169	1.24392018347741e-05\\
170	3.62354759220196e-06\\
171	6.39572344263151e-06\\
172	1.69829366991614e-06\\
173	1.69463695548489e-05\\
174	1.37818778218196e-06\\
175	5.46868917483678e-07\\
176	8.57807135157262e-07\\
177	5.71508662966513e-06\\
178	7.3230134649549e-06\\
179	9.45763086424023e-06\\
180	1.21479391898549e-05\\
181	2.04908236822732e-06\\
182	1.55151401332043e-05\\
183	1.49823398606641e-05\\
184	1.30022198942398e-05\\
185	5.51175283464156e-06\\
186	6.6100419412185e-06\\
187	5.81060363322172e-06\\
188	1.45899855782949e-06\\
189	5.87679699662336e-06\\
190	1.40882357776984e-06\\
191	9.63081859065018e-06\\
192	2.02874300841288e-06\\
193	9.8797510727916e-06\\
194	5.59444307311752e-06\\
195	3.44637552987526e-06\\
196	1.75031756337318e-06\\
197	5.73110966142765e-07\\
198	4.44150300228652e-06\\
199	1.57243998898104e-07\\
200	7.69016157634722e-06\\
201	3.7387127834812e-06\\
202	1.53813692970184e-05\\
203	1.35919477408167e-06\\
204	3.11687107155262e-06\\
205	4.69510843316423e-06\\
206	1.48169711890883e-05\\
207	1.25516480083132e-05\\
208	1.29633952528317e-05\\
209	1.48855615753757e-06\\
210	6.67356668991592e-06\\
211	1.48618912446039e-05\\
212	5.68350042313634e-06\\
213	1.01648523369302e-06\\
214	3.63349155368953e-06\\
215	8.24643476182371e-06\\
216	6.68164385686856e-06\\
217	1.35587850557117e-05\\
218	1.05711060994496e-05\\
219	3.85994263115211e-06\\
220	1.99158673961433e-05\\
221	2.13994040629967e-06\\
222	3.0867076579455e-06\\
223	5.39312842442671e-06\\
224	6.55467513520705e-06\\
225	6.38410829150934e-07\\
226	3.56894795021379e-06\\
227	6.3734814718247e-06\\
228	1.48589479330766e-05\\
229	8.01129128921035e-06\\
230	4.8401690942862e-06\\
231	6.65623739965274e-06\\
232	1.30365054060166e-06\\
233	2.45361257310076e-05\\
234	1.17961793777727e-05\\
235	7.71564016684503e-06\\
236	6.37994060834202e-06\\
237	1.13169595197127e-05\\
238	1.35650903463269e-05\\
239	2.19643344398848e-06\\
240	6.43573067713246e-06\\
241	9.86351490982196e-06\\
242	1.24635695769662e-05\\
243	1.71889162247866e-05\\
244	1.35437205921757e-05\\
245	9.02739110435202e-06\\
246	2.57663382987466e-07\\
247	2.05841495800382e-06\\
248	1.95787118722947e-06\\
249	4.33829752473011e-06\\
250	6.70062422839651e-06\\
251	1.62805787463204e-05\\
252	1.04515647175437e-05\\
253	7.37953534356685e-06\\
254	1.05030010140138e-05\\
255	3.70515205329325e-06\\
256	6.24909659595677e-06\\
257	1.87907549243048e-05\\
258	1.26977474347589e-07\\
259	5.97452935548843e-06\\
260	5.52095617421117e-06\\
261	2.53506177244223e-05\\
262	7.54673370545987e-06\\
263	2.23802428129141e-05\\
264	1.14588847805654e-05\\
265	3.67368446848801e-06\\
266	2.00296803951061e-06\\
267	5.99586094788326e-06\\
268	1.98578416873638e-06\\
269	7.23270725351555e-06\\
270	2.23953574531796e-05\\
271	7.27517081508273e-06\\
272	4.66268458854807e-06\\
273	8.05573256032303e-06\\
274	1.7325266582404e-05\\
275	3.05049283870875e-06\\
276	1.2530259871647e-05\\
277	9.09508023863061e-06\\
278	7.58021133788478e-06\\
279	1.68904302416282e-05\\
280	1.43522568422198e-05\\
281	7.47849495731645e-06\\
282	1.81632182833869e-06\\
283	5.1689791338177e-06\\
284	1.01498135128553e-06\\
285	1.26723697830874e-05\\
286	3.07216637085092e-06\\
287	4.00798721222786e-06\\
288	9.30624854955653e-06\\
289	3.6806154402047e-06\\
290	3.84270746091686e-06\\
291	7.25435009946325e-06\\
292	3.02434018950581e-06\\
293	8.23790034176763e-06\\
294	1.09530735604214e-05\\
295	7.88223275348474e-06\\
296	9.36718277966264e-06\\
297	1.53982640895833e-05\\
298	1.62471144264144e-06\\
299	1.05089108001049e-05\\
300	2.72669791190339e-06\\
};
\addlegendentry{$\|u_,^T b\|$};

\addplot [color=mycolor3,solid]
  table[row sep=crcr]{%
1	13.1001099646064\\
2	2.9221621755932e-06\\
3	7.54166164128994\\
4	1.03481288569629e-05\\
5	2.2893548399468\\
6	3.83103218576862e-06\\
7	0.27363790284643\\
8	3.01281355037039e-06\\
9	0.00776851519267626\\
10	3.39477057893798e-06\\
11	0.00283314786707786\\
12	0.000875463008179208\\
13	6.65848966767627e-06\\
14	1.02625558448781e-05\\
15	0.000106404985427922\\
16	8.62468129252588e-06\\
17	0.000194303852496009\\
18	1.26162644634634e-05\\
19	2.47504007818291e-06\\
20	1.94663612917755e-06\\
21	5.54809608560283e-05\\
22	5.23159906519124e-06\\
23	7.82011805572049e-07\\
24	1.89907337686356e-05\\
25	2.35174154186202e-05\\
26	1.88526772257664e-05\\
27	2.02533471480898e-05\\
28	9.99395714209652e-06\\
29	6.4934555577816e-06\\
30	2.98046224033838e-05\\
31	2.60722477238097e-05\\
32	1.11757847294757e-05\\
33	1.02242493682498e-06\\
34	1.83592830183074e-05\\
35	4.09731372644051e-06\\
36	5.22611987472586e-06\\
37	9.98974777052136e-06\\
38	8.27735681381789e-06\\
39	5.97338830089294e-06\\
40	4.31048699733542e-06\\
41	1.56357576129006e-05\\
42	9.54921391446836e-06\\
43	5.45888512752287e-06\\
44	1.53211817582515e-05\\
45	6.16593464857186e-06\\
46	1.0335196329094e-06\\
47	1.94120801470183e-05\\
48	1.83037351550561e-06\\
49	1.2948650532869e-05\\
50	4.39225626171303e-06\\
51	4.34352395428089e-06\\
52	1.48121472601243e-05\\
53	1.48835195201355e-07\\
54	4.74425163384266e-06\\
55	8.13877679123681e-06\\
56	3.85267295621738e-06\\
57	3.88866814448767e-06\\
58	9.6395163448141e-06\\
59	9.65921613681112e-06\\
60	7.20541225847203e-06\\
61	1.17955545803195e-05\\
62	9.4435847977406e-06\\
63	1.54025341035508e-05\\
64	2.3561335770611e-07\\
65	1.31258175186191e-05\\
66	5.65704959003889e-06\\
67	1.16968656679368e-05\\
68	5.44291050146996e-06\\
69	6.34739491826917e-06\\
70	5.41149650438083e-06\\
71	8.85147067956824e-06\\
72	1.17549689867905e-05\\
73	6.57071541109941e-07\\
74	2.97922228708085e-06\\
75	1.81021914840275e-06\\
76	5.43351382402326e-07\\
77	1.17710290283024e-05\\
78	3.14107135739399e-06\\
79	1.05144000159155e-05\\
80	3.5423955238894e-06\\
81	1.17996782486175e-05\\
82	2.20962964259629e-06\\
83	1.27088476935747e-05\\
84	5.18659636888127e-06\\
85	3.83918572528438e-06\\
86	1.15216434754042e-05\\
87	4.10776310020346e-06\\
88	1.47523470460953e-06\\
89	1.17551788010009e-06\\
90	8.31753727081719e-06\\
91	7.32845101834487e-06\\
92	8.04889052529074e-06\\
93	5.06094715837136e-06\\
94	5.2411058785754e-07\\
95	3.11621207267285e-08\\
96	5.2022409240118e-06\\
97	7.33038136661239e-06\\
98	4.75913322225355e-08\\
99	2.12732985200789e-06\\
100	8.29030761421271e-06\\
101	5.16828284479809e-06\\
102	4.23810818249437e-06\\
103	3.02559801840241e-06\\
104	2.85334695032495e-06\\
105	4.60839392087999e-06\\
106	1.31078554709138e-06\\
107	3.88380983983461e-06\\
108	7.54988440025521e-08\\
109	3.68896114225081e-07\\
110	1.51370807669957e-06\\
111	6.26375985991615e-06\\
112	3.41300396241792e-07\\
113	2.03558499867716e-06\\
114	6.9810928310602e-06\\
115	2.61694431090507e-06\\
116	2.39415499738924e-06\\
117	2.63205804185672e-07\\
118	4.60567791080863e-06\\
119	1.42670643270163e-06\\
120	7.08772075976374e-06\\
121	1.29417383074124e-06\\
122	7.85762786048843e-07\\
123	5.40203981002491e-07\\
124	1.27356121722936e-06\\
125	1.89566291265014e-06\\
126	4.08065873205854e-06\\
127	8.31207537355105e-07\\
128	3.33511904961059e-06\\
129	1.61322019466559e-06\\
130	7.2659089037086e-07\\
131	3.65097223009592e-08\\
132	2.94342835882588e-07\\
133	6.88542933779956e-08\\
134	7.85829481774608e-07\\
135	1.2660914750981e-06\\
136	1.50682599359131e-06\\
137	3.34073426228043e-07\\
138	1.11540578880825e-06\\
139	7.29807827996864e-07\\
140	1.09061775015878e-06\\
141	2.91813224959081e-07\\
142	1.08490884349714e-06\\
143	3.9905693225914e-07\\
144	1.28884831874544e-06\\
145	2.52093621720684e-07\\
146	9.62861664472597e-07\\
147	1.80439728662135e-07\\
148	3.77436937804704e-07\\
149	1.19361234840465e-07\\
150	1.45357459862064e-08\\
151	4.74353684202025e-07\\
152	7.4350177696296e-07\\
153	2.42172407679403e-07\\
154	9.78904354347054e-08\\
155	4.30149254222745e-07\\
156	1.4487816728595e-07\\
157	7.78867046205556e-09\\
158	4.95424381083973e-07\\
159	9.32341580675566e-09\\
160	2.10219426475741e-07\\
161	3.86733746408528e-07\\
162	6.19451540146408e-07\\
163	2.86354883995837e-07\\
164	3.34472471691383e-07\\
165	1.02425621108107e-07\\
166	2.6670209927462e-08\\
167	2.96627171703784e-07\\
168	7.46907506054065e-07\\
169	3.39358683668946e-07\\
170	9.53770303915857e-08\\
171	1.59116006211157e-07\\
172	3.95213148977891e-08\\
173	3.93668133763167e-07\\
174	3.16241116941288e-08\\
175	1.13969697325411e-08\\
176	1.6974298819826e-08\\
177	1.13040304694847e-07\\
178	1.35262230031158e-07\\
179	1.65936763409785e-07\\
180	2.0467551174741e-07\\
181	3.436313799135e-08\\
182	2.50484173718729e-07\\
183	2.22338394334064e-07\\
184	1.86536720176442e-07\\
185	7.81476244293179e-08\\
186	8.66092682025657e-08\\
187	7.31699995237464e-08\\
188	1.77816166797417e-08\\
189	7.03456821103416e-08\\
190	1.61441128421376e-08\\
191	1.03031330743442e-07\\
192	2.10525404816279e-08\\
193	9.98217888374328e-08\\
194	5.24402569640828e-08\\
195	3.1321054614372e-08\\
196	1.53756767145826e-08\\
197	4.89188928048338e-09\\
198	3.63668593581293e-08\\
199	1.21276472230588e-09\\
200	5.75006376067589e-08\\
201	2.69842195651984e-08\\
202	1.03540302705566e-07\\
203	8.88154514091188e-09\\
204	1.96702758728648e-08\\
205	2.86880663420827e-08\\
206	8.69990353412147e-08\\
207	6.93258504236315e-08\\
208	6.94463240188156e-08\\
209	7.70612128180915e-09\\
210	3.22682560331797e-08\\
211	6.91821204566978e-08\\
212	2.56311301146555e-08\\
213	4.46187456163631e-09\\
214	1.5320548370256e-08\\
215	3.22429644213573e-08\\
216	2.53936025481854e-08\\
217	5.04613395482025e-08\\
218	3.67342846380007e-08\\
219	1.26383454817112e-08\\
220	6.33507814554329e-08\\
221	6.74305046436503e-09\\
222	9.3551517379943e-09\\
223	1.4795439438428e-08\\
224	1.73712081053715e-08\\
225	1.69102366561169e-09\\
226	8.93378258306696e-09\\
227	1.47491320228436e-08\\
228	3.26483159862008e-08\\
229	1.75364066986116e-08\\
230	1.05463064878082e-08\\
231	1.30499489221835e-08\\
232	2.35184647981218e-09\\
233	4.4072965320592e-08\\
234	2.11825801309346e-08\\
235	1.28302064760611e-08\\
236	9.95226912034166e-09\\
237	1.64709245775331e-08\\
238	1.96402596048377e-08\\
239	3.09619076860359e-09\\
240	8.30741082013793e-09\\
241	1.17731941563449e-08\\
242	1.48264631673358e-08\\
243	1.97598387617405e-08\\
244	1.47137414277735e-08\\
245	9.10373233898685e-09\\
246	2.47915237282595e-10\\
247	1.8460388198585e-09\\
248	1.7042379942434e-09\\
249	3.68104366130701e-09\\
250	5.47212098140951e-09\\
251	1.16945870160676e-08\\
252	7.06613172136226e-09\\
253	4.98131592070431e-09\\
254	7.03059440753795e-09\\
255	2.22290062357826e-09\\
256	3.52587718695521e-09\\
257	9.59326962315951e-09\\
258	6.34128277676636e-11\\
259	2.97577829805758e-09\\
260	2.50416905963425e-09\\
261	1.04747863404013e-08\\
262	3.04481843626301e-09\\
263	7.8655072811388e-09\\
264	3.92208743095686e-09\\
265	1.24190256358231e-09\\
266	6.43656158803866e-10\\
267	1.65823540094615e-09\\
268	5.18864871696498e-10\\
269	1.71611145427395e-09\\
270	5.15966815690495e-09\\
271	1.61647175261847e-09\\
272	9.39249691137204e-10\\
273	1.41722866677663e-09\\
274	2.99369842642394e-09\\
275	4.23694477383222e-10\\
276	1.69826333815966e-09\\
277	1.23139208800316e-09\\
278	9.84539137572127e-10\\
279	1.75227623513542e-09\\
280	1.39036447959163e-09\\
281	5.94437007515812e-10\\
282	1.38244722535525e-10\\
283	3.59594952670179e-10\\
284	6.38333820336665e-11\\
285	6.76849774916147e-10\\
286	1.60012009498641e-10\\
287	1.41053789705335e-10\\
288	3.0647038795103e-10\\
289	1.00495711328955e-10\\
290	9.55777133065781e-11\\
291	1.52385468155077e-10\\
292	5.29918956598921e-11\\
293	8.96645568496189e-11\\
294	1.16193664154086e-10\\
295	2.9928228751044e-11\\
296	3.34738326170615e-11\\
297	4.5868897571112e-11\\
298	4.47505622299115e-12\\
299	4.41369588186155e-12\\
300	5.59500001023731e-13\\
};
\addlegendentry{$\|f u_,^T b\|$};

\end{axis}
\end{tikzpicture}%
\end{document}
\caption{Plot of Fourier coefficients and singular values for the fifth $\mathbf{A}_5, \mathbf{b}_{err5}$ and sixth $\mathbf{A}_6, \mathbf{b}_{err6}$ value pair .}
\label{fig:picard56}
\end{figure}


\subsection{DSVD}
A modification of the filter function given in equation~\ref{eq:filter} to
\begin{equation}
f_i = \frac{\sigma_i}{\sigma_i + \lambda}
\label{eq:filt2}
\end{equation}
gives the damped svd, which introduces less filtering. However the same effect can also be achived by reducing the regularization parameter. Filtering following equation~\ref{eq:filt2} has the drawback that often the L-curve is very noisy, which leads to curvature selection often incorrectly selecting a noise spike instead of the proper edge of the L-curve.
Therefore dsvd has not been pursued further.
A different approach is to truncate the svd after if the singular values are smaller the regularization parameter. Working with unfiltered matrices until they become stable will also lead to a noisy L-curve and has therefore not been pursued further.
  
\subsection{Conjugate gradients}
When solving normalized problems ($\mathbf{A}^T\mathbf{Ax} = \mathbf{A}^T\mathbf{b}$) using the conjugate gradient (CG) algorithm solution components associated with large singular values appear first. While running the CG-algorithm the damping decreases from only using the largest singular values initially to using more and more in higher iterations. The listing below shows the way CG was implemented:
\begin{lstlisting}[language=Matlab]
it = length(b);
r = A'*b;
p = r;
x = 0*p;
rsOld = r'*r;
for n = 1:it
    Asp = A'*(A*p);            % form A'*A without computing A'*A.
    vAlpha = rsOld/(p'*Asp);    % step length.
    x = x + vAlpha*p;           % approximate solution.
    r = r - vAlpha*Asp;         % residual
    rsNew = r'*r;
    vBeta = (rsNew)/(rsOld);    % improvement of previous step.
    p = r + vBeta*p;            % search direction.
    
    rsOld = rsNew;             % squared residual update.
    
    xVec(:,n) = x;                % store the solution
    resNormVec(n) = norm(A*x-b);  % compute residual norm.
    solNormVec(n) = norm(x);    % compute the solution norm.
end
\end{lstlisting}
The code snipped above shows that in each iteration the residual norm and size of the solution are recoded. Observing the behavior of the algorithm reveals that  

Figure~\ref{fig:cgA1} shows the results obtained from running the CG-algorithms on the $\mathbf{A}_1, \mathbf{b}_{err1}$ pair. In comparison to figure~\ref{fig:A1LTihk} CG performs equally well as Thikonov regularization. The regularization parameter with CG is the number of iterations. As the norm values for both norms are not continuous, using the L-curve criterion for selecting a suitable number of iterations is difficult, as it is very hard to find the edge of the L-curve. Using filtered approximations of the norm functions as shown in figure~\ref{fig:cgA1} on the left leads to the smoothed L-curve shown in figure~\ref{fig:Lsmooth}. A meaningful curvature function can now be computed. The approximated edge of the L is indicated by a star as before.
\begin{figure}
% This file was created by matlab2tikz.
% Minimal pgfplots version: 1.3
%
%The latest updates can be retrieved from
%  http://www.mathworks.com/matlabcentral/fileexchange/22022-matlab2tikz
%where you can also make suggestions and rate matlab2tikz.
%
\documentclass[tikz]{standalone}
\usepackage{pgfplots}
\usepackage{grffile}
\pgfplotsset{compat=newest}
\usetikzlibrary{plotmarks}
\usepackage{amsmath}

\begin{document}
\definecolor{mycolor1}{rgb}{0.00000,0.44700,0.74100}%
\definecolor{mycolor2}{rgb}{0.85000,0.32500,0.09800}%
\definecolor{mycolor3}{rgb}{0.92900,0.69400,0.12500}%
\definecolor{mycolor4}{rgb}{0.49400,0.18400,0.55600}%
\definecolor{mycolor5}{rgb}{0.46600,0.67400,0.18800}%
\definecolor{mycolor6}{rgb}{0.30100,0.74500,0.93300}%
\definecolor{mycolor7}{rgb}{0.63500,0.07800,0.18400}%
%
\begin{tikzpicture}

\begin{axis}[%
width=2in,
height=2in,
scale only axis,
xmode=log,
xmin=1,
xmax=100,
xminorticks=true,
xlabel={$\lambda$},
ymode=log,
ymin=0.01,
ymax=100000,
yminorticks=true,
legend style={legend cell align=left,align=left,draw=white!15!black}
]
\addplot [color=mycolor1,solid]
  table[row sep=crcr]{%
1	9.9220799897762\\
2	5.22384771653712\\
3	1.22073459635496\\
4	0.113087520456134\\
5	0.0536016899010133\\
6	0.0241459991267795\\
7	0.0241407558538278\\
8	0.0183473894364139\\
9	0.0183460787368072\\
10	0.0183445014152279\\
11	0.018279043918033\\
12	0.018278998392579\\
13	0.0182789770791641\\
14	0.0181735688030919\\
15	0.0181735688029681\\
16	0.0181567476909605\\
17	0.0181564478596819\\
18	0.0181564478283597\\
19	0.0181564414636345\\
20	0.018156441463592\\
21	0.0181564414632127\\
22	0.0181564413075875\\
23	0.0181564412702185\\
24	0.018156441132594\\
25	0.0181564339875968\\
26	0.0181528233250997\\
27	0.0181528233250967\\
28	0.0181528233250979\\
29	0.0181528233250974\\
30	0.0181528233250955\\
31	0.0181528233083045\\
32	0.018152823289098\\
33	0.0181528232889588\\
34	0.0181528232854777\\
35	0.0181526447440435\\
36	0.0181526444912152\\
37	0.018152644418462\\
38	0.0181526444167262\\
39	0.0181526444167278\\
40	0.0181526444135345\\
41	0.0181526444131782\\
42	0.0181526441211895\\
43	0.0181526441142586\\
44	0.0181526441142607\\
45	0.0181526441142589\\
46	0.0181526441142599\\
47	0.0181526441142512\\
48	0.0181466351262467\\
49	0.0181463248100883\\
50	0.0181462378009243\\
51	0.0181462377886996\\
52	0.0181462377886854\\
53	0.0181462377885454\\
54	0.0181462377862419\\
55	0.0181462377856304\\
56	0.018146237784584\\
57	0.0181462377836297\\
58	0.0181462377835978\\
59	0.0181462377836032\\
60	0.0181462377836241\\
61	0.0181462377836145\\
62	0.0181462193716606\\
63	0.0181462192525478\\
64	0.0181462192523438\\
65	0.0181462192523162\\
66	0.0181462192465276\\
67	0.0181462181960199\\
68	0.0181462181749204\\
69	0.0181462181664436\\
70	0.0181461952395225\\
71	0.0181461795508129\\
72	0.0181459008965547\\
73	0.0181458320589595\\
74	0.0181457383212846\\
75	0.0181239119321869\\
76	0.0181225924481826\\
77	0.0181104637089701\\
78	0.0181104631263809\\
79	0.0181104630703552\\
80	0.0181104630702709\\
81	0.0181104630703027\\
82	0.0181104630704289\\
83	0.0181104630705017\\
84	0.0181104630704215\\
85	0.0181104630704432\\
86	0.0181104630703411\\
87	0.0181104630704194\\
88	0.0181104630703396\\
89	0.0181104630704904\\
90	0.0181104630704603\\
91	0.0181104630704297\\
92	0.0181104630704509\\
93	0.018110463050969\\
94	0.0181104629782807\\
95	0.0181104629782702\\
96	0.0181104629782669\\
97	0.0181104629782259\\
98	0.0181104629775648\\
99	0.0181104629773879\\
100	0.0181104629756746\\
};
\addlegendentry{$\|\mathbf{Ax} - \mathbf{b}\|$};

\addplot [color=mycolor2,solid]
  table[row sep=crcr]{%
1	13.1637696637118\\
2	15.0307112849308\\
3	16.6355536620982\\
4	17.0367059668325\\
5	17.135439133901\\
6	17.2425842184358\\
7	17.2427283164185\\
8	17.2716708578138\\
9	17.2717017186667\\
10	17.2718500696266\\
11	17.2845475920743\\
12	17.2845875934804\\
13	17.2846029447901\\
14	17.3734436446022\\
15	17.3734436517489\\
16	20.0161988789642\\
17	20.1043433559446\\
18	20.1043523879112\\
19	20.1062180851961\\
20	20.1062180788003\\
21	20.1062182097041\\
22	20.1062674782991\\
23	20.1062781630555\\
24	20.1063167053371\\
25	20.1085010214977\\
26	41.107396819893\\
27	41.1073968570845\\
28	41.1073968571067\\
29	41.1073968575586\\
30	41.1073968577031\\
31	41.1075390487128\\
32	41.1077028362766\\
33	41.107704009064\\
34	41.1077337461163\\
35	57.2476730556261\\
36	57.285401499402\\
37	57.2962626714349\\
38	57.2965216090896\\
39	57.2965216951864\\
40	57.2969993065189\\
41	57.2970524284632\\
42	57.3407470804104\\
43	57.3417845801515\\
44	57.3417845801697\\
45	57.3417847189991\\
46	57.3417847897929\\
47	57.3417855429398\\
48	2532.42784341523\\
49	2662.0694487955\\
50	2698.41955625002\\
51	2698.42466881185\\
52	2698.42467290575\\
53	2698.42473975567\\
54	2698.42569701528\\
55	2698.42594191528\\
56	2698.42639409438\\
57	2698.42679537423\\
58	2698.42679537456\\
59	2698.42679558551\\
60	2698.42679629285\\
61	2698.42679703638\\
62	2706.11935953165\\
63	2706.16914393549\\
64	2706.16922500801\\
65	2706.16923178794\\
66	2706.17165100515\\
67	2706.61074399856\\
68	2706.61956296054\\
69	2706.62309851171\\
70	2716.20860630884\\
71	2722.77093321977\\
72	2839.71932175165\\
73	2868.71797267656\\
74	2908.27060617762\\
75	12579.0163256388\\
76	13169.7584674439\\
77	18603.1714544719\\
78	18603.4326018489\\
79	18603.457623403\\
80	18603.457623403\\
81	18603.457623403\\
82	18603.4576234045\\
83	18603.4576234047\\
84	18603.4576234047\\
85	18603.4576234066\\
86	18603.4576234066\\
87	18603.4576234066\\
88	18603.4576234066\\
89	18603.4576234068\\
90	18603.4576234068\\
91	18603.4576234068\\
92	18603.4576234069\\
93	18603.4670974675\\
94	18603.505504999\\
95	18603.5055165411\\
96	18603.5055275706\\
97	18603.5055672328\\
98	18603.5058943612\\
99	18603.5059868414\\
100	18603.5069356017\\
};
\addlegendentry{$\|\mathbf{Lx}\|$};

\addplot [color=mycolor3,solid,forget plot]
  table[row sep=crcr]{%
1	13.1637696637118\\
2	13.5230739600013\\
3	13.9021867722929\\
4	14.2975772591311\\
5	14.7089038578762\\
6	15.1366113591204\\
7	15.5811262451614\\
8	16.0431948718219\\
9	16.5235122442382\\
10	17.0228954724509\\
11	17.5514976178047\\
12	18.1111946216294\\
13	18.7081441751654\\
14	19.3448342674945\\
15	20.0237977793013\\
16	20.7479420602408\\
17	21.5106709526693\\
18	22.3140383818767\\
19	23.1604877334471\\
20	24.0526208075883\\
21	24.9932913789099\\
22	25.9854863660396\\
23	27.0326688996395\\
24	28.3096042481032\\
25	29.8481016817626\\
26	31.6896499980377\\
27	33.8091623289854\\
28	36.2543212301866\\
29	39.0808533806381\\
30	42.3510077005631\\
31	46.1384418227755\\
32	50.5319305558649\\
33	55.6381460341613\\
34	61.5863360165639\\
35	68.5332049783264\\
36	76.5882967557939\\
37	85.9543914113665\\
38	96.8767885188175\\
39	109.652215191537\\
40	124.642072287663\\
41	142.285772310625\\
42	163.157200863902\\
43	187.932148478166\\
44	217.443260403828\\
45	252.720765985515\\
46	295.045142398835\\
47	346.011271975648\\
48	407.638775416134\\
49	476.636220502235\\
50	553.050821454676\\
51	638.275733244325\\
52	733.576935523181\\
53	840.073067032175\\
54	958.567107441436\\
55	1089.83826396902\\
56	1234.62665644068\\
57	1393.61662151144\\
58	1567.41882512588\\
59	1756.55143973819\\
60	1961.42072482511\\
61	2183.45806631318\\
62	2423.16851585585\\
63	2680.91609022225\\
64	2956.94770700582\\
65	3251.35871132703\\
66	3564.07597755841\\
67	3894.84200585835\\
68	4243.20354799855\\
69	4608.49396279272\\
70	4989.8250269819\\
71	5386.01842287091\\
72	5795.66430296145\\
73	6216.31529836468\\
74	6686.13676861931\\
75	7211.8311355396\\
76	7764.582339867\\
77	8343.13228573784\\
78	8937.14099133794\\
79	9543.91404852202\\
80	10160.4480191138\\
81	10783.4474442003\\
82	11409.3478894132\\
83	12034.3447819526\\
84	12654.4276682748\\
85	13265.4197468925\\
86	13863.0221199582\\
87	14442.8621679804\\
88	15000.6136641727\\
89	15531.9222261125\\
90	16032.5207276652\\
91	16498.2862573247\\
92	16925.2963498196\\
93	17309.8882303932\\
94	17648.7024991827\\
95	17938.7358854164\\
96	18177.4885749413\\
97	18362.8714091387\\
98	18494.4404995464\\
99	18571.3195722705\\
100	18603.5069356017\\
};
\addplot [color=mycolor4,solid,forget plot]
  table[row sep=crcr]{%
1	9.9220799897762\\
2	6.31839115855031\\
3	4.03182537965502\\
4	2.59005342423534\\
5	1.68784978752735\\
6	1.1184446170206\\
7	0.755542779545944\\
8	0.520316769174402\\
9	0.365613428826239\\
10	0.262133246528053\\
11	0.191764334672344\\
12	0.143141080201171\\
13	0.109021166120885\\
14	0.0847241924439057\\
15	0.0671828083053853\\
16	0.0543579374578572\\
17	0.0448769332215152\\
18	0.0378038730364523\\
19	0.0324939333781141\\
20	0.0284984402223147\\
21	0.0254918895888\\
22	0.023256531740175\\
23	0.0216120377338661\\
24	0.0204331501464233\\
25	0.0195799074078521\\
26	0.0189719270406119\\
27	0.0185691600505054\\
28	0.0183403363097431\\
29	0.0182367045389883\\
30	0.0181868016148446\\
31	0.0181684726598975\\
32	0.0181584384973729\\
33	0.0181566790887819\\
34	0.0181552189982918\\
35	0.0181540560786327\\
36	0.0181531883467721\\
37	0.0181525119396662\\
38	0.0181520267343604\\
39	0.0181517326814746\\
40	0.0181514618094823\\
41	0.0181512141173558\\
42	0.0181509627113694\\
43	0.0181507071113561\\
44	0.0181504473174316\\
45	0.0181501833195823\\
46	0.0181499150813013\\
47	0.0181496425776693\\
48	0.0181493653627012\\
49	0.0181491025545761\\
50	0.0181488549955842\\
51	0.0181485880033762\\
52	0.0181482936897116\\
53	0.0181479526162648\\
54	0.0181475647847386\\
55	0.0181471301980487\\
56	0.0181466488595563\\
57	0.0181461207729587\\
58	0.0181455459423104\\
59	0.0181449243720547\\
60	0.0181442560669904\\
61	0.0181435407467614\\
62	0.0181427784165257\\
63	0.0181419691409997\\
64	0.0181411129268467\\
65	0.0181402097807167\\
66	0.0181392597096183\\
67	0.0181382627209472\\
68	0.0181372188253384\\
69	0.0181361280309222\\
70	0.0181349903460784\\
71	0.0181338058529558\\
72	0.0181325746109018\\
73	0.0181312975204187\\
74	0.018129965207331\\
75	0.0181285774874853\\
76	0.0181272040495168\\
77	0.0181258491131265\\
78	0.0181245515032951\\
79	0.0181233112095893\\
80	0.0181221282204304\\
81	0.018121002524607\\
82	0.01811993411145\\
83	0.0181189229707168\\
84	0.0181179690927951\\
85	0.018117072468647\\
86	0.0181162330897771\\
87	0.0181154509482329\\
88	0.0181147260071964\\
89	0.0181140582596128\\
90	0.0181134476991587\\
91	0.0181128943200529\\
92	0.0181123981170466\\
93	0.0181119590837639\\
94	0.0181115772160768\\
95	0.0181112525105893\\
96	0.0181109849276155\\
97	0.0181107744395704\\
98	0.0181106205993976\\
99	0.0181105232947977\\
100	0.0181104629756746\\
};
\addplot [color=mycolor4,only marks,mark=asterisk,mark options={solid},forget plot]
  table[row sep=crcr]{%
28	41.1073968571067\\
};
\addplot [color=mycolor6,only marks,mark=asterisk,mark options={solid},forget plot]
  table[row sep=crcr]{%
28	0.0181528233250979\\
};
\addplot [color=mycolor3,only marks,mark=asterisk,mark options={solid},forget plot]
  table[row sep=crcr]{%
28	36.2543212301866\\
};
\addplot [color=mycolor1,only marks,mark=asterisk,mark options={solid},forget plot]
  table[row sep=crcr]{%
28	0.0183403363097431\\
};
\end{axis}
\end{tikzpicture}%
\end{document}
% This file was created by matlab2tikz.
% Minimal pgfplots version: 1.3
%
%The latest updates can be retrieved from
%  http://www.mathworks.com/matlabcentral/fileexchange/22022-matlab2tikz
%where you can also make suggestions and rate matlab2tikz.
%
\documentclass[tikz]{standalone}
\usepackage{pgfplots}
\usepackage{grffile}
\pgfplotsset{compat=newest}
\usetikzlibrary{plotmarks}
\usepackage{amsmath}

\begin{document}
\definecolor{mycolor1}{rgb}{0.00000,0.44700,0.74100}%
\definecolor{mycolor2}{rgb}{0.85000,0.32500,0.09800}%
%
\begin{tikzpicture}

\begin{axis}[%
width=2in,
height=2in,
scale only axis,
xmode=log,
xmin=0.01,
xmax=10,
xminorticks=true,
xlabel={$\|\mathbf{Ax} - \mathbf{b}\|$},
ymode=log,
ymin=10,
ymax=100000,
yminorticks=true,
ylabel={$\|\mathbf{Lx}\|$}
]
\addplot [color=mycolor1,solid,forget plot]
  table[row sep=crcr]{%
9.9220799897762	13.1637696637118\\
5.22384771653712	15.0307112849308\\
1.22073459635496	16.6355536620982\\
0.113087520456134	17.0367059668325\\
0.0536016899010133	17.135439133901\\
0.0241459991267795	17.2425842184358\\
0.0241407558538278	17.2427283164185\\
0.0183473894364139	17.2716708578138\\
0.0183460787368072	17.2717017186667\\
0.0183445014152279	17.2718500696266\\
0.018279043918033	17.2845475920743\\
0.018278998392579	17.2845875934804\\
0.0182789770791641	17.2846029447901\\
0.0181735688030919	17.3734436446022\\
0.0181735688029681	17.3734436517489\\
0.0181567476909605	20.0161988789642\\
0.0181564478596819	20.1043433559446\\
0.0181564478283597	20.1043523879112\\
0.0181564414636345	20.1062180851961\\
0.018156441463592	20.1062180788003\\
0.0181564414632127	20.1062182097041\\
0.0181564413075875	20.1062674782991\\
0.0181564412702185	20.1062781630555\\
0.018156441132594	20.1063167053371\\
0.0181564339875968	20.1085010214977\\
0.0181528233250997	41.107396819893\\
0.0181528233250967	41.1073968570845\\
0.0181528233250979	41.1073968571067\\
0.0181528233250974	41.1073968575586\\
0.0181528233250955	41.1073968577031\\
0.0181528233083045	41.1075390487128\\
0.018152823289098	41.1077028362766\\
0.0181528232889588	41.107704009064\\
0.0181528232854777	41.1077337461163\\
0.0181526447440435	57.2476730556261\\
0.0181526444912152	57.285401499402\\
0.018152644418462	57.2962626714349\\
0.0181526444167262	57.2965216090896\\
0.0181526444167278	57.2965216951864\\
0.0181526444135345	57.2969993065189\\
0.0181526444131782	57.2970524284632\\
0.0181526441211895	57.3407470804104\\
0.0181526441142586	57.3417845801515\\
0.0181526441142607	57.3417845801697\\
0.0181526441142589	57.3417847189991\\
0.0181526441142599	57.3417847897929\\
0.0181526441142512	57.3417855429398\\
0.0181466351262467	2532.42784341523\\
0.0181463248100883	2662.0694487955\\
0.0181462378009243	2698.41955625002\\
0.0181462377886996	2698.42466881185\\
0.0181462377886854	2698.42467290575\\
0.0181462377885454	2698.42473975567\\
0.0181462377862419	2698.42569701528\\
0.0181462377856304	2698.42594191528\\
0.018146237784584	2698.42639409438\\
0.0181462377836297	2698.42679537423\\
0.0181462377835978	2698.42679537456\\
0.0181462377836032	2698.42679558551\\
0.0181462377836241	2698.42679629285\\
0.0181462377836145	2698.42679703638\\
0.0181462193716606	2706.11935953165\\
0.0181462192525478	2706.16914393549\\
0.0181462192523438	2706.16922500801\\
0.0181462192523162	2706.16923178794\\
0.0181462192465276	2706.17165100515\\
0.0181462181960199	2706.61074399856\\
0.0181462181749204	2706.61956296054\\
0.0181462181664436	2706.62309851171\\
0.0181461952395225	2716.20860630884\\
0.0181461795508129	2722.77093321977\\
0.0181459008965547	2839.71932175165\\
0.0181458320589595	2868.71797267656\\
0.0181457383212846	2908.27060617762\\
0.0181239119321869	12579.0163256388\\
0.0181225924481826	13169.7584674439\\
0.0181104637089701	18603.1714544719\\
0.0181104631263809	18603.4326018489\\
0.0181104630703552	18603.457623403\\
0.0181104630702709	18603.457623403\\
0.0181104630703027	18603.457623403\\
0.0181104630704289	18603.4576234045\\
0.0181104630705017	18603.4576234047\\
0.0181104630704215	18603.4576234047\\
0.0181104630704432	18603.4576234066\\
0.0181104630703411	18603.4576234066\\
0.0181104630704194	18603.4576234066\\
0.0181104630703396	18603.4576234066\\
0.0181104630704904	18603.4576234068\\
0.0181104630704603	18603.4576234068\\
0.0181104630704297	18603.4576234068\\
0.0181104630704509	18603.4576234069\\
0.018110463050969	18603.4670974675\\
0.0181104629782807	18603.505504999\\
0.0181104629782702	18603.5055165411\\
0.0181104629782669	18603.5055275706\\
0.0181104629782259	18603.5055672328\\
0.0181104629775648	18603.5058943612\\
0.0181104629773879	18603.5059868414\\
0.0181104629756746	18603.5069356017\\
};
\addplot [color=mycolor1,only marks,mark=asterisk,mark options={solid},forget plot]
  table[row sep=crcr]{%
0.0181528233250979	41.1073968571067\\
};
\end{axis}
\end{tikzpicture}%
\end{document}
\caption{Result of conjugate gradient iterations on the $\mathbf{A}_1$, $\mathbf{b}_{err1}$ pair. 
The yellow and purple lines show smoothed version of the orange and blue lines.}
\label{fig:cgA1}
\end{figure}
\begin{figure}
% This file was created by matlab2tikz.
% Minimal pgfplots version: 1.3
%
%The latest updates can be retrieved from
%  http://www.mathworks.com/matlabcentral/fileexchange/22022-matlab2tikz
%where you can also make suggestions and rate matlab2tikz.
%
\documentclass[tikz]{standalone}
\usepackage{pgfplots}
\usepackage{grffile}
\pgfplotsset{compat=newest}
\usetikzlibrary{plotmarks}
\usepackage{amsmath}

\begin{document}
\definecolor{mycolor1}{rgb}{0.00000,0.44700,0.74100}%
\definecolor{mycolor2}{rgb}{0.85000,0.32500,0.09800}%
%
\begin{tikzpicture}

\begin{axis}[%
width=2in,
height=2in,
scale only axis,
xmode=log,
xmin=10,
xmax=100000,
xminorticks=true,
xlabel={$\|\mathbf{Ax} - \mathbf{b}\|$},
ymode=log,
ymin=0.01,
ymax=10,
yminorticks=true,
ylabel={$\|\mathbf{Lx}\|$}
]
\addplot [color=mycolor1,solid,forget plot]
  table[row sep=crcr]{%
13.1637696637118	9.9220799897762\\
13.5230739600013	6.31839115855031\\
13.9021867722929	4.03182537965502\\
14.2975772591311	2.59005342423534\\
14.7089038578762	1.68784978752735\\
15.1366113591204	1.1184446170206\\
15.5811262451614	0.755542779545944\\
16.0431948718219	0.520316769174402\\
16.5235122442382	0.365613428826239\\
17.0228954724509	0.262133246528053\\
17.5514976178047	0.191764334672344\\
18.1111946216294	0.143141080201171\\
18.7081441751654	0.109021166120885\\
19.3448342674945	0.0847241924439057\\
20.0237977793013	0.0671828083053853\\
20.7479420602408	0.0543579374578572\\
21.5106709526693	0.0448769332215152\\
22.3140383818767	0.0378038730364523\\
23.1604877334471	0.0324939333781141\\
24.0526208075883	0.0284984402223147\\
24.9932913789099	0.0254918895888\\
25.9854863660396	0.023256531740175\\
27.0326688996395	0.0216120377338661\\
28.3096042481032	0.0204331501464233\\
29.8481016817626	0.0195799074078521\\
31.6896499980377	0.0189719270406119\\
33.8091623289854	0.0185691600505054\\
36.2543212301866	0.0183403363097431\\
39.0808533806381	0.0182367045389883\\
42.3510077005631	0.0181868016148446\\
46.1384418227755	0.0181684726598975\\
50.5319305558649	0.0181584384973729\\
55.6381460341613	0.0181566790887819\\
61.5863360165639	0.0181552189982918\\
68.5332049783264	0.0181540560786327\\
76.5882967557939	0.0181531883467721\\
85.9543914113665	0.0181525119396662\\
96.8767885188175	0.0181520267343604\\
109.652215191537	0.0181517326814746\\
124.642072287663	0.0181514618094823\\
142.285772310625	0.0181512141173558\\
163.157200863902	0.0181509627113694\\
187.932148478166	0.0181507071113561\\
217.443260403828	0.0181504473174316\\
252.720765985515	0.0181501833195823\\
295.045142398835	0.0181499150813013\\
346.011271975648	0.0181496425776693\\
407.638775416134	0.0181493653627012\\
476.636220502235	0.0181491025545761\\
553.050821454676	0.0181488549955842\\
638.275733244325	0.0181485880033762\\
733.576935523181	0.0181482936897116\\
840.073067032175	0.0181479526162648\\
958.567107441436	0.0181475647847386\\
1089.83826396902	0.0181471301980487\\
1234.62665644068	0.0181466488595563\\
1393.61662151144	0.0181461207729587\\
1567.41882512588	0.0181455459423104\\
1756.55143973819	0.0181449243720547\\
1961.42072482511	0.0181442560669904\\
2183.45806631318	0.0181435407467614\\
2423.16851585585	0.0181427784165257\\
2680.91609022225	0.0181419691409997\\
2956.94770700582	0.0181411129268467\\
3251.35871132703	0.0181402097807167\\
3564.07597755841	0.0181392597096183\\
3894.84200585835	0.0181382627209472\\
4243.20354799855	0.0181372188253384\\
4608.49396279272	0.0181361280309222\\
4989.8250269819	0.0181349903460784\\
5386.01842287091	0.0181338058529558\\
5795.66430296145	0.0181325746109018\\
6216.31529836468	0.0181312975204187\\
6686.13676861931	0.018129965207331\\
7211.8311355396	0.0181285774874853\\
7764.582339867	0.0181272040495168\\
8343.13228573784	0.0181258491131265\\
8937.14099133794	0.0181245515032951\\
9543.91404852202	0.0181233112095893\\
10160.4480191138	0.0181221282204304\\
10783.4474442003	0.018121002524607\\
11409.3478894132	0.01811993411145\\
12034.3447819526	0.0181189229707168\\
12654.4276682748	0.0181179690927951\\
13265.4197468925	0.018117072468647\\
13863.0221199582	0.0181162330897771\\
14442.8621679804	0.0181154509482329\\
15000.6136641727	0.0181147260071964\\
15531.9222261125	0.0181140582596128\\
16032.5207276652	0.0181134476991587\\
16498.2862573247	0.0181128943200529\\
16925.2963498196	0.0181123981170466\\
17309.8882303932	0.0181119590837639\\
17648.7024991827	0.0181115772160768\\
17938.7358854164	0.0181112525105893\\
18177.4885749413	0.0181109849276155\\
18362.8714091387	0.0181107744395704\\
18494.4404995464	0.0181106205993976\\
18571.3195722705	0.0181105232947977\\
18603.5069356017	0.0181104629756746\\
};
\addplot [color=mycolor1,only marks,mark=asterisk,mark options={solid},forget plot]
  table[row sep=crcr]{%
36.2543212301866	0.0183403363097431\\
};
\end{axis}
\end{tikzpicture}%
\end{document}
% This file was created by matlab2tikz.
% Minimal pgfplots version: 1.3
%
%The latest updates can be retrieved from
%  http://www.mathworks.com/matlabcentral/fileexchange/22022-matlab2tikz
%where you can also make suggestions and rate matlab2tikz.
%
\documentclass[tikz]{standalone}
\usepackage{pgfplots}
\usepackage{grffile}
\pgfplotsset{compat=newest}
\usetikzlibrary{plotmarks}
\usepackage{amsmath}

\begin{document}
\definecolor{mycolor1}{rgb}{0.00000,0.44700,0.74100}%
\definecolor{mycolor2}{rgb}{0.85000,0.32500,0.09800}%
%
\begin{tikzpicture}

\begin{axis}[%
width=2in,
height=2in,
scale only axis,
xmin=0,
xmax=5,
xlabel={$\lambda$},
ymin=0,
ymax=1.6,
ylabel={curvature $\kappa$}
]
\addplot [color=mycolor1,solid,forget plot]
  table[row sep=crcr]{%
0	0.0167766085529283\\
0.693147180559945	0.0189397073821273\\
1.09861228866811	0.0216845123547695\\
1.38629436111989	0.0251338466241147\\
1.6094379124341	0.0294551222455297\\
1.79175946922805	0.0347141201312574\\
1.94591014905531	0.0411358294830195\\
2.07944154167984	0.0489979922502466\\
2.19722457733622	0.0586573788047913\\
2.30258509299405	0.0704933342643364\\
2.39789527279837	0.0850528219765524\\
2.484906649788	0.103012367659215\\
2.56494935746154	0.125268574791724\\
2.63905732961526	0.152938345218258\\
2.70805020110221	0.187425251291993\\
2.77258872223978	0.230551855549908\\
2.83321334405622	0.284463673622252\\
2.89037175789616	0.351705132450472\\
2.94443897916644	0.435064038396026\\
2.99573227355399	0.537420781579193\\
3.04452243772342	0.66106340340424\\
3.09104245335832	0.806901765226731\\
3.13549421592915	0.968712821007155\\
3.17805383034795	1.13888477744322\\
3.2188758248682	1.30425840649112\\
3.25809653802148	1.44725341951913\\
3.29583686600433	1.54760268980728\\
3.3322045101752	1.58978950902484\\
3.36729582998647	1.56963700073904\\
3.40119738166216	1.49242142415332\\
3.43398720448515	1.37219067779103\\
3.46573590279973	1.2254633055599\\
3.49650756146648	1.06850353391227\\
3.52636052461616	0.913648003265563\\
3.55534806148941	0.76919536951297\\
3.58351893845611	0.639329346476154\\
3.61091791264422	0.525568696370867\\
3.63758615972639	0.427718430900058\\
3.66356164612965	0.344654945121758\\
3.68887945411394	0.274831313723736\\
3.71357206670431	0.216593462406468\\
3.73766961828337	0.168386690213755\\
3.76120011569356	0.128816328297235\\
3.78418963391826	0.0966832265180771\\
3.80666248977032	0.0709011146772565\\
3.8286413964891	0.050674348305042\\
3.85014760171006	0.0351319086381279\\
3.87120101090789	0.0236203084406923\\
3.89182029811063	0.0150334182899669\\
3.91202300542815	0.00872120008819557\\
3.93182563272433	0.00440568664462655\\
3.95124371858143	0.00187916799998884\\
3.97029191355212	0.000701588295402476\\
3.98898404656427	0.000111027575184678\\
4.00733318523247	0.000121963003343445\\
4.02535169073515	0.000263591701467969\\
4.04305126783455	0.000305254485782384\\
4.06044301054642	0.000344934425959933\\
4.07753744390572	0.000382418622116778\\
4.0943445622221	0.000417530543782299\\
4.11087386417331	0.000451395173195552\\
4.12713438504509	0.000483591019735598\\
4.14313472639153	0.000513525190282487\\
4.15888308335967	0.00054345423427432\\
4.17438726989564	0.000572934296102966\\
4.18965474202643	0.000601861416301532\\
4.20469261939097	0.000629529837159277\\
4.21950770517611	0.000654994510124112\\
4.23410650459726	0.000677026012442185\\
4.24849524204936	0.000694044212684741\\
4.26267987704132	0.00070405219758426\\
4.27666611901606	0.000704492395955336\\
4.29045944114839	0.00070136011625095\\
4.30406509320417	0.000695693353607055\\
4.31748811353631	0.000688598930024761\\
4.33073334028633	0.000679414880955092\\
4.34380542185368	0.000670271820961804\\
4.35670882668959	0.000661303989668554\\
4.36944785246702	0.000652674141694752\\
4.38202663467388	0.000644608211774994\\
4.39444915467244	0.000637418700512449\\
4.40671924726425	0.000631547199697562\\
4.4188406077966	0.000627627299752586\\
4.43081679884331	0.00062658294786184\\
4.44265125649032	0.000628883725460201\\
4.45434729625351	0.000636073444424473\\
4.46590811865458	0.000650581793624616\\
4.47733681447821	0.000676279762049829\\
4.48863636973214	0.00071956517639409\\
4.49980967033027	0.000791462286493019\\
4.51085950651685	0.000911923826964741\\
4.52178857704904	0.00111931416286843\\
4.53259949315326	0.00149326287906961\\
4.54329478227	0.00221625629300766\\
4.55387689160054	0.00376660912440634\\
4.56434819146784	0.00766104887302846\\
4.57471097850338	0.0203916439561719\\
4.58496747867057	0.0890087106509088\\
};
\addplot [color=mycolor1,only marks,mark=asterisk,mark options={solid},forget plot]
  table[row sep=crcr]{%
3.3322045101752	1.58978950902484\\
};
\end{axis}
\end{tikzpicture}%
\end{document}
\caption{Smoothed L-curve and $\kappa$ of this new L-curve.}
\label{fig:Lsmooth}
\end{figure}
\begin{figure}
\centering
% This file was created by matlab2tikz.
% Minimal pgfplots version: 1.3
%
%The latest updates can be retrieved from
%  http://www.mathworks.com/matlabcentral/fileexchange/22022-matlab2tikz
%where you can also make suggestions and rate matlab2tikz.
%
\documentclass[tikz]{standalone}
\usepackage{pgfplots}
\usepackage{grffile}
\pgfplotsset{compat=newest}
\usetikzlibrary{plotmarks}
\usepackage{amsmath}

\begin{document}
\definecolor{mycolor1}{rgb}{0.00000,0.44700,0.74100}%
\definecolor{mycolor2}{rgb}{0.85000,0.32500,0.09800}%
%
\begin{tikzpicture}

\begin{axis}[%
width=2in,
height=2in,
at={(0.758333in,0.48125in)},
scale only axis,
xmode=log,
xmin=0.151305499283386,
xmax=0.229020390187819,
xminorticks=true,
xlabel={$\|\mathbf{Ax} - \mathbf{b}\|$},
ymode=log,
ymin=1,
ymax=10000000000,
yminorticks=true,
ylabel={$\|\mathbf{Lx}\|$}
]
\addplot [color=mycolor1,solid,forget plot]
  table[row sep=crcr]{%
0.229020390187819	1.87070651536815\\
0.166990135839937	12.347546744138\\
0.166803269858904	34.370060445427\\
0.166803269857116	34.3700608098966\\
0.166773842325852	692.858545301545\\
0.16677384232524	692.858560219614\\
0.166773842325242	692.858560221952\\
0.165865249901931	209285.709602614\\
0.165865249902473	209285.709602614\\
0.165865249902437	209285.709612124\\
0.16586524990144	209285.709612143\\
0.165865249595725	209285.779967113\\
0.165592387890634	6320377.81016467\\
0.165592387889711	6320377.81016467\\
0.16559238788626	6320377.81016468\\
0.165592387881759	6320377.84127387\\
0.165592387884898	6320377.84127559\\
0.165592387862977	6320377.84131125\\
0.165592387860553	6320377.84131125\\
0.165592387869212	6320377.84954973\\
0.16557704813057	9627364.54640455\\
0.163389185242913	991494746.524078\\
0.163389184907265	991494746.524174\\
0.163389185745043	991494746.52418\\
0.16338918547278	991494746.524183\\
0.163389184612521	991494746.589091\\
0.163389184860328	991494746.589249\\
0.163389119995933	991523925.908413\\
0.163389119995325	991523925.912109\\
0.16338911988474	991523925.917647\\
0.163389120172619	991523925.917653\\
0.163389119707019	991524019.955072\\
0.163389120051949	991524019.955073\\
0.163389119496573	991524019.956041\\
0.163389119686054	991524019.956099\\
0.163389120101733	991524020.098339\\
0.163389119584881	991524020.098339\\
0.162339989060568	1512697891.98287\\
0.155581636073555	5275973439.26416\\
0.155581636798178	5275973735.26251\\
0.155581632726632	5275974826.85604\\
0.155581632726632	5275974826.85604\\
0.155581633755317	5275974827.3604\\
0.155581633057074	5275974827.3604\\
0.155581633425259	5275974827.37486\\
0.155581631730438	5275974827.38534\\
0.155581634869217	5275974839.91232\\
0.155581634869217	5275974839.91232\\
0.1555816347605	5275974839.95995\\
0.155581634032189	5275974840.00177\\
0.155581610335348	5275987998.54008\\
0.155581610764023	5275988317.47693\\
0.155581611129951	5275988317.47693\\
0.155581607798909	5275988317.48405\\
0.155581609187919	5275988317.48446\\
0.155581612333917	5275988317.48458\\
0.155581608787268	5275988317.5056\\
0.152573042251006	7097631192.08292\\
0.152558392566925	7107007359.47524\\
0.152461953199479	7168819145.59281\\
0.152461952067157	7168819147.98662\\
0.152461942596944	7168826065.16808\\
0.152461517162585	7169097816.56795\\
0.152461518120252	7169098463.39825\\
0.15246151735158	7169098466.4877\\
0.152461160087139	7169327656.16758\\
0.15246116090171	7169327671.21577\\
0.152461159798107	7169327671.21579\\
0.152461160669095	7169327671.22283\\
0.152461158369073	7169327737.42981\\
0.151518310899631	7780188329.09899\\
0.151518309750697	7780188816.6109\\
0.151456504427181	7820592497.96512\\
0.15145230896004	7823336542.95159\\
0.151305532659405	7919445523.449\\
0.151305531511215	7919445523.4492\\
0.151305532119187	7919445524.1106\\
0.151305531510335	7919445524.11232\\
0.151305532082251	7919446682.35396\\
0.151305530265246	7919446682.35439\\
0.151305529994997	7919446682.46127\\
0.151305531263123	7919446682.46183\\
0.15130553111002	7919446682.46322\\
0.15130553111002	7919446682.46322\\
0.151305530798759	7919446772.59789\\
0.151305528498315	7919447789.27872\\
0.151305527696188	7919447885.64415\\
0.15130552866033	7919447885.64679\\
0.151305530595215	7919447885.64698\\
0.151305529622663	7919447885.6477\\
0.15130552943387	7919447885.6477\\
0.15130552943387	7919447885.6477\\
0.151305528948322	7919447886.16516\\
0.151305502835901	7919465534.88737\\
0.151305500735983	7919465543.46152\\
0.151305502522872	7919465569.93895\\
0.15130550365024	7919465570.19492\\
0.15130549947697	7919466378.93199\\
0.151305499283386	7919466379.122\\
0.151305502053325	7919466583.65514\\
};
\addplot [color=mycolor1,only marks,mark=asterisk,mark options={solid},forget plot]
  table[row sep=crcr]{%
0.151305530265246	7919446682.35439\\
};
\end{axis}
\end{tikzpicture}%
\end{document}
% This file was created by matlab2tikz.
% Minimal pgfplots version: 1.3
%
%The latest updates can be retrieved from
%  http://www.mathworks.com/matlabcentral/fileexchange/22022-matlab2tikz
%where you can also make suggestions and rate matlab2tikz.
%
\documentclass[tikz]{standalone}
\usepackage{pgfplots}
\usepackage{grffile}
\pgfplotsset{compat=newest}
\usetikzlibrary{plotmarks}
\usepackage{amsmath}

\begin{document}
\definecolor{mycolor1}{rgb}{0.00000,0.44700,0.74100}%
\definecolor{mycolor2}{rgb}{0.85000,0.32500,0.09800}%
%
\begin{tikzpicture}

\begin{axis}[%
width=2in,
height=2in,
at={(0.758333in,0.48125in)},
scale only axis,
xmode=log,
xmin=0.0174574471131566,
xmax=0.0205051570776924,
xminorticks=true,
xlabel={$\|\mathbf{Ax} - \mathbf{b}\|$},
ymode=log,
ymin=0.01,
ymax=10000000000,
yminorticks=true,
ylabel={$\|\mathbf{Lx}\|$}
]
\addplot [color=mycolor1,solid,forget plot]
  table[row sep=crcr]{%
0.0205051570776924	0.326680000196856\\
0.0178522980583452	0.444290499754514\\
0.0177832328843643	1.4858840786064\\
0.0177832090191115	1.48646907088544\\
0.017691004655269	76.7696612417168\\
0.0176910046331534	76.7696794886441\\
0.0176909367460143	76.8673274690314\\
0.0175053065879238	6859.82408568443\\
0.0175053065879012	6859.8240856852\\
0.0175053065870278	6859.82411797467\\
0.0175053065860789	6859.82415360207\\
0.0175052566795015	6875.30348446564\\
0.0174841753260121	183365.488793526\\
0.0174841753261225	183365.488799106\\
0.0174841753261991	183365.488799321\\
0.0174841752348072	183366.277680114\\
0.0174841752346929	183366.277680228\\
0.0174841752348415	183366.27768059\\
0.0174841752348401	183366.27768059\\
0.0174841752351742	183366.277686948\\
0.0174765833413199	6621639.20493049\\
0.0174765833498356	6621639.20495112\\
0.0174765833437554	6621639.2095971\\
0.0174765833461569	6621642.27749889\\
0.0174765829478307	6621979.58123787\\
0.0174738150968662	9033124.76742856\\
0.0174738149665839	9033239.26937821\\
0.0174738149611898	9033241.94086775\\
0.0174650954751422	16627763.7050237\\
0.0174650954808502	16627763.7050354\\
0.0174650954703367	16627763.7051576\\
0.0174650954921543	16627763.708545\\
0.017465095490255	16627763.708545\\
0.0174650955060193	16627763.708545\\
0.0174650954946116	16627763.708545\\
0.0174650955065838	16627763.708545\\
0.017465095487287	16627763.708545\\
0.0174650954765902	16627763.7085501\\
0.0174650954829762	16627763.7085507\\
0.0174650954836031	16627763.7085658\\
0.0174650954818382	16627763.7230478\\
0.0174650954718573	16627763.7230478\\
0.0174650954718573	16627763.7230478\\
0.0174650955059133	16627764.3275163\\
0.0174650954852719	16627764.3275266\\
0.0174650954858851	16627764.3275417\\
0.0174650954936994	16627764.3286254\\
0.0174650954940209	16627764.3286254\\
0.0174650925212342	16636753.0263361\\
0.017465092558798	16636753.3711398\\
0.0174650925359754	16636755.2399845\\
0.0174650925246761	16636755.2399849\\
0.0174650925070553	16636755.2417381\\
0.0174650925340003	16636755.2426454\\
0.0174650925579845	16636755.2437481\\
0.0174650925476148	16636823.3765994\\
0.0174650925055906	16636823.3768517\\
0.0174650925455183	16636823.3768575\\
0.0174636549792529	225871232.662138\\
0.0174636551957565	225872783.292211\\
0.0174636554438491	225872867.339743\\
0.0174636555951997	225877524.920736\\
0.0174636553628181	225877800.236176\\
0.0174636523672546	226296705.960142\\
0.0174635328037053	244909703.892389\\
0.0174574571687097	1194417320.40523\\
0.0174574563252205	1194418237.89759\\
0.0174574554284868	1194418268.14316\\
0.0174574559497688	1194418304.914\\
0.0174574575964584	1194423749.9839\\
0.0174574547287141	1194522589.67961\\
0.0174574483672123	1195465585.94301\\
0.0174574481892611	1195474595.93759\\
0.0174574489084936	1195474595.93761\\
0.0174574489084936	1195474595.93761\\
0.0174574489084936	1195474595.93761\\
0.0174574492131685	1195474595.93761\\
0.0174574493286395	1195474595.93761\\
0.0174574493286395	1195474595.93761\\
0.0174574485841133	1195474595.93784\\
0.0174574485841133	1195474595.93784\\
0.0174574484849557	1195474595.93784\\
0.017457448020606	1195474595.96856\\
0.0174574483172658	1195474595.9686\\
0.0174574483014087	1195474595.9686\\
0.0174574483014087	1195474595.9686\\
0.0174574483014087	1195474595.9686\\
0.0174574483014087	1195474595.9686\\
0.0174574483014087	1195474595.9686\\
0.0174574472386037	1195474595.96864\\
0.0174574471131566	1195474595.96864\\
0.0174574486037465	1195474595.96864\\
0.0174574486743451	1195474595.96864\\
0.0174574484021248	1195474595.9817\\
0.0174574492527223	1195474595.9871\\
0.0174574473622785	1195474595.98723\\
0.0174574479796539	1195474597.32141\\
0.0174574485946848	1195474597.32142\\
0.0174574493943787	1195474597.32142\\
0.0174574493690364	1195474597.32142\\
};
\addplot [color=mycolor1,only marks,mark=asterisk,mark options={solid},forget plot]
  table[row sep=crcr]{%
0.0174650954765902	16627763.7085501\\
};
\end{axis}
\end{tikzpicture}%
\end{document}
% This file was created by matlab2tikz.
% Minimal pgfplots version: 1.3
%
%The latest updates can be retrieved from
%  http://www.mathworks.com/matlabcentral/fileexchange/22022-matlab2tikz
%where you can also make suggestions and rate matlab2tikz.
%
\documentclass[tikz]{standalone}
\usepackage{pgfplots}
\usepackage{grffile}
\pgfplotsset{compat=newest}
\usetikzlibrary{plotmarks}
\usepackage{amsmath}

\begin{document}
\definecolor{mycolor1}{rgb}{0.00000,0.44700,0.74100}%
\definecolor{mycolor2}{rgb}{0.85000,0.32500,0.09800}%
%
\begin{tikzpicture}

\begin{axis}[%
width=2in,
height=2in,
at={(0.758333in,0.48125in)},
scale only axis,
xmode=log,
xmin=0.0001,
xmax=1,
xminorticks=true,
xlabel={$\|\mathbf{Ax} - \mathbf{b}\|$},
ymode=log,
ymin=0.01,
ymax=100000000,
yminorticks=true,
ylabel={$\|\mathbf{Lx}\|$}
]
\addplot [color=mycolor1,solid,forget plot]
  table[row sep=crcr]{%
0.471789297598319	0.885747357209502\\
0.0267353251717842	1.17633615533863\\
0.00112381113121549	1.23558761351457\\
0.000885258163012169	1.24413514411972\\
0.000885158604863073	1.24413845837224\\
0.000885152318166375	1.24414370000063\\
0.000884841959964588	1.24817427790217\\
0.000884841959964488	1.24817427791635\\
0.000884841959950662	1.24817425719791\\
0.000884824725413784	1.40146991862623\\
0.000884824285291882	1.40893473877401\\
0.000884824285291898	1.40893473881276\\
0.000884820430572719	1.51615586853093\\
0.000884820418295433	1.51661917682551\\
0.000883344305924437	137.192852650042\\
0.000883288645407155	142.337745341164\\
0.00088262472429371	203.682240323734\\
0.000882624724294111	203.682240323735\\
0.000882624724292667	203.682240323736\\
0.000882624715526116	203.683050885706\\
0.000882624706567131	203.683879948177\\
0.000882624706565797	203.683879951238\\
0.000882624706491586	203.683887054599\\
0.00088262468832142	203.68557292251\\
0.000882624688300224	203.685574576306\\
0.000875707452532171	12129.8041685515\\
0.000874926010059626	13492.5380251416\\
0.000874926007920108	13492.5417168657\\
0.000874926002847029	13492.5508628366\\
0.000874926002763647	13492.5508628366\\
0.000874926002815624	13492.5508628366\\
0.000874926002530391	13492.5508628423\\
0.000874926002725744	13492.5508628761\\
0.000874926002526761	13492.5508628764\\
0.000874926002681885	13492.5508722221\\
0.000874926002567619	13492.5508722221\\
0.000874926002598376	13492.5508722221\\
0.000874926002607254	13492.5508722221\\
0.000874926002680589	13492.5508723095\\
0.000874926002683413	13492.5508723095\\
0.000874926002653458	13492.550872317\\
0.000874926002919322	13492.5508726544\\
0.000874925999764453	13492.5562549174\\
0.000874925948614305	13492.6559839845\\
0.000874611215479282	48258.9974899855\\
0.000873614662098382	192830.535938103\\
0.000872882254988897	299873.989828875\\
0.000872880947230159	300065.296357882\\
0.000872880949147806	300065.29662128\\
0.000872880945719057	300065.296638332\\
0.000872880944952617	300065.296639587\\
0.000872880947475808	300065.313865956\\
0.000872880944104782	300065.314107216\\
0.000872880944797183	300065.315440185\\
0.00087288094397415	300065.315665459\\
0.000872880947498919	300065.324843558\\
0.000872880945645295	300065.324843847\\
0.000872880945408558	300065.324845464\\
0.000872880928697937	300067.655432138\\
0.000872880532257285	300125.8324753\\
0.000872880512053981	300128.817822199\\
0.000872874935167362	300943.988524391\\
0.000872874931354199	300944.278370145\\
0.000872874929959224	300944.278370147\\
0.000872874929959224	300944.278370147\\
0.000872874931325457	300944.278370203\\
0.000872874931429278	300944.278370204\\
0.000872874930200961	300944.278370204\\
0.000872874930137549	300944.278370204\\
0.000872874930113536	300944.278377524\\
0.000872874932793405	300944.279018128\\
0.000872874932870273	300944.279018389\\
0.000872874932870273	300944.279018389\\
0.000872874932198061	300944.279018389\\
0.000872874932252709	300944.279018389\\
0.00087287493226598	300944.279018389\\
0.000872874933774309	300944.279018438\\
0.00087287493394782	300944.279018438\\
0.000872874931195336	300944.279018438\\
0.000872874932497408	300944.279018439\\
0.00087287493261657	300944.279018439\\
0.000872874932446128	300944.279018439\\
0.00087287493473564	300944.279018441\\
0.000872874929247646	300944.279018937\\
0.000877345043516297	3177145.91339947\\
0.000880849087537883	5615897.26157736\\
0.000884676302579829	8294716.85006759\\
0.00096442078872218	66803832.8086112\\
0.000964421004736472	66803852.1810208\\
0.000964421102453123	66803977.7304055\\
0.000965000399504186	67247439.9942944\\
0.000965000308763836	67247464.3150963\\
0.000965001109554747	67248072.9411962\\
0.000965001048852418	67248107.0257837\\
0.000965001094927996	67248107.0264495\\
0.000965001290390202	67248107.0447308\\
0.000965001408338703	67248107.2618853\\
0.000965001206634034	67248107.5010851\\
0.000965001238194624	67248116.3069594\\
0.000965001183896248	67248116.3069724\\
};
\addplot [color=mycolor1,only marks,mark=asterisk,mark options={solid},forget plot]
  table[row sep=crcr]{%
0.000882624706567131	203.683879948177\\
};
\end{axis}
\end{tikzpicture}%
\end{document}
% This file was created by matlab2tikz.
% Minimal pgfplots version: 1.3
%
%The latest updates can be retrieved from
%  http://www.mathworks.com/matlabcentral/fileexchange/22022-matlab2tikz
%where you can also make suggestions and rate matlab2tikz.
%
\documentclass[tikz]{standalone}
\usepackage{pgfplots}
\usepackage{grffile}
\pgfplotsset{compat=newest}
\usetikzlibrary{plotmarks}
\usepackage{amsmath}

\begin{document}
\definecolor{mycolor1}{rgb}{0.00000,0.44700,0.74100}%
\definecolor{mycolor2}{rgb}{0.85000,0.32500,0.09800}%
%
\begin{tikzpicture}

\begin{axis}[%
width=2in,
height=2in,
scale only axis,
xmode=log,
xmin=0.0001,
xmax=10000,
xminorticks=true,
xlabel={$\|\mathbf{Ax} - \mathbf{b}\|$},
ymode=log,
ymin=1,
ymax=100,
yminorticks=true,
ylabel={$\|\mathbf{Lx}\|$}
]
\addplot [color=mycolor1,solid,forget plot]
  table[row sep=crcr]{%
325.618682118172	6.43353764156103\\
178.944698884963	12.3823573964926\\
56.1550887658752	16.8953396116949\\
10.2339557527921	18.1701664404548\\
5.69032375884836	18.2770005850107\\
3.57062097941321	18.403800686646\\
1.1252566745652	18.7019078388213\\
1.12226472933483	18.702314721229\\
0.230505102254373	18.8542753642863\\
0.230494644428051	18.8543058477654\\
0.172281248463876	18.869928134217\\
0.159873278271278	18.8776953835756\\
0.0916080052370302	18.9458284360697\\
0.0824569615539435	18.956075705185\\
0.0306087481087545	18.996887354572\\
0.0306040013460867	18.9968999495647\\
0.030600266524901	18.996895717363\\
0.0248841445941383	19.0175009196578\\
0.0248840471894563	19.0175015134883\\
0.0248839973317798	19.017503619875\\
0.0248827457459675	19.0175186996731\\
0.00943763944969404	19.1901374402228\\
0.0094281035917642	19.1902467381028\\
0.00942804514105531	19.1902476472893\\
0.00942801756802288	19.1902479764179\\
0.00941657095491969	19.1902952401505\\
0.00906388642950325	19.1946889698583\\
0.00788570949692592	19.2214836543195\\
0.00763228506703693	19.2291663429763\\
0.00763194612588802	19.2291772322795\\
0.00140005969051997	19.4288146102859\\
0.00113282769065014	19.4321756961138\\
0.00113282760464188	19.4321756809832\\
0.00113282720725327	19.4321756523276\\
0.00113282720305588	19.4321756530633\\
0.0011328271330532	19.4321756658192\\
0.00112885336204244	19.4322132937751\\
0.00112885271518879	19.4322133179843\\
0.00108054870205049	19.434061213713\\
0.000999398929420708	19.4415363738107\\
0.000999318170776651	19.441546178826\\
0.000999310278740157	19.4415477316222\\
0.000999309997643247	19.4415477554979\\
0.00099930992651127	19.4415477592373\\
0.000999309926438932	19.4415477594776\\
0.000999309580955998	19.4415479906784\\
0.000999298489427019	19.4415554524246\\
0.00099929106071583	19.4415556029425\\
0.000999289384864246	19.4415557036028\\
0.000999289291216114	19.4415557265304\\
0.0009992892489551	19.4415557354273\\
0.000999270809304266	19.4415579759496\\
0.000999236905334391	19.4415622701688\\
0.000852954997441731	19.5319386181085\\
0.000825063897832487	19.5624059838157\\
0.000825063897674229	19.5624059839559\\
0.00082506370496105	19.5624061762942\\
0.000825063324325356	19.5624065306663\\
0.000825062986787066	19.5624068237378\\
0.000825061424026786	19.5624078962239\\
0.000825061423809585	19.5624078963519\\
0.000825061423658916	19.5624078960615\\
0.000825061423090966	19.5624078974085\\
0.000825061083197956	19.5624081475296\\
0.00082506108153405	19.5624081485958\\
0.000825061060799751	19.5624081712089\\
0.000825061060421893	19.5624081716381\\
0.000825061060168832	19.5624081712504\\
0.000808682130499051	19.6593932721198\\
0.000799327895964074	19.7819595541305\\
0.0007992707891405	19.7828509413229\\
0.000798705789976761	19.7917624647398\\
0.000798702543738342	19.7918141771301\\
0.000798702543697955	19.7918141773127\\
0.000798702543715161	19.7918141775624\\
0.000798702543311312	19.7918141687644\\
0.000798702536861667	19.7918140195845\\
0.000798702480042134	19.7918125823902\\
0.00079870239751696	19.7918105030565\\
0.00079870239749908	19.7918105024171\\
0.000798702396850761	19.7918104962908\\
0.000798702391791676	19.7918104486616\\
0.000798702391484473	19.7918104448455\\
0.000798702387191448	19.7918104078169\\
0.000798702387196335	19.7918104078186\\
0.000798702387194678	19.7918104078221\\
0.00079870238717861	19.7918104078222\\
0.000798702377718933	19.7918107411096\\
0.000798702350588735	19.7918116813089\\
0.000798702349932346	19.7918117063472\\
0.000798702330686359	19.7918124095268\\
0.000798702212329311	19.7918167343567\\
0.000798702211167212	19.7918167773602\\
0.000798693905653706	19.7921915331897\\
0.00079869369960169	19.7922027097513\\
0.000798693699562511	19.7922027116862\\
0.000798693699534586	19.7922027126582\\
0.000798693699526242	19.7922027134527\\
0.000798693699431636	19.7922027202679\\
0.000798693699377476	19.7922027216808\\
};
\addplot [color=mycolor1,only marks,mark=asterisk,mark options={solid},forget plot]
  table[row sep=crcr]{%
0.000798693699526242	19.7922027134527\\
};
\end{axis}
\end{tikzpicture}%
\end{document}
% This file was created by matlab2tikz.
% Minimal pgfplots version: 1.3
%
%The latest updates can be retrieved from
%  http://www.mathworks.com/matlabcentral/fileexchange/22022-matlab2tikz
%where you can also make suggestions and rate matlab2tikz.
%
\documentclass[tikz]{standalone}
\usepackage{pgfplots}
\usepackage{grffile}
\pgfplotsset{compat=newest}
\usetikzlibrary{plotmarks}
\usepackage{amsmath}

\begin{document}
\definecolor{mycolor1}{rgb}{0.00000,0.44700,0.74100}%
\definecolor{mycolor2}{rgb}{0.85000,0.32500,0.09800}%
%
\begin{tikzpicture}

\begin{axis}[%
width=2in,
height=2in,
at={(0.758333in,0.48125in)},
scale only axis,
xmode=log,
xmin=0.0001,
xmax=10,
xminorticks=true,
xlabel={$\|\mathbf{Ax} - \mathbf{b}\|$},
ymode=log,
ymin=2.65945956450861,
ymax=3.0005989157135,
yminorticks=true,
ylabel={$\|\mathbf{Lx}\|$}
]
\addplot [color=mycolor1,solid,forget plot]
  table[row sep=crcr]{%
3.55113850720628	2.65945956450861\\
1.30826865206281	2.88901341923953\\
0.225727323028886	2.98341650747697\\
0.00813419794471326	2.99915050492251\\
0.00806600768556215	2.99915130502746\\
0.00740511742279889	2.99917939383245\\
0.00143222646183879	2.99992430425325\\
0.00133327676931131	2.99993080724588\\
0.000750870876807221	2.99997971973674\\
0.000750036859309753	2.99998436856338\\
0.000691109377663926	2.99998698729311\\
0.000690623893054078	2.99999037786\\
0.000233302711788067	3.00006214643064\\
0.000232956621985923	3.00006057216515\\
0.000232947980633233	3.00006056913888\\
0.000185907497086888	3.00007416060569\\
0.000184495917829518	3.00007474415578\\
0.000182961179151621	3.00007609454589\\
0.000182799311979011	3.00007724519372\\
0.0001820335387111	3.00007579263366\\
0.00018199966818117	3.00007579793098\\
0.000179854675481559	3.00007360241548\\
0.000171868584015347	3.00008123480892\\
0.00017178936984776	3.00008127267861\\
0.000166774383537448	3.00008722189777\\
0.000166702830670936	3.00008798178158\\
0.000166701872232647	3.0000879681401\\
0.000166697620273927	3.00008812701578\\
0.000165092740623189	3.00009155116631\\
0.000165083869732065	3.00009158401162\\
0.000165083575177338	3.00009154452705\\
0.000161385065240146	3.00010985738432\\
0.000161020928875923	3.00011264613533\\
0.000160991625624416	3.00011287928063\\
0.000160984475258503	3.00011312581826\\
0.000160979196446798	3.00011300395105\\
0.000160153684594614	3.00012037729633\\
0.000160128754158477	3.00012028243796\\
0.000160127453077664	3.0001202562749\\
0.000159469506822067	3.00012708084178\\
0.000159468843744611	3.00012714778853\\
0.000159448503791738	3.00012705299873\\
0.000159432881711012	3.00012697986645\\
0.000159383426400637	3.00012808134504\\
0.000158338518447016	3.00014286185199\\
0.000158187770893746	3.00014548455986\\
0.000158187299726534	3.00014549276863\\
0.000158116035705626	3.00014651807309\\
0.000158104569738225	3.00014642371369\\
0.000158078681716008	3.00014703515469\\
0.000158001900191441	3.00014763565995\\
0.000157916483549647	3.00014927307685\\
0.000157054642149113	3.00016862664021\\
0.000157052096290164	3.00016880169154\\
0.000156999392152526	3.00016997161289\\
0.000156987973195369	3.00017024779036\\
0.000156983436364483	3.0001703579966\\
0.000156957457019428	3.00017099192553\\
0.000156576099108085	3.00018133558926\\
0.000156575609245473	3.00018131966681\\
0.000156575556379719	3.00018131978283\\
0.00015657551286277	3.00018134036318\\
0.000156572982808391	3.0001813019676\\
0.000156077209102752	3.00019790929894\\
0.000155692791432727	3.00021470917683\\
0.000155444253523982	3.00022604829856\\
0.000155444235928024	3.00022604900421\\
0.000155444230612176	3.00022604963817\\
0.000155438379785562	3.00022655184741\\
0.000155330547078103	3.00023190936258\\
0.000155273753834664	3.00023543722568\\
0.000154984869980154	3.00025319815667\\
0.000154982012757529	3.00025350803095\\
0.000154977601808475	3.00025379925275\\
0.00015446832076411	3.00029455982031\\
0.000154297976369724	3.00030951274075\\
0.000153631398551433	3.0003776914699\\
0.000153627411833251	3.00037828790845\\
0.000153626529599492	3.00037837159001\\
0.000153626526596813	3.00037836801054\\
0.000153608611500208	3.00037991309146\\
0.000153602259414569	3.00038063854073\\
0.000153601952341506	3.00038070608377\\
0.000153601869487614	3.00038070346454\\
0.00015358681899823	3.00038261055456\\
0.000153586800851986	3.00038260834263\\
0.00015358542753467	3.00038284684627\\
0.0001528628323931	3.0004812755553\\
0.000152862829433809	3.00048127213873\\
0.000152862711967331	3.00048129194601\\
0.000152861202010705	3.00048152089748\\
0.000152843099629059	3.00048427434856\\
0.000152832185878327	3.00048615755315\\
0.000152769881319994	3.00049601241336\\
0.000152736391669966	3.0005009358908\\
0.000152730038591307	3.00050203247092\\
0.000152717747896762	3.00050399205403\\
0.000152344353445185	3.00056857008546\\
0.000152240407162277	3.00058927688376\\
0.000152190277426246	3.0005989157135\\
};
\addplot [color=mycolor1,only marks,mark=asterisk,mark options={solid},forget plot]
  table[row sep=crcr]{%
0.000152344353445185	3.00056857008546\\
};
\end{axis}
\end{tikzpicture}%
\end{document}
\caption{L-curve plots of the original CG-results on the five remaining pairs in increasing order, with edge approximation indicated by a star. The approximation comes from smoothed data, which is not shown.}
\label{fig:cgRest}
\end{figure}
L curve plots from CG application to the remaining five data sets are shown in figure~\ref{fig:cgRest}. L-curve shapes only appear for the sets two, three and four. For $\mathbf{A}_5, \mathbf{b}_{err5}$ and $\mathbf{A}_6, \mathbf{b}_{err6}$ the curve is an inverted A. Which indicates that the conjugate gradient method can only minimize residual only if $\|\mathbf{Lx}\|$ becomes large, which implies undesired tracking of the noise contributions.  
For $\mathbf{A}_2, \mathbf{b}_{err2}$, $\mathbf{A}_3, \mathbf{b}_{err3}$ and $\mathbf{A}_4, \mathbf{b}_{err4}$, the CG-algorithms produces an L-shaped curve. 
For the second case the reduction of the solution norm is insufficient. In the third and fourth case results are comparable to Thikonov regularization, however the approximations of the proper number of iterations could be more precise.

